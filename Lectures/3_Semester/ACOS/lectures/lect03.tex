\section{Целочисленная арифметика}

\subsection{Уровни абстракции <<железа>>}
\begin{itemize}
    \item <<p>> и <<n>> полупроводники, проводники, конденсаторы и резисторы
    \item Транзисторы и диоды
    \item Логические элементы
    \item Арифметическо-логические юниты
    \item Центральный процессор
    \item Компьютерная система
\end{itemize}

Компьютер состоит из элементарных блоков, из которых строятся электрические
элементы. Они в свою очередь образуют логические элементы.

Ключевая часть компьютера - центральный процессор. А главная часть ЦП - арифметическо-логический блок, который выполняет все вычисления.

\subsection{Типы сигналов}

\begin{Def}
    \underline{Сигналы} --- это изменения напряжения в электрической сети. Их можно регистрировать
и дальше обрабатывать.
\end{Def}

Сигналы бывают двух видов:
\begin{itemize}
    \item \textbf{Аналоговый.} Сила сигнала изменяется непрерывно с изменением напряжения.
    \item \textbf{Цифровой.} Есть всего два состояния - высокий, либо низкий.
\end{itemize}


\subsection{Дискретные сигналы}

\begin{itemize}
    \item \textbf{Логический <<0>>.} Все, что ниже определенного значения
        напряжения --- это нулевой уровень.
    \item \textbf{Логическая <<1>>.}
    \item \textbf{Неопределенное состояние.} Напряжение между 0 и 1, не приводящее к изменению состояния.
    \item \textbf{Hi-Z состояние.} Физически какой-то контакт внутри процессора отключается от общей схемы, чтобы не оказывать влияние на все остальные компоненты.
\end{itemize}

\subsection{Арифметико-логический юнит}

Базовый юнит центрального процессора для реализации операций над числами:
\begin{itemize}
	\item Битовая логика
	\item Побитовые сдвиги
	\item Сложение и вычитание
	\item Умножение и деление
\end{itemize}

 Управление ходом программы --- тоже арифметические операции. 
 
 \subsection{Представление чисел и переполнения}
 Было на семинарах...
 
 \subsection{Значения из нескольких байтов}
 
 \begin{itemize}
 	\item Минимальный адресуемый обьем данных --- 8 бит.
 	\item За один такт процессора он может адресовать одно \textbf{машинное слово} процессора - 32 или 64 бита.
 \end{itemize}

\subsection{Big-Endian и Little-Endian}

\begin{itemize}
	\item \textbf{Big-Endian:}
	Выглядит как человеческое представление байтов.
	сначала старшие байты, потом - младшие
	Пример: большинство RISC процессоров по умолчанию.
	\item \textbf{Little-Endian:}
	сначала младшие байты, потом старшие
	Пример: Процессоры Intel и ARM.
\end{itemize}

\subsection{Ковертирование endian-ов}

Беззнаковые типы можно конвертировать просто побитовыми сдвигами. Достаточно переставить байты в обратном порядке.

Со знаковыми типами есть проблемы, потому что там первый бит отвечает за знак.

\subsection{Упаковка структур}

Размер любой структуры должен быть кратен размеру машинного слова. Это связано с оптимизацией времени чтения/записи. 

Поэтому если в структурке хранится два поля - на 16 бит и на 8 бит, то стуктура будет занимать 32 бита, а не 16 + 8 = 24. Ведь размер машинного слова в x86 равен 32 битам.

Если же прописать у структуры специальный атрибут, компилятор уложит поля вплотную.

\subsection{Где используется атрибут packed?}

\begin{itemize}
	\item Бинарные данные
	\item Сетевые пакеты
	\item Драйвера
\end{itemize}

В остальных случаях использование packed ненужно и даже вредно.
