\section{Операционные системы. Файловые системы.}

\subsection{Цель операционных систем}

Глобальная цель --- абстрагировать все оборудование, на котором работают
пользовательские приложения через некоторую прослойку.

Разработчики не должны писать софт под каждый компьютер по отдельности, 
операционные системы помогают решить такие проблемы как:
\begin{itemize}
	\item Вычислительные системы могут иметь разные архитектуры процессора.
	\item Разные модели памяти.
\end{itemize}

\subsection{Основные компоненты}

\begin{itemize}
	\item Ядро.
	\item Минимальный набор библиотек и инструкментов.
	\item Высокоуровневые библиотеки и фреймворки.
	\item Окружение рабочего стола.
\end{itemize}

\subsection{Абстракции уровня API}

\underline{Стандартные библиотеки}: C17, C++17.

\underline{Взаимодействие с ОС}: Portable Operating System Interface --- POSIX (в основном системные вызовы).

\subsection{Реализации абстракций уровня API}

\underline{Стандартные библиотеки}: libstdc++ (Linux, *BSD), msvcp.dll (Windows) 
или libc++ (MacOS).

\underline{Реализации API POSIX стандарта}: glibc, bionic, ...


\subsection{Использование библиотек}

\begin{itemize}
	\item Внутри программы библиотеки загружаются с помощью ld.so.
	\item Содержимое библиотек живет внутри адресного пространства процесса.
	\item Функции должны вызываться с помощью Procedure Linkage Table.
\end{itemize}


\subsection{Взаимодействие ядром}

Системные вызовы можно использовать с помощью прерываний. 

При этом не для всех системных вызовов нужно менять режим исполнения ядра. 
Системные вызовы \_\_vdso\_clock\_gettime, ... (x86\_64) функционируют как обычные функции.
\textit{Узнать больше: man 7 vdso.}

\subsection{Подсистемы ядра}

\begin{itemize}
	\item Драйвера
	\item Управление памятью.
	\item Планировщик процессов и потоков.
	\item Межпроцессорное взаимодействие
	\item Поддержка файловых систем.
\end{itemize}

\subsection{UNIX Виртуальная файловая система}

В отличае от Windows и других стремных систем тут нет дисков. \textbf{Вся файловая система 
организована в виде дерева.} Совершенно неважно, как вы подключили устройство, вы всегда
работаете единообразным образом.

\subsection{Типы файлов в UNIX}

\begin{itemize}
	\item Обычный файл
	\item Каталог
	\item Устройство. Бывают блочные утройства и символьные устройства.
	\item Символические ссылки.
	\item Именованый канал (pipe).
	\item Сокет.
\end{itemize}

\subsection{Обычные файлы}

\begin{itemize}
	\item Просто хранит данные.
	\item Произвольный доступ к данным с помощью lseek.
\end{itemize}

\subsection{Каталоги}

\begin{Def}
	\underline{Каталог} --- просто файл определенного формата, чтобы хранить содержимое
	директории.
\end{Def}

\begin{itemize}
	\item Директория состоит из struct dirent.
	\item POSIX определяет содержимое struct dirent:
		\begin{itemize}
			\item inode, т.е. уникальный идентификатор определенного файла.
			\item размер записи.
			\item имя файла.
		\end{itemize}
\end{itemize}

\subsection{Ссылки}

\textbf{Символическая ссылка} ведет себя практически точно также, как и тот файл, на который
она ссылается.

\begin{Def}
	\underline{Жесткая ссылка} --- это другое имя того же файла. Это тоже файл, имеющий тот
	же inode (идентификатор файла). Жесткая ссылка увеличивает количество ссылок на файл.
	Когда ссылок станет 0, файл будет реально удален.
\end{Def}

\subsection{Устройства}

Блочные устройства --- диски. Символьное устройство отличается от блочного тем, что
тут мы можем только писать и читать, но не перемещаться по содержимому.


\subsection{FIFO и сокеты}

FIFO файлы используются для простейшего межпроцессорного взаимодействия.

Сокет --- более сложная концепция. Использовать ее надо через системные вызовы (например, connect).

\subsection{Файловые системы}

Каждая файловая система:
\begin{itemize}
	\item подмонтируется к некоторому поддереву.
	\item самодостаточна
\end{itemize}

\subsection{Типы физических файловых систем}

\begin{itemize}
	\item Дисковые: FAT, NTFS, ...
	\item Без структуры: SWAP	
	\item Cетевые
	\item Виртуальные
\end{itemize}

\subsection{Адресация файлов}

C каждым файлом связано два числа: inode и номер устройства.
\begin{Def}
	\underline{inode} --- ключ файла внутри одной файловой системы
\end{Def}

\subsection{Физическая память}

\begin{itemize}
	\item HDD --- дешево, но есть накладные расходы по времени на перемещение головки.
	\item SSD --- флэш память. Работают сильно быстрее, чем HDD. Но есть ограничение на 
	количество циклов записи, поэтому их надо менять.
\end{itemize}


\subsection{Концепции файловой системы}

\begin{itemize}
	\item Уровень пользователя: доступ к файлу по его имени (string).
	\item Уровень процесс: открыть файловый дескриптор (int).
	\item Уровень ядра: запись в виртуальной файловой системе (st\_dev и inode).
	\item Уровень файловой системы: поиск данных по inode.
\end{itemize}

\subsection{Linux файловые системы для дисков}

\begin{itemize}
	\item ext2/3/4 --- самые надежные.
	\item XFS --- подходит для хранения больших файлов.
	\item ReiserFS --- наиболее подходящая для большого количества маленьких файлов.
\end{itemize}


