\section{Прерывания и системные вызовы}

\begin{Def}
	\underline{Прерывание} --- это сигнал от программного или аппаратного обеспечения,
	сообщающий процессору о наступлении какого-либо события, требующего немедленного внимания.
\end{Def}

\subsection{Обработка прерывания в x86}

\begin{itemize}
	\item Каждое устройство имеет свой собственный номер прерывания (IRQ).
	\item Процессор получает сигнал INT и прекращает исполнение текущекого контекста.
	\item Каждое прерывание имеет свой адрес в векторе прерываний.
	\item Прерывания бывают не только аппаратные. Его можно сымитировать, 
	вызвав прерывание программно.
\end{itemize}

\subsection{Маска прерывания}

Иногда хочется запретить прерывания в каком-то месте. Для этого существует маска прерываний.

\subsection{Номера прерываний}

\begin{itemize}
	\item 0 --- нажатие на кнопку включения или Reset
	\item 1...15 --- Аппаратные прерывания. Тут задействованы реальные железяки.
	\item 16+ --- программные прерывания, вызванные через int.
\end{itemize}

\subsection{Базовый вектор прерываний}

Кто прописывает нам функции, которые будут вызваны при прерывании?
\begin{itemize}
	\item BIOS --- содержит весь минимальный набор функций, нужный после включения.
	\item Обработка железяк (клавиатура, ...).
	\item Функции для обращения к данным на диске и загрузки операционной системы.
\end{itemize}

\subsection{Обработка прерываний}

\begin{itemize}
	\item Сохранить EIF на стек
	\item Проставить 1 в флаг IF
	\item Перейти на инструкцию по адресу IDTR + offset
\end{itemize}

\underline{Перед вызовом обработчика прерывания:}
\begin{itemize}
	\item Сбросить флаги
	\item Переключить текущее адресное пространство
	\item Заменяется стек
\end{itemize}

\underline{Выполнение функции обработчика прерывания:}
\begin{itemize}
	\item Сохранить текущее значение регистров и флагов
	\item что-то поделать...
	\item бернуть состояние регистров и флагов обратно
\end{itemize}

\subsection{Ядро}

При запуске операционной системы есть одна программа - ядро. Она может делать все что угодно.

\subsection{Режимы исполнения (x86)}

\textbf{Обычный режим:}
\begin{itemize}
	\item Процесс имеет доступ только к своей собственной памяти.
	\item Виртуальное адресное пространство каждого процесса начинается с адреса 0.
	\item Процесс не может взаимодействовать с устройствами.
\end{itemize}

\textbf{Превилегированный режим:}
\begin{itemize}
	\item Полный доступ к физической (не виртуальной) памяти.
	\item Полный доступ к портам ввода/вывода.
	\item Некоторые дополнительные команды, не доступные в обычном режиме исполнения.
\end{itemize}

\begin{Def}
	\underline{Системный вызов} --- механизм для взаимодействия обычных процессов с ядром. 
	Он позволяет обратиться к функциональности ядра, чтобы сделать недоступные в обычном
	режиме операции.
\end{Def}

\subsection{Ядро}

\begin{itemize}
	\item Обычный ELF файл.
	\item Запускается в привилегированном режиме.
	\item Отвечает за взаимодействие с железом.
\end{itemize}

\subsection{Запуск ядра}

\begin{itemize}
	\item Проинициализировать устройства.
	\item Проинициализировать вектор прерываний.
	\item Найти, загрузить и запустить все драйвера.
	\item Понизить привилегии процессора.
	\item Запустить первый пользовательский процесс --- init.
\end{itemize}

\subsection{Взаимодействие с ядром (x86)}

\begin{itemize}
	\item Все процессы непривилегированы.
	\item Единственный способ получить доступ к железу --- переключиться в 
	режим ядра.
	\item Все прерывания переключают процессор в привилегированный режим.
	\item Используйте int для того, чтобы получить доступ к системным вызовам.
\end{itemize}

\subsection{Типы архитектуры ядер}

\begin{itemize}
	\item \textbf{Монолитные ядра.} Одна большая программа, которая исполняется 
	в привилегированном режиме. Примеры: Linux.
	\item \textbf{Микроядерная.} Только небольшая часть ядра запускается в привилегированном
	режиме. Большая часть подсистем работает как пользовательские процессы. Примеры: Minix3.
	\item \textbf{Гибридная.} Модульная, но не одна большая программа, запускаемая в
	привилегированном режиме. Примеры: Windows, Mac, BSD или Linux.
\end{itemize}

\subsection{Системные вызовы}

Системный вызов можно осуществить с помощью:
\begin{itemize}
	\item int 0x80 (Linux, BSD)
	\item sysenter/sysleave инструкции (Intel)
	\item syscall/sysret инструкции (AMD64/x86\_64)
\end{itemize}

\subsection{INT 0x80}

\begin{itemize}
	\item В eax сохраняется номер системного вызова
	\item Параметры сохраняются в ebx, ecx, ...
	\item Возвращаемое значение сохраняется в eax
	\item Конвенции вызовов отличаются от принятых в языке C!
\end{itemize}






