\section{Команды процессора}

\subsection{Процессор}

Процессор умеет только манипулировать с целыми числами. Значение адреса - тоже число.

\subsection{Процессор с точки зрения пользователя}

У процессора есть:
\begin{itemize}
	\item Регистры - eax, ebx, ...
	\item Флаги - ZF, CF, SF
	\item Указатель на текущую команду (PC - pointer count)
\end{itemize}

\subsection{Команды x86}

nop - 0x00
halt - 0x10
call Dest - 0x80, Dest - 5 байтов
...

Команды кодируются переменным количеством байт. Количество байт может доходить до 6.

\subsection{Представление структур программы в виде простых инструкций}

Было на семинарах, на ассембелере писать уже должны уметь...

\subsection{CISC}

\begin{Def}
	\underline{CISC} --- \textbf{C}omplex \textbf{I}nstruction \textbf{S}et \textbf{C}omputing
\end{Def}

Процессоры: x86, Z80, PDP-11/VAX

Это тип архитектуры процессора, где:
\begin{itemize}
	\item Команд много, на все случаи жизни
	\item Кодируются переменным количеством байт
	\item Разные режимы адресации
	\item Упрощают жизнь программисту на ассемблере
\end{itemize}

Например, инструкция \textbf{loop} Address позволяет реализовать цикл одной командой.

\subsection{Конвейер}

\begin{itemize}
	\item Загрузка инструкции
	\item Декодирование команды
	\item Выполнение
	\item Доступ к памяти
	\item Записать результат в регистр
\end{itemize}

\subsection{Проблемы ковейеризации}

\begin{itemize}
	\item Инструкции имеют разную длину в байтах
	\item Разные инструкции выполняются за разное число тактов
	\item Работа с памятью и с регистрами задействует разные блоки процессора
\end{itemize}

Одна из задач компилятора с точки зрения оптимизации - минимизировать время простоя при конвейеризации. Для этого компилятор может менять порядок команд, если это не изменит результата, но поможет с ковейеризацией.

\subsection{RISC}

\begin{Def}
	\underline{RISC} --- \textbf{R}educed \textbf{I}nstruction \textbf{S}et \textbf{C}omputing
\end{Def}

Процессоры: ARM, AVR, MIPS, PowerPC

\begin{itemize}
	\item Только простейшие инструкции
	\item Инструкции фиксированной длины
	\item Почти у всех инструкций одинаковое время выполнения
	\item Адресация только по регистрам. Не можем читать напрямую из памяти
\end{itemize}

\subsection{Современные RICS процессоры}

\begin{itemize}
	\item \textbf{PowerPC:} Playstation, XBox, суперкомпьютеры IBM
	\item \textbf{MIPS:} WiFi/Bluetooth-адаптеры, процессоры в телевизорах
	\item \textbf{ARM:} Смартфоны
	\item \textbf{AVR:} Arduino
\end{itemize}

\subsection{Вещественная арифметика, IEEE 754}

Было на семинарах. Очень очень лень это все дублировать...



