\lecture{5}{Хеширование строк. Алгоритм Рабина-Карпа.}
\subsection{Хеширование строк. Вычисление полиномиального хеша методом Горнера.}
Научимся хешировать строки. Пусть нам дана строка $s = s_0, s_1, \ldots, s_{n-1}$.
Можно использовать следующие хеш-фунции:
\begin{itemize}
  \item $h_1(s) = (s_0 + s_1 a + s_2 a^2 + \ldots + s_{n-1} a^{n-1}) \textrm{ mod } M$
  \item $h_2(s) = (s_0 a^{n-1} + s_1 a^{n-2} + s_2 a^{n-3} + \ldots + s_{n-1}) \textrm{ mod } M$
\end{itemize}

Теперь выберем константы $a$ и $M$. В качестве $M$ удобно брать степени двойки (например, $2^{32}$), 
чтобы взятие остатка было равносильно беззнаковому переполнению.

Наша хеш-функция должна изменяться при изменении одного символа в строке, т.е. чтобы все значения
$s \cdot a^{u}, 0 \leq s < M$ были различны. Этого можно добиться, если $a$ и $M$ будут взаимно простыми.
\begin{theorem}[]
  \begin{enumerate}
    \item Если $a$ и $M$ не взаимно простые, то 
      \[
        \{ s \cdot a  \textrm{ mod } M, 0 \leq s < M \} \neq \{0, \ldots, M - 1\}
      .\] 
    \item Если $a$ и $M$ взаимно простые, то
      \[
        \{ s \cdot a \textrm{ mod } M, 0 \leq s < M \} = \{ 0, \ldots, M - 1 \}
      .\] 
  \end{enumerate}
\end{theorem}
\begin{proof}
   \begin{enumerate}
     \item $a$ и $M$ не являются взаимно простыми, тогда у них есть какой-то общий делитель 
       $d > 1 \colon a = d \cdot x, M = d \cdot y$. Тогда:
       \begin{gather*}
         s \cdot a = M \cdot k + r \\
         r = s \cdot a - M \cdot k \\
         r = s \cdot d \cdot x - d \cdot y \cdot k \\
         r = d(s \cdot x - y \cdot k) 
       \end{gather*}
       Получаем требуемое.
     \item Доказательство от противного. Пусть это множество $\{ s \cdot a \textrm{ mod } M, 0 \leq s < M \}$
       имеет меньше $M$ различных элементов. Тогда $\exists i < j \colon ia \equiv ja (\textrm{ mod } M)$.
       Следовательно, $(j - i)a = M \cdot u$, т.е.  $j - i$ делится на  $M$ ($a$ и $M$ взаимно простые).
       Но $0 < j - i < M$, получаем противоречие.
   \end{enumerate}
\end{proof}

Метод Горнера для вычисление хеш-функции:
\[
  h_2(s) = \left( \left( \left( s_0 a + s_1 \right)a + s_2  \right)a + \ldots + s_{n-2}  \right) a + s_{n-1} 
.\] 
\subsection{Алгоритм Рабина-Карпа.}
Пусть нам нужно найти подстроку $p$ длины $m$ в тексте $t$ длины $n$.

Алгоритм очень простой:
\begin{enumerate}
  \item Считаем хеш шаблона $p$.
  \item Считаем хеш всех окон размера $m$ в тексте $t$, сдвигаем окно. Этот хеш считает на основе
    предыдущего.
  \item Если получили совпадение хешей, сравниваем строки.
\end{enumerate}

В лучшем случае (хеш шаблона не найден) $O(n + m)$.

Найдено $k$ совпадений хеш-функции, тогда асимптотик $O(n + m + km)$.

\subsection{Быстрое вычисление хеша подстроки на базе хеш-значений префиксов.}
Пусть нам известны значения хешей для всех префиксов.
Т.е. $h(s[0 \ldots r -1]) = s_0 x^{r-1} + s_1 x^{r-2} + \ldots + s_{r-1}$.
Научимся быстро считать хеш подстроки, ответ простой:
\[
  h(s[l\ldots r - 1]) = h(s[0\ldots r-1]) - x^{r-l} \cdot h(s[0\ldots l-1]) = s_{l} x^{r - l - 1} + \ldots + 
  s_{r - 1} \textrm{ mod } M
.\] 


