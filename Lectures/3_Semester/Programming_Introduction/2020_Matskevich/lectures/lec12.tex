\lecture{12}{Длинная арифметика.}
\subsection{Основные понятия.}
\begin{definition}
  \highlight{Длинная арифметика} --- набор алгоритмов для поразрядной работы с числами
  произвольной длины.
\end{definition}

\begin{remark}
  Для хранения больших чисел можно использовать массив цифр по по некоторому основанию $c$. 
  Знак числа можно хранить отдельным флагом.
  \[
    A = a_0 + a_1 \cdot c + a_2 \cdot c + \ldots + a_{n-1} \cdot c^{n - 1}
  .\] 
\end{remark}

Также для удобства можно считать, что у нас нет лидирующих нулей и перед всеми алгоритмами
делать нормировку наших чисел (откидывать лидирующие нули, если они появляются).

\begin{remark}
  Пусть длина двух больших чисел $m$ и $n$, соответственно. 
  <<Школьное>> вычитание и сложение будет работать за $O(\max(n, m))$. Умножение же будет работать за
  $O(nm)$.
\end{remark}

\subsection{Алгоритм Карацубы.}
Пусть у нас есть два $n$-значных числа $A$ и $B$ в десятичной системе счисления, 
\[
A = ax + b, B = cx + d
,\] 
где $x = 10^{\frac{n}{2}}$ (примем, что $n$ --- чётное).

Получается, в $a$ мы имеем $\frac{n}{2}$ старших разрядов, в $b$ --- младших.
Распишем произведение:
\[
  AB = (ax + b) \cdot (cx + d) = acx^2 + (ad + bc) \cdot x + bd
.\] 
Заменим сумму произведений:
\[
  (ad + bc) = (a + b) \cdot (c + d) - ac - bd
.\] 
Получим:
\[
  AB = acx^2 + ((a + b) \cdot (c + d) - ac - bd) \cdot x + bd
.\] 
Итого мы свели наше умножение в задачу умножения $ac$, $bd$, $(a + b) \cdot (c + d)$, а также сложения, 
которое выполняется за $\Theta(n)$.
Это большие числа, но их разрядность в два раза меньше и их можно посчитать рекурсивно.
Размер задачи уменьшился в два раза, количество подзадач равно трём, значит по мастер-теореме имеем следующую
асимптотику:
 \[
   T(n) = 3T\left(\frac{n}{2}\right) + \Theta(n) = \Theta\left(n^{\log_{2}^3}\right)
.\] 

\subsection{Деление.}
Самый простой вариант: бинарный поиск по ответу, но асимптотика тут оставляет желать лучшего.

Более эффективным вариантом является <<школьное>> деление в столбик двух многочленов. 
Асимптотика тут уже лучше: $O(n^2)$. Далее приведена формализация этого деления из Кнута.

Пусть у нас есть два многочлена $u(x)$ и $v(x)$. Мы можем представить их в таком виде:
\[
  u(x) = q(x) \cdot v(x) + r(x)
,\] 
где $deg(r) < deg(v)$.
\begin{remark}
  Вышеизложенное разложение единственно.
\end{remark}
\begin{proof}
  Пусть это не так и уравнению удовлетворяют две разные пары многочленов $(q_1(x), r_1(x)), (q_2(x), r_2(x))$.
  Тогда справедливы следующие равенства:
  \begin{gather*}
    q_1(x) \cdot v(x) + r_1(x) = q_2(x) \cdot v(x) + r_2(x) \\
    (q_1(x) - q_2(x)) \cdot v(x) = r_2(x) - r_1(x) \\
  \end{gather*}
  Из предположения $q_1(x) \neq q_2(x) \implies q_1(x) - q_2(x) \neq 0 \implies$
  \begin{gather*}
    deg((q_1 - q_2) \cdot v) = deg(q_1 - q_2) + deg(v) \geq deg(v) > deg(r_2 - r_1)
  \end{gather*}
  Пришли к противоречию, т. к. $deg((q_1 - q_2) \cdot v)$ не может быть больше $deg(r_2 - r_1)$.
\end{proof}

Пусть нам даны два многочлена вида:
\begin{gather*}
  u(x) = u_{m} x^{m} + \ldots + u_1 x + u_0 \\
  v(x) = v_{n} x^{n} + \ldots + v_1 x + v_0
\end{gather*}
Результатом алгоритма будут многочлены из соотношения выше:
\begin{gather*}
  q(x) = q_{m-n} x^{m-n} + \ldots + q_0 \\
  r(x) = r_{n-1} x^{n-1} + \ldots + r_0
\end{gather*}



Далее буквально изложен алгоритм деления многочленов в столбик, он итеративный по $k$, 
для простоты понимания напишем, за что отвечают индексы:
\begin{itemize}
  \item $k$ --- позиция текущего разряда в частном, пробегает значения от $m - n$ до $0$.
  \item $j$ --- позиция разрядов многочлена $u$, из которых необходимо вычесть на текущем разряде,
    чтобы получить остаток, пробегает значения от $n + k - 1$ до $k$.
\end{itemize}
\begin{figure}[ht]
    \centering
    \incfig{poly}
    \caption*{Иллюстрация коэффициентов}
    \label{fig:poly}
\end{figure}

Далее сам алгоритм:
\begin{enumerate}
  \item Итерируемся по $k = m - n, m - n - 1, \ldots, 0$.
  \item На каждом шаге вычисляем $q_{k} = \frac{u_{n + k}}{v_{n}}$. И пересчитываем остаток:
    $u_{j} = u_{j} - q_{k} \cdot v_{j - k}$ для $j = n + k - 1, n + k - 2, \ldots, k$.
\end{enumerate}

