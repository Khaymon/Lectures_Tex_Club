\lecture{14}{Комбинаторные игры.}
\subsection{Игра с камнями.}
\paragraph{Условие} Есть $N$ камней, игрок может брать от $1$ до $K$ камней. Побеждает игрок,
взявший последний камень.

Разбор в зависимости от числа $N$:
\begin{itemize}
  \item $N = 0$ --- проигрыш 
  \item $N = 1, \ldots, k$ --- выигрыш
  \item $N = k + 1$ --- проигрыш
  \item $N = k + 2, \ldots, 2k + 1$ --- выигрыш (берём от $1$ до $k$ камней соответственно)
  \item $N = 2(k + 1)$ --- проигрыш
  \item $\ldots$
\end{itemize}

\subsection{Модификация игры с камнями.}
\paragraph{Условие} Есть $N$ камней, игрок может брать от $1$ до $K$ камней. Проигрывает игрок,
взявший последний камень.

Разбор в зависимости от числа $N$:
\begin{itemize}
  \item $N = 0$ --- выигрыш
  \item $N = 1$ --- проигрыш
  \item $N = 2, \ldots, k + 1$ --- выигрыш (можно взять от $1$ до $k$ камня соответственно и свести для
    оппонента ситуацию к $N = 1$)
  \item $N = k + 2$ --- проигрыш
  \item $N = k + 3, \ldots, 2k + 2$ --- выигрыш
  \item $2(k + 1) + 1$ --- проигрыш
  \item $\ldots$
\end{itemize}

\subsection{Метод симметричной стратегии.}
\paragraph{Условие} Игра в монеты за круглым столом. Игроки по очереди кладут круглые монеты на круглый стол так, чтобы они не пересекались. Игрок, который не может сделать ход, проигрывает.

У первого игрока есть выигрышная стратегия, вот она:
\begin{itemize}
  \item Первым ходом кладём монету в центр
  \item Далее повторяем каждый ход оппонента симметрично относительно центра
\end{itemize}

Это выигрышная стратегия, т. к. после хода первого игрока у нас картинка расположения монет всегда 
симметрична относительно центра. А это значит, что если второй игрок нашёл, куда поставить, то и 
симметричное место будет свободно.

Некоторые классификации игр:
\begin{definition}
  \highlight{Комбинаторные} игры --- игры с полной информацией, которые ведут два игрока, делая ходы 
  по очереди.
\end{definition}
\begin{definition}
  \highlight{Справедливая} игра --- игра, в которой возможные ходы каждого игрока совпадают.
\end{definition}
\begin{definition}
  \highlight{Нормальная} игра --- игра, в которой игрок, который не может сделать ход, проигрывает.
\end{definition}

\subsection{Игры на графе.}
Пусть некоторая игра ведётся на некотором графе $G$. Т. е. текущее состояние игры --- некоторая вершина
графа, и из каждой вершины рёбра идут в те вершины, в которые можно пойти следующим ходом.

\begin{definition}
  \highlight{Терминальные вершины} --- вершины, из которых нельзя сделать ход.
\end{definition}

\begin{definition}
  \highlight{Выигрышная игра} --- вне зависимости от действий второго игрока, первый выигрывает.
\end{definition}
\begin{definition}
  \highlight{Проигрышная игра} --- вне зависимости от действий первого игрока, второй игрок может
  выиграть.
\end{definition}
\begin{definition}
  \highlight{Ничейная игра} --- игра, которая не является ни выигрышной, ни проигрышной.
\end{definition}
\begin{remark}
  Ничейные игры продолжаются бесконечно, граф ничейной игры должен содержать циклы.
\end{remark}

\begin{remark}
  Если из некоторой вершины есть ребро в проигрышную вершину, то он выигрышная.
\end{remark}
\begin{remark}
  Если из некоторой вершины все рёбра исходят в выигрышные вершины, то это вершина проигрышная.
\end{remark}
\begin{remark}
  Если граф не содержит циклы, то все его вершины можно разбить на два непересекающихся множества 
  выигрышных и проигрышных вершин.
\end{remark}

Алгоритм построения такого разбиения:
\begin{itemize}
  \item Граф без циклов, значит можно запустить топологическую сортиовку.
  \item Обходим граф в порядке, обратном топологической сортировки и определяем, какая вершина перед нами.
\end{itemize}

\subsection{Стратегия}
\begin{definition}
  \highlight{Стратегия} --- отображение из множества нетерминальных вершин в множество вершин графа.
\end{definition}
\begin{definition}
  \highlight{Выигрышная стратегия} для первого игрока --- стратегия, следуя которой, первый игрок
  выигрывает, вне зависимости от действий второго игрока.
\end{definition}

\begin{remark}
  В любой игре на ациклическом графе существует стратегия $s$, такая что для игр, начинающихся в
  выигрышных вершинах, она является выигрышной.
\end{remark}
\begin{proof}
  Достаточно переходить по графу по первоначальному правилу (из проигрышной в любую выигрышную, 
  из выигрышной --- в проигрышную.
\end{proof}

\subsection{Ретроспективный анализ.}
Пусть $W$--- множество выигрышных вершин, $L$ --- проигрышных.
\begin{remark}
  В графе с циклами нельзя использовать алгоритм поиска $W$, $L$ с помощью топологической сортировки.
\end{remark}

Таким образом, приходим к алгоритму ретро-анализа:
\begin{enumerate}
  \item Начинаем обход с терминальных вершин с помощью dfs или bfs.
  \item Для каждого ребра $(u, v)$ исходя из определённых ранeе правил получаем:
    \begin{gather*}
      v \in L  \implies u \in W \\   
      v \in W \implies u \in L, \text{ если $(u,v)$ --- последнее среди нерасмотренных ребер.}
    \end{gather*}
  \item Дополнительно нужно для каждой вершины хранить количество необработанных исходящих ребер
    (нужно для второго условия в пункте 2).
\end{enumerate}

\begin{remark}
  Время работы такого алгоритма $O(\left| E \right|)$.
\end{remark}

Данный алгоритм можно модифицировать так, чтобы он выяснял не только тип вершины, но и минимальное число
ходов до выигрыша (максимальное число ходов до проигрыша). Для этого вместо dfs воспользуемся bfs и будем
для каждого уровня вершин хранить число до победы (до проигрыша). Получается, мы разделим все наши
вершины на классы вида (напрямую определение выигрышных и проигрышных позиций):
\begin{enumerate}
  \item $L_0 = \left\{ u \mid \text{из $u$ не выходит ни одной дуги} \right\} $ 
  \item $W_1 = \left\{ u \mid \exists v \in L_0 \colon uv \in E \right\} $
  \item $L_2 = \left\{ u \mid \forall v \colon uv \in E \to v \in W_1 \right\} $
  \item $\ldots$ 
  \item $L_{2i} = \left\{ u \mid \forall v \colon uv \in E \to v \in W_{2i - 1} \right\} $
  \item $W_{2i+1} = \left\{ u \mid \exists v \in L_{2i} \colon uv \in E \right\} $
\end{enumerate}

\begin{remark}
  $W = \bigcup L_{i}, L = \bigcup L_{i}$.
\end{remark}

\begin{remark}
  Множество ничейных позиций $D = V \setminus (P \cup N)$.
\end{remark}












