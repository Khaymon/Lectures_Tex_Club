\begin{prop}
	Формы $g^1, \ldots, g^s \in \Lambda_1$ линейно независимы $\Leftrightarrow g^1\wedge\ldots\wedge g^s \ne 0$.
\end{prop}

\begin{proof}
	Пусть набор форм линейно независим. Дополним $g^1, \ldots, g^s$ до базиса $E^*$. Рассмотрим двойственный базис $e_1, \ldots, e_n$ пространства $E$.
	
	Посмотрим 
	\begin{multline*}
		g^1\wedge\ldots\wedge g^s(e_1, \ldots, e_s)=s!\asym(g^1\otimes\ldots\otimes g^s)(e_1,\ldots, e_s)=\\=\sum\limits_{\pi\in S_s}\sgn(\pi)\pi(g^1\otimes\ldots\otimes g^s)(e_1,\ldots, e_s)=\sum\limits_{\pi\in S_s}\sgn(\pi)g^1(e_{\pi(1)})\ldots g^s(e_{\pi(s)})=1,
	\end{multline*}
	так как только в одном случае слагаемое не нуль, когда перестановка ничего не переставляет.
	
	Пусть набор форм линейно зависим: $g^s = \sum\limits_{i=1}^{s-1}\alpha_ig^i\Rightarrow g^1\wedge\ldots\wedge g^{s-1}\wedge\left(\sum\limits_{i=1}^{s-1}\alpha_ig^i\right)=0$, так как $f\wedge f = 0$.
\end{proof}

\begin{prop}
	Если $x_i=\xi_i^je_j, i=1,\ldots, p$, и $\{f^k\}_{k=1}^n$ --- двойственный базис к $\{e_j\}_{j=1}^n$, то $f^{i_1}\wedge\ldots\wedge f^{i_p}=\det(\xi_l^{i_k})_{l=1, k=1}^{p,p}$.
\end{prop}

\begin{proof}
	Распишем $f^{i_1}\wedge\ldots\wedge f^{i_p}(x_1, \ldots, x_p)=p!\asym(f^{i_1}\otimes\ldots\otimes f^{i_p})(x_1, \ldots, x_p)=\\=\sum\limits_{\pi\in S_p}\sgn(\pi)(\pi(f^{i_1}\otimes\ldots\otimes f^{i_p}))(x_1, \ldots, x_p)=\sum\limits_{\pi\in S_p}\sgn(\pi)(f^{i_1}\otimes\ldots\otimes f^{i_p})(x_{\pi(1)}\ldots, x_{\pi(p)})=\\=\sum\limits_{\pi\in S_p}\sgn(\pi) (f^{i_1}(x_{\pi(1)})\ldots f^{i_p}(x_{\pi(p)}))=\sum\limits_{\pi\in S_p}\sgn(\pi)(\xi_{\pi(1)}^{i_1}\ldots \xi_{\pi(p)}^{i_p})=\det(\xi_l^{i_k})_{l=1, k=1}^{p,p}$
\end{proof}

\begin{prop}
	Формы $\{f^{i_1}\wedge\ldots\wedge f^{i_p}:1\leqslant i_1<\ldots<i_p\leqslant n\}$ образуют базис пространства $\Lambda_p$.
\end{prop}

\begin{proof}
	Из утверждения 5 $\{f^{i_1}\otimes\ldots\otimes     f^{i_p}:1\leqslant i_1<\ldots<i_p\leqslant n\}$ --- базис пространства $\Omega_p^0$. Тогда, если $W\in \Lambda_p$, то это антисимметрическая форма, значит \\$W\in \Omega_p^0\Rightarrow W=\sum\limits_{\vec{i}}w_{\vec{i}}f^{i_1}\otimes\ldots\otimes f^{i_p}$. Применим операцию антисимметризации к обеим частям равенства, причем $W$ уже антисимметричная форма:
	$$\asym(W)=W=\sum\limits_{\vec{i}}w_{\vec{i}}\asym f^{i_1}\otimes\ldots\otimes f^{i_p}= \dfrac{1}{p!}\sum\limits_{\vec{i}}w_{\vec{i}}f^{i_1}\wedge\ldots\wedge f^{i_p}.$$
	Если хотя бы пара индексов совпадает, то слагаемое равно нулю. 
	
	$$W = \sum\limits_{1\leqslant i_1 < \ldots < i_p \leqslant n}\underbrace{\left(\sum\limits_{\pi\in S_p}\dfrac{1}{p!}\sgn(\pi_p)w_{\pi(1)\ldots\pi(p)}\right)}_{=\widetilde{w}_{i_1 \ldots i_p}}f^{i_1}\wedge\ldots\wedge f^{i_p}.$$
	Получили, что любой элемент пространства, можно разложить в линейную комбинацию $f^i$, осталось установить, что они линейно независимы.
	 Для этого возьмем $W$ и подействуем ею на векторах соответствующего двойственного базиса: 
	 $$W(e_{j_1}, \ldots, e_{j_p})=\sum\limits_{1\leqslant i_1 < \ldots < i_p \leqslant n}\widetilde{w}_{i_1 \ldots i_p}f^{i_1}\wedge\ldots\wedge f^{i_p}(e_{j_1},\ldots, e_{j_p})=\widetilde{w}_{j_1 \ldots j_p}$$
\end{proof}

\begin{corollary}
	$\dim\Lambda_p=C_n^p$
\end{corollary}

\begin{note}
	Также как из пространства $\Omega_p^0$ мы построили антисимметрические полиллинейные формы --- пространство $\Lambda_p$,можно построить пространство $\Omega_0^q$ --- пространство поливекторов, с базисом $\{e_{i_1}\wedge\ldots\wedge e_{i_q}, 1\leqslant i_1 < \ldots < i_q\leqslant n\}$
\end{note}


\subsection{Дифференциальные формы на области, операции над ними.}

Пусть $U$ --- область в $\R^n=E$.

\begin{Def}
	$p$ - формой (дифференциальной формой валентности (степени) $p$) на $U$ называется отображение $\Omega: U\to \Lambda_p$.
\end{Def}

Из утверждения $15\S 1\Rightarrow \Omega(x)=\sum\limits_{1\leqslant i_1 <\ldots < i_p\leqslant n} w_{i_1\ldots i_p}(x)f^{i_1}\wedge\ldots\wedge f^{i_p}$

\begin{Def}
	Внешнее дифференцирование $p$-формы $\Omega$ определяется как $(p+1)$-форма $d\Omega:U\to\Lambda_{p+1}$, по правилу $d\Omega=(p+1)\asym\Omega', \text{ где }\Omega':U\to L(E\to \Lambda_p), L$ --- пространство линейных операторов, то есть $\Omega'(x)\in L(E\to\Lambda_p)=\Lambda_{1,p}\subset\Omega_{p+1}^0$, где $\Lambda_{1, p}$ --- полилинейная форма $(p+1)$ аргумента, по последним $p$ аргументам антисимметричная.
\end{Def}

\begin{example}
	Пусть $p=0$. Тогда $\Lambda_0$ --- это пространство чисел и нуль-форма --- это функция. Посмотрим на функцию, которая дает координату: $x^j$. Ее дифференциал $d(x^j)=f^j$, так как $(x^j)':U\to L(E\to\R)$ представляет из себя умножение произвольного вектора на вектор $(0,\ldots, \underbrace{1}_{j}, \ldots,0)$.
\end{example}

Важный результат: $$\fbox{$d(x^j)=f^j$}$$ 

Откуда следует, что $\Omega(x)=\sum\limits_{1\leqslant i_1 <\ldots < i_p\leqslant n} w_{i_1\ldots i_p}(x)dx^{i_1}\wedge\ldots\wedge dx^{i_p}$, которое можно принять как соглашение.

Пусть $\Omega:=w(x)dx^{i_1}\wedge\ldots\wedge d x^{i_p}$, тогда $\Omega'(x)=\dfrac{\partial w}{\partial x_j}(x)dx^j\otimes dx^{i_1}\wedge\ldots\wedge d x^{i_p}$, (помним про соглашение Эйнштейна). \\
\begin{multline*}
	d\Omega(x)=(p+1)\asym(\dfrac{\partial w}{\partial x_j}(x)dx^j\otimes dx^{i_1}\wedge\ldots\wedge d x^{i_p})=\\=\sum\limits_{j=1}^n\dfrac{\partial w}{\partial x_j}(x)(p+1)\asym(dx^j\otimes p!\asym(dx^{i_1}\wedge\ldots\wedge dx^{i_p}))=\\=\sum\limits_{j=1}^n\dfrac{\partial w}{\partial x_j}(x)(p+1)!\asym(dx^j\otimes dx^{i_1}\wedge\ldots\wedge dx^{i_p})=\sum\limits_{j=1}^n\dfrac{\partial w}{\partial x_j}(x)dx^j\wedge dx^{i_1}\wedge\ldots\wedge dx^{i_p}=\\=dw(x)\wedge dx^{i_1}\wedge\ldots\wedge dx^{i_p},
\end{multline*}
где $dw(x)=\sum\limits_{j=1}^n\dfrac{\partial w}{\partial x_j}(x)dx^j$ --- дифференциал, который теперь понимается, как 1-форма.

Примем соглашение, что гладкость формы определяется гладкостью ее коэффициентов.

\begin{theorem}(Основные свойства операции внешнего дифференцирования).
	\begin{enumerate}
		\item(Линейность)\\ $d(\alpha\Omega+\beta\Pi)=\alpha d\Omega+\beta d\Pi, \alpha, \beta\in\R, \Omega, \Pi\in\Lambda_p^{(1)}(U)$
		\item(Дифференцирование внешнего произведения)\\  Если $\Omega\in\Lambda_p^{(1)}(U), \Pi\in\Lambda_q^{(1)}(U),$ то $d(\Omega\wedge\Pi)=d\Omega\wedge\Pi+(-1)^p\Omega\wedge d\Pi$
		\item (Отсутствие вторых дифференциалов) \\ Если $\Omega\in\Lambda_p^{(2)}(U)\Rightarrow d(d\Omega)=0$
	\end{enumerate}
\end{theorem}
\begin{proof} \ 
	\begin{enumerate}
		\item Первое свойство следует из линейности производной, операций  антисимметризации.
		\item Пусть $\Omega(x)=w(x)dx^{i_1}\wedge\ldots\wedge dx^{i_p}, \Pi(x)=\pi(x)dx^{j_1}\wedge\ldots\wedge dx^{j_q}$
		\begin{multline*}
			d(\Omega\wedge\Pi)=d(w(x)\pi(x)dx^{i_1}\wedge\ldots\wedge dx^{i_p}\wedge dx^{j_1}\wedge\ldots\wedge dx^{j_q})=\\=d(w(x)\pi(x))dx^{i_1}\wedge\ldots\wedge dx^{i_p}\wedge dx^{j_1}\wedge\ldots\wedge dx^{j_q}
			=\\=
			\pi(x)\sum\limits_{j=1}^n\dfrac{\partial w}{\partial x_j}(x)dx^j\wedge dx^{i_1}\wedge\ldots\wedge dx^{i_p}\wedge dx^{j_1}\wedge\ldots\wedge dx^{j_q}+\\+
			w(x)\sum\limits_{j=1}^n\dfrac{\partial \pi}{\partial x_j}(x)dx^j\wedge dx^{i_1}\wedge\ldots\wedge dx^{i_p}\wedge dx^{j_1}\wedge\ldots\wedge dx^{j_q}=\\=d\Omega(x)\wedge\Pi (x)+(-1)^p\Omega(x)\wedge d\Pi(x). 
		\end{multline*}
		\item $d(d\Omega)=d\left(\sum\limits_{j=1}^n\dfrac{\partial w}{\partial x_j}(x)dx^j\wedge dx^{i_1}\wedge\ldots\wedge dx^{i_p}\right)$, если $j$ совпадет с одним из индексов $i_1, \ldots, i_p$, то значение слагаемого будет нуль. Тогда
		\begin{multline*}
			d(d\Omega)=d\left(\sum\limits_{j\ne i_k}^n\dfrac{\partial w}{\partial x_j}(x)dx^j\wedge dx^{i_1}\wedge\ldots\wedge dx^{i_p}\right)=\\=\sum\limits_{l\ne j,i_k}\sum\limits_{j\ne i_k}\dfrac{\partial^2 w}{\partial x_l \partial x_j}dx^l\wedge dx^j\wedge dx^{i_1}\wedge\ldots\wedge  dx^{i_p}=\\=\sum\limits_{j<l; j,l\ne i_k} \left(\dfrac{\partial^2 w}{\partial x_j \partial x_l} - \dfrac{\partial^2 w}{\partial x_l \partial x_j}\right)dx^l\wedge dx^j\wedge dx^{i_1}\wedge\ldots\wedge  dx^{i_p}=0
		\end{multline*}
	\end{enumerate}
\end{proof}

Пусть $e_1, \ldots, e_n$ --- ортонормированный базис.

\begin{Def}
	Операция дополнения(звездочка Ходжа) задается на базисных формах: $*(w(x)dx^{i_1}\wedge\ldots\wedge dx^{i_p})=w(x)(-1)^{[\vec{i},\vec{j}]}dx^{j_1}\wedge\ldots\wedge dx^{j_{n-p}}$, где $(-1)^{[\vec{i}, \vec{j}]}$ --- знак перестановки $(i_1, \ldots, i_p, j_1, \ldots, j_{n-p})$ чисел $(1,2,\ldots, n)$.
	
	В $\R^3:$ зададим естественный порядок $ dx_1=dx, dx_2=dy, dx_3=dz$.
	
	$*dx = dy\wedge dz;\ *dy=-dx\wedge dz=dz\wedge dx;\ *dz=dx\wedge dy;\ *(dx\wedge dy\wedge dz)=1$
\end{Def}

\begin{prop}
	Если $\Omega\in\Lambda_p(U)$, то $*(*\Omega)=(-1)^{p(n-p)}\Omega$
\end{prop}

\begin{proof}
	Рассмотрим базисную форму $\Omega(x)=w(x)dx^{i_1}\wedge\ldots\wedge dx^{i_p},\\ *\Omega=w(x)(-1)^{[\vec{i},\vec{j}]}dx^{j_1}\wedge\ldots\wedge dx^{j_{n-p}};\ *(*\Omega)=w(x)(-1)^{[\vec{i},\vec{j}]}(-1)^{[\vec{j},\vec{i}]}dx^{i_1}\wedge\ldots\wedge dx^{i_p}=(-1)^{p(n-p)}\Omega$, так как $(-1)^{[\vec{j}, \vec{i}]}$ --- знак перестановки $(j_1, \ldots, j_{n-p}, i_1, \ldots, i_p)$, и $(-1)^{[\vec{j}, \vec{i}]}$, $(-1)^{[\vec{i}, \vec{j}]}$ отличаются на $(-1)^{p(n-p)}$
\end{proof}





