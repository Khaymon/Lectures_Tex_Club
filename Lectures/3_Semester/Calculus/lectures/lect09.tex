\section{Глава 11. Интегрирование дифференциальных форм. Теория поля.}
 \subsection{Элементы тензорной алгебры.}
Договоримся, что $E = \R^n, E^*$ --- сопряженное пространство, то есть пространство линейных функционалов (линейных форм, ковекторов). Вектора обозначаются $x_1, \ldots, x_p \in E$, $x_i=\sum\limits_{k=1}^n \xi_i^k e_k;$ формы обозначаются $y^1, \ldots, y^q\in E^*, y^j=\sum\limits_{l=1}^n \eta_l^j e^l$. 
\begin{Def}
	Полилинейной формой называется функция $U:E^p\times(E^*)^q\rightarrow \R$, линейная по каждому из своих аргументов, например, $U(\alpha x_1'+\beta x_1'', x_2, \ldots, x_p; y_1, \ldots, y^q)=\alpha U(x_1', x_2, \ldots, x_p;y^1, \ldots, y^q)+\beta U(x_1'', x_2, \ldots, x_p;y^1, \ldots, y^q), \alpha, \beta \in \R$.
\end{Def}
\begin{prop}
	Полилинейная форма однозначно определяется своими значениями на базисных элементах $E$ и $E^*$, то есть числами $w_{i_1, \ldots, i_p}^{j_1, \ldots, j_q} = U(e_{i_1}, \ldots, e_{i_p}; e^{j_1}, \ldots, e^{j_q})$, где $\{e_i\}_{i=1}^n$ --- базис $E$, $\{e^j\}_{j=1}^q$ --- двойственный базис $E^* (e^j(e_i)=\delta_i^j$ --- символ Кронекера).
\end{prop}
\begin{proof}
	\begin{multline*}
		U(x_1, \ldots, x_p; y^1, \ldots, y^q) = U(\sum\limits_{k_1=1}^n \xi_1^{k_1} e_{k_1}, \ldots, \sum\limits_{k_p=1}^n \xi_p^{k_p} e_{k_p}; \sum\limits_{l_1=1}^n \eta_{l_1}^1 e^{l_1}, \ldots, \sum\limits_{l_q=1}^n \eta_{l_q}^q e^{l_q})= \\ =
		\text{[пользуясь полилинейностью]} = \\ = \sum\limits_{k_1=1}^n \ldots \sum\limits_{k_p=1}^n\sum\limits_{l_1=1}^n\ldots\sum\limits_{l_q=1}^n \xi_1^{k_1}\ldots\xi_p^{k_p}\eta_{l_1}^1\ldots\eta_{l_q}^q U(e_{k_1}, \ldots, e_{k_p};e^{l_1}, \ldots, e^{l_q}) = \\ =
		\sum\limits_{k_1=1}^n \ldots \sum\limits_{k_p=1}^n\sum\limits_{l_1=1}^n\ldots\sum\limits_{l_q=1}^n \xi_1^{k_1}\ldots\xi_p^{k_p}\eta_{l_1}^1\ldots\eta_{l_q}^q w_{k_1, \ldots, k_p}^{l_1, \ldots, l_q} = \sum_{\overrightarrow{k}, \overrightarrow{l}}\xi^{\overrightarrow{k}}\eta_{\overrightarrow{l}}w_{\overrightarrow{k}}^{\overrightarrow{l}}.
	\end{multline*}
\end{proof}

\begin{Def}
	Набор чисел $\{w_{\overrightarrow{i}}^{\overrightarrow{j}}\}$ из утверждения 11.1.1 называется тензором полилинейной формы $U$.
\end{Def}

\begin{Def}
	Если $U, V$ --- формы, то $U+V, \alpha U$, где $\alpha\in\R$ определяются как $(U+~V)(x_1, \ldots, x_p; y^1, \ldots, y^q)=U(\ldots)+V(\ldots)$, и $(\alpha U)(\ldots)=\alpha(U(\ldots))$.
\end{Def}

\begin{prop}
	Множество полилинейных форм валентности $(p, q)$ образует линейное пространство $\Omega_p^q$.
\end{prop}

\begin{prop}
	Базис пространства $\Omega_p^q$ образуют формы $W_{j_1, \ldots, j_q}^{i_1, \ldots, i_p}$, задаваемые формулой $W_{j_1, \ldots, j_q}^{i_1, \ldots, i_p}(x_1, \ldots, x_p; y_1, \ldots, y^q)=\xi_1^{i_1}, \ldots, \xi_p^{i_p}\eta_{j_1}^1, \ldots, \eta_{j_q}^q$.
\end{prop}

\begin{Def}
	Тензорным произведением форм $U\in \Omega_{p_1}^{q_1}, V\in\Omega_{p_2}^{q_2}$ называется форма $U\otimes V\in \Omega_{p_1+p_2}^{q_1+q_2}$, задаваемая формулой
	\begin{multline*}
U\otimes V (x_1, \ldots, x_{p_1}, x_{p_1+1}, \ldots, x_{p_1+p_2}; y^1,\ldots, y^{q_1}, y^{q_1+1}, \ldots, y^{q_1+q_2})=\\=U(x_1, \ldots x_{p_1}; y^1, \ldots, y^{q_1})V(x_{p_1+1}, \ldots, x_{p_1+p_2}; y^{q_1+1}, \ldots, y^{q_1+q_2})
	\end{multline*} 
\end{Def}

\begin{prop}(Свойства тензорного призведения)
	\begin{enumerate}
		\item (Линейность) \\$(\alpha U_1+\beta U_2)\otimes V = \alpha(U_1\otimes V)+\beta(U_2\otimes V)$; \\
		$U\otimes (\alpha V_1+\beta V_2) = \alpha(U\otimes V_1)+\beta(U\otimes V_2), \alpha, \beta \in \R, U_1, U_2, U \in \Omega_{p_1}^{q_1}, V_1, V_2, V \in \Omega_{p_2}^{q_2}$
		\item (Ассоциативность) \\ $(U\otimes V)\otimes W = U\otimes (V\otimes W), U\in \Omega_{p_1}^{q_1}, V\in\Omega_{p_2}^{q_2}, W\in\Omega_{p_3}^{q_3}$
	\end{enumerate}
\end{prop}

\begin{prop}
	Базисные формы $W_{j_1, \ldots, j_q}^{i_1, \ldots, i_p} = e^{i_1}\otimes\ldots\otimes e^{i_p}\otimes e_{j_1}\otimes\ldots\otimes e_{j_q}$, где $e_j$ рассматриваются как элементы $(E^*)^*\cong E$
\end{prop}

Далее используем принцип Эйнштейна, согласно которому, будем опускать знак суммирования.

\textbf{Переход к другому базису.}

Пусть $\{e_i'\}_{i=1}^n$ --- новый базис $E$, \fbox{$e_i'=\alpha_i^je_j$} (здесь идет суммирование по $j$), $\{{f'}^j\}_{j=1}^n$ --- новый базис $E^*$, ${f'}^j= \beta_k^j f^k$, причем $\alpha_i^k \beta_k^j=\delta_i^j$. Получается, что $x=\xi^i e_i ={\xi'}^i e_i'$ --- вектор не зависит от того в каком базисе мы его записываем. Подставляя $e_i': x = \alpha_i^j{\xi'}^ie_j \Rightarrow \\ \xi^i = \alpha_i^j{\xi'}^i \Leftrightarrow {\xi'}^i=\beta_j^i\xi^j$. Для форм получаем: \fbox{$\eta_j'=\alpha_j^k\eta_k$}. Заметим, что базис в $E$ преобразуется так же как и координаты форм (выделены в рамках) --- это называется ковариантный закон. Если данное условие не выполняется, то выполняется контравариантный закон.
\begin{prop}
	Тензоры полилинейной формы принадлежащей $\Omega_p^q$ преобразуются при замене двойтсвенных базисов $p$ раз ковариантно и $q$ раз контравариантно, то есть ${w'}_{i_1, \ldots, i_p}^{j_1, \ldots, j_q}=\underbrace{\sum}_{p+q\text{ раз}}\beta^{j_1}_{l_1}\ldots\beta_{l_q}^{j_q}w_{k_1, \ldots, k_p}^{l_1, \ldots, l_q}\alpha_{i_1}^{k_1}\ldots\alpha_{i_p}^{k_p}$
\end{prop}

\begin{proof} По определению тензора формы:
	$${w'}_{i_1, \ldots, i_p}^{j_1, \ldots, j_q}=U({e'}_{i_1}, \ldots, {e'}_{i_p}; {f'}^{j_1}, \ldots, {f'}^{j_q}) = \alpha_{i_1}^{k_1}\ldots \alpha_{i_p}^{k_p}\beta_{l_1}^{j_1}\ldots\beta_{l_q}^{j_q}U(e_{k_1}, \ldots, e_{k_p};f^{l_1}, \ldots, f^{l_q})$$
\end{proof}

\begin{prop}
	Набор чисел $\{{w}_{i_1, \ldots, i_p}^{j_1, \ldots, j_q}\}$, $p$ раз ковариантный и $q$ раз контравариантный задает тензор полилинейной формы $U\in \Omega_p^q$ по формуле: $$U={w}_{i_1, \ldots, i_p}^{j_1, \ldots, j_q}f^{i_1}\otimes\ldots\otimes f^{i_p}\otimes e_{j_1}\otimes\ldots\otimes e_{j_q}$$
\end{prop}

\begin{proof}
	Пусть задана полилинейная форма $U(x_1, \ldots, x_p; y_1, \ldots, y^q)$, хотим доказать, что если мы переходим к другому базису, то значение формы будет тем же самым, так как векторы и ковекторы не зависят от базиса. Но эту форму мы определяли неким законом, который зависит от базиса. А теперь говорим что эта форма от базиса не зависит. Значит это надо проверить. Временно обозначим $\widetilde{U} = {w'}_{i_1, \ldots, i_p}^{j_1, \ldots, j_q}{f'}^{i_1}\otimes\ldots\otimes {f'}^{i_p}\otimes {e'}_{j_1}\otimes\ldots\otimes {e'}_{j_q}$ в измененном базисе. Тогда по предыдущему утверждению: $$\widetilde{U} = \beta_{l_1}^{j_1}\ldots\beta_{l_q}^{j_q}{w}_{k_1, \ldots, k_p}^{l_1, \ldots, l_q}\alpha_{i_1}^{k_1}\ldots\alpha_{i_p}^{k_p}\cdot\beta^{i_1}_{r_1}f^{r_1}\otimes\ldots\otimes\beta^{i_p}_{r_p}f^{r_p}\otimes\alpha_{j_1}^{s_1}e_{s_1}\otimes\ldots\otimes\alpha_{j_q}^{s_q}e_{s_q}.$$
	
	В этом выражении присутствуют такие конструкции: $\sum\limits_{i_1}\alpha_{i_1}^{k_1}\beta_{r_1}^{i_1} = \delta_{r_1}^{k_1}$. Тогда\\ $\widetilde{U} = w_{k_1\ldots k_p}^{l_1\ldots l_q}\delta_{r_1}^{k_1}\ldots\delta_{r_p}^{k_p}\delta_{l_1}^{s_1}\ldots\delta_{l_q}^{s_q}f^{r_1}\otimes\ldots\otimes f^{r_p}\otimes e_{s_1}\otimes\ldots\otimes e_{s_q}$. Суммирование символов Кронекера по $r$ и по $s$ в итоге даст $\widetilde{U} = w_{k_1\ldots k_p}^{l_1\ldots l_q}f^{k_1}\otimes\ldots\otimes f^{k_p}\otimes e_{s_1}\otimes\ldots\otimes e_{s_q}$.
\end{proof}

\textbf{Примеры}
\begin{enumerate}
	\item Вектор $x\in E$ с координатами $(\xi^1, \ldots, \xi^n)$ --- тензор валентности $(0, 1)$.
	
	Ковектор $y\in E^*$ с координатами $(\eta^1, \ldots, \eta^n)$ --- тензор валентности $(1, 0)$.
	\item Матрица линейного оператора $T: y= Tx, \eta^j=\tau_i^j\xi^i$, тензор $\{\tau_i^j\}$ валентности $(1,1)$, так как ${\eta'}^i=\beta_i^j\eta^i=\beta_i^j\tau_k^i\xi^k=\underbrace{\beta_i^j\tau^i_k\alpha_l^k}_{{\tau'}_l^j}{\xi'}^l$
\end{enumerate}

\begin{Def}
	$W\in\Omega_p^0$ называется симметрической, если она не изменяется при любой пререстановке ее аргументов.
\end{Def}

\begin{Def}
	$W\in\Omega_p^0$ называется антисимметрической или кососиммметрической, если при любой транспозиции пары ее аргументов, она меняет знак.
\end{Def}

\begin{prop}
	$W$ --- антисимметрическая $\Leftrightarrow$ она равна нулю, если какие-то два ее аргумента совпадают.
\end{prop}

\begin{proof}
	Пусть $W$ --- антисимметрическая, $x_k=x_l=a$. Тогда $W(x_1, \ldots, x_k, \ldots, x_l, \ldots, x_p)=-W(x_1, \ldots, x_k, \ldots, x_l, \ldots, x_p) = 0$.
	
	Пусть $W$ --- такая форма, что если два ее аргумента совпадают,то она нуль. 
	\begin{multline*}
		0=
		W(x_1, \ldots, x_k+x_l, \ldots, x_k+x_l, \ldots, x_p)=\\=
		\underbrace{W(x_1, \ldots, x_k, \ldots, x_k, \ldots, x_p)}_{=0}+
		W(x_1, \ldots, x_k, \ldots, x_l, \ldots, x_p)+\\+
		W(x_1, \ldots, x_l, \ldots, x_k, \ldots, x_p)+
		\underbrace{W(x_1, \ldots, x_l, \ldots, x_l, \ldots, x_p)}_{=0}
	\end{multline*}
Оставшиеся два слагаемых отличаются положением двух аргументов и в сумме равны 0, что и треюовалось доказать.
\end{proof}

\begin{corollary}[1]
	Если $x_1, \ldots, x_p$ --- линейно зависимы, то антисимметрическая $W(x_1, \ldots, x_p)=0$.
\end{corollary}

\begin{corollary}[2]
	Если $p>n$, то антисимметрическая $W(x_1, \ldots, x_p)\equiv0$
\end{corollary}

\begin{Def}
	Пусть $\pi_p=(i_1, \ldots, i_p)$ --- перестановка индексов $\{1, 2, \ldots, p\}$. Действие перестановки на форму $\pi_pW$ --- это форма. $\pi_pW(x_1, \ldots, x_p):=W(x_{i_1}, \ldots, x_{i_p})$
\end{Def}

\begin{prop}
	Если $W$ --- симметрическая, то $\pi_pW=W$ для $\forall \pi_p\in S_p$, где $S_p$ --- группа перестановок. (читать: для любой перестановки).
\end{prop}

\begin{prop}
	Если $W$ --- антисимметрическая, то $\pi_pW=\sgn(\pi_p)W$
\end{prop}

\begin{prop}
	$\forall W\in \Omega_p^0, \dfrac{1}{p!}\sum\limits_{\pi_p\in S_p}\pi_pW$ --- симметрическая, а $\dfrac{1}{p!}\sum\limits_{\pi_p\in S_p}\sgn(\pi_p)\pi_pW$ --- антисимметрическая
\end{prop}

\begin{proof}
	Пусть $\pi_p'$ --- некоторая транспозиция, подействуем ею на нашу сумму: 
	\begin{multline*}\pi_p'\left(\dfrac{1}{p!}\sum\limits_{\pi_p\in S_p}\sgn(\pi_p)\pi_pW\right)=\dfrac{1}{p!}\sum\limits_{\pi_p\in S_p}\sgn(\pi_p)\pi'_p\pi_pW = \\ =
	\dfrac{1}{p!}\sum\limits_{\pi_p\in S_p}\sgn^2(\pi'_p)\sgn(\pi_p)\pi'_p\pi_pW = \sgn(\pi'_p)\dfrac{1}{p!}\sum\limits_{\pi_p\in S_p}\sgn(\pi'_p\circ\pi_p)(\pi'_p\circ\pi_p)W, 
\end{multline*}
	так как $\sgn(\pi'_p\circ\pi_p)=\sgn(\pi'_p)\cdot\sgn(\pi_p)$. Суммируя по всем перестановкам, получаем исходную форму, со знаком впереди.
\end{proof}

\begin{Def}
	Симметризация формы $W\in\Omega_p^0: \sym W =\dfrac{1}{p!}\sum\limits_{\pi_p\in S_p}\pi_pW$. Антисимметризация формы $W\in\Omega_p^0: \asym W =\dfrac{1}{p!}\sum\limits_{\pi_p\in S_p}\sgn(\pi_p)\pi_pW$
\end{Def}

\begin{prop}
	$\sym W$ --- симметрическая форма, $\asym W$ --- антисимметрическая форма. $\sym , \asym $ --- линейные операторы на $\Omega_p^0$, сохраняющие симметрические и антисимметрические формы соответственно.
\end{prop}

\begin{Def}
	Если $U\in\Omega_p^0, V\in\Omega_p^0$ --- антисимметрические, то их внешним произведением называется $U\wedge V = \dfrac{(p+q)!}{p!q!}\asym(U\otimes V)$
\end{Def}


\begin{lemma}
	Если $U\in \Omega_p^0, V\in\Omega_q^0$, то $\asym(\asym U\otimes V)=\asym(U \otimes \asym V)=asym(U\otimes V)$
\end{lemma}

\begin{proof}Распишем:\\
	\mbox{$T = \asym(\asym U\otimes V)=\asym\left(\left(\dfrac{1}{p!}\sum\limits_{\pi\in S_p}\sgn(\pi)\pi U\right)\otimes V\right) = \dfrac{1}{p!}\sum\limits_{\pi\in S_p}\sgn(\pi)\asym((\pi U)\otimes V)$}. Рассмотрим отдельно на произвольном наборе векторов: 
	\begin{multline*}
		(\pi U)\otimes V(x_1, \ldots, x_{p+q}) = (\pi U)(x_1, \ldots, x_p)V(x_{p+1}, \ldots, x_{p+q}) =\\= U(x_{\pi(1)}, \ldots, x_{\pi(p)})V(x_{p+1}, \ldots, x_{p+q}) = \varpi_\pi (U\otimes V)(x_1, \ldots, x_{p+q}),
	\end{multline*}
	где $\varpi_\pi(i)=\pi(i), i=1, \ldots, p; \varpi_\pi(i)=i, i=p+1, \ldots, p+1$. 
	
	Тогда $T = \dfrac{1}{p!}\sum\limits_{\pi\in S_p}\sgn(\pi)\asym(\varpi_\pi(U\otimes V))\overset{\text{утв. 10}}{=} \dfrac{1}{p!}\sum\limits_{\pi\in S_p}\sgn(\pi)\sgn(\varpi_\pi)\asym(U\otimes V)$, так как  $\sgn(\pi)=\sgn(\varpi_\pi)$, то $T = \asym(U\otimes V)$.
\end{proof}

Введем обозначение $\Lambda_p$ --- линейное пространство антисимметрических форм из $\Omega_p^0$

\begin{theorem}(Основные свойства внешнего произведения)
	\begin{enumerate}
		\item (Дистрибутивность)\\$(\alpha U_1+\beta U_2)\wedge V = \alpha(U_1\wedge V)+\beta(U_2\wedge V); U\wedge(\alpha V_1+\beta V_2)=\alpha( U\wedge V_1)+\beta (U\wedge V_2),\\ (\forall \alpha, \beta \in \R), U_1, U_2, U\in \Lambda_p, V_1, V_2, V\in \Lambda_q$
		\item (Ассоциативность)\\$(U\wedge V)\wedge W = U \wedge (V\wedge W) = \dfrac{(p+q+r)!}{p!q!r!}\asym(U\otimes V\otimes W),\\ 
		U\in\Lambda_p, V\in\Lambda_q, W\in\Lambda_r$
		\item (Антикоммутативность) \\ $U\wedge V = (-1)^{pq}V\wedge U, U\in\Lambda_p, V\in\Lambda_q$
	\end{enumerate}
\end{theorem}

\begin{proof}
	\begin{enumerate}
		\item Следует из свойства 1 утверждения 4, определения, а также утверждения 12.
		
		\item \begin{multline*}
			(U\wedge V)\wedge W = \dfrac{((p+q)+r)!}{(p+q)!r!}\asym((U\wedge V)\otimes W)=\\ =\dfrac{(p+q+r)!}{(p+q)!r!}\asym\left(\dfrac{(p+q)!}{p!q!}\asym(U\otimes V)\otimes W\right) =\\= \dfrac{(p+q+r)!}{p!q!r!}\asym(\asym(U\otimes V)\otimes W)=\\=\text{[по лемме 1]}=\dfrac{(p+q+r)!}{p!q!r!}\asym((U\otimes V)\otimes W).
		\end{multline*}
	
		\item 
		$U\wedge V = \dfrac{(p+q)!}{p!q!}\asym(U\otimes V)=\dfrac{1}{p!q!}\sum\limits_{\pi\in S_{p+q}}\sgn(\pi)(\pi (U\otimes V))$.
		
		 Отдельно рассмотрим 
		 $T = \pi(U\otimes V)(x_1, \ldots, x_{p+q})=U\otimes V(x_{\pi(1)}, \ldots, x_{\pi(p+q)})=\\=U(x_{\pi(1)}, \ldots, x_{\pi(p)})V(x_{\pi(p+1)}, \ldots, x_{\pi(p+q)})$. Введем новую перестановку:
		 
		  $\pi'(i)=\pi(p+1), i=1, \ldots, q, \pi'(i)=\pi(i-q), i=q+1, \ldots, p+q$. 
		Тогда $T = U(x_{\pi'(q+1)}, \ldots, x_{\pi'(q+p)})V(x_{\pi'(1)}, \ldots, x_{\pi'(q)})=\pi'V\otimes U(x_1, \ldots, x_{p+q})$. 
		
		Вернемся к выражению $U\wedge V = \dfrac{1}{p!q!}\sum\limits_{\pi\in S_{p+q}}\sgn(\pi)(\pi'(V\otimes U))=\dfrac{(-1)^{pq}}{p!q!}\sum\limits_{\pi'\in S_{p+q}}\sgn(\pi')(\pi'(V\otimes U)) = (-1)^{pq}\dfrac{(p+q)!}{p!q!}\asym(V\otimes U)=(-1)^{pq}V\wedge U$, так как знаки перестановок $\pi$ и $\pi'$ отличаюстя на $(-1)^{pq}$.
	\end{enumerate}
\end{proof}















