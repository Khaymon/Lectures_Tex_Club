\subsection{Нормальные подгруппы}

\begin{definition}
	Пусть $G$ "--- группа, $H \le G$. Подгруппа $H$ называется \textit{нормальной} в $G$, если $\forall g \in G: gH = Hg$. Обозначение "--- $H \normal G$.
\end{definition}

\begin{note}
	Определение выше имеет ряд эквивалентных формулировок:
	\begin{enumerate}
		\item $G / H = H \bs G$
		\item $\forall g \in G: gHg^{-1} = H$
		\item $\forall g \in G: gH \subset Hg$ (поскольку тогда $\forall g \in G: Hg^{-1} \subset g^{-1}H)$
		\item $\forall g \in G: gHg^{-1} \subset H$
	\end{enumerate}
\end{note}

\begin{example}
	Рассмотрим несколько примеров нормальных подгрупп в соответствующих группах:
	\begin{enumerate}
		\item $\forall H \le G: H \normal G$, где $G$ "--- абелева группа
		\item $G \normal G$, $\{e\} \normal G$, где $G$ "--- произвольная группа
		\item $A_n \normal S_n$, поскольку $\forall \sigma \in S_n: \forall \tau \in A_n: \sgn(\sigma\tau\sigma^{-1}) = \sgn\tau = 1$, то есть $\forall \sigma \in S_n: \sigma A_n\sigma^{-1} \subset A_n$
	\end{enumerate}
\end{example}

\begin{proposition}
	Пусть $G$ "--- группа, $H_1 \normal G$, $H_2 \normal G$. Тогда $H_1 \cap H_2 \hm\normal G$.
\end{proposition}

\begin{proof}
	$\forall g \in G: g(H_1 \cap H_2)g^{-1} \subset (gH_1g^{-1}) \cap (gH_2g^{-1}) \hm= H_1 \cap H_2$.
\end{proof}

\begin{proposition}
	Пусть $G$ "--- группа, $H \normal G$, $K \le G$. Тогда:
	\begin{enumerate}
		\item $HK \le G$
		\item Если $K \normal G$, то $HK \normal G$
	\end{enumerate}
\end{proposition}

\begin{proof}
	Заметим, что $HK = \bigcup_{k \in K}Hk = \bigcup_{k \in K}kH = KH$. Воспользуемся этим свойством:
	\begin{enumerate}
		\item $(HK)(HK) = H(KH)K = H(HK)K = (HH)(KK) = HK$, поэтому $HK$ замкнуто относительно умножения, и, аналогично, $(HK)^{-1} = K^{-1}H^{-1} = KH = HK$
		\item $\forall g \in G: g(HK)g^{-1} = (gHg^{-1})(gKg^{-1}) = HK$
	\end{enumerate}
\end{proof}

\begin{proposition}
	Пусть $G$ "--- группа, $H \le G$, $|G : H| = 2$. Тогда $H \normal G$.
\end{proposition}

\begin{proof}
	По условию, $G / H = \{H, G \bs H\} = H \bs G$.
\end{proof}

\begin{note}
	Покажем, что $|S_n : A_n| = 2$ при $n \ge 2$. Действительно, сопоставление $\sigma \hm\mapsto (1, 2)\sigma$ осуществляет биекцию между $A_n$ и $S_n \bs A_n$, поэтому $S_n / A_n = \{A_n, (1, 2)A_n\}$.
\end{note}

\begin{definition}
	Пусть $G$ "--- группа, $g, x \in G$. Элементом, \textit{сопряженным} к $g$ с помощью $x$, называется $g^x := x^{-1}gx \in G$. Элементы $g_1, g_2 \in G$ называются \textit{сопряженными}, если $\exists x \in G: g_1 = g_2^x$.
\end{definition}

\begin{note}
	Пусть $G$ "--- группа, $H \le G$, $g \in G$. Будем обозначать $g^{-1}Hg$ как $H^g$.
\end{note}

\begin{proposition} Пусть $G$ "--- группа. Тогда:
	\begin{enumerate}
		\item $\forall g, x, y \in G: g^{xy} = (g^x)^y$
		\item $\forall g_1, g_2, x \in G: (g_1g_2)^x = g_1^xg_2^x$
	\end{enumerate}
\end{proposition}

\begin{proof} Произведем непосредственную проверку:
	\begin{enumerate}
		\item $(g^x)^y = y^{-1}(x^{-1}gx)y = (xy)^{-1}g(xy) = g^{xy}$
		\item $g_1^xg_2^x = (x^{-1}g_1x)(x^{-1}g_2x) = x^{-1}(g_1g_2)x = (g_1g_2)^x$
	\end{enumerate}
\end{proof}

\begin{proposition}
	Сопряженность является отношением эквивалентности в группе $G$.
\end{proposition}

\begin{proof} Произведем непосредственную проверку:
	\begin{itemize}
		\item (Рефлексивность) $\forall g \in G: g = g^e$
		\item (Симметричность) Если $g_2 = g_1^x$, то $g_2^{x^{-1}} = (g_1^x)^{x^{-1}} = g_1^e = g_1$
		\item (Транзитивность) Если $g_2 = g_1^x$, $g_3 = g_2^y$, то $g_3 = (g_1^x)^y = g_1^{xy}$
	\end{itemize}
\end{proof}

\begin{definition}
	Пусть $G$ "--- группа, $g \in G$. \textit{Классом сопряженности} элемента $g$ называется множество $g^G$, то есть класс эквивалентности по отношению сопряженности, содержащий $g$.
\end{definition}

\begin{proposition}
	Пусть $G$ "--- группа, $H \le G$. Тогда $H \normal G \hm\Leftrightarrow H$ "--- объединение некоторого количества классов сопряженности в $G$.
\end{proposition}

\begin{proof}
	$H \normal G \Leftrightarrow \forall g \in G: g^{-1}Hg \subset H \Leftrightarrow \forall g \in G: \forall h \hm\in H: h^g \in H \Leftrightarrow \forall h \in H: h^G \subset H \Leftrightarrow H = \bigcup_{h \in H}h^G$.
\end{proof}

\begin{exercise}
	Пусть $G$ "--- группа, $g_1, g_2 \in G$. Докажите, что тогда $g_1^Gg_2^G$ "--- объединение нескольких классов сопряженности, причем необязательно одного.
\end{exercise}

\begin{solution}
	Пусть $x, y \in G, x^{-1}g_1xy^{-1}g_2y \in g_1^Gg_2^G$. Нам достаточно показать, что если $a \hm\in G$ сопряжен с $x^{-1}g_1xy^{-1}g_2y$, то $a \in g_1^Gg_2^G$. Действительно, в силу сопряженности, $a \hm= z^{-1}x^{-1}g_1xy^{-1}g_2yz$ для некоторого $z \in G$, то есть $a = (xz)^{-1}g_1(xz)(yz)^{-1}g_2(yz) \in g_1^Gg_2^G$. Пример, когда класс сопряженности в $g_1^Gg_2^G$ действительно не один, легко строится на основе следующего примера.
\end{solution}

\begin{example}
	Пусть $\sigma \in S_n$. Представим $\sigma$ в виде произведения независимых циклов, $\sigma \hm= (a_1\dotsc a_k)(b_1\dotsc b_l)\dotsc$, и рассмотрим $\sigma^\tau$ для произвольного $\tau \in S_n$. Если $a_i' := \tau^{-1}(a_i), i \hm\in \{1, \dots, k\}$, то $\sigma^\tau(a_i') \hm= (\tau^{-1}\sigma\tau)(a_i') = a_{i + 1}'$. Значит, $\sigma^\tau = (a_1'\dotsc a_k')(b_1'\dotsc b_l')\dotsc$, и, следовательно, $\sigma^{S_n}$ состоит из перестановок того же циклического типа, что и $\sigma$, причем из всех, потому что по каждой такой перестановке легко восстанавливается соответствующая $\tau \in S_n$.
\end{example}