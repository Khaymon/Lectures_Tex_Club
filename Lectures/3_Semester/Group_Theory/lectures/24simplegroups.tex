\subsection{Простые группы}

\begin{definition}
	Пусть $G$ "--- группа, $|G| > 1$. $G$ называется \textit{простой}, если в ней нет нормальных подгрупп, отличных от $\{e\}$ и $G$.
\end{definition}

\begin{note}
	Пусть $G$ "--- конечная группа. Рассмотрим максимальный субнормальный ряд в $G = G_0 \ge G_1 \ge \dots \ge G_n = \{e\}$, в котором все группы различны. Тогда $\forall i \in \{1, \dotsc, n\}: G_{i-1}/G_i$ "--- простая, поскольку иначе в ряд можно было бы добавить такую подгруппу $G_{i-1} \ge H \ge G_i$, что $H/ G_i \normal G_{i-1} / G_i$.
\end{note}

\begin{proposition}
	Абелева группа $A$ "--- простая $\lra$ $A \cong \Z_p$, где число $p$ "--- простое.
\end{proposition}

\begin{proof}~
	\begin{itemize}
		\item[$\ra$] Рассмотрим $a \in A \backslash \{e\}$, $\ord{a} = n$, $p\mid n$. Тогда если $b = a^{\frac np} \in A$, то $\ord{b} = p$. Но $B := \gl b \gr \normal A$, поскольку $A$ "--- абелева, значит, $A = B$, то есть $A \cong \Z_p$.
		\item[$\la$] Если $A \cong \Z_p$, то $\forall B \le A: |B| = 1$ или $|B| = p$, то есть $B = \{e\}$ или $B = A$, поэтому $A$ "--- простая.
	\end{itemize}
\end{proof}

\begin{proposition}
	Пусть $G$ "--- конечная группа, $H \le G$, причем $|G : H| = 2$, и $h \in H$. Если $C_G(h) \ne C_H(h)$, то $h^G = h^H$. В противном случае, $|h^G| = 2|h^H|$.
\end{proposition}

\begin{proof}
	Заметим, что $h^H \subset h^G$. С другой стороны, если рассмотреть действие $H$ на себя сопряжениями, то $h^H$ будет орбитой $h$, откуда $|h^H| = \frac{|H|}{|C_H(h)|}$, и, аналогично, $|h^G| = \frac{|G|}{|C_G(h)|}$.
	
	Если $C_G(h) \ne C_H(h)$, то рассмотрим $g \in C_G(h) \backslash C_H(h)$. Тогда $G = H \sqcup gH$, поэтому $C_G(h) = (C_G(h) \cap H) \sqcup (C_G(h) \cap gH)$, причем сопоставление $x \mapsto gx$ осуществляет биекцию между этими двумя множествами. Значит, $|C_G(h)| = 2|C_G(h) \cap H|= 2|C_H(h)|$, тогда $|h^G| \hm= |h^H| \ra h^G \hm= h^H$. Если же $C_G(h) = C_H(h)$, то, по той же формуле для мощностей орбит, $|h^G| = 2|h^H|$.
\end{proof}

\begin{theorem}
	$\forall n \in \N, n \ge 5:$ группа $A_n$ "--- простая.
\end{theorem}

\begin{proof}
	Докажем сначала, что $A_5$ "--- простая. Пользуясь предыдущим утверждением, перечислим все классы сопряженности элементов $h \in A_5$ в $A_5$ и $S_5$:
	
	\def\arraystretch{1.5}
	\begin{center}
		\begin{tabular}{c|c|c|c}
			$h^{S_5}$ & $g \in C_G(h) \backslash C_H(h)$ & $h^{A_5}$ & |$h^{A_5}$| \\
			\hline
			$\id^{S_5}$ & $(12)$ & $\id^{S_5}$ & 1\\
			\hline
			$(123)^{S_5}$ & $(45)$ & $(123)^{S_5}$ & $\frac{A_5^3}3 = 20$\\
			\hline
			$((12)(34))^{S_5}$ & $(12)$ & $((12)(34))^{S_5}$ & $\frac{C_5^2C_3^2}2 = 15$ \\
			\hline
			\multirow{2}{*}{$(12345)^{S_5}$}& \multirow{2}{*}{---} & $(12345)^{A_5}$ & $\frac12\frac{P_5}{5} = 12$ \\
			\cline{3-4}
			&  & $(12354)^{A_5}$ & $\frac12\frac{P_5}{5} = 12$ \\
		\end{tabular}
	\end{center}
	\def\arraystretch{1}
	
	Остается непосредственно убедиться, что сумма мощностей никаких двух и более классов сопряженности в $A_5$ не является собственным делителем $|A_5| = 60$. Значит, в $A_5$ нет нормальных подгрупп, отличных от $\{id\}$ и $A_5$.
	
	Проведем теперь индукцию по $n$. База, $n = 5$, уже доказана. Пусть теперь $n \ge 6$. Рассмотрим $H \normal A_n$, $H \ne \{\id\}$. Рассмотрим два случая:
	\begin{enumerate}
		\item Пусть $\exists \sigma \in H, \sigma \ne \id: \exists i \in \{1, \dotsc, n\}: \sigma(i) = i$. Без ограничения общности можно считать, что $\sigma(n) = n$, то есть $\sigma \in A_{n-1}$. Рассмотрим $L := H \cap A_{n - 1} \normal A_{n - 1}$. По предположению индукции, $L = A_{n - 1}$, поскольку $L$ нетривиальна, и, следовательно, $(123) \in L \subset H \hm\ra \gl(123)^{A_n}\gr = A_n \le H$, то есть $H = A_n$.
		\item Пусть $\forall \sigma \in H, \sigma \ne \id: \forall i \in \{1, \dotsc, n\}: \sigma(i) \ne i$. Зафиксируем $\sigma \in H, \sigma \ne \id$. Без ограничения общности можно считать, что $\sigma(1) = 2$. Выберем $j \in \{3, \dotsc, n\}$, $i := \sigma^{-1}(j)$, $i \ne j$ и $k, l \in \{1, \dotsc, n\} \backslash \{1, 2, i, j\}$. Положим $\tau := \sigma^{(jkl)} \hm= (jlk)\sigma(jkl)$. Рассмотрим $\rho := \sigma\tau^{-1} \in H$ и заметим, что $\rho(2) = 2$, $\rho(l) = j$. Итак $\rho \ne \id$, но $\rho$ обладает неподвижной точкой --- противоречие. Значит, при $H \ne \{id\}$ этот случай невозможен.
	\end{enumerate}
	
	Таким образом, переход индукции доказан.
\end{proof}

\begin{note}
	Группа $A_4$ "--- уже не простая, поскольку $V_4 \normal A_4$.
\end{note}

\begin{note}
	Можно показать, что если $F$ "--- поле, $k \ge 2$, то группа $\PSL_k(F) \hm{:=} \SL_k(F) / Z(\SL_k(F))$ проста при $|F| \ge 4$ или $k \ge 3$ ($Z(\SL_k(F)) = \{\alpha E: \alpha \in F, \alpha^k = 1\}$).
\end{note}

\begin{exercise}
	Докажите, что $\PSL_2(\Z_2) \cong S_3$.
\end{exercise}

\begin{solution}
	Заметим, что $\SL_2(\Z_2)  = \GL_2(\Z_2)$, и $Z(\SL_2(\Z_2)) = \{E\}$, поэтому $\PSL_2(\Z_2) \hm\cong \GL_2(\Z_2)$. Легко убедиться, что $\GL_2(\Z_2)$ осуществляет перестановки трех ненулевых векторов в $\Z_2^2$, причем все такие перестановки, и $|\GL_2(\Z_2)| = 6$, поэтому сопоставление каждой матрице соответствующей перестановки "--- это изоморфизм между $\GL_2(\Z_2)$ и $S_3$.
\end{solution}

\begin{note}
	Можно также показать, что $\PSL_2(\Z_3) \cong A_4$ и, кроме того, $\PSL_2(\mathbb{F}_4) \hm\cong \PSL_2(\Z_5) \cong A_5$.
\end{note}

\begin{note}
	На данный момент классификация конечных простых групп считается завершенной. Она состоит из конечного числа бесконечных серий и конечного числа так называемых \textit{спорадических} групп, не попадающих ни в одну из серий.
\end{note}

\begin{proposition}
	Пусть $G$ "--- неабалева простая группа. Тогда $G' \hm= G$.
\end{proposition}

\begin{proof}
	$G' \ne \{e\}$ в силу неабелевости $G$, и $G' \normal G$, следовательно, $G' = G$.
\end{proof}

\begin{corollary}
	$\forall n \in \N, n \ge 5:$ группа $A_n$ неразрешима.
\end{corollary}

\begin{theorem}
	Группа $\mathrm{SO}_3 = \{A \in \mathcal{O}_3: \det{A} = 1\}$ "--- простая.
\end{theorem}

\begin{proof}
	Воспользуемся следующим фактом из линейной алгебры: $\forall A \in \mathrm{SO}_3: \exists S \in \mathrm{SO_3}:$
	\[S^{-1}AS = \begin{pmatrix}
		1 & 0 & 0 \\
		0 & \cos\alpha & -\sin\alpha\\
		0 & \sin\alpha & \cos\alpha\\
	\end{pmatrix}\]
	
	Рассмотрим $H \normal \mathrm{SO}_3, H \ne \{E\}$. Пусть $A \in H, A \ne E$ "--- вращение на угол $\alpha$ вокруг некоторой оси, тогда $A^{\mathrm{SO}_3}$ "--- это все вращения на угол $\alpha$, и $A^{\mathrm{SO}_3} \subset H$.
	
	Пусть $B(\beta)$ "--- вращение на угол $\beta$ вокруг оси, подобранной так, что $B(\beta)$ не коммутирует с $A$ при $\beta = \beta_0$, то есть $A^{B(\beta_0)} \ne A$. Положим $C(\beta) := (A^{B(\beta)})^{-1}A$. Заметим, что тогда $C(0) = E$ и $C(\beta_0) \ne E$, причем элементы $C$ "--- непрерывные функции $\beta$. Если $C(\beta)$ "--- это вращение на угол $\mu(\beta)$, то, поскольку след матрицы инвариантен относительно замены базиса и $1 + 2\cos\mu = \tr C(\beta)$, $\mu(\beta)$ "--- тоже непрерывная функция $\beta$. Значит, $H$ содержит хотя по по одному повороту на каждый угол из отрезка $[0, \mu(\beta_0)]$, $\mu(\beta_0) \ne 0$, и, следовательно, содержит все такие повороты. Но тогда $H$ содержит и все повороты на углы из отрезков $[0, 2\mu(\beta_0)], [0, 3\mu(\beta_0)], \dotsc, [0, 2\pi]$. Значит, $H = \mathrm{SO}_3$.
\end{proof}