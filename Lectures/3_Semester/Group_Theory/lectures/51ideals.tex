\section{Кольца и поля}

\subsection{Идеалы колец и факторкольцо}

\begin{reminder}
	\textit{Кольцом} называется множество $R$ с определенными на нем бинарными операциями $+$ и $\cdot$ такими, что:
	\begin{enumerate}
		\item $(R, +)$ "--- абелева группа
		\item $\forall a, b, c \in R: a(bc) = (ab)c$ (то есть $(R, \cdot)$ является \textit{полугруппой})
		\item $\forall a, b, c \in R: a(b + c) = ab + ac$, $(a + b)c = ac + bc$
	\end{enumerate}

	$R$ называется \textit{кольцом с единицей}, если в нем есть нейтральный элемент относительно умножения, обозначаемый через 1.
\end{reminder}

\begin{example}
	Рассмотрим несколько примеров колец:
	\begin{enumerate}
		\item $\Z$, $\Z_n$
		\item $F[x]$, где $F$ "--- произвольное поле
		\item $M_n(F)$, где $F$ "--- произвольное поле
	\end{enumerate}
\end{example}

\begin{reminder}
	\textit{Алгеброй} над полем $F$ называется множество $R$ с определенными на нем бинарными операциями $+$ и $\cdot$ и операцией умножения на скаляры из $F$ такими, что:
	\begin{enumerate}
		\item $(R, +, \cdot)$ "--- кольцо
		\item $R$ "--- линейное пространство над $F$
		\item $\forall \alpha \in F: \forall a, b \in A: \alpha(ab) = (\alpha a)b = a(\alpha b)$
	\end{enumerate}

	$R$ называется \textit{алгеброй с единицей}, если в ней есть нейтральный элемент относительно умножения, обозначаемый через 1.
\end{reminder}

\begin{note}
	Мы всегда предполагаем, что 0 и 1 в кольце или алгебре не совпадают.
\end{note}

\begin{definition}
	\textit{Подкольцом} кольца $R$ называется множество $S \subset R$, $S \ne \emptyset$, замкнутое относительно сложения, умножения и взятия обратного элемента относительно сложения. Обозначение "--- $S \le R$.
	
	\textit{Подалгеброй} алгебры $R$ над полем $F$ называется $S \hm\le R$ "--- подкольцо, замкнутое относительно умножения на скаляры из $F$. Если $R$ "--- кольцо (алгебра) с единицей, то $S$ называется \textit{подкольцом (подалгеброй) с единицей} при условии, что $1 \in S$.
\end{definition}

\begin{note}
	$M_n(F) \supset M_{n - 1}(F)$, но единицы в этих кольцах различны, поэтому $M_{n-1}(F)$ "--- подкольцо в $M_n(F)$, но не подкольцо с единицей.
\end{note}

\begin{note}
	Если $R$ "--- алгебра с единицей $1$ над полем $F$, то тогда $F \cdot 1 \le R$ "--- подалгебра, изоморфная $F$, причем $F \cdot 1$ содержится в центре алгебры.
\end{note}

\begin{definition}
	Пусть $R, S$ "--- кольца. \textit{Гомоморфизмом колец} $R$ и $S$ называется отображение $\phi: R \to S$ такое, что $\forall x, y \in R: \phi(x + y) \hm= \phi(x) + \phi(y)$, $\phi(xy) = \phi(x)\phi(y)$.
	
	Если $R, S$ "--- кольца с единицей, то $\phi$ называется \textit{гомоморфизмом колец с единицей} при условии, что $\phi(1) = 1$.
\end{definition}

\begin{note}
	Гомоморфизм алгебр над одним полем $F$ "--- это гомоморфизм колец, являющийся при этом линейным отображением.
\end{note}

\begin{definition}
	Пусть $R, S$ "--- кольца, $\phi : R \hm\to S$ "--- гомоморфизм $R$ и $S$. Тогда:
	\begin{itemize}
		\item \textit{Образом} $\phi$ называется $\im\phi := \phi(R)$
		\item \textit{Ядром} $\phi$ называется $\ke\phi := \phi^{-1}(0)$
	\end{itemize}
\end{definition}

\begin{definition}
	Пусть $R$ "--- кольцо (алгебра), $I \subset R$. $I$ называется \textit{идеалом} $R$, если:
	\begin{enumerate}
		\item $(I, +) \le (R, +)$ (в случае, когда $R$ "--- алгебра, $I$ должно быть подпространством в $R$)
		\item $\forall a \in R: aI \subset I, Ia \subset I$
	\end{enumerate}
	
	Обозначение "--- $I \normal R$.
\end{definition}

\begin{reminder}
	В любом кольце $R$ выполнено свойство $\forall a \in R: 0a = a0 = 0$.
\end{reminder}

\begin{proposition}
	Пусть $\phi: R \to S$ "--- гомоморфизм колец (алгебр). Тогда $\im\phi \le S$ и $\ke\phi \normal R$.
\end{proposition}

\begin{proof}~
	\begin{enumerate}
		\item Пусть $a, b \in \im\phi$, то есть $a = \phi(x), b = \phi(y), x, y \hm\in R$. Тогда $a + b = \phi(a + b), -a \hm= \phi(-x), ab = \phi(xy) \in \im\phi$ (а в случае гомоморфизма алгебр $\forall \alpha \in F: \alpha a = \alpha\phi(x) \hm= \phi(\alpha x) \in \im\phi$) поэтому $\im\phi \le S$.
		\item Пусть $x, y \in \ke\phi$, тогда $\phi(x + y) = \phi(x) + \phi(y) = 0$, $\phi(-x) \hm= -\phi(x) = 0$ (а в случае гомоморфизма алгебр $\forall \alpha \in F: \phi(\alpha x) \hm= \alpha\phi(x) = 0$). Наконец, $\forall a \in R: \phi(ax) \hm= \phi(a)\phi(x) \hm= \phi(a)0 = 0$ и, аналогично, $\forall a \in R: \phi(xa) = 0$. Значит, $\ke\phi \normal R$.
	\end{enumerate}
\end{proof}

\begin{definition}
	Пусть $R$ "--- кольцо, $I \normal R$, $I \ne R$. Тогда $R / I$ "--- аддитивная факторгруппа. Определим умножение на $R / I$ следующим образом: $\forall a, b \in R: (a + I)(b + I) \hm{:=} ab + I$.
\end{definition}

\begin{note}
	В отличии от случая факторгруппы, здесь операция задана не инвариантным образом, поэтому требуется проверить корректность определения. Пусть $a' \in a + I$, тогда $a' = a + x, x \in I$, поэтому $a'b = ab + xb \in ab + I$. Аналогично проверяется независимость от выбора представителя во втором множителе.
\end{note}

\begin{theorem}
	Пусть $R$ "--- кольцо, $I \normal R$, $I \ne R$. Тогда $(R / I, +, \cdot)$ "--- кольцо. Более того, отображение $\pi: R \to R/I$, $\forall a \in R: \pi(a) \hm= a + I$, является сюръективным гомоморфизмом колец.
\end{theorem}

\begin{proof}
	Проверим сначала, что $\phi$ сохраняет операции:
	\begin{itemize}
		\item (Сложение) $\forall a, b \in I: \pi(a + b) = a + b + I = (a + I) + (b + I) \hm= \pi(a) + \pi(b)$
		\item (Умножение) $\forall a, b \in I: \pi(ab) = ab + I = (a + I)(b + I) = \pi(a)\pi(b)$
	\end{itemize}

	Отображение $\pi$ сюръективно и сохраняет операции, из чего следует, что $R / I$ "--- кольцо, а $\pi$ "--- гомоморфизм колец.
\end{proof}

\begin{definition}
	Пусть $R$ "--- кольцо, $I \normal R$, $I \ne R$. Кольцо $R / I$ называется \textit{факторкольцом} $R$ по $I$.
\end{definition}

\begin{note}
	В терминах теоремы выше, $\ke\pi = I$ аналогично случаю факторгруппы. Значит, любой идеал является ядром некоторого гомоморфизма.
\end{note}

\begin{note}
	Если $R$ "--- кольцо с единицей 1, $I \normal R$. Легко видеть, что тогда $I \ne R \hm\lra 1 \not\in I$. При $I \ne R$ факторкольцо $I / R$ является кольцом с единицей $1 + I \ne I$.
\end{note}

\begin{theorem}[Основная теорема о гомоморфизмах колец]~
	\begin{enumerate}
		\item Пусть $R$ "--- кольцо, $I \normal R$, $I \ne R$. Тогда $\exists \pi: R \to R / I$ "--- эпиморфизм колец такой, что $\ke\phi \hm= I$.
		\item Пусть $R, S$ "--- кольца, $\phi: R \to S$ "--- гомоморфизм колец. Тогда $I \hm{:=} \ke\phi \normal R$, и, более того, $\im\phi \cong R / I$.
	\end{enumerate}
\end{theorem}

\begin{proof}
	Большая часть теоремы уже была доказана выше, остается доказать лишь последнее утверждение. Мы уже знаем, что $(\im\phi, +) \cong (R / I, +)$, причем этот изоморфизм групп имеет вид $\Theta: \im\phi \to R / I$, $\forall x \in \im\phi: \Theta(x) = \phi^{-1}(x)$. Проверим, что $\Theta$ сохраняет умножение: $\forall x, y \in \im\phi: \forall a \in \phi^{-1}(x): \forall b \in \phi^{-1}(y): ab \hm\in \phi^{-1}(xy) \ra \Theta(x) = a + I, \Theta(y) = b + I, \Theta(xy) = ab + I$, поэтому $\Theta(xy) = \Theta(x)\Theta(y)$.
\end{proof}

\begin{note}
	Аналогично случаю групп, $\Theta^{-1}$ имеет следующий вид: $\forall a + I \in R / I: \Theta^{-1}(a \hm+ I) = \phi(a)$. Кроме того, имеет место аналогичная коммутативная диаграмма:
	\[
	\begin{tikzcd}[row sep = huge]
		R \arrow{rr}{\phi} \arrow[swap]{dr}{\pi} && \im\phi \le S \\
		& R / I \arrow[swap]{ur}{\Theta} &
	\end{tikzcd}
	\]
\end{note}

\begin{note}
	Если $R$ "--- алгебра с единицей, то понятия идеала в $R$ как в кольце и как в алгебре эквивалентны, поскольку $F \cong F \cdot 1 \le R$. Дальнейшая теория для гомоморфизмов алгебр (вне зависимости от наличия единицы в $R$) строится аналогоично, но с некоторым дополнением.
\end{note}

\begin{definition}
	Пусть $V$ "--- линейное пространство над полем $F$, $U \le V$. Тогда $(U, +) \hm\le (V, +)$, поэтому можно определить факторгруппу $(V / U, +) = \{\overline{v} + U: \overline{v} \in V\}$. Определим умножение на скаляры из $F$ на $V / U$ следующим образом: $\forall \alpha \in F: \forall \overline{v} \in V: \alpha(\overline{v} + U) \hm{:=} \alpha v + U$.
\end{definition}

\begin{note}
	Корректность определения выше проверяется аналогично случаю колец. Далее аналогичным образом можно доказать, что $\pi: V \to V / U$, $\forall \overline{v} \in V: \pi(\overline{v}) = \overline{v} + U$ "--- сюръективное линейное отображение такое, что $\ke\pi = U$, поэтому $V / U$ "--- линейное пространство, называемое \textit{факторпространством}.
	
	Наконец, если $R$ "--- алгебра, $I \normal R$, $I \ne R$, то $R / I$ "--- алгебра, и также имеет место основная теорема о гомоморфизмах алгебр.
\end{note}

\begin{exercise}
	Пусть $V$ "--- конечномерное линейное пространство, $\phi \in \mathcal{L}(V)$, $U \le V$ инвариантно относительно $\phi$. Тогда в базисе, согласованном с $U$, матрица $\phi$ имеет вид:
	\[\phi \xleftrightarrow[e]{} A = \begin{pmatrix}B & C \\ 0 & D\end{pmatrix}\]
	
	Докажите, что $D$ "--- это матрица оператора $\psi \in \mathcal{L}(V / U)$ такого, что $\forall \overline{v} \in V: \psi(\overline{v} \hm+ U) = \phi(\overline{v}) + U$.
\end{exercise}

\begin{solution}
	Проверим сначала, что оператор $\psi$ определен корректно. Пусть $\overline{v} + U = \overline{w} + U$, тогда $\overline{v} - \overline{w} \in U$ и $\overline{v} = \overline{w} + \overline{u}, \overline{u} \in U$. Значит, $\psi(\overline{v} + U) = \phi(\overline{v}) + U = \phi(\overline{w} + \overline{u}) + U = \phi(\overline{w}) + U$.
	
	Пусть теперь $e = (\overline{e_1}, \dotsc, \overline{e_n})$ "--- такой базис в $V$, что $(\overline{e_1}, \dotsc, \overline{e_k})$ "--- базис в $U$. Легко убедиться, что тогда $(\overline{e_{k+1}} + U, \dotsc, \overline{e_n} + U)$ "--- базис в $V / U$. Значит, в данном базисе $D$ "--- это матрица оператора $\psi$.
\end{solution}

\begin{note}
	Аналогично случаю групп, имеют место теоремы об изоморфизмах колец и алгебр.
\end{note}

\begin{definition}
	Пусть $R$ "--- кольцо, $I \normal R$, $I \ne R$. $I$ называется \textit{максимальным идеалом}, если $\forall J \normal R, J \ne R: J \subset I$. 
\end{definition}
	
\begin{definition}
	Пусть $R$ "--- кольцо, $a \in R$. \textit{Главным идеалом, порожденным $a$,} называется наименьший по включению идеал, содержащий $a$. Обозначение "--- $(a)$.
\end{definition}

\begin{note}
	Если $R$ "--- коммутативное кольцо с единицей, то $(a) \hm= aR := \{ar: r \in R\}$, а если $R$ "--- некоммутативное кольцо с единицей, то $(a) = RaR := \{\sum_{i = 1}^nr_ias_i : \forall i \in \{1, \dotsc, n\}: r_i, s_i \in R\}$.
\end{note}

\begin{theorem}
	Пусть $R$ "--- коммутативное кольцо с единицей, $I \hm\normal R$, $I \ne R$. Тогда $I$ максимален $\lra$ $R / I$ "--- поле.
\end{theorem}

\begin{proof}~
	\begin{itemize}
		\item[$\la$] Пусть $I$ не максимален, то есть $\exists J \normal R, J \ne R: I \subsetneq J$. Легко показать, что тогда $J / I \normal R / I$, причем $J / I \ne \{I\}$ и $J / I \ne R / I$. Но в поле любой ненулевой идеал совпадает со всем полем, поскольку он содержит единицу поля, --- противоречие.
		\item[$\ra$] Пусть $I$ максимален, $a + I \in R / I$, $a \not\in I$. Рассмотрим идеал $J = I + (a) \normal R$. В силу максимальности $I$, $J = R$. Значит, $1 \in J$, то есть единица представима в виде $1 = x + ar, x \in I, r \in R$. Значит, $(a + I)(r + I) = (1 - x) + I = 1 + I$. Поскольку $1 + I$ "--- это единица в $R / I$, то $a + I$ обратим в $I$ и, в силу произвольности $a$, $R / I$ "--- поле.
	\end{itemize}
\end{proof}

\begin{note}
	Если $p$ "--- простое число, то $p\Z \normal \Z$ "--- максимальный идеал, поэтому $\Z / p\Z \hm= \Z_p$ "--- поле.
\end{note}

\begin{note}
	Некоммутативное кольцо с единицей без нетривиальных идеалов может содержать необратимые элементы, поэтому теорема выше для некоммутативных колец неверна.
\end{note}

\begin{exercise}
	Пусть $F$ "--- поле. Докажите, что в кольце $M_n(F)$ нет нетривиальных идеалов.
\end{exercise}

\begin{solution}
	Пусть $I \normal M_n(F)$ и $I \ne \{0\}$. Все матрицы одного ранга можно привести к одному и тому же упрощенному виду элементарными преобразованиями и $M_n(F)I, IM_n(F) \subset I$. Следовательно, если $I$ содержит одну матрицу ранга $k$, то $I$ содержит все матрицы ранга $k$.
	
	Поскольку $I \ne \{0\}$, то в $I$ есть матрица ненулевого ранга. Значит, $I$ содержит все матрицы этого ранга, и из них легко получить с помощью сложения матрицы произвольного ранга. Таким образом, $I$ содержит все матрицы произвольного ранга, то есть $I = M_n(F)$.
\end{solution}

\begin{definition}
	Кольцо, не имеющее нетривиальных идеалов, называется \textit{простым}.
\end{definition}