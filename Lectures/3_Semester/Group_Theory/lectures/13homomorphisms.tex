\subsection{Гомоморфизмы групп и факторгруппа}

\begin{definition}
	Пусть $G, H$ "--- группы. \textit{Гомоморфизмом групп} $G$ и $H$ называется отображение $\phi : G \hm\to H$ такое, что $\forall g_1, g_2 \hm\in G: \phi(g_1g_2) \hm= \phi(g_1)\phi(g_2)$. Сюръективный гомоморфизм называется \textit{эпиморфизмом}, инъективный гомоморфизм --- \textit{мономорфизмом}.
\end{definition}

\begin{note}
	Изоморфизм групп "--- это гомоморфизм, являющийся биекцией, то есть эпиморфизмом и мономорфизмом одновременно.
\end{note}

\begin{definition}
	Пусть $G, H$ "--- группы, $\phi : G \hm\to H$ "--- гомоморфизм $G$ и $H$. Тогда:
	\begin{itemize}
		\item \textit{Образом} $\phi$ называется $\im\phi := \{\phi(g): g \in G\} = \phi(G)$
		\item \textit{Ядром} $\phi$ называется $\ke\phi := \{g \in G: \phi(g) = e\} = \phi^{-1}(e)$
	\end{itemize}
\end{definition}

\begin{note}
	Далее мы часто будем обозначать $\phi(g)$ как $\overline{g}$.
\end{note}

\begin{example}
	Рассмотрим несколько примеров гомоморфизмов групп:
	\begin{enumerate}
		\item Любой изоморфизм групп, в частности, изоморфизм вида $\phi: G \to G$ --- \textit{автоморфизм} группы $G$
		\item $\phi: G \to H$, $\forall g \in G: \phi(g) = e$, где $G, H$ "--- произвольные группы
		\item Сопряжение при помощи $x \in G$, где $G$ "--- произвольная группа, поскольку $\forall g_1, g_2 \hm\in G: (g_1g_2)^x = g_1^xg_2^x$ (более того, сопряжение "--- это автоморфизм, поскольку существует обратное отображение: $\forall g \in G: \phi^{-1}(g) = g^{x^{-1}}$)
		\item $\det: \GL_n(F) \to F^*$, поскольку $\forall A, B \in \GL_n(F): \det(AB) \hm= \det A \det B$
		\item $\sgn: S_n \to \Q^*$, поскольку $\forall \sigma, \tau \in S_n: \sgn(\sigma\tau) \hm= \sgn\sigma\sgn\tau$
		\item Отображение $\phi: \Z \to \Z_n$ такое, что $\forall a \in \Z: \phi(a) = a + n\Z$, поскольку $\forall a, b \in \Z: \phi(a + b) \hm= (a + b) + n\Z = \phi(a) + \phi(b)$
	\end{enumerate}
\end{example}

\begin{proposition}
	Пусть $G, H$ "--- группы, $\phi: G \to H$ "--- гомоморфизм. Тогда $\phi(e) = e$ и $\forall g \in G: \phi(g^{-1}) = (\phi(g))^{-1}$.
\end{proposition}

\begin{proof}~
	\begin{itemize}
		\item $\overline{e}^2 = \overline{e^2} = \overline{e} \Rightarrow \overline{e} = e$
		\item $\overline{g^{-1}}\overline{g} = \overline{g^{-1}g} = \overline{e} = e \Rightarrow \overline{g}^{-1} = \overline{g^{-1}}$
	\end{itemize}
\end{proof}

\begin{proposition}
	Пусть $G, H$ "--- группы, $\phi: G \to H$ "--- гомоморфизм, $K := \ke\phi$. Тогда $\forall g \in G: \phi^{-1}(\overline{g}) = gK = Kg$.
\end{proposition}

\begin{proof}
	Пусть $a \in G$. Тогда $a \in \phi^{-1}(\overline{g}) \Leftrightarrow \overline{a} = \overline{g} \Leftrightarrow e \hm= \overline{g}^{-1}\overline{a} = \overline{g^{-1}a} \Leftrightarrow g^{-1}a \hm\in \ke\phi = K \Leftrightarrow a \in gK$. Аналогично доказывается, что $a \in \phi^{-1}(\overline{g}) \Leftrightarrow a \in Kg$.
\end{proof}

\begin{corollary}
	$\phi$ "--- мономорфизм $\Leftrightarrow \ke\phi = \{e\}$.
\end{corollary}

\begin{proposition}
	Пусть $G, H$ "--- группы, $\phi: G \to H$ "--- гомоморфизм. Тогда:
	\begin{enumerate}
		\item $\im\phi \le H$
		\item $\ke\phi \normal G$
	\end{enumerate}
\end{proposition}

\begin{proof}~
	\begin{enumerate}
		\item Если $h_1, h_2 \in \im\phi$, то $h_1 = \overline{g_1}$, $h_2 = \overline{g_2}$, откуда $h_1h_2 = \overline{g_1g_2} \hm\in \im\phi$ и $h_1^{-1} = \overline{g^{-1}} \in \im\phi$
		\item Если $g_1, g_2 \in \ke\phi$, то $\overline{g_1} = \overline{g_2} = e$, откуда $\overline{g_1g_2} = e$ и $\overline{g_1^{-1}} = e$, и, более того, $\forall g \in G: gK = Kg = \phi^{-1}(g)$
	\end{enumerate}
\end{proof}

\begin{note}
	Если $G, H$ "--- группы, $\phi: G \to H$ "--- гомоморфизм и $G' \le G$, то $\phi|_{G'}: G' \hm\to H$ "--- тоже гомоморфизм, поэтому $\phi(G') \hm= \im\phi|_{G'} \le H$. С другой стороны, если $H' \le H$, то существует гомоморфизм $\psi = \id|_{H'}: H' \to H$ такой, что $H' = \im\psi$.
\end{note}

\begin{definition}
	Пусть $G$ "--- группа, $K \normal G$. Определим операцию на $G / K$ следующим образом: $\forall g_1, g_2 \in G: g_1K\cdot g_2K \hm= g_1(Kg_2)K = g_1(g_2K)K = g_1g_2K$.
\end{definition}

\begin{note}
	Определение выше корректно, поскольку оно не зависит от выбора представителей классов смежности.
\end{note}

\begin{proposition}
	Пусть $G$ "--- группа, $K \normal G$. Тогда $(G / K, \cdot)$ "--- группа. Более того, отображение $\pi: G \to G / K$, $\forall g \in G: \pi(g) \hm= gK$, является эпиморфизмом.
\end{proposition}

\begin{proof}
	Проверим непосредственно, что множество $G / K$ является группой:
	\begin{itemize}
		\item (Ассоциативность) $\forall g_1K, g_2K, g_3K \in G / K: (g_1Kg_2K)g_3K \hm= (g_1g_2g_3)K \hm= g_1K(g_2Kg_3K)$
		\item (Нейтральный элемент) $\exists K \in G / K: \forall gK \in G / K: (gK)K \hm= K(gK) = gK$
		\item (Обратный элемент) $\forall gK \in G / K: \exists (gK)^{-1} = g^{-1}K \in G / K: (gK)(gK)^{-1} \hm= (gK)^{-1}(gK) = K$
	\end{itemize}
	
	Отображение $\pi$ "--- это гомоморфизм по определению, и его сюръективность очевидна: $\forall gK \in G / K: \exists g \in G: \pi(g) = gK$.
\end{proof}

\begin{definition}
	Пусть $G$ "--- группа, $K \normal G$. Группа $G / K$ называется \textit{факторгруппой} $G$ по $K$.
\end{definition}

\begin{note}
	$g \in \ke\pi \Leftrightarrow \pi(g) = gK = K \Leftrightarrow g \in K$, поэтому $\ke\pi = K$. Значит, любая нормальная подгруппа является ядром некоторого гомоморфизма, точно так же, как любая подгруппа является образом некоторого гомоморфизма.
\end{note}

\begin{theorem}[Основная теорема о гомоморфизме]~
	\begin{enumerate}
		\item Пусть $G$ "--- группа, $K \normal G$. Тогда $\exists \pi: G \to G / K$ "--- эпиморфизм такой, что $\ke\pi \hm= K$.
		\item Пусть $G, H$ "--- группы, $\phi: G \to H$ "--- гомоморфизм. Тогда $K \hm{:=} \ke\phi \normal G$, и, более того, $\im\phi \cong G / K$.
	\end{enumerate}
\end{theorem}

\begin{proof}
	Большая часть теоремы уже была доказана выше, остается доказать лишь последнее утверждение. Изобразим его на коммутативной диаграмме:
	\[
	\begin{tikzcd}[row sep = huge]
		G \arrow{rr}{\phi} \arrow[swap]{dr}{\pi} && \im\phi \le H \arrow[dashrightarrow, swap, xshift = -4pt]{dl}{\psi}\\
		& G / K \arrow[dashrightarrow, swap, xshift = 4pt]{ur}{\Theta} &
	\end{tikzcd}
	\]
	
	Нам требуется предъявить такой изоморфизм $\Theta: G / K \to \im\phi$, что $\Theta \circ \pi = \phi$. Построим $\psi := \Theta^{-1}$: $\forall \overline{g} \in \im\phi: \psi(\overline{g}) = \phi^{-1}(\overline{g}) \hm= gK \in G / K$. Проверим, что это изоморфизм:
	\begin{itemize}
		\item (Гомоморфизм) $\forall \overline{g_1}, \overline{g_2} \in \im\phi: \psi(\overline{g_1}\cdot\overline{g_2}) = \psi(\overline{g_1g_2}) = g_1g_2K \hm= (g_1K)(g_2K) = \psi(\overline{g_1})\psi(\overline{g_2})$
		\item (Сюръективность) $\forall gK \in G / K: gK = \psi(\overline{g})$
		\item (Инъективность) Достаточно показать, что $\ke\psi = \{e\}$: $\forall \overline{g} \hm\in \im\phi: \psi(\overline{g}) = K \hm\Rightarrow gK = K \Rightarrow g \in K \Rightarrow \overline{g} = \overline{e}$
	\end{itemize}
	
	Теорема доказана, но убедимся непосредственно в том, что диаграмма коммутативна, то есть $\Theta \circ \pi = \phi$, или $\Theta^{-1} \circ \phi = \psi \circ \phi = \pi$: $\forall g \hm\in G: \psi(\phi(g)) = \psi(\overline{g}) = gK = \pi(g)$.
\end{proof}

\begin{note}
	Гомоморфный образ группы, будь во имя коммунизма изоморфен факторгруппе по ядру гомоморфизма!
\end{note}

\begin{theorem}[Первая теорема об изоморфизме]
	Пусть $G$ "--- группа, $K \normal G$, $H \le G$. Тогда $HK = KH \le G$, $K \cap H \normal H$ и $HK / K \cong H / (K \cap H)$.
\end{theorem}

\begin{proof}
	Первое утверждение теоремы уже было доказано, поэтому докажем оставшиеся два. Для этого рассмотрим канонический эпиморфизм $\pi: G \hm\to G / K$ и $\forall g \in G$ обозначим $\overline{g} := \pi(g)$.
	
	Пусть $\phi := \pi|_H : H \to G/ K$. Тогда $\ke\phi = \ke\pi \cap H = K \cap H$, откуда $K \cap H \normal H$. $\im\phi = \{\overline{h}: h \in H\} = \{hK: h \hm\in H\} = HK / K$, поскольку $HK / K \hm= \{hkK: h \in H, k \hm\in K\} = \{hK: h \in H\}$. По основной теореме о гомоморфизме, $HK / K \cong H / (K \cap H)$.
\end{proof}

\begin{theorem}[Вторая теорема об изоморфизме, или теорема о соответствии]
	Пусть $G$ "--- группа, $K \normal G$. Тогда каждой подгруппе $H$ такой, что $K \le H \le G$, соответствует подгруппа $\overline{H} = H / K \hm\le G / K = \overline{G}$, причем соответствие $H \mapsto \overline{H}$ "--- биекция между подгруппами вида $K \le H \le G$ и подгруппами $\overline{H} \le \overline{G}$. Более того, если $K \le H \le G$, то $H \normal G \Leftrightarrow \overline{H} \normal \overline{G}$, и в этом случае $G / H \cong \overline{G} / \overline{H}$.
\end{theorem}

\begin{proof}~
	\begin{enumerate}
		\item Рассмотрим канонический эпиморфизм $\pi: G \hm\to G / K = \overline{G}$. Тогда $\forall K \le H \le G: \pi(H) =\overline{H} = H / K \le G / K$. С другой стороны, $\forall L \le \overline{G}: \pi^{-1}(L) = \bigcup_{gK \in L}gK \hm\le G$. Проверим, что $\pi$ осуществляет требуемую биекцию. Действительно, $\pi^{-1}\circ\pi \hm= \id$, поскольку $\forall K \le H \le G: \pi^{-1}(\pi(H)) = H$ ($H$ "--- объединение нескольких левых смежных классов по $K$), и $\pi\circ\pi^{-1} = \id$, поскольку $\forall L \le \overline{G}: \pi(\pi^{-1}(L)) = L$.
		
		\item Если $H \normal G$, то $\forall g \in G: gH = Hg$, поэтому, применяя эпиморфизм $\pi$, получаем, что $\forall \overline{g} \in \overline{G}: \overline{g}\overline{H} = \overline{H}\overline{g}$, то есть $\overline{H} \normal \overline{G}$. Пусть теперь, наоборот, $\overline{H} \normal \overline{G}$. Рассмотрим канонический эпиморфизм $\pi': \overline{G} \to \overline{G} / \overline{H}$. Тогда $\phi := \pi'\circ\pi: G \hm\to \overline{G} \hm\to \overline{G} / \overline{H}$ "--- тоже эпиморфизм, причем $\ke\phi = \pi^{-1}(\pi'^{-1}(\overline{H})) \hm= \pi^{-1}(\overline{H}) = H$. Значит, $H \normal G$, и, по основной теореме о гомоморфизме, $\overline{G} / \overline{H} \cong G / H$.
	\end{enumerate}
\end{proof}

\begin{exercise}
	Укажите в явном виде изоморфизм $G / H \hm\to \overline{G} / \overline{H}$ из предыдущей теоремы.
\end{exercise}

\begin{solution}
	В общем случае, при гомоморфизме групп $\phi: G_1 \to G_2$ изоморфизм из основной теоремы имеет вид $g_1\ke\phi \mapsto \phi(g_1)$. В теореме выше это отображение принимает следующий вид: $gH \hm\mapsto \phi(g) = \pi'(\pi(g)) = \overline{g}\overline{H}$.
\end{solution}

\begin{proposition}
	Пусть $G$ "--- конечная группа, $H, K \le G$. Тогда $|HK| = \frac{|H||K|}{|H\cap K|}$.
\end{proposition}

\begin{proof}
	Рассмотрим отображение $\delta: H\times K \to G$ такое, что $\forall h \hm\in H: \forall k \in K: \delta(h, k) = hk$. Тогда $\delta(h_1, k_1) = \delta(h_2, k_2) \hm\Leftrightarrow h_1k_1 = h_2k_2 \hm\Leftrightarrow h_2^{-1}h_1 = k_2k_1^{-1} = x \in H \cap K$, то есть $h_1 = h_2x$, $k_1 = x^{-1}k_2$. Значит, $|HK||H \cap K| = |H \times K| = |H||K|$.
\end{proof}

\begin{example} Рассмотрим несколько примеров применения теорем об изоморфизме:
	\begin{enumerate}
		\item Множество $V_4 = \{e, (12)(34), (13)(24), (14)(23)\} \subset S_4$ "--- это подгруппа в $S_4$ (\textit{четверная группа Клейна}), причем $V_4 \normal S_4$. Рассмотрим $S_3 \le S_4$, $S_3 \cap V_4 = \{e\}$. По первой теореме об изоморфизме, $S_3V_4 / V_4 \cong S_3 / \{S_3 \cap V_4\} = S_3$. Поскольку $|S_3V_4| = 24$, то $S_3V_4 = S_4$ и $S_4 / V_4 \cong S_3$.
		\item Рассмотрим $\Z_n = \Z / n\Z$. По второй теореме об изоморфизме, $\forall \overline{H} \le Z_n: \overline{H} \mapsto H$, $n\Z \le H \le \Z$. Поскольку $\forall H \le \Z: H = k\Z$, это возможно только тогда, когда $k\mid n$. Значит, все подгруппы в $\Z_n$ "--- это $\overline{k\Z} \hm= k\Z_n$, $k\mid n$. Более того, $\Z_n / k\Z_n \cong \Z / k\Z = \Z_k$.
	\end{enumerate}
\end{example}