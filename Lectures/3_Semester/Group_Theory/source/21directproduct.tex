\section{Теоретико-групповые конструкции}

\subsection{Прямое произведение групп}

\begin{definition}
	Пусть $A, B$ "--- группы. Тогда их \textit{(внешним) прямым произведением} называется группа $G = A \times B$ со следующей операцией: $\forall (a_1, b_1), (a_2, b_2) \in G: (a_1, b_1)(a_2, b_2) \hm= (a_1a_2, b_1b_2)$.
\end{definition}

\begin{note}
	Конечно, $G$ действительно является группой: ассоциативность очевидна, нейтральный элемент в $G$ "--- это $e = (e_A, e_B)$, и $\forall (a, b) \in G: (a, b)^{-1} = (a^{-1}, b^{-1})$.
\end{note}

\begin{proposition}
	Пусть $A, B$ "--- группы, $G = A\times B$. Тогда:
	\begin{enumerate}
		\item $A \times \{e_B\} \normal G$, $\{e_A\} \times B \normal G$
		\item $A \times B \cong B \times A$
		\item Если $C$ "--- группа, то $(A \times B) \times C \cong A \times (B \times C)$
	\end{enumerate}
\end{proposition}

\begin{proof}~
	\begin{enumerate}
		\item Очевидно, $A \times \{e_B\} \le G$, и, более того, $\forall (a, e_B) \in A \times \{e_B\}: \forall (a', b') \in G: (a', b')(a, e_B)(a'^{-1}, b'^{-1}) = (a^{a'^{-1}}, e_B) \in A \times \{e_B\}$, поэтому $A \times \{e_B\} \normal G$ (вторая часть утверждения доказывается аналогично)
		\item Предъявим изоморфизм: $(a, b) \mapsto (b, a)$
		\item Предъявим изоморфизм: $((a, b), c) \mapsto (a, (b, c))$
	\end{enumerate}
\end{proof}

\begin{note}
	В силу <<ассоциативности>> прямого произведения, мы будем опускать скобки в прямых произведениях трех и более групп, записывая их в следующем виде:
	\[A_1 \times \dotsb \times A_k = \prod_{i = 1}^kA_i = \{(a_1, \dotsc, a_k): a_1 \in A_1, \dotsc, a_k \in A_k\}\]
\end{note}

\begin{proposition}
	Пусть $A, B$ "--- группы, $A_1 \normal A, B_1 \normal B$. Тогда $A_1 \times B_1 \normal A \times B$, при этом $(A \times B) / (A_1 \times B_1) \cong (A / A_1) \times (B / B_1)$.
\end{proposition}

\begin{proof}
	Пусть $\pi_A: A \to A/A_1$, $\pi_B: B \to B/B_1$ "--- канонические эпиморфизмы. Рассмотрим $\pi := \pi_A \times \pi_B: (A \times B) \hm\to (A / A_1) \times (B / B_1)$ "--- такое отображение, что $\forall (a, b) \hm\in A \times B: \pi((a, b)) = (\pi_A(a), \pi_B(b))$. Тогда $\pi$ "--- это гомоморфизм, причем сюръективный в силу сюръективности $\pi_A, \pi_B$. $\im\pi = (A / A_1) \times (B / B_1)$, $\ke\pi \hm= \ke\pi_A \times \ke\pi_B = A_1 \times B_1$, и, следовательно, $A_1 \times B_1 \hm\normal A \times B$. Наконец, по основной теореме о гомоморфизме, $(A \hm\times B) / (A_1 \hm\times B_1) \hm\cong (A / A_1) \times (B / B_1)$, причем изоморфизм имеет следующий вид: $(a, b)(A_1 \hm\times B_1) \hm\mapsto (aA_1, bB_1)$.
\end{proof}

\begin{note}
	Легко видеть, что $A \times \{e_B\} \cong A$, $\{e_A\} \times B \cong B$, поэтому далее мы будем отождествлять эти подгруппы с $A$ и $B$. Тогда, по утверждению выше:
	\[(A \times B)/A \cong (A / A) \times (B / \{e_B\}) \cong \{A\} \times B \cong B\]
\end{note}

\begin{note}
	Если группы $A$ и $B$ "--- абелевы, их прямое произведение часто обозначается как $A \oplus B$.
\end{note}

\begin{theorem}
	Пусть $G$ "--- группа, $A, B \normal G$, причем $A \cap B = \{e\}$ и $AB = G$. Тогда $G \cong A \times B$.
\end{theorem}

\begin{proof}
	Заметим сначала, что $\forall a \in A: \forall b \in B: ab = ba$, поскольку $aba^{-1}b^{-1} \hm= (aba^{-1})b^{-1} = a(ba^{-1}b^{-1}) \in A \cap B \ra ab(ba)^{-1} \hm= e \ra ab = ba$. Построим изоморфизм $\phi: A \times B \to G$ следующим образом: $\phi((a, b)) = ab$. Проверим свойства изоморфизма:
	\begin{itemize}
		\item (Гомоморфизм) $\forall (a_1, b_1), (a_2, b_2) \in A \times B: \phi((a_1, b_1)(a_2, b_2)) \hm= (a_1a_2)(b_1b_2) \hm= (a_1b_1)(a_2b_2) = \phi((a_1, b_1))\phi((a_2, b_2))$
		\item (Инъективность) $\ke\phi = \{(a, b) \in A \times B: ab = e\} = \{(a, b) \hm\in A \times B: a = b^{-1}\} \hm= \{(e, e)\}$
		\item (Сюръективность) $\im\phi = AB = G$
	\end{itemize}
	
	Значит, $\phi$ действительно является изоморфизмом $G$ и $A \times B$.
\end{proof}

\begin{note}
	В ситуации выше группа $G$ называется \textit{внутренним прямым произведением} своих подгрупп $A$ и $B$.
\end{note}

\begin{definition}
	Пусть $G$ "--- группа, $A \normal G$, $B \le G$. Если $A \cap B \hm= \{e\}$ и $AB = G$, то $G$ называется \textit{полупрямым произведением} $A$ и $B$. Обозначение "--- $G = A \sd B$.
\end{definition}

\begin{note}
	Рассмотрим канонический эпиморфизм $\pi: G \to G/A$ (в обозначениях определения выше). Тогда, по первой теореме об изоморфизме, $G / A = AB/A \cong B / (A \cap B) \cong B$, как и в случае внутреннего прямого произведения.
\end{note}

\begin{example} Рассмотрим несколько полупрямых произведений:
	\begin{enumerate}
		\item $S_n = A_n \sd \gl(12)\gr$ (если $n \ge 2$)
		\item $S_4 = V_4 \sd S_3$
	\end{enumerate}
\end{example}

\begin{note}
	Пусть $G$ "--- группа, $A \normal G$, $B \le G$, и $G = A \sd B$. Рассмотрим действие $B$ на $A$ сопряжениями: $\forall b \in B: \forall a \in A: b(a) \hm= bab^{-1} \in A$, поскольку $A \normal G$. Данному действию соответствует гомоморфизм $\phi: B \to S(A)$. Более того, поскольку $\forall b \in B: bAb^{-1} = A$ и сопряжение с помощью $b^{-1}$ "--- это автоморфизм $G$, то это также автоморфизм $A$. Значит, на самом деле $\phi: B \to \Aut{A} \le S(A)$. Структура полупрямого произведения однозначно (точнее, не более чем однозначно) задается группами $A, B$ и гомоморфизмом $\phi$: $(a_1b_1)(a_2b_2) \hm= a_1b_1a_2b_2 = a_1b_1a_2b_1^{-1}b_1b_2 = a_1\phi_{b_1}(a_2)b_1b_2$.
\end{note}

\begin{definition}
	Пусть $A, B$ "--- группы, $\phi: B \to \Aut{A}$. Определим на $G = A \times B$ операцию следующим образом: $\forall (a_1, b_1), (a_2, b_2) \hm\in G: (a_1, b_1)(a_2, b_2) = (a_1\phi_{b_1}(a_2), b_1b_2)$. Полученная конструкция называется \textit{полупрямым произведением $A$ и $B$, заданным гомоморфизмом $\phi$}. Обозначение "--- $G = A \sd_\phi B$.
\end{definition}

\begin{exercise}
	Докажите, что $G = A \sd_\phi B$ является группой, причем $A \sd_\phi \{e_B\} \hm\cong A$, $\{e_A\} \sd_\phi B \cong B$ и $G = A \sd B$ (в смысле первого определения).
\end{exercise}

\begin{proof}[Решение]
	Проверим непосредственно, что $G$ "--- группа:
	\begin{itemize}
		\item (Ассоциативность) Достаточно убедиться в ассоциативности операции в первой координате: $\forall a_1, a_2, a_3 \in A: \forall b_1, b_2, b_3 \in B: a_1\phi_{b_1}(a_2\phi_{b_2}(a_3)) = a_1\phi_{b_1}(a_2)\phi_{b_1}(\phi_{b_2}(a_3)) \hm= a_1\phi_{b_1}(a_2)\phi_{b_1b_2}(a_3)$
		\item (Нейтральный элемент) $\exists (e, e) \in G: \forall (a, b) \in G: (e, e)(a, b) \hm= (e\phi_e(a), eb) \hm= (a, b), (a, b)(e, e) = (a\phi_b(e), be) = (a, b)$
		\item (Обратный элемент) $\forall (a, b) \in G: \exists (a, b)^{-1} = (\phi_{b^{-1}}(a^{-1}), b^{-1}) \hm\in G: (a, b)(a, b)^{-1} \hm= (a\phi_b(\phi_{b^{-1}}(a^{-1})), e) = (e, e), (a, b)^{-1}(a, b) \hm= (\phi_{b^{-1}}(a^{-1})\phi_{b^{-1}}(a), e) = (e, e)$
	\end{itemize}
	
	Легко видеть, что $A' = A \sd_\phi \{e\} = \{(a, e): a \in A\} \cong A$ и $B' \hm= \{e\} \sd_\phi B \hm = \{(e, b): b \hm\in B\} \cong B$. Очевидно, $A' \cap B' = \{(e, e)\}$ и $A'B' = G$, значит, остается проверить, что $A' \normal G$. Действительно, если $(a', e) \in A'$ и $(a, b) \in G$, то вторая координата $(a, b)^{-1}(a', e)(a, b)$ равна $e$, то есть $(a, b)^{-1}(a', e)(a, b) \in A'$.
\end{proof}

\begin{note}
	Если группа $G$ является полупрямым произведением $A \normal G$ и $B \le G$, и $\phi: B \to \Aut{A}$ "--- соответствующий действию $B$ на $A$ сопряжениями гомоморфизм, то, как и в случае прямого произведения, $A \sd_\phi B \cong G$, и изоморфизм имеет вид $(a, b) \mapsto ab$.
	
	Отметим также, что прямое произведение является частным случаем полупрямого: гомоморфизм, описанный выше, каждому элементу $b \in B$ сопоставляет $\phi_b = \id$.
\end{note}