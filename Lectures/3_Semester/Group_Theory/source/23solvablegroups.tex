\subsection{Разрешимые группы}

\begin{definition}
	Группа $G$ называется \textit{разрешимой}, если $\exists n \hm\in \N: G^{(n)} = \{e\}$. Наименьшее $n \in \N$, для которого это выполнено, называется \textit{степенью разрешимости} $G$.
\end{definition}

\begin{note}
	Разрешимые группы степени 1 "--- это абелевы группы. Разрешимые группы степени 2 часто называют \textit{метаабелевыми}.
\end{note}

\begin{note}
	Если $G$ "--- конечная группа, то последовательность вида $G \ge G' \ge G'' \ge \dotsc$ обязательно стабилизируется, но необязательно на $\{e\}$.
\end{note}

\begin{proposition}
	Пусть $G$ "--- разрешимая группа, $H \le G$. Тогда $H$ тоже разрешима.
\end{proposition}

\begin{proof}
	Достаточно заметить, что $H \le G \ra H' \le G' \hm\ra \dotsb \ra H^{(n)} \le G^{(n)} \hm= \{e\}$.
\end{proof}

\begin{theorem}
	Пусть $G$ "--- группа, $K \normal G$. Тогда $G$ разрешима $\lra$ $K$ и $G / K$ разрешимы.
\end{theorem}

\begin{proof}~
	\begin{itemize}
		\item[$\ra$] Если $G$ разрешима, то $K \le G$ разрешима, и, поскольку при каноническом эпиморфизме $\pi: G \to G/K$ выполнено равенство $\pi(G') = (G / K)'$, то, по индукции, $(G / K)^{(n)} = \pi(G^{(n)}) = \{K\}$
		\item[$\la$] Если $K^{(m)} = \{e\}$ и $(G / K)^{(m)}$, то, снова рассматриавя канонический эпиморфизм, получаем, что $\pi(G^{(n)}) = (G / K)^{(n)} \hm= \{K\} \ra G^{(n)} \le \ke\pi = K \ra G^{(n + m)} \le K^{(m)} \hm= \{e\}$
	\end{itemize}
\end{proof}

\begin{corollary}
	Пусть $G$ "--- группа, $K_1, K_2 \normal G$ разрешимы. Тогда группа $K_1K_2$ разрешима.
\end{corollary}

\begin{proof}
	Заметим, что $K_1 \normal K_1K_2$ и, по первой теореме об изоморфизме, $K_1K_2/K_1 \cong K_2/(K_1 \cap K_2)$. Группы $K_1$ и $K_1K_2 / K_1$ разрешимы, поэтому $K_1K_2$ разрешима.
\end{proof}

\begin{corollary}
	Пусть $G$ "--- конечная группа. Тогда в $G$ существует наибольшая по включению разрешимая нормальная подгруппа $K$.
\end{corollary}

\begin{proof}
	Пусть $K_1, \dotsc, K_m \normal G$ "--- это все разрешимые нормальные подгруппы в $G$. Тогда, обобщая предыдущее следствие на случай $m$ нормальных подгрупп в $G$ по индукции, получаем, что $K = K_1\dotsm K_m \normal G$ "--- нормальная разрешимая подгруппа, причем $K_1, \dotsc, K_m \le K$.
\end{proof}

\begin{theorem}
	Пусть $G$ "--- группа. Тогда следующие утверждения эквивалентны:
	\begin{enumerate}
		\item $G$ разрешима
		\item В $G$ существует ряд $G = G_0 \ge G_1 \ge \dotsc \ge G_n = \{e\}$ такой, что $\forall i \in \{1, \dotsc, n\}: G_i \normal G$ и $G_{i-1}/G_i$ "--- абелева (нормальный ряд с абелевыми факторами)
		\item В $G$ существует ряд $G = G_0 \ge G_1 \ge \dotsc \ge G_n = \{e\}$ такой, что $\forall i \in \{1, \dotsc, n\}: G_i \normal G_{i-1}$ и $G_{i-1}/G_i$ "--- абелева (субнормальный ряд с абелевыми факторами)
	\end{enumerate}
\end{theorem}

\begin{proof}~
	\begin{itemize}
		\item ($1 \ra 2$) Достаточно расмотреть ряд $G \ge G' \ge \dotsc \ge G^{(n)} = \{e\}$, и по свойствам коммутантов все необходимые свойства будут выполнены.
		\item ($2 \ra 3$) Заметим, что нормальный ряд также является и субнормальным: если $\forall i \hm\in \{1, \dotsc, n\}: G_i \normal G$, то $\forall i \in \{1, \dotsc, n\}: G_i \normal G_{i-1}$.
		\item ($3 \ra 1$) Докажем, что $\forall i \in \{0, \dotsc, n\}: G^{(i)} \le G_i$, по индукции. База, $i = 0$, тривиальна. Пусть теперь $H := G^{(i)} \le G_i$. Поскольку $H / G_{i+1}$ "--- абелева, то, первой теореме об изоморфизме, $H/(H \cap G_{i+1}) \cong HG_{i+1}/G_{i+1} \le G_i/G_{i+1}$ тоже является абелевой. Тогда $G^{(i+1)} = H' \le H \cap G_{i+1} \le G_{i+1}$.
	\end{itemize}
\end{proof}

\begin{proposition}
	Если $G$ "--- $p$-группа, то $G$ разрешима.
\end{proposition}

\begin{proof}
	Пусть $|G| = p^n$. Будем вести индукцию по $n$. Если $n = 1$, то $G$ "--- циклическая, и, в частности, $G$ "--- абелева и потому разрешима. Пусть теперь $n > 1$. Рассмотрим $Z(G) \ne \{e\}$. Если $Z(G) = G$, то $G$ "--- абелева и потому разрешима. В противном случае $|Z(G)|, |G/Z(G)| < p^n$, и это $p$-группы меньшего порядка. Значит, по предположению индукции, $Z(G)$ и $G/Z(G)$ разрешимы, но тогда и $G$ разрешима.
\end{proof}

\begin{note}
	В доказательстве утверждения выше мы построили нормальный ряд с абелевыми факторами, но снизу вверх: $H_0 = \{e\}$, $H_1 = Z(G)$, далее, по теореме о соответствии, $Z(G/H_1) = H_2/H_1$ для некоторой $H_1 \normal H_2 \le G$, $Z(G/H_2) = H_3/H_2$ для некоторой $H_2 \normal H_3 \le G$, и так далее.
\end{note}

\begin{exercise}
	Докажите, что группа $S_4$ разрешима.
\end{exercise}

\begin{proof}[Решение]
	Покажем, что ряд $S_4 \ge A_4 \ge V_4 \ge \{e\}$ "--- это субнормальный ряд с абелевыми факторами:
	\begin{itemize}
		\item $|S_4 : A_4| = \frac{24}{12} = 2$, поэтому $A_4 \normal S_4$ и $S_4/A_4 \cong \Z_2$ "--- абелева
		\item $V_4 \normal S_4$, поэтому $V_4 \normal A_4$, и, поскольку $|A_4 : V_4| = \frac{12}{4} = 3$, $A_4/V_4 \cong \Z_3$ "--- абелева
		\item $\{e\} \normal V_4$ и $V_4/\{e\} \cong V_4$ "--- абелева
	\end{itemize}
	
	По теореме выше, из существования субнормального ряда с абелевыми факторами следует разрешимость $S_4$.
\end{proof}

\begin{theorem}
	Пусть $G$ "--- $p$-группа, $|G| = p^n$. Тогда $\forall k \in \N, k \le n: \exists H \le G: |H| = p^k$.
\end{theorem}

\begin{proof}
	Будем вести индукцию по $k$. Рассмотрим $Z(G)$, $|Z(G)| = p^l$, $l > 1$. Порядок всех элементов $Z(G)$ кроме нейтрального делится на $p$, поэтому $\exists g \in Z(G): \ord{g} = p$. Положим $L := \gl g\gr$, $|L| = p$, и, поскольку $L \le Z(G)$, $L \normal G$. Если $k = 1$, мы уже добились требуемого. Если $k > 1$, то, по предположению индукции (и теореме о соответствии), $\exists H / L \le G / L: |H / L| = p^{k-1}$ для некоторой $H \le G$ такой, что $|H| = p^k$.
\end{proof}

\begin{note}
	На самом деле, в силу все той же теоремы о соответствии, доказательство теоремы гарантирует существование нормальной подгруппы заданного порядка в $G$.
\end{note}

\begin{exercise}
	Докажите, что в группе $A_4$ нет подгруппы, порядок которой равен 6.
\end{exercise}

\begin{proof}[Решение]
	Предположим, что это не так и $\exists H \le A_4: |H| \hm= 6$. Тогда $|A_4 : H| = 2$, поэтому $H \normal A_4$, то есть $H$ состоит из нескольких классов сопряженности. Но классы сопряженности в $A_4$ имеют размер 1, 3, 4 и 4 --- противоречие.
\end{proof}