\documentclass[a4paper, 12pt]{article}

\usepackage{cmap}					
\usepackage{mathtext} 		% русские буквы в формулах		
\usepackage[T2A]{fontenc}			
\usepackage[utf8]{inputenc}			
\usepackage[english,russian]{babel}

\usepackage{amsmath,amsfonts,amssymb,amsthm,mathtools}
\usepackage{icomma} % 'умная' запятая

\mathtoolsset{showonlyrefs=true} % Показывать номера только у тех формул, на которые есть \eqref{} в тексте.

\usepackage{euscript}
\usepackage{mathrsfs}
\usepackage{etoolbox}


\usepackage{hyperref}
\usepackage[usenames,dvipsnames,svgnames,table,rgb]{xcolor}
\hypersetup{				% Гиперссылки
	unicode=true,           % русские буквы в раздела PDF\\
	pdfstartview=FitH,
	pdftitle={Заголовок},   % Заголовок
	pdfauthor={Автор},      % Автор
	pdfsubject={Тема},      % Тема
	pdfcreator={Создатель}, % Создатель
	pdfproducer={Производитель}, % Производитель
	pdfkeywords={keyword1} {key2} {key3}, % Ключевые слова
	colorlinks=true,       	% false: ссылки в рамках; true: цветные ссылки
	linkcolor=blue,          % внутренние ссылки
	citecolor=green,        % на библиографию
	filecolor=magenta,      % на файлы
	urlcolor=NavyBlue,           % на URL
}

%%% Страница 
\usepackage{extsizes} % Возможность сделать 14-й шрифт
\usepackage{geometry}  
\geometry{left=20mm,right=20mm,top=25mm,bottom=30mm} % задание полей текста

\usepackage{titleps}      % колонтитулы
\newpagestyle{main}{
	\setheadrule{.4pt}                      
	\sethead{теория формальных систем и алгоритмов}{}{\hyperlink{intro}{\;Назад к содержанию}}
	\setfootrule{.4pt}                       
	\setfoot{2020 г. осень \; ФПМИ МФТИ}{}{\thepage} 
}      
\pagestyle{main}  

%%%  Текст
\setlength\parindent{0pt}         % Устанавливает длину красной строки 0pt
\sloppy                                        % строго соблюдать границы текста
\linespread{1.3}                           % коэффициент межстрочного интервала
\setlength{\parskip}{0.5em}                % вертик. интервал между абзацами
%\setcounter{secnumdepth}{0}                % отключение нумерации разделов
%\setcounter{section}{-1}         % Чтобы сделать нумерацию лекций с нуля
\usepackage{multicol}				          % Для текста в нескольких колонках
%\usepackage{soul}
\usepackage{soulutf8} % Модификаторы начертания

%% Перенос знаков в формулах 
\newcommand*{\hm}[1]{#1\nobreak\discretionary{}
{\hbox{$\mathsurround=0pt #1$}}{}}

\usepackage{graphicx}  
\graphicspath{{images/}{images2/}}  % папки с файлами
\setlength\fboxsep{3pt} % Отступ рамки \fbox{} от рисунка
\setlength\fboxrule{1pt} % Толщина линий рамки \fbox{}
\usepackage{wrapfig} 
\usepackage{caption} % Создание пустых заголовков

%% Работа с таблицами
\usepackage{array,tabularx,tabulary,booktabs} 
\usepackage{longtable}
\usepackage{multirow}

%% Новые функции
\DeclareMathOperator{\sgn}{\mathop{sgn}}
\DeclareMathOperator{\M}{\mathop{mod}}
\DeclareMathOperator{\cont}{\mathop{cont}}

\newcommand{\tit}[1]{%
\textit{#1}%
}

\newcommand{\tbf}[1]{%
\textbf{#1}%
}

\newcommand{\wt}[1]{%
\widetilde{ #1 }%
}

\newcommand{\pair}[2]{%
(#1, #2)%
}

\newcommand{\abs}[1]{%
\left| #1 \right|%
}

\newcommand{\av}[1]{%
\left \langle #1 \right \rangle%
}

\newcommand{\wholepart}[1]{%
\left \lfloor #1 \right \rfloor%
}

\newcommand{\br}[1]{%
\left( #1 \right)%
}

\newcommand{\sqrbr}[1]{%
\left[ #1 \right]%
}

\newcommand{\figbr}[1]{%
\left\{ #1 \right\}%
}

\newcommand{\eqmod}[3]{%
#1 \equiv #2 ~ \M ~ #3%
}



%% Сокращения команд
\def\RR{\mathbb{R}}
\def\QQ{\mathbb{Q}}
\def\ZZ{\mathbb{Z}}
\def\NN{\mathbb{N}}
\def\FF{\mathbb{F}}
\def\CC{\mathbb{C}}
\def\Zp{\mathbb{N}_0}


\let\Rar\Rightarrow
\let\rar\rightarrow
\let\LRar\Leftrightarrow
\let\hkar\hookrightarrow
\let\drar\dashrightarrow

\let\INF\infty

\let\degree\circ

\let\lor\vee
\let\land\wedge

\let\leq\leqslant  
\let\geq\geqslant

\let\eps\varepsilon
\let\al\alpha
\let\phi\varphi
\let\Om\Omega
\let\om\omega
\let\lam\lambda
\let\Lam\Lambda
\let\Th\Theta
\let\t\theta
\let\De\Delta
\let\ga\gamma
\let\Ga\Gamma  
\let\XOR\oplus

% Сокращенния в тексте
\newcommand{\Dekstra}{\ensuremath{алгоритм~Дейкстры~}}
\newcommand{\BFS}{\ensuremath{BFS~}}
\newcommand{\DFS}{\ensuremath{DFS~}}
\newcommand{\Ford}{\ensuremath{алгоритм~Форда-Беллмана~}}
\newcommand{\Floid}{\ensuremath{алгоритм~Флойда-Уоршелла~}}
\newcommand{\Krascal}{\ensuremath{алгоритм~Краскала~}}
\newcommand{\MST}{\ensuremath{миностов~}}
\newcommand{\beginproof}{\noindent \tbf{Доказательство:} }
\newcommand{\beginprooff}{\noindent \tbf{Доказательство л 1.1:} }
\newcommand{\conclude}{\noindent \tbf{Вывод:} }

%% быстрое оформление
\theoremstyle{plain} % стиль по умолчанию
\newtheorem{theorem}{Теорема}[section]
\newtheorem{proposition}[theorem]{Утверждение}
\newtheorem{lemma}{Лемма}[section]

 
\theoremstyle{definition} % "Определение"
\newtheorem{corollary}{Следствие}[theorem]
\newtheorem{problem}{Задача}[section]
\newtheorem{definition}{Определение}[section]
 
\theoremstyle{remark} % "Примечание"
\newtheorem*{remark}{Замечание}
\newtheorem*{example}{Пример}
