\section{\Large{Лекция 4}}
\subsection{Уравнение в симметричной форме}

\begin{Def} \fcolorbox{g}{g}{ДУ в симметричной форме}
\begin{equation}
    P(x , y) dx + Q(x, y) dy = 0
\end{equation}
\end{Def}

\subsubsection{ДУ в симметричной форме с разделяющимеся переменными}

Если возможно представление $P(x, y) = M_1(x) N_1(y), Q(x, y) = M_2(x) N_2(y) $, то переменные легко разделяются.

\begin{equation}
    \frac{M_1(x)}{M_2(x)} dx =  - \frac{N_2(y)}{N_1} dy
\end{equation}

\begin{example}
\begin{equation}
    (x^2 - 1)y dx  - x(x^2+1)dy = 0
\end{equation}
Разделим переменные:
\begin{equation}
    \frac{x^2 - 1}{x(x^2+1)} dx = \frac{dy}{y}
\end{equation}
Проинтегрируем:
\begin{equation}
    y = \frac{C(x^2+1)}{x}
\end{equation}
Так же учтем частные решения: $y = 0$ , $x = 0$ (последнее не выражается через предыдущую запись)
\end{example}

\subsubsection{Однородное ДУ в симметрической формой}
\begin{Def}
Уравнение
\begin{equation}
     P(x , y) dx + Q(x, y) dy = 0
\end{equation}
Однородное, если $P, Q$ -- однородные функции относительно $x, y$, более того, показатель однородности должен совпадать: $P(\al x, \al y) = \al^n P(x, y ), ~ Q(\al x, \al y) = \al^n Q(x, y )$
\end{Def}
Перепишем уравнение в виде:
\begin{equation}
    \frac{dy}{dx} = - \frac{P(x, y)}{Q (x, y)} = - \frac{P( \al x, \al y)}{Q (\al x, \al y)}
\end{equation}
Отношение двух однородных функций с одинаковым показателем однородности, есть однородная функция с показателем однородности = 0 
\begin{note}
Далее лектор не стал показывать как решать такие уравнения, однако техающий считает важным привести полный метод решения. \\
Обознаим  как $G = - P (x, y) / Q(x, y)$, сделаем заммену $y(x) = t(x) x, ~ y' = t'x + t$, тогда получаем уравнение 
\begin{equation}
    t'x + t = G(x, xt)
\end{equation}
т.к. $G$ однородна (причем с нулевым показателем) $G(x, tx) = G(1, t) = \widetilde{G(t)}$:
\begin{equation}
\frac{x dt}{dx} + t = \widetilde{G(t)}
\end{equation}
Разделим переменные:
\begin{equation}
    \frac{dt}{\widetilde{G(t)} - t} = \frac{dx}{x}
\end{equation}
интегрируя и потенциируя: $\widetilde{G(t)} - t = Cx , C =const$ Теперь достаточно сделать обратную замену
\end{note}
\subsection{ДУ в полных дифференциалах}

\begin{Def} \fcolorbox{g}{g}{ДУ в полных дифференциалах}
\begin{equation}
    P(x , y) dx + Q(x, y) dy = 0
\end{equation}
В полных дифференциал, если его левая часть -- полный дифференциал некоторой функции $U(x, y)$ на рассматриваемой области: $U_x ' = P, U_y ' = Q$
\end{Def}

т.к. правая часть 0, получаем, что $U = C$, значит решение легко найти

\begin{theorem} \fcolorbox{o}{o}{Критерий ур. в полных дифференциалах} \\
пусть $G = \figbr{ a < x < b, c < y < d}$ тогда чтобы ур. $P(x, y)dx + Q(x, y) dy = 0$ представляло из себя ур. в полных дифференциалах необходимо и достаточно:
\begin{align}
    & \frac{\partial P}{\partial y} , \frac{\partial Q}{\partial x} \in C(G) \\
    & \frac{\partial P}{\partial y} - \frac{\partial Q}{\partial x}  = 0 
\end{align}
\end{theorem}
\beginproof \\
\tit{необходимость}: найдется $U(x, y) = P(x, y) dx + Q(x, y) dy$ : $U'_x = P, ~ U'_y \hm{=} Q$ т.к. $P'_y , Q'_x \in C(G)$, тогда $U''_{xy} , U''_{yx}$ непрерывны, следовательно равны, значит $P'_{y} = Q'_{x}$ \\
\tit{достаточность}: рассмотрим 
\begin{equation}
    U(x, y) = g(y) + \int_{x_0}^{x} P(t, y)dt
\end{equation}
задача состоит в том, чтобы так подобрать $g(y)$, что $U''_{xy}=  U''_{yx}$, то есть:
\begin{equation}
    \frac{\dif}{\dif y} U(x, y) =  g'(y) + \frac{\dif}{\dif y} \int_{x_0}^{x} P(t, y)dt
\end{equation}
Интеграл по конечкому промежутку от непрерывной функции, тогда возможно преобразование:
\begin{equation}
     \frac{\dif}{\dif y}\int_{x_0}^{x} P(t, y)dt  = \int_{x_0}^{x}  \frac{\dif}{\dif y} P(t, y)dt = \int_{x_0}^{x} P(t, y)dt = \int_{x_0}^{x}  \frac{\dif}{\dif t} Q(t, y) dt
\end{equation}
Последний переход обусловен равными частными производными по условиям теоремы
\begin{equation}
\int_{x_0}^{x}  \frac{\dif}{\dif t} Q(t, y) dt = Q(x, y) - Q(x_0 , y)    
\end{equation}
тогда $U'_y =  Q(x, y) - Q(x_0, y) +  g'(y)$ -- что получили\\
$U'_y = Q (x , y)$ -- по условию теоремы\\
значит $g'(y) =  Q(x_0 , y)$ -- это разрашимое ур. , досточтоно проинтегрировать, в конечном итоге:
\begin{equation}
    U = \int_{x_0}^{x} P(a, y) da + \int_{x_0}^{y} Q(x, b) db
\end{equation}
Удовлетворяет всем условиям

\subsubsection{интегрирующий множитель}

\begin{Def} \fcolorbox{g}{g}{интегрирющий множитель} \\
функция двух перепменных $\nu(x, y)$ -- \tit{интегрирующий множитель} , если
\begin{itemize}
    \item  $\mu(x, y)$ -- не 0 в $G$
    \item существует такая $U(x, y)$ , что $U = const$ в области $G$ и
    \item $dU = \mu P dx  + \mu Q dy $ ($U'_x = \mu P , U'_y = \mu Q$)
\end{itemize}
\end{Def}
Составим уравнение для поиска $\mu$ :
\begin{equation} \label{eq:fing_mu}
    \mu (P_y' - Q_x') = Q \mu_x - P \mu_y
\end{equation}
В общем случае это уравнение не проще исходного, но иногда можно рассмотреть частные случаи, на пример $\mu$ зависит только от одной переменной, на пример $\mu(x)$, тогда \eqref{eq:fing_mu} упрощается до ур. с разделяющимися переменными (если $P'_y - Q_y '$ не зависит от $y$):
\begin{equation}
    \mu(x) (P_y' - Q_x') = - P \frac{\partial \mu}{\partial x}
\end{equation}
Так же иногда интегрирующий множетель находится в виде $\mu(x^{\al} y^{\beta}), \mu(ax \pm by), \mu(ax^2 \pm by^2)$ 
\begin{problem}
$y' + y(2 + \frac{\ln y}{x})$ -- решить, методом интегрирующего множителя, интегрирующий множитель искать в виде $\mu(x^{\al} y^{\beta})$
\end{problem}

\subsection{Линейные ОДУ}
\begin{Def} \fcolorbox{g}{g}{линейное ОДУ $n$-ного порядка} \\
$a_0(x)y^{(n)}(x) + a_1(x)y^{(n-1)}(x) + ... + a_n(x)y(x) - b(x) = 0$ -- общий вид линейного ОДУ n-ного порядка ($a_0(x)$ -- не тождественный 0 на $G$, в таком случае после деления на этот коэффициент можно считать $a_0(x) = 1$)
\end{Def}
В случае $b(x) = 0$ (тождественно) -- однородное линейное уравнение
\begin{Def} \fcolorbox{g}{g}{линейный оператор}
опрератор $A$, заданный на $E$ называется линейным, если:  $\forall y_1, y_2$  $A(c_1y_1 + c_2 y_2) = c_1A(y_1)+ c_2A(y_2), c_1, c_2 \in \CC$
\end{Def}
\begin{theorem} \fcolorbox{o}{o}{ I принцип суперпозиции}
пусть $y_1(x) , y_2(x)$ -- решения линейного, однородного ОДУ, тогда любая их линейная комбинация $y_3 = c_1 y_1(x) + c_2 y_2(x)$ -- так же решение того же уравнения
\end{theorem}

\begin{theorem} \fcolorbox{o}{o}{ II принцип суперпозиции }
если $y_1(x)$ -- решение линейного ОДУ: $L(y_1(x)) = b_1(x)$, $y_2(x)$ -- решение линейного ОДУ: $L(y_2(x)) = b_2(x)$, то $y_3 = y_1(x) + y_2(x)$ -- решение $L(y) = b_1(x) + b_2(x)$, где $L(y) = a_0(x)y^{(n)} + a_1(x)y^{(n-1)} + ... + a_{n-1}(x)y^{(1)}$
\end{theorem}
\beginproof: оба доказательства следует из линейности оператора диффиереницирования.
По теоремам 4.3 и 4.2 получаем, что любое решение $L(y) =b (x)$ представляется в виде суммы $y(x) = y_0(x) + y_{1}(x)$, где $y_0(x)$ -- решение соответствующего однородного линейного ОДУ :$L(y) = 0$, $y_1(x)$ -- произвольное решение $L(y) = b(x)$.

\begin{theorem} \fcolorbox{o}{o}{существование и единственность  решение ОДУ порядка n}\\
пусть $y^{(n)} =  f(x, y, y' , y'', ... ,y^{(n-1)})$ -- ОДУ\\
все частные производные первого порядка $f$ по $x, y, y' , ... , y^{(n-1)}$. непрерывны на обдасти $\Om$ \\
точка $(x_0, y_0 , y'(x_0), y''(x_0) , ... , y^{(n-1)}(x_0) ) \in \Om$, тогда при НУ :  $y(x_0) = y_0 ,  y'(x_0) = y_1 , ... , y^{(n-1)}(x_0)$ уравнение имеет ровно одно решение

\end{theorem}




