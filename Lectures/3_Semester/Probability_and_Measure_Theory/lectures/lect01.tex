\section{Вводная Лекция}

\subsection{Теория Вероятностей}

\begin{definition}
	\textbf{Вероятность} --- $\sigma$-аддитивная мера вероятностного пространства 
	$(\Omega, \F, \Pb)$, где
	\begin{enumerate}
		\item[$\Omega$] --- пространство элементарных исходов.
		\item[$\F$]  --- $\sigma$-алгебра на $\Omega$
		\item[$\Pb$] --- $\sigma$-аддитивная мера на $\F$
	\end{enumerate}
\end{definition}

\subsection{Предмет исследования теории вероятности}

Случайный эксперимент
\begin{enumerate}
	\item Отсутствие детерменированной устойчивости

		Пример: Есть студент, корректен ли вопрос "Какова вероятность того что студент А получит отл 10 на экзамене по теории меры?". Ответ: нет, вопрос безграмотен. Студент А уникален и будет сдавать максимум 3 раза, поэтому о повторяемости не может идти речи. Так же студент А будет меняться, приобретая знания, поэтому и о детерминированном результате говорить не приходится.

	\item Повторяемость

		Должна быть возможность повторить эксперимент.

	\item Статистическая устойчивость

		Вы проводите серии испытаний
		\begin{center}
		\begin{tabular}{ c c c }
			1 серия & $10^4$ & $\frac{N(A)}{N}$ \\
			2 серия & $10^4$ & $\frac{N(A)}{N}$ \\
			\vdots  & \vdots &\\
		\end{tabular}
		\end{center}
		Испытания, например, встречаете ли вы девушку, которая вам нравится, на остановке в этот день.
		И частоты  $\frac{N(A)}{N}$ должны быть \textbf{близки} .Понятие \textbf{близки} плохо переваривается математиками, так как оно относится к понятию (физической) модели, в контексте которой существует близость.
		
\end{enumerate}

	\begin{tabular}{p{0.5\textwidth}|p{0.5\textwidth}}
			Случайный Эксперимент (жизнь) & Математическая Модель (математика) \\
		\hline
		\hline
		Результат эксперимента\\ Пример: кидаете кости & Элементарный исход $ \{\omega_i\} = \Omega $  \\
		\hline
		Множество благоприятных результатов $A$ & событие $A \subseteq \Omega$ (множество событий обозначается так: $\{A\} = \F$) \\
		\hline
		$\frac{N(A)}{N}$ --- частота события\\ N - количество экспериментов\\ N(A) - количество благоприятных результатов & $p(A)$ - вероятность \\
		\hline
	\end{tabular}

$\F = \{A\}$ --- пространство событий. Оно должно быть замкнуто относительно теоретико-множественный операций. Например, пересечение двух событий будет множеством и это множество тоже будет событием.

$p$ ---  вероятность, должна обладать следующими свойствами:
\begin{enumerate}
		\item $0 \leq p(A) \leq 1$
		\item  $p(\Omega) = 1$
		\item $p(A\sqcup B) = P(A) + P(B)$
\end{enumerate}

\subsection{Классическая теория вероятности}

 Это ТВ, которая исследует конечные математические  модели с равновероятными мат исходами.

$\Omega = \{\omega\}^n_{i=1}$

$$ 1 = p(\Omega) = p(\overset{n}{\underset{i=1}{\bigsqcup}}\omega_i)=\sum^n_{i=1}p(\omega_i) $$

$$ 1 = cn, p(\omega_i) = \frac1n = \frac1{|\Omega|} $$

$$ p(A) = p\left(\underset{i:w_i \in A}{\bigsqcup}\omega_i\right) = \sum_{i: w_i \in A} \frac1n = \frac{|A|}{|\Omega|}$$

Тот же вывод еще раз:
$$
\left\{
\begin{aligned}
	& p(\omega_i)=\ldots=p(\omega_n)\\
	& \sum_{i = 1}^{n} p(\omega_i) = 1\\
\end{aligned}
\right.
\Rightarrow
\Pb(A) = \sum_{i: \omega_i \in A}p(\omega_i) = \frac{|A|}{|\Omega|}
$$
 \begin{example}
 	Круглый стол,  $N$ гостей. Какая вероятность  $\Pb(\text{A и B сядут рядом})$
 	Предлагается такая модель:
 	\begin{enumerate}
 		\item $\omega_i = (i_1, \ldots, i_n) \qquad 1 \leq i \leq N$
 		$$\Pb(\text{A рядом с B}) = \frac{N*2*(N-2)!}{N!} = \frac{2}{N-1}$$
 		Первого сажаем куда угодно ($N$ мест), у второго 2 места (справа или слева) и дальше умножаем на $(N-2)!$ --- количество оставшихся свободных мест

 		\item $\omega_i = \{i_1\} \qquad 1 \leq i_1 \leq N-1$ (Нумерация с позиции куда сел А)
 		$$\Pb(\text{A рядом с B}) = \frac{2}{N-1}$$
 		Ура, ответы совпали!
 	 \end{enumerate}
 \end{example}

