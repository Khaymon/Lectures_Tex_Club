\newpage
\lecture{4}{Лекция 4}
\section{Стохастический анализ}
Непрерывность, дифференцируемость и интегрируемовть процессов рассматривается по
времени, исход рассматривают как параметр.
\\
Вспомним сходимости случайных величин:
\begin{itemize}
    \item в среднеквадратичном
    \item почти наверное
    \item по вероятности
    \item по распределению
\end{itemize}
\begin{Def}
    Говорят, что последовательность случайных величин $X_n$ \textbf{сходится в
      среднеквадратичном смысле} к случайной величине $X$:
    \begin{align*}
      & X_n \tosk{n \to \infty} X \Leftrightarrow X = \LIM{n \to \infty} X_n \Leftrightarrow \lim_{n \to \infty} \EE(X_n- X)^2 = 0
    \end{align*}
\end{Def}
\begin{lemma}
    ~
    \\
    Пусть $\left\{ X_m \right\}_{m = 1}^\infty$, $\left\{ Y_n \right\}_{n =
      1}^\infty$~--- случайные величины, заданные на одном вероятностном
    пространстве. \textit{Это подразумевается, если не указано иного.} Тогда
    верно:
    \begin{align*}
      \lim_{n, m \to \infty} \EE X_mY_n = \EE \LIM{m \to \infty} X_m \EE \LIM{n \to \infty} Y_n = \EE XY
    \end{align*}
\end{lemma}
\begin{Proof}
    \begin{align*}
      & X_mY_n - XY = \left( X_m - X \right)\left( Y_n - Y \right) + \left( X_m - X \right)Y + \left( Y_n - Y \right)X
    \end{align*}
    \begin{align*}
      & \left| \EE X_mY_n - \EE XY \right| \leq \left| \EE \left( X_m - X \right)\left( Y_n - Y \right) \right| + \left| \EE \left( X_m - X \right)Y \right| + \left| \EE \left( Y_n - Y \right)X \right|
    \end{align*}
    Применим неравенство Коши-Буняковского.
    \begin{align*}
      & \left| \EE \left( X_m - X \right)\left( Y_n - Y \right) \right| \leq \sqrt{\EE(X_m - X)^2\EE(Y_n - Y)^2} \To{m, n \to \infty} 0
    \end{align*}
    \begin{align*}
      & \left| \EE \left( X_m - X \right)Y \right| \leq \sqrt{\EE(X_m - X)^2\EE Y^2} \To{m \to \infty} 0
    \end{align*}  
    \begin{align*}
      & \left| \EE \left( Y_n - Y \right)X \right| \leq \sqrt{\EE(Y_n - Y)^2\EE X} \To{n \to \infty} 0
    \end{align*}
    Значит,
    \begin{align*}
      & \left| \EE X_mY_n - \EE XY \right| \to 0 \Rightarrow \EE X_mY_m \To{m, n \to \infty} \EE XY
    \end{align*}
\end{Proof}
\begin{Note}
    Если $Y_n \equiv Y \equiv 1$, то
    \begin{align*}
      & X_m \tosk{m \to \infty} X \Rightarrow \EE X_m \To{m \to \infty} \EE X
    \end{align*}
\end{Note}
\begin{lemma}
    (без доказательства)
    \\
    Среднеквадратичная сходимость обладает свойством фундаментальности по Коши.
    \\
    Пусть $\left\{ X_k \right\}_{k=1}^\infty$~--- случайная последовательность,
    причем $\forall k \ \EE X_k^2 < \infty$. Тогда сходимость в СК к некоторой
    случайной величине $X$, $\EE X < \infty$ равносильна тому, что для любых её
    подпоследовательностей $\left\{ Y_n \right\}_{n=1}^{\infty}$, $\left\{ Z_m
    \right\}_{m=1}^\infty$ выполнено
    \begin{align*}
      \LIM{m, n \to \infty}(Y_m - Z_n)^2 = 0
    \end{align*}
\end{lemma}
\begin{lemma}
    ~
    \\
    Пусть
    \begin{align*}
      \left\{ X_k \right\}_{k=1}^\infty, \ \forall k \ \EE X_k^2 < \infty; \ \exists c \in \RR: \ \forall \left\{ Y_n \right\}_{n=1}^\infty, \ \left\{ Z_m \right\}_{m=1}^\infty \ \EE Y_nZ_m \To{m,n \to \infty} c
    \end{align*}
    Тогда существует случайная величина $X$:
    \begin{align*}
      & X_k \tosk{k \to \infty} X, \ \EE X^2 < \infty
    \end{align*}
\end{lemma}
\begin{Proof}
    \begin{align*}
      & \left( Y_m - Z_n \right)^2 = Y_m^2 - 2Y_mZ_n + Z_n^2
    \end{align*}
    \begin{align*}
      & \lim_{m,n \to \infty}\EE\left( Y_m - Z_n \right)^2 = c - 2c + c = 0
    \end{align*}
    По лемме $2$ $X_k$ сходится в СК.
\end{Proof}
\begin{Def}
    процесс $\left\{ X(t), \ t \in T \right\}$ называется \textbf{процессом
      второго порядка} ($L_2$-процессом), если $\forall t \in T \ \EE X^2(t) <
    \infty$.
\end{Def}
\begin{Def}
    Случайный процесс $\left\{ X(t), \ t \in T \right\}$ второго порядка
    называется \textbf{непрерывным в среднеквадратичном смысле} (СК-непрерывным)
    в $t_0$, если
    \begin{align*}
      & X(t_0+\varepsilon) \tosk{\varepsilon \to 0} X(t_0)
    \end{align*}
\end{Def}
\begin{theorem}
    Критерий СК-непрерывности в точке.
    \\
    Процесс $\left\{ X(t), \ t \in T \right\}$ второго порядка СК-непрерывен в
    $t_0$ тогда и только тогда, когда $K_x(t,s)$ непрерывна в $(t_0,t_0)$, что
    равносильно одновременным непрерывности $m_X(t)$ в $t_0$ и $R_X(t,s)$ в
    $(t_0,t_0)$.
\end{theorem}
\begin{Proof}
    ~
    \\
    \begin{itemize}
        \item Первая равносильность.
        \\
        Пусть $X(t)$ СК-непрерывен в $t_0$. Тогда по определению
        \begin{align*}
          & X(t_0 +\varepsilon_1) \tosk{\varepsilon_1 \to 0} X(t_0), \ X(t_0 +\varepsilon_2) \tosk{\varepsilon_2 \to 0} X(t_0)
        \end{align*}
        Из леммы $1$
        \begin{align*}
          & K_X(t_0+\varepsilon_1, t_0+\varepsilon_2) = \EE X(t_0+\varepsilon_1)X(t_0 + \varepsilon_2) \To{\varepsilon_1, \varepsilon_2 \to 0} \EE X^2(t_0) = K_X(t_0,t_0) < \infty
        \end{align*}
        Необходимость непрерывности доказана.
        \\
        Обратно, если ковариационная функция непрерывна, то
        \begin{align*}
          & \EE\left( X(t_0+\varepsilon) - X(t_0) \right)^2 = \EE\left( X(t_0+\varepsilon) - X(t_0) \right)\left( X(t_0+\varepsilon) - X(t_0) \right) = K_X(t_0+\varepsilon, t_0 +\varepsilon) - \\
          & - 2K(t_0+\varepsilon, t_0) + K_X(t_0,t_0) \To{\varepsilon \to 0} 0
        \end{align*}
        По определению это означает СК-непрерывность $X(t)$ в $t_0$.
        \\
        Достаточность непрерывности доказана.
        \item Вторая равносильность.
        \\
        Если $m_X(t)$ непрерывно в $t_0$, как и $R_X(t,s)$ в $(t_0,t_0)$, то в
        силу того, что $K_X(t,s) = R_X(t,s) + m_X(t)m_X(s)$, непрерывно и
        $K_X(t,s)$ в точке $(t_0, t_0)$.
        \\
        Необходимость непрерывности $K_X(t,s)$ доказана.
        \\
        Обратно, если $K_X(t,s)$ непрерывна, то
        \begin{align*}
          & X(t_0+\varepsilon) \tosk{\varepsilon \to 0} X(t_0)
        \end{align*}
        \begin{align*}
          & \EE\left( X(t_0+\varepsilon) - X(t_0) \right)^2 = \EE\left( \cent{X}(t_0+\varepsilon) - \cent{X}(t_0) + m_X(t_0+\varepsilon) - m_X(t_0) \right)^2 = \\
          & = \EE\left( \cent{X}(t_0+\varepsilon) - \cent{X}(t_0)\right)^2 + 2 \EE\left( \cent{X}(t_0+\varepsilon) - \cent{X}(t_0)\right)\left( m_X(t_0+\varepsilon) - m_X(t_0) \right) + \\
          & + \left( m_X(t_0+\varepsilon) - m_X(t_0) \right)^2 = \EE\left( \cent{X}(t_0+\varepsilon) - \cent{X}(t_0)\right)^2 + \left( m_X(t_0+\varepsilon) - m_X(t_0) \right)^2 \To{\varepsilon \to 0} 0
        \end{align*}
        значит, $m_X(t)$ непрерывно в $t_0$;
        \begin{align*}
          & K_{\cent{X}}(t,s) = R_X(t,s)
        \end{align*}
        непрерывно в $(t_0,t_0)$ в силу
        \begin{align*}
          & \cent{X}(t_0+\varepsilon) \tosk{\varepsilon \to 0} \cent{X}(t_0)
        \end{align*}
        значит, $R_X(t,s)$ непрерывна в $(t_0,t_0)$.
        \\
        Достаточность непрерывности $K_X(t,s)$ доказана.
    \end{itemize}
\end{Proof}
\begin{Def}
    Случайный процесс $\left\{ X(t), \ t \in T \right\}$ второго порядка
    называется \textbf{дифференцируемым в среднеквадратичном смысле}
    (СК-дифференцируемым) \textbf{в точке $t_0 \in T$}, если существует
    случайная величина $\eta$, $\EE \eta^2 < \infty$:
    \begin{align*}
      & \frac{X(t_0+\varepsilon) - X(t_0)}{\varepsilon} \tosk{\varepsilon \to 0} \eta
    \end{align*}
    Величина $\eta$ называется \textbf{производной в среднеквадратичном смысле}
    (СК-производной) процесса $X(t)$ в точке $t_0$ и обозначается $\eta =
    X'(t_0)$.
\end{Def}
\begin{Def}
    Случайный процесс $\left\{ X(t), \ t \in T \right\}$ второго порядка
    называется \textbf{дифференцируемым в среднеквадратичном смысле}
    (СК-дифференцируемым), если $\forall t_0 \in T$ он дифференцируем в $t_0$.
    Процесс $\eta(t)$ называется его \textbf{СК-производной}.
\end{Def}
\begin{theorem}
    Критерий СК-дифференцируемости в точке.
    \\
    Процесс $\left\{ X(t), \ t \in T \right\}$ второго порядка СК-дифференцируем
    в $t_0$ тогда и только тогда, когда
    \begin{align*}
      & \lim_{^{\varepsilon \to 0}_{\delta \to 0}}\frac{1}{\delta \varepsilon}\left( K_X(t_0+\varepsilon, t_0+\delta) - K_X(t_0+\varepsilon, t_0) - K_X(t_0, t_0+\delta) + K_X(t_0+\varepsilon, t_0+\delta)\right) < \infty
    \end{align*}
    что равносильно одновременным дифференцируемости $m_X$ в $t_0$ и
    существованию
    \begin{align*}
      & \lim_{^{\varepsilon \to 0}_{\delta \to 0}}\frac{1}{\delta \varepsilon}\left( R_X(t_0+\varepsilon, t_0+\delta) - R_X(t_0+\varepsilon, t_0) - R_X(t_0, t_0+\delta) + R_X(t_0+\varepsilon, t_0+\delta)\right) < \infty
    \end{align*}
\end{theorem}
\begin{Proof}
    ~
    \\
    \begin{itemize}
        \item Первая равносильность.
        \\
        Пусть $X(t)$ СК-дифференцируем в $t_0$. Введем обозначения:
        \begin{align*}
          & Y_\varepsilon = \frac{X(t_0+\varepsilon) - X(t_0)}{\varepsilon}, \ Y_\delta = \frac{X(t_0+\delta) - X(t_0)}{\delta}
        \end{align*}
        Заметим, что
        \begin{align*}
          & \EE Y_\varepsilon Y_\delta = \frac{1}{\delta \varepsilon}\left( K_X(t_0+\varepsilon, t_0+\delta) - K_X(t_0+\varepsilon, t_0) - K_X(t_0, t_0+\delta) + K_X(t_0+\varepsilon, t_0+\delta)\right)
        \end{align*}
        \begin{align*}
          & Y_\varepsilon \tosk{\varepsilon \to 0} X'(t_0), \ Y_\delta \tosk{\delta \to 0} X'(t_0)
        \end{align*}
        Тогда по лемме $1$ необходимо
        \begin{align*}
          & \EE Y_\varepsilon Y_\delta \To{^{\varepsilon \to 0}_{\delta \to 0}} \EE (X'(t_0))^2 < \infty
        \end{align*}
        Необходимость конечности предела доказана.
        \\
        Обратно, если предел существует и конечен, то по лемме $3$ существует
        случайная величина $\eta$: $\EE \eta^2 < \infty$, $Y_{\varepsilon}
        \tosk{\varepsilon \to 0} \eta$, что по определению дает
        СК-дифференцируемость в $t_0$.
        \\
        Достаточность конечности предела доказана.
        \item Вторая равносильность.
        \\
        Если $m_X(t)$ дифференцируемо в $t_0$, а предел из условия для
        $R_X(t,s)$ существует, то в силу того, что $K_X(t,s) = R_X(t,s) +
        m_X(t)m_X(s)$, этот предел существует и для $K_X(t,s)$.
        \\
        Необходимость предела для $K_X(t,s)$ доказана.
        \\
        Обратно, если предел существует, то $X(t)$ СК-дифференцируем в $t_0$, а
        значит,
        \begin{align*}
          & \EE\left( Y_\varepsilon - X'(t_0) \right)^2 = \EE \left( \frac{X(t_0 +\varepsilon) - X(t_0)}{\varepsilon} - X'(t_0) \right)^2 = \EE \left( \frac{\cent{X}(t_0 +\varepsilon) - \cent{X}(t_0)}{\varepsilon} - \cent{X'}(t_0) \right)^2 + \\
          & + \EE \left( \frac{m_X(t_0 +\varepsilon) - m_X(t_0)}{\varepsilon} - \EE X'(t_0) \right)^2 \To{\varepsilon \to 0} 0
        \end{align*}
        В силу дифференцируемости $m_X(t)$ в $t_0$ и СК-дифференцируемости
        $\cent{X}(t)$ в $t_0$ предел для $K_{\cent{X}}(t,s)$ конечен в
        $(t_0,t_0)$. Но
        \begin{align*}
          & K_{\cent{X}}(t,s) = R_X(t,s)
        \end{align*}
        значит, предел для $R_X(t,s)$ существует и конечен.
        \\
        Достаточность предела для $K_X(t,s)$ доказана.
    \end{itemize}
\end{Proof}
\begin{Note}
    Достаточным условием существования обобщенной смешанной производной от
    $K_X(t,s)$ в $(t_0,t_0)$ (предела из условия критерия) является
    непрерывность хотя бы одной из смешанных производных
    \begin{align*}
      & \frac{\partial}{\partial t}\left( \frac{\partial}{\partial s} \left( K_X(t,s) \right) \right), \ \frac{\partial}{\partial s}\left( \frac{\partial}{\partial t} \left( K_X(t,s) \right) \right)
    \end{align*}
    в $(t_0,t_0)$. Тогда обе производные равны между собой и с обобщенной.
\end{Note}
\begin{Def}
    Пусть процесс $\left\{ X(t), \ t \in T \right\}$ второго порядка определен
    на $[a,b] \subseteq T$. Построим разбиение $a = t_0 < t_1 < \dots < t_{n-1}
    < t_n = b$, $\tau_i \in [t_{i-1}, t_i)$; $\Delta = \dst \max_{i \in \{1,
      \dots, n\}}\left| t_i-t_{i-1} \right|$ называется его мелкостью. Выберем
    на каждом $[t_{i-1}, t_i]$ произвольно $\tau_i$. Если при $n \to \infty$,
    $\Delta \to 0$ существует предел в СК
    \begin{align*}
      & \sum_{i=1}^n X(\tau_i)(t_i-t_{i-1}) \tosk{^{\Delta \to 0}_{n \to \infty}} \eta
    \end{align*}
    не зависящий от $\{t_i\}$, $\{\tau_i\}$, то эта величина называется
    \textbf{интегралом Римана в среднеквадратичном смысле от процесса $X(t)$ в
      пределах от $a$ до $b$} (СК интегралом Римана от $X(t)$ от $a$ до $b$) и
    записывается как
    \begin{align*}
      & \eta = \int_a^bX(t) dt
    \end{align*}
    а процесс~--- \textbf{интегрируемым по Риману в среднеквадратичном смысле в
      пределах от $a$ до $b$} (СК-интегрируемым по Риману).
\end{Def}
\begin{theorem}
    Критерий существования интеграла Римана в среднеквадратичном смысле. (без
    доказательства)
    \\
    Процесс $\{X(t), \ t \in T\}$ второго порядка СК-интегрируем по Риману на
    $[a,b] \subseteq T$ тогда и только тогда, когда существует конечный интеграл
    Римана
    \begin{align*}
      & \int_a^b\int_a^b K_X(t,s) dt ds
    \end{align*}
    что равносильно существованию конечных интегралов Римана
    \begin{align*}
      & \int_a^b\int_a^b R_X(t,s) dt ds, \ \int_a^b m_X(t) dt
    \end{align*}  
\end{theorem}
\begin{Note}
    Определение интеграла Римана в СК от $X$ обобщается на несобственный случай
    следующим образом:
    \begin{align*}
      & \int_a^\infty X(t) dt = \LIM{b \to \infty} \int_a^b X(t)dt
    \end{align*}
    \begin{align*}
      & \int_{-\infty}^b X(t)dt = \LIM{a \to -\infty} \int_a^bX(t)dt
    \end{align*}
    \begin{align*}
      & \int_{-\infty}^\infty X(t) dt = \LIM{^{a\to -\infty}_{b \to \infty}} \int_a^b X(t)dt
    \end{align*}
\end{Note}
