\newpage 
\lecture{7}{Теорема Крамера и прикладные аспекты теории}
\begin{theorem} (Крамера)
\\
Любому СК-непрерывному процессу $X(t)$ с математическим ожиданием $m_X$
соответствует случайный процесс с ортогональными приращениями $V(\lambda)$
такой, что с вероятностью $1$ выполняется равенство
\begin{align*}
  & X(t) = m_X \int_{-\infty}^{+\infty} e^{i\lambda t}dV(\lambda)
\end{align*}
где интеграл есть СК-интеграл Римана-Стилтьеса.
\end{theorem}
\begin{Proof}
    Рассмотрим множество случайных величин $H\left(\cent{X}\right)$ из сечений
    $\cent{X}(t) = X(t)-m_X$, их линейных комбинаций $\dst
    \sum_{i=1}^n\alpha_i\cent{X}(t_i)$ и всевозможные СК-пределы
    последовательностей таких комбинаций.
    \\
    Введем на этом множестве скалярное произведение
    \begin{align*}
      & \left(\eta_1, \eta_2\right) = \EE \eta_1\oL{\eta_2} \ \forall \eta_1, \eta_2 \in H \left(\cent{X}\right)
    \end{align*}
    Если расстояние, порожденное таким скалярным произведением, ноль, считаем
    элементы тождественными. Получили гильбертово пространство, все его элементы
    имеют нулевые математические ожидания.
    \\
    Обозначим как $L_2(S)$, где $S$~--- спектральная функция процесса $X(t)$,
    множество всех неслучайных комплексных $g(\lambda)$ таких, что
    \begin{align*}
      & \int_{-\infty}^{+\infty}\left| g(\lambda) \right|dS(\lambda)
    \end{align*}
    Если ввести скалярное произведение
    \begin{align*}
      & (g_1,g_2) = \int_{-\infty}^{+\infty} g_1(\lambda)\oL{g_2(\lambda)}dS(\lambda)
    \end{align*}
    то получим гильбертово пространство $H(S)$. Если расстояние, порожденное
    таким скалярным произведением, ноль, считаем элементы тождественными.
    \\
    Установим соответствие между пространствами.
    \begin{align*}
      & \cent{X}(t) \in H \left( \cent{X} \right) \leftrightarrow e^{i t \lambda} \in H(S)
    \end{align*}
    По теореме Хинчина
    \begin{align*}
      & \EE\cent{X}(t)\oL{\cent{X}(s)} = \int_{-\infty}^{+\infty}e^{it\lambda}e^{-is\lambda}dS(\lambda)
    \end{align*}
    и скалярное произведение сохраняется.
    \begin{align*}
      & \eta = \sum_{i=1}^{n}\alpha_i\cent{X}(t_i) \in H \left( \cent{X} \right) \leftrightarrow g(\lambda) = \sum_{j=1}^n \alpha_j e^{i t_j \lambda} \in H(S)
    \end{align*}
    По теореме Хинчина
    \begin{align*}
      & \left( \eta_1, \eta_2 \right) = \left( g_1, g_2 \right)
    \end{align*}
    и скалярное произведение сохраняется.
    \begin{align*}
      & \EE\left| \eta_1 - \eta_2 \right|^2 = \int_{-\infty}^{+\infty}\left| g_1(\lambda) - g_2(\lambda) \right|^2dS(\lambda)
    \end{align*}
    Пусть $\{\eta_n\}_{n=1}^{\infty}$~---последовательность случайных величин
    вида $\eta_i$, $\tosk \eta$. Тогда из
    \begin{align*}
      & \EE\left| \eta_m - \eta_n \right|^2 = \int_{-\infty}^{+\infty}\left| g_m(\lambda) - g_m(\lambda) \right|^2dS(\lambda) \to 0
    \end{align*}
    \begin{align*}
      & \exists g(\lambda): \ g_n(\lambda) \to g(\lambda)
    \end{align*}
    \begin{align*}
      & \eta \leftrightarrow g(\lambda)
    \end{align*}
    Аналогично можно построить и в обратном направлении. Отсюда следует
    выполнимость свойства о скалярном произведении и расстоянии.
    \\
    Введем
    \begin{align*}
      & g(t) = \chi_{t \leq 0}
    \end{align*}
    \begin{align*}
      & \forall \lambda_0 \in \RR \ g(\lambda - \lambda_0) = \chi_{\lambda \leq \lambda_0} \in H(s) \leftrightarrow V(\lambda_0) \in H\left( \cent{X} \right)
    \end{align*}
    Пусть $\lambda_0 < \lambda_1$, рассмотрим
    \begin{align*}
      &V(\lambda_1) - V(\lambda_0) = g(\lambda-\lambda_1) - g(\lambda - \lambda_0) = \chi(\lambda)_{(\lambda_0, \lambda_1]}
    \end{align*}
    Пусть теперь $\lambda_1 < \lambda_2 < \lambda_3 < \lambda_4$ и рассмотрим
    \begin{align*}
      & \left( V(\lambda_4) - V(\lambda_3)\right)\oL{\left( V(\lambda_2) - V(\lambda_1) \right)} = \int_{-\infty}^{+\infty}\left( g(\lambda-\lambda_4) - g(\lambda-\lambda_3) \right)\oL{\left( g(\lambda-\lambda_2) - g(\lambda-\lambda_1) \right)}dS(\lambda) = 0
    \end{align*}
    поскольку множители равны единице на непересекающихся интервалах, и
    $V(\lambda)$~--- процесс с ортогональными приращениями.
    \\
    Если взять совпадающие интервалы,
    \begin{align*}
      & \EE \left| V(\lambda_2) - V(\lambda_1)\right|= \int_{-\infty}^{+\infty}\chi(\lambda)_{(\lambda_1, \lambda_2]}dS(\lambda) = \int_{\lambda_1}^{\lambda_2}dS(\lambda) = S(\lambda_2) - S(\lambda_1)
    \end{align*}
    Пусть некоторое разбиение
    \begin{align*}
      & -A = \lambda_1 < \lambda_2 < \dots < \lambda_n = A
    \end{align*}
    Тогда
    \begin{align*}
      & \eta = \sum_{j=1}^n e^{it\lambda_j}\left( V(\lambda_{j+1}) - V(\lambda_j) \right) \leftrightarrow g(\lambda) = \chi(\lambda)_{(\lambda_j, \lambda_{j+1}]}e^{it\lambda_j} \To{\Delta \lambda \to 0} e^{it\lambda} 
    \end{align*}
    \begin{align*}
      & \eta = \sum_{j=1}^n e^{it\lambda_j}\left( V(\lambda_{j+1}) - V(\lambda_j) \right) \To{^{\Delta \lambda \to 0}_{A \to \infty}} \int_{-\infty}^{\infty}e^{it\lambda}dV(\lambda)
    \end{align*}
    \begin{align*}
      & \cent{X}(t) = \int_{-\infty}^{\infty}e^{it\lambda}dV(\lambda)
    \end{align*}  
\end{Proof}
\begin{Note}
    Для всяких линейных комбинаций имеет место соотношение
    \begin{align*}
      & \eta = \int_{-\infty}^{\infty}g(\lambda)dV(\lambda)
    \end{align*}
    Кроме того, это верно для $\forall \eta \in H\left( \cent{X} \right)$ и
    $g(\lambda)\in H(S)$
\end{Note}
\subsection{Физический смысл спектральной плотности}
Пусть дан центрированный СК-непрерывный второго порядка процесс~---
\textbf{сигнал}
\begin{align*}
  & X(t) = \int_{-\infty}^{\infty}e^{it\lambda}dV(\lambda)
\end{align*}
Мощность сигнала:
\begin{align*}
  & N(X) = \lim_{t \to \infty} = \int_{0}^{T}\left| X(t) \right|^2 dt
\end{align*}
Тогда
\begin{align*}
  & N\left( e^{it\lambda}dV(\lambda) \right) = \left| dV(\lambda) \right|^2
\end{align*}
\begin{align*}
  & \EE\left| dV(\lambda) \right|^2 = dS(\lambda) = \rho(\lambda)d\lambda
\end{align*}
Средняя мощность, отвечающая гармонической компоненте с частотой $\lambda$.
\begin{Note}
    Спектральная плотность~--- плотность распределения средней мощности сигнала
    $X(t)$ в частотной области.
    \\
    На интервале $(\lambda_0, \lambda_1)$ она равна $S(\lambda_1)-S(\lambda_0)$,
    а на $\RR$ $\EE \left| X(0) \right|^2$.
\end{Note}
\subsection{Белый шум}
\begin{Def}
Стационарный процесс
\begin{align*}
  & \rho_{\lambda_0}(\lambda) = C\chi(\lambda)_{(-\lambda_0, \lambda_0)}
\end{align*}
где $C$ не зависит от $\lambda$, может моделировать сигнал, излучающий с
одинаковой мощностью в данном диапазоне частот.
\\
Такой процесс называется \textbf{белым шумом}.
\end{Def}
Его корреляционная функция
\begin{align*}
  & R_{\lambda_0}(t) = 2C\frac{\sin \lambda_0 t}{t}, \ R_{\lambda_0}(0) = 2C\lambda_0
\end{align*}
Часто называют белым шумом процесс, излучающий на всем $\RR$.
\\
У такого процесса корреляционная функция ($\lambda_0 = \infty$)
\begin{align*}
  & R_{\lambda_0}(t) = 2C\pi \delta(t)
\end{align*}
У шума сечения независимы, дисперсия бесконечна, распределения любые.
\begin{Def}
    На практике часто применяется \textbf{гауссовский белый шум}~--- такой, у
    которого распределения нормальны.
\end{Def}
\begin{Note}
    Гауссовский белый шум получается при обобщенном дифференцировании
    винеровского процесса (по распределению).
\end{Note}
\begin{Def}
    Последовательность $\IID$ случайных блужданий называется \textbf{дискретным
      белым шумом}.
\end{Def}
\subsection{Линейная теория стационарных процессов}
Пусть $X(t)$~--- стационарный СК-непрерывный процесс с нулевым математическим
ожиданием, тогда по теореме Крамера
\begin{align*}
  & X(t) = \int_{-\infty}^{+\infty}e^{it\lambda} dV_X(\lambda), \ \EE \left| dV_X(\lambda) \right|^2 = dS_X(\lambda)
\end{align*}
Пусть $\cL$~--- линейное преобразование над случайными величинами. Пусть
\begin{align*}
  & Y(t) = (\cL X)(t) = \cL\left( \int_{-\infty}^{+\infty}e^{it\lambda} dV_X(\lambda)\right) = \int_{-\infty}^{+\infty}\cL\left( e^{it\lambda}\right) dV_X(\lambda) = \int_{-\infty}^{+\infty} e^{it\lambda} dV_Y(\lambda)
\end{align*}
Если 
\begin{align*}
  & \cL\left( e^{it\lambda}\right) = \Phi(\lambda)e^{it\lambda}
\end{align*}
то
\begin{align*}
  & dS_Y(\lambda) = \left| \Phi(\lambda) \right|^2 dS_X(\lambda)
\end{align*}
\begin{align*}
  & \rho_Y(\lambda) = \left| \Phi(\lambda) \right|^2 \rho_X(\lambda)
\end{align*}
При линейных преобразованиях процессы могут изменяться очень сложным образом, но
в частотной области всё меняется очень просто. Композиция линейных
преобразований~--- умножения этих функций в частотной области.
\begin{Def}
    $\Phi(\lambda)$~--- \textbf{частотная характеристика} преобразования $\cL$.
\end{Def}
Для определения частотной характеристики достаточно рассмотреть, как
преобразование воздействует на гармонику.
\\
\begin{example}
    Пусть $X(t)$ СК-дифференцируем и
    \begin{align*}
      & \exists \int_{-\infty}^{+\infty}\lambda^2dS_X(\lambda)< \infty
    \end{align*}
    Тогда существует СК-интеграл
    \begin{align*}
      & \int_{-\infty}^{+\infty}\lambda e^{it\lambda} dV_X(\lambda)
    \end{align*}
    и
    \begin{align*}
      & \frac{X(t+h) - X(t)}{h} - \int_{-\infty}^{+\infty}i\lambda e^{it\lambda} dV_X(\lambda) = \int_{-\infty}^{+\infty}\left( \frac{e^{ih\lambda}-1}{h}-i\lambda \right)e^{it\lambda dV_X(\lambda)}
    \end{align*}
     \begin{align*}
      & \EE\left| \frac{X(t+h) - X(t)}{h} - \int_{-\infty}^{+\infty}i\lambda e^{it\lambda} dV_X(\lambda) \right|^2 = \left| \int_{-\infty}^{+\infty}\left( \frac{e^{ih\lambda}-1}{h}-i\lambda \right)e^{it\lambda dV_X(\lambda)} \right|^2
     \end{align*}
     Устремляя $h \to 0$, получаем
     \begin{align*}
      & X'(t) = \int_{-\infty}^{+\infty}i\lambda e^{it\lambda} dV_X(\lambda)
     \end{align*}
     \begin{align*}
       & \cL\left( e^{i\lambda t} \right) = i\lambda e^{it\lambda} \Rightarrow \left| \Phi(\lambda) \right|^2 = \left| i\lambda \right|^2 = \left| \lambda \right|^2
    \end{align*}
\end{example}
\begin{Def}
    \begin{align*}
      & \int_{-\infty}^{+\infty}\lambda^{2k}dS_X(\lambda)< \infty
    \end{align*}
    называется $2k$-спектральным моментом.
\end{Def}
\begin{theorem} (без доказательства)
    \\
    $2k$-спектральный момент существует и конечен тогда и только тогда, когда
    столько раз в нуле дифференцируема корреляционная функция.
\end{theorem}