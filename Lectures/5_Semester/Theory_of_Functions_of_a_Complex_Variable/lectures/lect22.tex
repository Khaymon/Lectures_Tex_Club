\begin{flushright}
    \textit{Лекция 22 (от 17.11)}
\end{flushright}
\section{$\S 27.$ Аналитическое продолжение.}
\Def
Путь задана $f: D \mapsto \CC$, $D \subseteq G \subseteq \CC$, $G$~--- область,
$g: G \mapsto \CC$ регулярна. Если $\forall z \in D \ f(z) = g(z)$, то $g$
называется \textbf{аналитическим продолжением} $f$ на $G$.
\Example
\begin{align*}
  & e^x \to e^z = e^xe^{iy}, \ D = \RR \subseteq \CC \to G = \CC
\end{align*}
\Example
\begin{align*}
  & \sin x \to \sin z = \frac{e^{iz}-e^{-iz}}{2}, \ D = \RR \subseteq \CC \to \CC
\end{align*}
\Def
Пусть $a \in \CC$, $f: B_r(a) \mapsto \CC$, $r>0$~--- регулярная. Тогда $\left(
    B_r(a), f \right)$ называется \textbf{элементом аналитической функции с
  центром в точке $a$}.
\Def
Два элемента $\left( B_r(a), f \right)$ и $\left( B_\rho(b), g \right)$
называются \textbf{эквивалентными}, если $a = b$, $\forall z \in B_{r_0}(a) \
f(z) = g(z)$, $r_0 = \min \left\{ r, \rho \right\}$.
\Def
Пусть $\left( B_r(a), f \right)$~--- элемент аналитической функции, тогда
говорят, что элемент $\left( B_\rho(b), g \right)$ является его
\textbf{непосредственным аналитическим продолжением (НАП)}, если $B_r(a) \cap
B_\rho(b) \neq \varnothing$, $\forall z \in B_r(a) \cap B_\rho(b) \ f(z) =
g(z)$.
\Note
Если определены $B_r(a), B_\rho(b), f$, то по теореме единственности однозначно
определен и $\left( B_\rho(b), g \right)$.
\Def
Пусть $\left( B_r(a), f \right)$~--- элемент. Говорят, что $\left( B_\rho(b), g
\right)$ есть \textbf{аналитическое продолжение элемента $\left( B_r(a), f
  \right)$ вдоль конечной цепочки элементов (кругов)}, если $\exists \left\{
    \left( B_{r_k}(z_k), f_k \right) \right\}_{k=0}^n$ такое, что $\left(
    B_{r_0}(z_0), f_0 \right) \sim \left( B_r(a), f \right)$, $\left(
    B_{r_n}(z_n), f_n \right) \sim \left( B_\rho(b), g \right) $, $\forall k \in
\left\{ 1,\dots, n\right\}$ $\left( B_{r_k}(z_k), f_k \right)$ является
непосредственным аналитическим продолжением $\left( B_{r_{k-1}}(z_{k-1}),
    f_{k-1} \right)$.
\Example
\begin{align*}
  & f_1(z) = \sum_{n=0}^{\infty}z^n
\end{align*}
сходится на $B_1(0)$ и расходится при $\left| z \right| \geq 1$. Пусть $\left(
    B_1(0), f_1 \right)$~--- элемент; тогда
\begin{align*}
  & f_2 = \frac{1}{1-z}
\end{align*}
регулярна в $\CC \setminus \left\{ 1 \right\}$, и для $a \in \CC \setminus [1;
+\infty)$ элемент $\left( B_{\left| a-1 \right|}(a), f_2 \right)$ будет НАП
$\left( B_1(0), f_1 \right)$; для $a_2 \in (1; +\infty)$ элемент $\left(
    B_{\left| a_2-1 \right|}(a_2), f_2 \right)$ не будет НАП $\left( B_1(0), f_1
\right)$, но будет аналитическим продолжением вдоль конечной цепочи кругов.
\Example
Пусть
\begin{align*}
  & f_s(z) = \sqrt{\left| z \right|}\exp\left( \frac{i}{2}\argt_s(z) \right), \ \argt_s(z) \in \left( s-\frac{\pi}{2}, s+\frac{\pi}{2} \right)
\end{align*}
Для элемента $\left( B_1(1), f_0 \right)$ элемент $\left( B_1(i),
    f_{\frac{\pi}{2}} \right)$ будет НАП, а для него, в свою очередь, элемент
$\left( B_1(-1), f_\pi \right)$ будет НАП.
\\
Для элемента $\left( B_1(1), f_0 \right)$ элемент $\left( B_1(-i),
    f_{-\frac{\pi}{2}} \right)$ будет НАП, а для него, в свою очередь, элемент
$\left( B_1(-1), f_{-\pi} \right)$ будет НАП.
\begin{align*}
  & \forall z \in B_1(-1) \ f_\pi(z) = f_{-\pi}(z)
\end{align*}
\section{$\S 28.$ Полные аналитические функции $\Ln z$, $\sqrt[n]{z}$ и римановы
  поверхности.}
