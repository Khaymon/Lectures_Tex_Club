\begin{flushright}
    \textit{Лекция 2 (от 08.09)}
\end{flushright}
\section{$\S 3.$ Дифференцирование}
Пусть $z_0 \in \CC$, $B_r(z_0)$ и $f: B_r(z_0) \mapsto \CC$
\Def
Пусть $f: B_r(z_0) \mapsto \CC$. Если
\begin{equation}\label{(3.1)}
    \exists \dst \lim_{z \mapsto z_0} \frac{f(z)-f(z_0)}{z-z_0}
\end{equation}
то он называется \textbf{производной функции $f$ в точке $z_0$} (записывается
$f'(z_0)$).
\\
Можем расписать предел:
\begin{equation}\label{(3.2)}
    \forall \varepsilon > 0 \exists \delta(\varepsilon)>0: \ z \in B_\delta(z_0), \ \left| \frac{f(z) - f(z_0)}{z-z_0} - f'(z_0) \right| < \varepsilon
\end{equation}
\begin{equation}\label{(3.3)}
    f(z) = f(z_0) + f'(z_0)(z-z_0) + \alpha(z-z_0)
\end{equation}
\begin{equation}\label{(3.4)}
    \lim_{z \to z_0}\frac{\alpha(z-z_0)}{z-z_0} = 0
\end{equation}
\begin{align*}
  & f(z_0) = u(x,ym) + iv(x,y)
\end{align*}
\Def
Говорят, что $f: B_r(z_) \mapsto \CC$ \textbf{дифференцируема в точке $z_0$},
если
\begin{equation}\label{(3.5)}
    \exists A \in \CC: \ f(z) = f(z_0)+A(z-z_0)+o(z-z_0)
\end{equation}
\lemma
$f: B_r(z_0) \mapsto \CC$ дифференцируема в $z_0$ $\Leftrightarrow$ $\exists
f'(z_0)$, $A = f'(z_0)$
\pr
Очевидно.
\\
Обозначим: $\Delta x = x-x_0$, $\Delta y = y - y_0$, $\Delta z = z-z_0 = \Delta
x + i \Delta y$, $\Delta u = u(x,y)-u(x_0,y_0)$, $\Delta v = v(x,y)-v(x_0,y_0)$,
$\Delta f = f(x,y)-f(x_0,y_0) = \Delta u + i \Delta v$.
\theorem
$f: B_r(z_0) \mapsto \CC$ дифференцируема в $z_0$ т огда и только тогда, когда
\begin{itemize}
    \item $u(x,y)$, $v(x,y)$ дифференцируемы в $(x_0,y_0)$ (в $\RR^2$)
    \item выполняется условие Коши-Римана (УКР):
    \begin{equation}\label{(3.6)}
        \left\{ \begin{matrix}
                \dst \frac{\partial u}{\partial x} = \dst \frac{\partial v}{\partial y} \\
                \dst \frac{\partial v}{\partial x} = - \dst \frac{\partial u}{\partial y}
            \end{matrix} \right.
    \end{equation}
\end{itemize}
При этом
\begin{equation}\label{(3.7)}
    f'(z_0) = \frac{\partial u}{\partial x}(x_0,y_0) + i\frac{\partial v}{\partial x}(x_0,y_0) = \frac{\partial v}{\partial y}(x_0,y_0) - i\frac{\partial u}{\partial y}(x_0,y_0)
\end{equation}
\pr
Докажем в обе стороны.
\begin{itemize}
    \item Необходимость.
    \begin{align*}
      & \exists f'(z_0) = a+ib = A \in \CC
    \end{align*}
    Значит, по определению дифференцируемости
    \begin{align*}
      & \Delta f = A \Delta z + \alpha(\Delta z); \ \alpha(\Delta z) = \alpha_1(\Delta x, \Delta y) + \alpha_2(\Delta x, \Delta y)
    \end{align*}
    \begin{align*}
      & \left\{ \begin{matrix}
              \Delta u = a \Delta x - b \Delta y + \alpha_1(\Delta x, \Delta y) \\
              \Delta v = b \Delta x + a \Delta y + \alpha_2(\Delta x, \Delta y)
          \end{matrix} \right.
    \end{align*}
    \begin{align*}
      & \left| \alpha_1 \right| \leq \left| \alpha \right|, \ \left| \alpha_2 \right| \leq \left| \alpha \right|
    \end{align*}
    \begin{align*}
      & \lim_{(\Delta x; \Delta y) \to (0;0)}\left( \frac{\left| \alpha_1(\Delta x, \Delta y) \right|}{\sqrt{\left( \Delta x \right)^2 + \left( \Delta y \right)^2}} \right) \leq \lim_{\Delta z \to 0}\left( \frac{\left| \alpha(\Delta z) \right|}{\left| \Delta z \right|} \right) = 0
    \end{align*}
    Значит, $u$ дифференцируема, причем $\dst \frac{\partial u}{\partial x} =
    a$, $\dst \frac{\partial u}{\partial y} = -b$.
    \\
    Аналогично для $v$, причем $\dst \frac{\partial v}{\partial x} = b$, $\dst
    \frac{\partial v}{\partial y} = a$.
    \\
    Видим выполнимость УКР.
    \\
    Выполнимость \eqref{(3.7)} очевидна.
    \item Достаточность.
    \\
    Пусть $u$, $v$ дифференцируемы в $(x_0,y_0)$ и выполняется УКР. Тогда
    \begin{align*}
      & \Delta f = \Delta u + i \Delta v = \frac{\partial u}{\partial x} \Delta x + \frac{\partial u}{\partial y} \Delta y + \alpha_1(\Delta x, \Delta y) + i \left( \frac{\partial v}{\partial x} \Delta x + \frac{\partial v}{\partial y} \Delta y + \alpha_2(\Delta x, \Delta y)\right) = \\
      & \frac{\partial u}{\partial x} \Delta x - \frac{\partial v}{\partial x} \Delta y + \alpha_1(\Delta x, \Delta y) + i \left( \frac{\partial v}{\partial x} \Delta x + \frac{\partial u}{\partial x} \Delta y + \alpha_2(\Delta x, \Delta y)\right) = \left( \frac{\partial u}{\partial x} + i \frac{\partial v}{\partial x}\right)\cdot \\
      & \cdot \left( \Delta x + i \Delta y \right)+ \alpha_1(\Delta x, \Delta y) + i \alpha_2(\Delta x, \Delta y)
    \end{align*}
    Значит,
    \begin{align*}
      & \exists f'(z_0) = \frac{\partial u}{\partial x} + i \frac{\partial v}{\partial x}
    \end{align*}
\end{itemize}
\Example
\begin{align*}
  & w = z^2
\end{align*}
\begin{itemize}
    \item[I] Найдем производную по определению.
    \begin{align*}
      & \frac{\Delta w}{\Delta z} = \frac{\left( z_0-\Delta z \right)^2 - z_0^2}{\Delta z} = \frac{2z_0\Delta z + \left( \Delta z \right)^2}{\Delta z} = 2z_0 \Delta z \us{\Delta z \to 0}{\to} 2z_0
    \end{align*}
    \item[II] Найдем производную с использованием теоремы.
    \begin{align*}
      & z^2 = x^2-y^2+2ixy \Rightarrow u = x^2-y^2, \ v = 2xy
    \end{align*}
    \begin{align*}
      & u_x = 2x = v_y; \ u_y = -2y = - v_x
    \end{align*}
    Видим выполнимость УКР, поэтому
    \begin{align*}
      & w' = u_x+iv_x = 2x + 2iy = 2z
    \end{align*}
\end{itemize}
\Example
\begin{align*}
  & w = \left| z \right|^2
\end{align*}
\begin{align*}
  & \left| z \right|^2 = x^2+y^2 \Rightarrow u = x^2+y^2, \ v = 0
\end{align*}
\begin{align*}
  & u_x = 2x \neq 0 = v_y; \ u_y = 2y \neq 0 = - v_x
\end{align*}
(за исключением точки $(0;0)$). Видим выполнение УКР только в точке $(0;0)$, а
значит, и функция дифференцируема только в этой точке (с производной, равной
$0$).
\Example
\begin{align*}
  & w = \ol{z}
\end{align*}
\begin{align*}
  & \ol{z} = x - iy \Rightarrow u = x, \ v = -y
\end{align*}
\begin{align*}
  & u_x = 1 \neq -1 = v_y; \ u_y = 0 = - v_x
\end{align*}
Видим, что УКР не выполняется ни в одной точке $\CC$, а значит, функция нигде не
дифференцируема.
\prop
Если $f(z)$, $g(z)$ дифференцируемы, то:
\begin{itemize}
    \item дифференцируема и сумма:
    \begin{align*}
      & \left( f+g \right)' = f'+g'
    \end{align*}
    \item дифференцируемо и произведение:
    \begin{align*}
      & \left( fg \right)' = f'g+fg'
    \end{align*}
    \item дифференцируемо и частное (когда $g(z) \neq 0$):
    \begin{align*}
      & \left( \frac{f}{g} \right)' = \frac{f'g+fg'}{g^2}
    \end{align*}
\end{itemize}
\pr
Доказательства легко провести через УКР с использованием действительного
анализа.
\section{$\S 4.$ Регулярные и гармонические функции}
\example
$B_r(z_0)$~--- односвязная область как в $\CC$, так и в $\CCC$.
\example
$K = \left\{ z: 0 < r_1 < \left| z \right| < r_2 < \infty \right\}$~--- не
односвязная область ни в $\CC$, ни в $\CCC$.
\example
$B_r(\infty)$~--- односвязная область в $\CCC$, но не в $\CC$.
\Def
Если $u:G \mapsto \RR^1$, $G \subseteq \RR^2$~--- область, $u \in C^2(G)$,
$\Delta u = 0$ ($\Delta = \dst \frac{\partial^2}{\partial x^2} + \dst
\frac{\partial^2}{\partial y^2}$), то $u$ называется \textbf{гармонической}.
\Def
\begin{itemize}
    \item[а)]$f: G \mapsto \CC$, $G \subseteq \CC$ называется
    \textbf{регулярной} (аналитической, голоморфной), если $\forall z \in G
    \exists f'(z)$ (если $f$ дифференцируема в каждой точке $G$).
    \item[б)]$f: G \mapsto \CC$, $G \subseteq \CC$ называется
    \textbf{регулярной в точке $z_0 \in G$}, если $\exists r > 0: B_r(z_0)
    \subseteq G$, $f$ регулярна на $B_r(z_0)$.
\end{itemize}
\Def
Функция, заданная на произвольном множестве, \textbf{регулярна} на нем, если она
регулярна в каждой его точке.
\Example
\begin{align*}
  & w = z^n, \ n \geq 2, \ n \in \NN
\end{align*}
Производная существует в любой точке:
\begin{align*}
  & \frac{\left( z+\Delta z \right)^n - z^n}{\Delta z} = \frac{n\Delta z z^{n-1} + \left( \Delta z \right)^2\left( P(z, \Delta z) \right)}{\Delta z} \us{\Delta z \to 0}{\to} nz^{n-1}
\end{align*}
Значит, функция всюду дифференцируема и регулярна на всей $\CC$.
\\
Аналогично можно доказать регулярность всех многочленов и рациональных функций.
\Def
\begin{align*}
  & e^z = e^x\left( \cos y + i \sin y \right), \ z = x+iy
\end{align*}
Функция периодична с периодом $2 i \pi$; ее компоненты $u = e^x \cos y$, $v =
e^x \sin y$.
Выполняются УКР:
\begin{align*}
  & \left\{ \begin{matrix}
          u_x = e^x \cos y = v_y \\
          v_x = e^x \sin y = -u_y
      \end{matrix} \right.
\end{align*}
Значит, $f'(z) = u_x + iv_x = u+iv = e^z$ (аналогично $(e^{iz})' = ie^{iz}$,
$(e^{-iz})' = -ie^{iz}$).
\\
Эти функции всюду дифференцируемы, а значит, и регулярны.
\Def
Тригонометрические функции:
\begin{align*}
  & \cos z = \frac{e^{iz} + e^{-iz}}{2}, \ \sin z = \frac{e^{iz} -e^{-iz}}{2i}
\end{align*}
Для них выполняются известные свойства тригонометрических функций:
\begin{align*}
  & \cos^2 z + \sin^2 z = 1
\end{align*}
\begin{align*}
  & \sin(z_1+z_2) = \sin z_1 \cos z_2 + \sin z_2 \cos z_1, \ \cos(z_1+z_2) = \cos z_1 \cos z_2 - \sin z_1 \sin z_2
\end{align*}
\begin{align*}
  & \left( \sin z \right)' = \cos z, \ \left( \cos z \right)' = - \sin z
\end{align*}
Но ограниченность не выполняется.
\\
Раскроем функцию через $x$ и $y$.
\begin{align*}
  & \sin z = \sin(x+iy) = \sin x \cos iy + \cos x \sin iy = \sin x \ch y + i \cos x \sh y \\
  & \Downarrow \\
  & u(x,y) = \sin x \ch y, \ v = \cos x \sh y
\end{align*}
Тогда
\begin{align*}
  & \left| \sin z \right| = \sqrt{\sin^2 x \ch^2 yy + \cos^2 x \sh^2 y} = \sqrt{(1-\cos^2 x)\ch^2 y + (\ch^2 y - 1)\cos^2 x} = \\
  & = \sqrt{\ch^2 y - \cos^2 x} \us{y \to \infty}{\sim} \ch x \us{y \to \infty}{\to} \infty
\end{align*}
\theorem
Пусть $f: G \mapsto \CC$ регулярна в $G$, $f = u + iv$, $u,v \in C^2(G)$. Тогда
$u$ и $v$ являются гармоническими.
\pr
\begin{align*}
  & \Delta u = \frac{\partial}{\partial x}\left( \frac{\partial u}{\partial x} \right)+ \frac{\partial}{\partial y} \left( \frac{\partial u}{\partial y} \right) = \frac{\partial^2 v}{\partial x \partial y} - \frac{\partial^2 v}{\partial y \partial x} = 0
\end{align*}
Переход объясняется условием Коши-Римана для регулярной $f$.
\\
Аналогично можно доказать гармоничность $v$.
\\
Пара функций, связанных УКР, называется \textbf{сопряженными гармоническими}.
\theorem
Пусть $G$~--- односвязная область, $u: G \mapsto \RR$ гармоническая. Тогда
сущетвует регулярная $f$, такая, что $\Real f(z) = u(x,y)$.
\pr
Пусть
\begin{align*}
  & P(x,y) = -\frac{\partial u}{\partial y}, \ Q(x,y) = \frac{\partial u}{\partial x}
\end{align*}
Хотим найти $v$, такую, что
\begin{align*}
  & \left\{ \begin{matrix}
          \dst \frac{\partial v}{\partial x} = P(x,y) \\
          \dst \frac{\partial v}{\partial y} = Q(x,y) 
      \end{matrix} \right.
\end{align*}
(из УКР на $f$).
\begin{align*}
  & dv = Pdx+Qdy
\end{align*}
Для односвязной $G$ и потенциального векторного поля $(P,Q)$, т.`е. такого, что
выполняется условие:
\begin{align*}
  & \frac{\partial P}{\partial x} -\frac{\partial Q}{\partial x}' = - \frac{\partial^2 u}{\partial y^2} - \frac{\partial^2 u}{\partial x^2} = 0
\end{align*}
В таком случае
\begin{align*}
  & v(x,y) = \int_{(x_0,y_0)}^{(x,y)} P dx + Q dy + const
\end{align*}
Полученая функция $v$ гармоническая. тогда $f(z) = u+iv$ будет регулярной в
каждой точке $G$.
\\
Аналогично можно построить регулярную функцию $f$ по данной гармонической $v$.
\Example
Пусть
\begin{align*}
  & u = xy, \ \Delta u = 0
\end{align*}
Значит, $u$ гармоническая. УКР:
\begin{align*}
  & v_x = -u_y = -x \Rightarrow v(x,y) = \frac{-x^2}{2} + \varphi(y)
\end{align*}
\begin{align*}
  & v_y = u_x = y \Rightarrow v(x,y) = \frac{-x^2}{2} + \frac{y^2}{2} + C
\end{align*}
Значит,
\begin{align*}
  & \varphi'(y) = y = v_y
\end{align*}
Тогда
\begin{align*}
  & v(x,y) = \frac{y^2-x^2}{2} + C
\end{align*}
Значит,
\begin{align*}
  & f(x,y)= xy+i\left( \frac{y^2-x^2}{2} + C \right) = -\frac{i}{2} \left( x^2-y^2+2ixy \right) + iC = -\frac{iz}{2} + iC
\end{align*}
