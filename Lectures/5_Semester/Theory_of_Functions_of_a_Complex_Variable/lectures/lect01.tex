\begin{flushright}
    \textit{Лекция 1 (от 07.09)}
\end{flushright}
\section{$\S 1.$ Комплексные числа}
\Def
Пусть $z = \left( x;y \right) \in \RR^2$. Пусть определены операции:
\begin{enumerate}
    \item \textbf{Сложение:} $z_1 + z_2 = \left( x_1+x_2; y_1+y_2 \right)$
    \item \textbf{Умножение на действительное число:} $\lambda z = \left(\lambda
        x, \lambda y \right)$
    \item \textbf{Расстояние:} $\rho(z_1, z_2) = \left| \left| z_1 - z_2 \right|
    \right| = \sqrt{\left( x_1 - x_2 \right)^2 + (y_1 - y_2)^2}$
\end{enumerate}
Добавим операцию
умножения друг на друга:
\begin{equation}
  z_1 \cdot z_2 = \left( x_1 \cdot x_2 - y_1 \cdot y_2; x_1 \cdot y_2 + x_2 \cdot y_1 \right)
\end{equation}
Будем называть это \textbf{комплексными числами} $\CC$.

Пусть $1 \sim (1; 0)$~--- \textbf{единица}, $i \sim (0;1)$~--- \textbf{мнимая
  единица}.

Тогда $z = 1 \cdot x + i \cdot y \Leftrightarrow z = x + iy$~---
\textbf{алгебраическая запись}.
\section{$\S 2.$ Последовательность. Предел. Ряды. Функции. Расширенная
  комплексная плоскость}
