\begin{flushright}
    \textit{Лекция 16 (от 27.10)}
\end{flushright}
\theorem (Руше)
Пусть $G$~--- односвязная область, $\ogamma$~--- замкнутая положительно
ориентированная кусочно гладкая кривая в этой области. Пусть $f,g: G \mapsto
\CC$ регулярны, и
\begin{equation}\label{(20.3)}
    \forall z \in \ogamma \ \left| f(z) \right| > \left| g(z) \right|
\end{equation}
Тогда $f(z)$ и $h(z) = f(z)+g(z)$ имеют внутри $\ogamma$ одинаковое число нулей
с учетом их порядка.
\pr
\begin{align*}
  & \forall z \in \ogamma \ \left| f(z) \right| > \left| g(z) \right| \Rightarrow \forall z \in \ogamma \ f(z) \neq 0
\end{align*}
\begin{align*}
  & \forall z \in \ogamma \ \left| h(z) \right|\geq \left| f(z) \right| - \left| g(z) \right| > 0 \Rightarrow \forall z \in \ogamma h(z) \neq 0
\end{align*}
\begin{align*}
  & \Delta_{\ogamma}\argt h(z) = \Delta_{\ogamma} \argt \left( f(z)\left( 1+\frac{g(z)}{f(z)} \right) \right) = \Delta_{\ogamma}\argt f(z) +\Delta_{\ogamma}\argt \left( 1+\frac{g(z)}{f(z)} \right)
\end{align*}
\begin{align*}
  & \ogamma: w = 1 + \frac{g(z)}{f(z)} \Rightarrow \left| w-1 \right| = \left| \frac{g(z)}{f(z)} \right| < 1
\end{align*}
Пусть $\Gamma = w(\ogamma) \subseteq B_1(1)$; $0 \not \in B_1(1)$ и эта область
односвязна, значит, по теореме $3$ $\S 14$
\begin{align*}
  & \Delta_{\ogamma}\argt \left( 1+\frac{g(z)}{f(z)} \right) = 0
\end{align*}
и тогда
\begin{align*}
  & N_h = \frac{1}{2\pi}\Delta_{\ogamma}\argt h(z) = \frac{1}{2\pi}\Delta_{\ogamma}\argt f(z) = N_f
\end{align*}
\theorem (Гаусса)
Многочлен
\begin{align*}
  & P_n(z) = c_0+zc_1+z^2c_2+\dots+z^nc_n
\end{align*}
имеет в $\CC$ ровно $n$ корней с учетом их порядка.
\pr
Рассмотрим $f(z) = c_nz^n$, $g(z) = P_n(z) - f(z)$. Как известно,
\begin{align*}
  & \left| \frac{g(z)}{f(z)} \right| \us{z \to \infty}{\to} 0 \Rightarrow \exists R_0 > 0: \ \forall R \geq R_0 \ \forall z \in \gamma_R \ \left| f(z) \right| > \left| g(z) \right|
\end{align*}
Значит, по теореме Руше функция $P_n(z)$ имеет столько же нулей, сколько и $f(z)
= c_nz^n$, на $B_R(0)$, с учетом порядка, т.~е. ровно $n$ штук.
\Example
Функция Жуковского
\begin{align*}
  & w = \frac{1}{2}\left( z+\frac{1}{z} \right)
\end{align*}
У функции $\pm i$~--- нули, $0$~--- полюс первого порядка.
\\
Рассматривая $R>1$, получим
\begin{align*}
  & \Delta_{\gamma_R}\argt w(z) = 2 \pi (N-P) = 2\pi
\end{align*}
\begin{align*}
  & \Delta_{\gamma_{\frac{1}{R}}}\argt w(z) = 2 \pi (N-P) = -2\pi
\end{align*}
\section{$\S 21.$ Геометрические принципы.}
\lemma (об открытости)
Пусть $f$ регулярна в $G$, $z_0 \in G$, $w_0 = f(z_0)$.
\\
Пусть при $n \geq 2$
\begin{equation}\label{(20.1)}
    f'(z_0) = \dots = f^{(n-1)}(z_0) = 0, \ f^{(n)}(z_0) \neq 0
\end{equation}
Тогда $\exists B_\delta(z_0)$ и $B_\varepsilon(w_0)$, такие, что $\forall w_1
\in B_{\varepsilon}(w_0)$ уравнение $f(z) = w_1$ имеет в круге $B_\delta(z_0)$
ровно $n$ решений.
\pr
Заметим, что $f(z) \neq const$, $f'(z) \neq const$. Тогда по теореме
единственности нули функции $f(z)-w_0$ и $f'(z)$ изолированы, поэтому
\begin{align*}
  & \exists \delta > 0: \ \forall z \in \ol{\os{\circ}B_\delta(z_0)}\setminus\{z_0\} \ f(z)-w_0 \neq 0, \ f'(z) \neq 0
\end{align*}
Пусть $\gamma = \left\{ z: \left| z-z_0 \right| = \delta \right\}$ положительно
ориентирована, и $\forall z \in \gamma \ f(z)\neq w_0$. Положим $0 < \varepsilon
= \inf \left\{ w_0 - f(z) \mid z \in \gamma \right\}$, $\Gamma = f(\gamma)$.
заметим, что $\Gamma$ замкнута и $w_0 \not \in \Gamma$.
\\
Тогда
\begin{align*}
& \forall w_1 \in \os{\circ}{B}_\varepsilon(w_0), \ \forall z \in \gamma \ \left| w_1-w_0 \right| < \varepsilon \leq \left| w_0 - f(z) \right|
\end{align*}
Пусть $F(z) = w_0 - f(z)$, $G(z) = w_1-w_0$. Тогда $H(z) = F(z)+G(z) =
w_1-f(z)$; значит, по теореме Руше (в силу $\left| F(z) \right|> \left| G(z)
\right|$) функции $w_0 - f(z)$ и $w_1-f(z)$ имеют одинаковое число нулей внутри
$\gamma$. 
\\
Заметим, что $f(z) = w_0$ имеет единственный нуль~--- $z_0$~--- порядка $n$.
Значит, с учетом порядка $f(z) = w_1$ имеет $n$ решений.
\\
Но 
\begin{align*}
& \forall z \in B_\delta(z_0) \ f'(z) = (f(z)-w_1)' \neq 0
\end{align*}
Значит, все нули будут иметь первый порядок, соответственно, их ровно $n$ штук.
\corollary
Пусть $f$ регулярна в $G \subseteq \CC$. Тогда $\forall z_0 \in G$ условие
$f'(z_0) \neq 0$ необходимо и достаточно для однолистности $f$ <<в малом>>~--- в
некоторой достаточно малой окрестности $z_0$, но не во всей $G$.
\Example
$w=e^z$ удовлетворяет условиям следствия, но не однолистна во всей комплексной
плоскости.
\theorem (принцип сохранения области)
Пусть $f$ регулярна в области $G$, причем $f(z) \neq const$. Тогда $f(G)$~---
также область.
\pr
Докажем открытость.
\begin{align*}
& \forall z_0 \in G \ w_0 = f(z_0) \in G^* = f(G)
\end{align*}
Поскольку $G$~--- область, 
\begin{align*}
& \exists \delta_1: \ B_{\delta_1}(z_0) \subseteq G
\end{align*}
и по лемме $21.1$
\begin{align*}
& \exists \delta \in (0; \delta_1], \ \exists \varepsilon > 0: \ \forall w_1 \in B_\varepsilon (w_0) \ \exists z_1 \in B_\delta(z_0): \ f(z_1) = w_1; \ f(B_\delta(z_0)) \supseteq B_\varepsilon(w_0)
\end{align*}
Значит, $B_\varepsilon(w_0)\subseteq G^*$, т.~е. $w_0$~--- внутренняя точка $G^*$.
\\
Докажем связность.
\begin{align*}
  & \forall w_1, w_2 \in G^* \ \exists z_1,z_2 \in G: \ \exists \gamma_{z_1z_2} \subseteq G, \ \Gamma_{w_1w_2} = f(\gamma_{z_1z_2}) \subseteq G^*
\end{align*}
\theorem (принцип максимума модуля регулярной функции)
Пусть $f: G \mapsto \CC$ регулярна в ограниченной области $G$, непрерывна на ее
замыкании и непостоянна. Тогда
\begin{align*}
  & \max_{z \in \ol{G}} \left| f(z) \right| = \max_{z \in \partial G} \left| f(z) \right|
\end{align*}
\pr
Путь, от противного,
\begin{align*}
  & \exists z_0 \in G: \ \left| f(z_0) \right| = \max_{z \in \ol{G}}\left| f(z) \right|
\end{align*}
Пусть $w_0 = f(z_0)$. По теореме $21.1$
\begin{align*}
  & \exists \varepsilon > 0, \ B_\varepsilon(w_0) \subseteq f(G) = G^*
\end{align*}
\begin{align*}
  & w_1 \in B_\varepsilon(w_0): \left| w_1 \right| > \left| w_0 \right|; \ w_1 = w_0\left( 1+\frac{\varepsilon}{2\left| w_0 \right|} \right)
\end{align*}
\begin{align*}
  & \exists z_1 \in G: \ f(z_1) = w_1; \ \left| f(z_1) \right| > \left| f(z_0) \right|
\end{align*}
Противоречие.
\corollary (принцип минимума модуля регулярной функции)
Пусть $f: G \mapsto \CC$ регулярна в ограниченной области $G$, непрерывна на ее
замыкании и непостоянна; $\forall z \in G \ f(z) \neq 0$. Тогда
\begin{align*}
  & \min_{z \in \ol{G}} \left| f(z) \right| = \min_{z \in \partial G} \left| f(z) \right|
\end{align*}
\Note
Для случая неограниченной $G$ вместо $\min$ и $\max$ используется $\inf$ и
$\sup$.
\lemma (Шварца)
Пусть $f: B_1(0) \mapsto \CC$ регулярна, $\forall z \in B_1(0) \ \left| f(z)
\right| \leq 1$, $f(0) = 0$. Тогда выполняется
\begin{equation}\label{(21.2)}
    \forall z \in B_1(0) \ \left| f(z) \right| \leq z
\end{equation}
Если в \eqref{(21.2)} достигается равенство при $z_0 \neq 0$, то
\begin{align*}
  & \exists \alpha \in \RR: \ \forall z \in \ol{B_1(0)} \ f(z) = e^{i\alpha}z
\end{align*}
\pr
$f(0) = 0$, а значит, существует регулярная в $B_1(0)$ функция $g(z)$: $f(z) =
zg(z)$. Функция $g(z) = \dst \frac{f(z)}{z}$, но при этом регулярна также и в
нуле.
\\
Рассмотрим произвольное $r \in (0;1)$ и $z: \ \left| z \right|<r$. По теореме
$2$
\begin{align*}
  & \left| g(z) \right| \leq \max \left\{ \left| \frac{f(\zeta)}{\zeta} \right| : \left| \zeta \right| =r \right\} \leq \frac{1}{r}
\end{align*}
Пусть $z_0 \in B_1(0)$; $\forall r \in (\left| z_0 \right|, 1)$ $\left| g(z_0)
\right| \leq \dst \frac{1}{r}$, а значит, $\left| g(z_0) \right|\leq 1$, и
$\forall z \in B_1(0) \ \left| f(z) \right|\leq \left| z \right|$.
\\
Пусть равенство в \eqref{(20.2)} достигается в некоторой точке $z_1$ (т.~е.
$\left| g(z) \right| = 1$). Поскольку точка лежит внутри области, то либо
возникает противоречие с принципом максимума, либо $g(z) = const = e^{i
  \alpha}$.
\theorem (принцип максимума и минимума гармонической функции)
Пусть $u: G \mapsto \CC$ гармоническая на $G \subseteq \RR^2$ и непостоянная,
непрерывная на $\ol{G}$. Тогда $\sup$ и $\inf$ функции $u$ на $G$ и $\ol{G}$
совпадают.
\pr
Допустим, $\exists z_0 = (x_0, y_0)\in G$, на которой $u$ достигает максимума.
Тогда $\exists \varepsilon > 0: \ B_\varepsilon(z_0)\subseteq G$; по теореме $2$
$\S 4$ сущствует регулярная $f$ в $B_\varepsilon(z_0)$ такая, что $\Real f(z) =
u(z)$.
\\
$w_0 = f(z_0)$; по теореме $21.1$ $f(B_\varepsilon(z_0))$~--- область, т.~е.
$\exists r > 0$: $B_r(w_0) \subseteq f(B_{\varepsilon}(z_0))$. Пусть $w_1 \in
B_r(w_0)$, $\Real w_1 > \Real w_0$.
\\
Значит,
\begin{align*}
  \exists z_1 \in B_\varepsilon(z_0): \ w_1 = f(z_1), \ \Real f(z_1) > \Real f(z_0) \Rightarrow u(z_1)>u(z_0)
\end{align*}
Противоречие.
\\
В силу $\inf u(z) = - \sup(-u(z))$, гармоничности $-u(z)$ и выполнимости
принципа максимума выполняется и принцип миминума.
\theorem (о среднем для гармонической функции)
Пусть $u: B_R(z) \mapsto \RR$ гармоническая и непрерывная на замыкании
этого круга, непостоянная. Тогда
\begin{equation}\label{(21.3)}
    u(a) = \frac{1}{2\pi}\int_0^{2\pi}u(a+Re^{i\varphi})d\varphi
\end{equation}
\pr
По теореме $2$ $\S 4$ существует $f: B_R(a) \mapsto \CC$~--- регулярная, причем
$\forall z \in B_R(a) \ \Real f(z) = u(z)$, $0<\rho < R$, $\gamma_\rho =
\left\{ z: \left| z-a \right| = \rho\right\}$.
\begin{align*}
  & f(a) = \frac{1}{2 i \pi}\int_{\gamma_\rho}\frac{f(\zeta)}{\zeta - a}d\zeta = \frac{1}{2i\pi}\int_0^{2\pi}\frac{f(a+\rho e^{i\varphi})}{\rho e^{i\varphi}} i \rho e^{i\varphi} d \varphi = \frac{1}{2\pi}\int_0^{2\pi}f(a+\rho e^{i\varphi})d \varphi
\end{align*}
\begin{align*}
  & \Real f(a) = u(a) = \frac{1}{2\pi}\int_0^{2\pi}\Real f(a+\rho e^{i\varphi})d \varphi = \frac{1}{2\pi}\int_0^{2\pi}u (a+\rho e^{i\varphi})d \varphi
\end{align*}
Устремляя $\rho \to R$, получаем \eqref{(21.3)}.
