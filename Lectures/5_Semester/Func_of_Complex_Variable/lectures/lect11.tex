\begin{flushright}
    \textit{Лекция 11 (от 12.10)}
\end{flushright}
\Def
Пусть в области $G$ задано семейство гладких кривых $\left\{ \gamma_\alpha
\right\}$, $\alpha \in [a,b]$, заданных при помощи параметра $t$: $z(t,
\alpha)$, $t \in [0;1]$. Если $z(t, \alpha)$ и $z'_t(t, \alpha)$ непрерывны на
$[0;1] \times [a,b]$, то говорят, что в области $G$ $\gamma_{\alpha}$
\textbf{задает непрерывную деформацию кривой $\gamma_a$ в кривую $\gamma_b$}.
\\
Если $\forall \alpha \in [a;b] \ z(0, \alpha) = z_0, \ z(1, \alpha) = z_1$, то
это называют \textbf{гомотетией $\gamma_a$ в $\gamma_b$}.
\theorem
Пусть в $\CC \setminus \{0\}$ задана непрерывная деформация $\gamma_a$ в
$\gamma_b$ через семейство $\left\{ \gamma_\alpha \right\}_{\alpha \in [a;b]}$.
Пусть $\exists A \in \CC, \ A \neq 0$ такое, что $\forall \alpha \in [a;b] \
z(1,\alpha) = Az(0, \alpha)$ и $I(\alpha) = \Delta_{\gamma_\alpha}\argt z =
\Delta_{[0;1]}\argt z(t, \alpha)$. Тогда $\forall \alpha \in [a,b] \ I(\alpha) =
const$. В частности, $\Delta_{\gamma_a}\argt z = \Delta_{\gamma_b}\argt z$.
\pr
По теореме $14.1$ $\forall \alpha \in [a;b] \ \exists \varphi(t, \alpha)$,
непрерывная по $t \in [0;1]$. $\varphi(t, \alpha) \in \Arg z (t, \alpha)$.
\begin{align*}
  & I(\alpha) = \Delta_{\gamma_\alpha}\argt z = \varphi(1, \alpha) - \varphi(0, \alpha) \in \Arg \frac{z(1, \alpha)}{z(0, \alpha)}
\end{align*}
\begin{align*}
  & \frac{z(1, \alpha)}{z(0, \alpha)} = A \neq 0 \Rightarrow \forall \psi_0 \in \Arg A \hookrightarrow I(\alpha) = \psi_0+2\pi k(\alpha), \ k(\alpha) \in \ZZ
\end{align*}
Значит, $I(\alpha)$ ступенчатая; но по определению приращения аргумента через
интеграл она непрерывна. Значит, она постоянна.
\\
$\forall z(t) \in C[0;1]$ таких, что $\forall t \in [0;1] \ z(t) \neq 0$ и $r =
\dst \min_{t \in [0;1]} \left| z(t) \right|$ положим $\varepsilon\in \left( 0;
    \dst \frac{r}{2} \right)$. Тогда существует \textbf{гладкая
  $\varepsilon$-аппроксимация $z(t)$}: $\forall t \in [0;1] \ z_\varepsilon(t)
\neq 0$, $\left| z(t) - z_\varepsilon(t) \right| \leq \varepsilon$,
$z_\varepsilon(0) = z(0)$, $z_\varepsilon(1) = z(1)$.
\\
Докажем это, построив пример. Действительно, по теореме Вейерштрасса
\begin{align*}
  & \exists P_n(t): \ \forall t \in [0;1] \left| z(t) - P_n(t) \right| < \frac{\varepsilon}{2}
\end{align*}
Положим
\begin{align*}
  & z_\varepsilon(t) = P_n(t) + \left( z(0)-P_n(0) \right)\left( 1-t \right) + \left( z(1)-P_n(1) \right)t
\end{align*}
Для такой функции выполняется $z(0) = z_\varepsilon(0)$, $z(1) =
z_\varepsilon(1)$; также
\begin{align*}
  & \abs{z(t) - z_\varepsilon(t)} \leq \abs{z(t) - P_n(t)} + \left| z(0)-P_n(0) \right| \left( 1-t \right) + \left| z(1)-P_n(1) \right|t < \frac{\varepsilon}{2} + \frac{\varepsilon}{2}(1-t) + \frac{\varepsilon}{2}(t) = \varepsilon
\end{align*}
и, наконец,
\begin{align*}
  & \abs{z_\varepsilon(t)} \geq \left| z(t) \right| - \left| z(t) - z_\varepsilon(t) \right| \geq r - \varepsilon \geq \frac{r}{2} > 0
\end{align*}
\Def
\textbf{Приращением аргумента непрерывной функции $z(t)$ вдоль непрерывной
  кривой $\gamma$ на $[0;1]$} называется $\Delta_{[0,1]}\argt z(t) =
\Delta_{[0,1]}\argt z_\varepsilon(t)$ для любой $\varepsilon$-аппроксимации
$z(t)$ при малых $\varepsilon$ ($\varepsilon \in \left( 0, \dst \frac{r}{2}
\right)$).
\prop
Определение $[14.4]$ не зависит от выбора $z_\varepsilon$.
\pr
Пусть $z_\varepsilon$~--- гладкая $\varepsilon$-аппроксимация,
$\tilde{z}_\varepsilon$~--- другая гладкая $\varepsilon$-аппорксимация.
Тогда
\begin{align*}
  & z(t, \alpha) = \alpha z_\varepsilon(t) + (1-\alpha)\tilde{z}_\varepsilon(t), \ \alpha \in [0;1], \ t \in [0;1]
\end{align*}
\begin{align*}
  & \abs{z(t, \alpha)} \geq \left| z(t) \right| - \left| z(t, \alpha) - z(t) \right| > r - \varepsilon \geq \frac{r}{2} > 0
\end{align*}
Значит, задана непрерывная деформация, и по теореме $14.3$ получаем, что
\begin{align*}
  & \Delta_{[0;1]}\argt z_\varepsilon(t) = \Delta_{[0;1]}\argt \tilde{z}_\varepsilon(t)
\end{align*}
\corollary
Пусть $\os{\circ}{\gamma}$~--- замкнутая непрерывная кривая, $0 \not \in
\os{\circ}{\gamma}$; тогда $\exists k(\os{\circ}{\gamma})\in \ZZ$:
$\Delta_{\os{\circ}{\gamma}}\argt z = 2 \pi k(\os{\circ}{\gamma})$.
\section{$\S 15.$ Регулярные ветви многозначных функций $\{\sqrt[n]{z}\}$ и $\Ln
  z$.}
Определим на $\CC\setminus\{0\}$:
\begin{equation}\label{(15.1)}
    \left\{ \sqrt[n]{z} \right\} = \sqrt[n]{\left| z \right|}\exp\left( \frac{i}{n} \Arg z \right)
\end{equation}
\begin{equation}\label{(15.2)}
    \left\{ \Ln z \right\} = \ln\left| z \right| + i \Arg z
\end{equation}
\begin{equation}\label{(15.3)}
    \Arg z= \left\{ \argt z + 2 \pi k \mid k \in \ZZ \right\}
\end{equation}
Из теоремы $2 \ \S 5$ (следствия $1$) можем видеть, что на $\CC \setminus \left(
    -\infty; 0 \right]$ существуют регулярные ветви:
\begin{equation}\label{(15.4)}
    \begin{split}
        & g_0(z) = \sqrt[n]{\left| z \right|}\exp\left( \frac{i}{n} \argm z \right) \\
        & h_0(z) = \ln\left| z \right| + i \argm z \\
        & \argm z \in \left( -\pi; \pi \right)
    \end{split}
\end{equation}
Это \textbf{главные регулярные ветви $\left\{\sqrt[n]{z}\right\}$ и $\Ln z$}.
\\
Рассмотрим \textbf{простейший случай}.
\\
Фиксируем $\varphi_0 \in \left( -\pi; \pi \right)$, $n \geq 2$.
\\
Пусть область
\begin{align*}
  & G_{1, \varphi_0} = \left\{ z \neq 0 \mid z = re^{i\varphi}, \ r > 0, \ \varphi \in \left( \frac{\varphi_0}{n}; \frac{\varphi_0 + 2\pi}{n} \right) \right\}
\end{align*}
$w = z^n$ однолистна на $G_{1, \varphi_0} \mapsto \CC \setminus
\lambda_{\varphi_0}$, гле $\lambda_{\varphi_0} = \left\{ z \mid z = re^{i
      \varphi_0}, \ r> 0\right\}\cup \{0\}$
\\
По следствию $1$ $\S 5$
\begin{equation}\label{(15.5)}
    \begin{split}
        & g_*(z) = \sqrt[n]{\left| z \right|}\exp\left( \frac{i}{n} \arg_* z \right) \\
        & \arg_* z \in \left( \varphi_0; \varphi_0+2\pi \right)
    \end{split}
\end{equation}
дает регулярную ветвь.
Пусть область
\begin{align*}
  & G_{2, \varphi_0} = \left\{ z \mid \Img z \in \left( \varphi_0; \varphi_0 + 2\pi \right) \right\}
\end{align*}
$w = e^z$ однолистна на $G_{2, \varphi_0}$, и
\begin{equation}\label{(15.6)}
    \begin{split}
        & h_*(z) = \ln\left| z \right| + i \arg_* z \\
        & \arg_* z \in \left( \varphi_0; \varphi_0+2\pi \right)
    \end{split}
\end{equation}
дает регулярную ветвь.
\lemma
Пусть $G$~--- область в $\CC \setminus \lambda_{\varphi_0}$. Тогда все
непрерывные ветви $\left\{ \sqrt[n]{z} \right\}$ и $\Ln z$ в $G$ являются
регулярными и удовлетворяют соотношениям:
\begin{equation}\label{(15.7)}
    g_k(z) = g_*(z) \exp\left( \frac{i}{n} 2 \pi k\right), \ k \in \{0, \dots, n-1\}
\end{equation}
\begin{equation}\label{(15.8)}
    h_k(z) = h_*(z) + 2i \pi k, \ k \in \ZZ
\end{equation}
\pr
Пусть $g(z)$~--- непрерывная ветвь $\left\{\sqrt[n]{z}\right\}$ в $G$; $g^n(z) \equiv
    z$, $g_*^n(z) \equiv z$. Деля,
    \begin{align*}
      & \left( \frac{g^n(z)}{g_*^n(z)} \right)^n = 1 \Rightarrow \frac{g(z)}{g_*(z)} \in \left\{ \sqrt[n]{1} \right\} = \exp \left( \frac{i}{n} 2 \pi k(z) \right), \ k(z) \in \left\{ 0, \dots, n-1 \right\}
    \end{align*}
В левой части функция непрерывная, в правой~--- ступенчатая, а значит, $k(z) =
k_* = const$, откуда видно, что $g(z)$ удовлетворяет \eqref{(15.7)}.
\\
Пусть $h(z)$~--- непрерывная ветвь $\Ln z$ в $G$; $e^{h(z)} \equiv z$,
$e^{h_*(z)} \equiv z$. Деля,
\begin{align*}
  & e^{h(z)-h_*(z)} = 1 \Rightarrow h(z) - h_*(z) = 2i\pi k(z)
\end{align*}
Значит, $k(z) = k_* = const$, откуда видно, что $h(z)$ удовлетворяет
\eqref{(15.8)}.
\lemma
В любом кольце $K_{r,R}(0) = \left\{ z \mid r < \left| z \right| < R \leq
    +\infty \right\}$ нет непрерывных ветвей $\left\{ \sqrt[n]{z} \right\}$ или
$\Ln z$.
\pr
Предположим противное: $\exists K_{r, R}(0)$: $\left\{ \sqrt[n]{z} \right\}$
имеет в нем непрерывную ветвь $g(z)$. 
\\
Тогда рассмотрим область $G = K \setminus \left( -\infty;0 \right]\subseteq \CC
\setminus \left( -\infty; 0 \right]$. Эта область удовлетворяет условиям леммы
$1$, но по этой лемме $\exists k_* \in \left\{ 0, \dots, n-1 \right\}$: $g(z) =
g_0(z)\exp \left( \dst \frac{i}{n} 2 \pi k_* \right)$, $z \in G$. Заметим, что
$\forall x \in \left( -R;-r \right)$
\begin{align*}
& g(x+i0) = \sqrt[n]{\left| z \right|}\exp \left( \frac{i \pi}{n} + \frac{2 i \pi k_*}{n}\right) \\
& g(x-i0) = \sqrt[n]{\left| z \right|}\exp \left( \frac{-i \pi}{n} + \frac{2 i \pi k_*}{n}\right) \\
\end{align*}
Они не равны, противоречие с предположением о непрерывности.
\\
Аналогично можно провести доказательство для $\Ln z$.
\corollary
Если $0 \in G$ или $\infty \in G$, то в $G$ не существует непрерывной ветви
$\left\{ \sqrt[n]{z} \right\}$ или $\Ln z$.
\\
Рассмотрим \textbf{общий случай}.
\\
Пусть $G \subseteq \CC$~--- односвязая область, $0 \not \in G$.
\lemma
В односвязной области $G$, $0 \not \in G$ существуют непрерывные ветви $\Arg z$.
\pr
Фиксируем $z_0 \in G$, $\psi_0 \in \Arg z_0$.
\\
Пусть $\gamma_{z_0z}$~--- гладкая кривая из $z_0$ в $z$, $ \gamma \subseteq G$.
Пусть
\begin{align*}
  & \varphi_{\gamma_{z_0z}} = \psi_0 + \Delta_{\gamma_{z_0z}}\argt z
\end{align*}
\begin{align*}
  & \varphi_{\gamma_{z_0z}} \in \Arg z
\end{align*}
\begin{align*}
  & \Delta_{\gamma_{z_0z}}\argt z = \Img \int_{\gamma_{z_0z}}\frac{d \zeta}{\zeta}
\end{align*}
Знаем, что $\dst \frac{1}{z}$ регулярна в $G$, а значит, по теореме Коши
\begin{align*}
  & \forall \os{\circ}{\gamma} \subseteq G \ \int_{\os{\circ}{\gamma}}\frac{d\zeta}{\zeta} = 0
\end{align*}
Тогда
\begin{align*}
  & \Phi(z) = \int_{\gamma_{z_0z}}\frac{d \zeta}{\zeta}
\end{align*}
регулярна в $G$;
\begin{equation}\label{(15.9)}
    \varphi_{\gamma_{z_0z}} = \varphi_0(z) = \psi_0 + \Delta_{\gamma_{z_0z}}\argt z \in \Arg z
\end{equation}
будет непрерывной, и, зная это, легко построить
\begin{equation}\label{(15.10)}
    \tilde{g}(z) = \sqrt[n]{\left| z \right|} \exp\left( \frac{i}{n} \varphi_0(z) \right)
\end{equation}
\begin{equation}\label{(15.11)}
    \tilde{h}(z) = \ln \left| z \right| + i \varphi_0(z)
\end{equation}
Это непрерывные ветви $\left\{ \sqrt[n]{z} \right\}$ и $\Ln z$ соответственно в
$G$.
\theorem
В односвязной $G$, $0 \not \in G$ существуют непрерывные ветви $\left\{
    \sqrt[n]{z} \right\}$ и $\Ln z$, являющиея регулярными ветвями и
удовлетворяюшщие соотношениям:
\begin{equation}\label{(15.12)}
    \tilde{g}_k(z) = \tilde{g}(z)\exp\left( \frac{2i\pi k}{n} \right), \ k \in \left\{ 0, \dots, n-1 \right\}
\end{equation}
\begin{equation}\label{(15.13)}
    \tilde{h}_k(z) = \tilde{h}(z) + 2 i \pi k, \ k \in \ZZ
\end{equation}
\pr
Всякая ветвь такова по лемме $1$. Докажем регулярность $\tilde{g}(z)$.
\\
Рассмотрим $z_1 \in G$: $\exists B_r(z_1) \subseteq G$, $0 \not \in B_r(z_1)$
$\Rightarrow$ $\exists \varphi_0 \in [-\pi, \pi): \ B_r(z_1) \subseteq \CC
\setminus \lambda_{\varphi_0}$.
\\
Значит, по лемме $1$ $\tilde{g}(z)$ регулярна в $B_r(z_1)$, а значит, и в $z_1$;
в силу произвольности выбора $z_1$ можем сказать, что функция регулярна в $G$.
\\
Аналогично можем провести доказательство для $\tilde{h}(z)$.
