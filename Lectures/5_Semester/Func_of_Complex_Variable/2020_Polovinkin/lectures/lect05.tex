\begin{flushright}
    \textit{Лекция 5 (от 21.09)}
\end{flushright}
% \begin{center}
\section{$\S 7.$ Интегральная теорема Коши}
% \end{center}
\theorem (Коши)
Пусть $G$~--- односвязная область, пусть $f: G \mapsto \CC$
регулярна. Тогда для любой простой замкнутой кусочно гладкой кривой
$\overset{\circ}{\gamma} \subseteq G$ выполняется
\begin{equation} \label{(7.1)}
  \int_{\overset{\circ}{\gamma}}f(z)dz = 0
\end{equation}
\lemma (Гурса)
Пусть $G$~--- область, $f: G \mapsto \CC$ регулярна. Тогда для
любого треугольника из $G$ верно
\begin{equation} \label{(7.2)}
  \int_{\partial \triangle} f(z) dz = 0
\end{equation}
\pr
\begin{align*}
  & \triangle ABC \subseteq G
\end{align*}
\begin{align*}
  & I = \int_{\partial \triangle ABC}f(z)dz
\end{align*}
Разобьем треугольник средними линиями:
\begin{align*}
  & \triangle ABC = \bigcup_{k=1}^{4}\triangle_k
\end{align*}
Тогда
\begin{align*}
  & I = \sum_{k=1}^4\int_{\partial \triangle_k}f(z)dz
\end{align*}
Докажем, что
\begin{align*}
  & \exists k_0: \left| \int_{\partial \triangle_{k_0}} f(z)dz\right|\geq\frac{\left| I \right|}{4}
\end{align*}
Очевидно от противного: т.~к. триангуляция с ориентацией, то если бы все были
меньше, то нельзя было бы набрать $I$.
\\
Назовем этот треугольник $\triangle^1$, а $\triangle ABC = \triangle^0$.
Аналогично построению $\triangle^1$ из $\triangle^0$ можем построить
бесконечную последовательность $\triangle^{N+1}$ из $\triangle^N$, и для них
\begin{align*}
  & \left| \int_{\partial \triangle^N} f(z)dz\right|\geq\frac{\left| I \right|}{4^N}
\end{align*}
\begin{align*}
  & P_N = \frac{P_0}{2^N}
\end{align*}
где $P_N$~--- периметр $N$-го треугольника.
В силу компактности
\begin{align*}
  & \exists z_0 \in \bigcap_{N=1}^{\infty}\triangle^N
\end{align*}
Т.~к. $f$ дифференцируема в $z_0$, то по определению дифференцируемости
\begin{align*}
  & \exists B_{\delta_0}(z_0): \ \forall z \in B_{\delta_0}(z_0) \ f(z) = f(z_0)+f'(z_0)(z-z_0) + o(z-z_0)
\end{align*}
\begin{align*}
  & o(z-z_0): \ \forall \varepsilon > 0 \ \exists \delta_1 \leq \delta_0: \ \forall z \in B_{\delta_1}(z_0) \ \left| o(z-z_0) \right| \leq \varepsilon\left| z-z_0 \right|
\end{align*}
\begin{align*}
  & \int_{\partial \triangle^N}f(z)dz = f(z_0)\int_{\partial \triangle^N}dz+f'(z_0)\int_{\partial \triangle^n}zdz - z_0 f'(z_0)\int_{\partial \triangle^N} dz + \int_{\partial \triangle^N}o(z-z_0)dz = \\
  & = \int_{\partial \triangle^N}o(z-z_0)dz
\end{align*}
причем полагаем $N$ таким, что
\begin{align*}
  & \forall z \in \triangle^N \ \left| z-z_0 \right| < \delta_1
\end{align*}
Тогда
\begin{align*}
  & \left| \int_{\partial \triangle^N}f(z)dz \right| \leq \int_{\partial \triangle^N} \left| o(z-z_0) \right|\cdot \left| dz \right| \leq \varepsilon\int_{\partial\triangle^N}\left| z-z_0 \right|\cdot\left| dz \right| \leq \varepsilon P^2_N \leq \varepsilon \frac{P^2_0}{4^N}
\end{align*}
\begin{align*}
  & \frac{\left| I \right|}{4^N} \leq \varepsilon \frac{P^2_0}{4^N}
\end{align*}
\begin{align*}
  & I = 0
\end{align*}
ч.~т.~д.
\corollary (леммы)
Пусть $G$~--- односвязная область, а $\gamma_{br} \subseteq
G$~--- замкнутая жорданова ломаная; $f:G \mapsto \CC$ регулярна. Тогда
\begin{equation} \label{(7.3)}
  \int_{\gamma_{br}} f(z)dz = 0
\end{equation}
\pr
По теореме Жордана существует триангуляция с ориентацией для $\gamma_{br}$;
тогда 
\begin{align*}
  & \int_{\gamma_{br}} f(z)dz = \sum_{k=1}^K\int_{\partial\triangle^K}f(z)dz = 0
\end{align*}
ч.~т.~д.
\pr (теоремы Коши)
Для любой ломаной $\gamma_{br}\subseteq G$ выполняется \eqref{(7.3)}.
Тогда по теореме $3$ $\S 6$ (п. $2$) $f dz$~--- полный дифференциал,
соответственно, по теореме $3$ $\S 6$ (п. $1$) выполняется \eqref{(7.1)}.
\note
Нельзя так просто убрать односвязность области:
\begin{align*}
  & \int_{\left| z \right| = 1} \frac{dz}{z} = 2 \pi i \neq 0
\end{align*}
но $\dst \frac{1}{z}$ регулярна в $\CC \setminus \{0\}$.
\Def
\textbf{Областью с кусочно гладкой границей $\Gamma$} называется область,
граница которой $\Gamma = \dst \bigcup_{k=1}^K\Gamma_k$, $\Gamma_k$~--- гладкая
кривая конечной пложительной длины и одного из классов:
\begin{itemize}
    \item \textbf{правильная компонента}~--- $\forall z \in \Gamma_k, \ \forall
    B_r(z_0): \ B_r(z_0)\cap \overline{G} \neq \varnothing, \ B_r(z_0) \cap
    \overline{\CC\setminus G} \neq \varnothing$
    \item \textbf{разрез}~--- $\forall z \in \Gamma_k$ (за исключением концов)
    $\exists \varepsilon > 0: \ B_{\varepsilon}(z) \setminus \Gamma \subseteq G$
\end{itemize}
\Def
Пусть $G$~--- область с кусочно гладкой границей $\Gamma$; говорят, что на
$\Gamma$ задана \textbf{положительная ориентация}, если:
\begin{itemize}
    \item для любой правильной компоненты $\Gamma_k$ направление обхода таково,
    что $G$ остается слева;
    \item любой разрез $\Gamma_k$ обходится дважды в противоположных
    направлениях.
\end{itemize}

Говорят, что \textbf{$f$ непрерывна на $\ol{G}$ с кусочно гладкой границей
  $\Gamma$}, если
\begin{itemize}
    \item если $\Gamma_k$~--- правильная компонента, то $\forall \Gamma_k, \ z_0
    \in \Gamma_k \ \dst \lim_{z \overset{G}{\to} z_0}f(z) = f(z_0)$
    \item если $\Gamma_k$~--- разрез, то $\forall \Gamma_k, \ z_0 \in \Gamma_k,
    \ z_0 \to z_0^+, z_0^-$ ($2$ края разреза, может быть больше краев)
    $f(z_0^+) = \dst \lim_{z \overset{B_+}{\to} z_0}f(z), f(z_0^-) = \dst
    \lim_{z\overset{B_-}{\to}z_0}f(z)$ ($B_+$, $B_-$~--- части, на которые
    разделена разрезами область, может быть больше частей.)
\end{itemize}
\theorem
Пусть $G$~--- ограниченная односвязная область с положительно ориентированной
кусочно гладкой границей $\Gamma$. Пусть $f: G \mapsto \CC$ регулярна в $G$,
непрерывна на $\overline{G}$. Тогда
\begin{equation}\label{(7.4)}
  \int_{\Gamma}f(z)dz = 0
\end{equation}
\Def
Ограниченная область $G$ называется \textbf{звездной}, если
\begin{itemize}
    \item $\Gamma = \left\{z \mid z = z_0 + z_1(t), \ t \in [\alpha, \beta], \
        z_1(\alpha) = z_1(\beta) \right\}$; $z_0$ называется \textbf{центром},
    $z_1(t)$ непрерывно дифференцируема;
    \item $\forall \lambda \in [0;1) \ \Gamma_{\lambda} = \left\{ z \mid z = z_0 +
    \lambda z_1(t), t \in [\alpha, \beta], z_1(\alpha) = z_1(\beta), \lambda \in
    [0;1)\right\} \subseteq G$
\end{itemize}
Докажем теорему $2$ для звездной области и конечного объединения звездных
областей.
\pr
Пусть $G$~--- звездная, тогда можем принять $z_0=0$: пусть $G$ такова, что $z_0
\neq 0$, тогда примем $\tilde{z} = z-z_0$, $\tilde{G} = G-z_0$, $\tilde{\Gamma}
= \Gamma - z_0$. Тогда
\begin{align*}
  & \int_{\Gamma}f(z)dz = \int_{\tilde{\Gamma}}f(\tilde{z}+z_0)d\tilde{z} = \int_{\tilde{\Gamma}}\tilde{f}(\tilde{z})d\tilde{z}
\end{align*}
Рассмотрим звездную $G$, такую, что $z_0 = 0$, $\Gamma_{\lambda} = \{z \mid z =
\lambda z_1(t), t\in[\alpha, \beta], z_1(\alpha) = z_1(\beta)\}$,
$\Gamma_{\lambda}\subseteq G$
По теореме Коши 
\begin{align*}
  & \forall \lambda \in (0;1) \ \int_{\Gamma_\lambda}f(\zeta)d\zeta = 0
\end{align*}
\begin{align*}
  & \zeta = \lambda z, \ \zeta \in \Gamma_{\lambda} \Leftrightarrow z \in \Gamma
\end{align*}
\begin{align*}
  & 0 = \int_{\Gamma_\lambda}f(z)dz = \int_{\Gamma}f(\lambda z)\lambda dz \Rightarrow \forall \lambda \in (0;1) \ \int_{\Gamma}f(\lambda z)dz = 0
\end{align*}
$f$ равномерно непрерывна на $\overline{G}$: $\forall \varepsilon > 0 \ \exists
\delta(\varepsilon) >0: \ \forall z_1, z_2 \in \overline{G}: \ \left| z_1-z_2
\right| < \delta(\varepsilon) \hookrightarrow \left| f(z_1) - f(z_2) \right| < \varepsilon$
\begin{align*}
  & \forall z \in \Gamma \ \left| z-\lambda z \right| = (1-\lambda) \left| z \right| \leq (1-\lambda)C_0, \ C_0 = \max_{z \in \overline{G}}\left| z \right| < +\infty
\end{align*}
Пусть $\lambda_{\varepsilon}\in(0;1): \ (1 -
\lambda_{\varepsilon})C_0<\delta(\varepsilon)$; тогда
\begin{align*}
  & \left| \int_{\Gamma}f(z)dz \right|= \left| \int_{\Gamma}\left( f(z)-f(\lambda_{\varepsilon}z) \right)dz \right|\leq \int_{\Gamma} \left| f(z) - f(\lambda_{\varepsilon}z) \right| \cdot \left| dz \right| \leq \varepsilon \int_{\Gamma} \left| dz \right| = \varepsilon l(\Gamma) = \varepsilon \cdot const
\end{align*}
ч.~т.~д.
\theorem (обобщенная Коши) Пусть $G$~--- ограниченная область с кусочно гладкой
и положительно ориентированной границей $\Gamma$. Пусть $f: G \mapsto
\overline{G}$ регулярна на $G$ и непрерывна на $\overline{G}$. Тогда
\begin{equation} \label{(7.5)}
  \int_{\Gamma}f(z)dz = 0
\end{equation}
\pr
Пусть $\Gamma_1$~--- внешняя граница, $\Gamma_k$~--- внутренние (правильные
копоненты и разрезы). Зададим для любой $\Gamma_k$, не выходящей на границу,
разрез $\gamma_k$ с двумя обходами, соединяющий с границей. Получим тогда
$\tilde{G} = G \setminus \left(\dst \bigcup_{k=1}^N \gamma_k \right)$. Тогда по
теореме $2$
\begin{align*}
  & 0 = \int_{\tilde{\Gamma}}f(z)dz = \int_{\Gamma} f(z)dz + \sum_{k=1}^N\left( \int_{\gamma_k} f(z) dz + \int_{\gamma_k^{-1}} \right) = \int_{\Gamma} f(z) dz
\end{align*}
ч.~т.~д.
\\