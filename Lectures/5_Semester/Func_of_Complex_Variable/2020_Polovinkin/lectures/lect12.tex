\begin{flushright}
    \textit{Лекция 12 (от 13.10)}
\end{flushright}
\corollary
Пусть $G$ односвязна, $0 \not \in G$. Тогда для регулярной $h_k \in \Ln z$
справедлива формула Ньютона-Лейбница
\begin{equation}\label{(15.14)}
    h_k(z) = h_k(a) + \int_{\gamma_{az}}\frac{d\zeta}{\zeta}
\end{equation}
где $a \in G$, $\gamma_{az} \subseteq G$.
\pr
\begin{align*}
  & h_k(z) = h_0(z) + 2i \pi k
\end{align*}
\begin{align*}
  & h_0(z) = \ln \left| z \right| + i \left( \psi_0 + \Delta_{\gamma_{az}}\argt z \right)
\end{align*}
\begin{align*}
  & \Real \int_{\gamma_{az}} \frac{d\zeta}{\zeta} = \ln \left| z \right| - \ln \left| a \right|
\end{align*}
Действительно, пусть $z(t) = \gamma_{az}$, тогда
\begin{align*}
  & \Real \int_{\gamma_{az}} \frac{d\zeta}{\zeta} = \Real \int_0^1 \frac{xx'+yy'}{x^2+y^2} d \tau = \Real \int_0^1 \frac{d \sqrt{x^2+y^2}}{\sqrt{x^2+y^2}} = \ln \left| z(1) \right| - \ln \left| z(0) \right| = \ln \left| z \right| - \ln \left| a \right|
\end{align*}
\begin{align*}
  & h_k(z) = \ln \left| z \right| - \ln \left| a \right| + \ln\left| a \right| + i \left( \psi_0 + 2 \pi k \right) + i \Delta_{\gamma_{az}}\argt z = h_k(a) + \int_{\gamma_{az}}\frac{d\zeta}{\zeta}
\end{align*}
\section{$\S 16$. Регулярные ветви $\{\sqrt[n]{f(z)}\}$ и $\Ln f(z)$.}
\sug
$G$~--- область, $f$ регулярна на $G$, $\forall z \in G \ f(z) \neq 0$.
\begin{equation}\label{(16.1)}
    \left\{ \sqrt[n]{f(z)} \right\} = \sqrt[n]{\left| f(z) \right|} \exp \left( \frac{i}{n}\Arg f(z) \right)
\end{equation}
\begin{equation}\label{(16.2)}
    \Ln f(z) = \ln \left| f(z) \right| +i \Arg f(z)
\end{equation}
\begin{equation}\label{(16.3)}
    \Arg f(z) = \left\{ \argt f(z) + 2 \pi k \mid k \in \ZZ \right\}
\end{equation}
\Def
Пусть $\gamma$~--- непрерывная кривая в $G$, $f$ удовлетворяет предположению
$1$. Пусть $z(t)$~--- параметризация $\gamma$, $t \in [0;1]$. Пусть $\Gamma =
f(\gamma): \ w = f(z(t)), \ t \in [0;1]$. Тогда \textbf{приращением аргумента
  $f(z)$ вдоль кривой $\gamma$} называется
\begin{equation}\label{(16.4)}
    \Delta_{\gamma}\argt f(z) = \Delta_\Gamma\argt w = \Delta_{[0;1]}\argt f(z(t))
\end{equation}
\lemma
Пусть $f$, $f_1$, $f_2$ удовлетворяют предположению $1$ в области $G$. Тогда
\begin{enumerate}
    \item для любой непрерывной $\gamma \subseteq G$ выполняется логарифмическое
    свойство:
    \begin{equation}\label{(16.5)}
        \Delta_{\gamma}\argt (f_1f_2) = \Delta_\gamma\argt f_1 + \Delta_{\gamma}\argt f_2
    \end{equation}
    \item если $\gamma \subseteq G$ разбита точкой $A \in \gamma$ на части
    $\gamma_1$, $\gamma_2$, то
    \begin{equation}\label{(16.6)}
        \Delta_{\gamma}\argt f = \Delta_{\gamma_1}\argt f + \Delta_{\gamma_2}\argt f
    \end{equation}
    \item если $\gamma$~--- кусочно гладкая кривая в $G$, то
    \begin{equation}\label{(16.7)}
        \Delta_{\gamma}\argt f(z) = \Img \int_\Gamma \frac{dw}{w} = \Img \int_{\gamma} \frac{f'(\zeta)}{f(\zeta)}d\zeta
    \end{equation}
\end{enumerate}
\pr
Утверждения леммы очевидны из определения приращения аргумента функции.
\lemma
Пусть $G$ односвязна, $\os{\circ}{\gamma} \subseteq G$ замкнута и непрерывна.
Пусть $f$ удовлетворяет предположению $1$. Тогда
\begin{equation}\label{(16.8)}
    \Delta_{\os{\circ}{\gamma}}\argt f(z) = 0
\end{equation}
\pr
Пусть $\os {\circ}{\gamma} \subseteq G$ гладкая и замкнутая, параметризованная
$z(t)$.
\\
$\forall z_0 \in \os{\circ}{\gamma}$ существует непрерывная деформация $z(t,
\alpha)$ такая, что $z(t,a) = z(t)$, $z(t,b) = z_0$. По теореме $3$ $\S 14$
$I(\alpha) = const$, а значит,
\begin{align*}
  \Delta_{[0;1]}\argt f(z) = \Delta_{[0;1]}\argt f(z_0) = 0
\end{align*}
\lemma
Пусть $f$ удовлетворяет предположению $1$ в области $G$. Если в области $G$
сществуют регулярные ветви $h_0(z)$ или $g_0(z)$~--- регулярные ветви $\Ln f(z)$
или $\left\{ \sqrt[n]{f(z)} \right\}$, то все их непрерывные ветви регулярны и
удовлетворяют соотношениям:
\begin{equation}\label{(16.9)}
    h_k(z) = h_0(z) + 2i \pi k, \ k \in \ZZ
\end{equation}
\begin{equation}\label{(16.10)}
    g_k(z) = g_0(z) \exp\left( \frac{i}{n} 2 \pi k\right), \ k \in \{0, \dots, n-1\}
\end{equation}
\pr
Доказательство полностью аналогично доказательству леммы $15.1$.
\lemma
Пусть $f$ удовлетворяет предположению $1$ в $G$. Если в $G$ существует
регулярная ветвь $h(z)$ многозначной функции $\Ln z$, то $\forall a,b \in G$
\begin{equation}\label{(16.11)}
    h(b) = h(a) + \ln \left| \frac{f(b)}{f(a)} \right| + i \Delta_{\gamma_{ab}}\argt f(z)
\end{equation}
\pr
\begin{align*}
  & h(z) = \ln \left| f(z) \right| + i \Img h(z)
\end{align*}
В силу регулярности $h$ $\Img h$ будет гармонической функцией от $x, y$. Значит,
\begin{align*}
  & \varphi(z) = \Img h(z) \in \Arg f(z)
\end{align*}
Пусть $a \in G$, $\psi_0 \in \Arg f(a)$, $h(a) = \ln \left| f(a) \right| + i
\psi_0$. Пусть $\gamma_{az}$~--- гладкая кривая с параметризацией $z(t)$, $t \in
[0;1]$, $\varphi(z(t))$~--- гладкая ветвь $\Arg f(z(t))$.
\begin{align*}
  & \Delta_{\gamma_{az}}\argt f(z) = \varphi(z(t))-\varphi(z(0)) = \Img h(z) - \Img h(a)
\end{align*}
\begin{align*}
  & h(b) = \ln \left| f(b) \right| + i \Img h(b)
\end{align*}
\begin{align*}
  & h(a) = \ln \left| f(a) \right| + i \Img h(a)
\end{align*}
\begin{align*}
  & h(b) - h(a) = \ln \left| \frac{f(b)}{f(a)} \right| + i \left( \Img h(b) - \Img h(a) \right)
\end{align*}
Отсюда следует \eqref{(16.11)}.
\theorem
Пусть $f$ удовлетворяет предположению $1$ в $G$. Тогда у  $\Ln f(z)$ существуют
регулярные ветви в $G$ тогда и только тогда, когда выполняется \eqref{(16.8)}
для любой замкнутой $\os{\circ}{\gamma}\subseteq G$.
\pr
Докажем в обе стороны.
\begin{itemize}
    \item Необходимость.
    \\
    Если есть регулярные ветви $h(z) \in \Ln z$  $G$, то из \eqref{(16.11)} при
    $a = b$ получаем \eqref{(16.8)}.
    \item Достаточность.
    \\
    Пусть выполнено \eqref{(16.8)}. Фиксируем $a \in G$, $h(a) \in \Ln f(a)$.
    Рассмотрим
    \begin{equation}\label{(16.12)}
        h(z) = h(a) + \ln \left| \frac{f(z)}{f(a)} \right| + i \Delta_{\gamma_{az}}\argt f(z)
    \end{equation}
    В действительности приращение аргумента может зависеть от $\gamma$, поэтому
    корректнее расссматривать это как $h(\gamma_{az})$.
    \\
    По \eqref{(16.8)}
    \begin{align*}
      & \Delta_{\gamma_{az}}\argt f(z) = \Delta_{\tilde{\gamma}_{az}}\argt f(z)
    \end{align*}
    \begin{align*}
      & \os{\circ}{\gamma} = \gamma_{az}+\tilde{\gamma}^{-1}_{az}
    \end{align*}
    Значит, $h(z)$~--- функция лишь точки $z$.
    \begin{equation}\label{(16.13)}
        \exists \psi_0 \in \Arg f(a): \ h(a) = \ln \left| f(a) \right| + i \psi_0
    \end{equation}
    Из \eqref{(16.12)} следует, что
    \begin{align*}
      & h(z) = \ln \left| f(z) \right| + i \left( \psi_0 + \Delta_{\gamma_{az}}\argt f(z) \right) \in \Ln f(z)
    \end{align*}
    \begin{align*}
      & \Real \int_{\gamma_{az}}\frac{f'(\zeta)}{f(\zeta)}d\zeta = \ln \left| f(z) \right| - \ln \left| f(a) \right|
    \end{align*}
    \begin{align*}
      & \Delta_{\gamma_{az}}\argt f(z) = \Img \int_{\gamma_{az}}\frac{f'(\zeta)}{f(\zeta)}d\zeta
    \end{align*}
    Отсюда следует, что
    \begin{align*}
      & h(z) = \ln \left| f(a) \right| + \ln \left| f(z) \right| - \ln \left| f(a) \right| + i \Delta_{\gamma_{az}} \argt f(z) + i \psi_0 = \ln \left| f(a) \right| + \int_{\gamma_{az}}\frac{f'(\zeta)}{f(\zeta)}d \zeta +  i \psi_0 
    \end{align*}
    Функция
    \begin{align*}
      & \varphi(z) = \int_{\gamma_{az}}\frac{f'(\zeta)}{f(\zeta)}d\zeta
    \end{align*}
    регулярна по теореме $3$ $\S 6$, а значит, и $h(z)$ регулярна.
\end{itemize}
\lemma
Пусть $f$ удовлетворяет предположению $1$ в $G$, $n \in \NN$, $n \geq 2$. Если в
$G$ существует регулярная ветвь $g(z)$ многозначной функции $\left\{
    \sqrt[n]{f(z)} \right\}$, то $\forall a, b \in G$
\begin{equation}\label{(16.14)}
    g(b) = g(a) \sqrt[n]{\left| \frac{f(b)}{f(a)} \right|}\exp \left( \frac{1}{n}\Delta_{\gamma_{ab}}\argt f(z) \right)
\end{equation}
\pr Доказательство разбиваем на два этапа.
\begin{enumerate}
    \item Пусть для начала $G$ односвязна.
    \\
    По лемме $2$ $\forall \os{\circ}{\gamma} \subseteq G$ выполняется
    \eqref{(16.8)} и тогда, по лемме $4$:
    \begin{align*}
      & \exists h(z) = h(a) + \ln \left|\frac{f(z)}{f(a)}\right|+ i\left(\Delta_{\gamma_{az}}\argt f(z)\right) \in \Ln f(z)
    \end{align*}
    регулярные ветви.
    \begin{align*}
      & g_0(z) = \exp \left(\frac{i}{n} h(z)\right)
    \end{align*}
    \begin{align*}
      & \exists \psi_0 \in \Arg f(a): \ h(z) = \ln \left| f(z) \right| + i \left( \psi_0 + \Delta_{\gamma_{az}}\argt f(z) \right)
    \end{align*}
    Тогда
    \begin{align*}
      & g_0(z) = \sqrt[n]{\left| f(z) \right|}\exp \left( \frac{i}{n} \left( \psi_0 + \Delta_{\gamma_{az}}\argt f(z) \right) \right)\in \left\{ \sqrt[n]{f(z)} \right\}
    \end{align*}
    регулярная ветвь.
    \\
    По лемме $3$ $\exists k \in \left\{ 0, \dots, n-1 \right\}: \ g(z) = g_0(z)$.
    Отсюда следует \eqref{(16.14)}.
    \item $G$~--- область.
    \begin{align*}
      &\forall a, b \in G, \ \forall \gamma_{ab} \subseteq G
    \end{align*}
    можем выбрать такие $a=z_0, z_1, \dots, z_{K-1}, z_K = b \in \gamma_{ab}$,
    что $\left| z_k - z_{k-1} \right|< l_k$, где $l_k$~--- расстояние до
    границы.
    \begin{align*}
      & \exists \left\{ B_\varepsilon(z_k) \right\}_{k=0}^K \subseteq G: \ z_{k+1} \in B_{\varepsilon}(z_k)
    \end{align*}
    Каждый из $B_{\varepsilon}(z_k)$ односвязен, поэтому выполняется
    \eqref{(16.14)}:
    \begin{align*}
      & \forall k \in \left\{ 0, \dots, K-1 \right\} \ \frac{g(z_{k+1})}{g(z_k)} = \sqrt[n]{\frac{f(z_{k+1})}{f(z_k)}} \exp \left( \frac{i}{n} \Delta_{\gamma_{z_kz_{k+1}}}\argt f(z) \right)
    \end{align*}
    Перемножая по всем $k$, получаем
    \begin{align*}
      & \frac{g(b)}{g(a)} = \sqrt[n]{\left| \frac{f(b)}{f(a)} \right|}\exp \left( \frac{i}{n} \sum_{k=0}^{K-1} \Delta_{\gamma_{z_kz_{k+1}}} \argt f(z) \right) = \sqrt[n]{\frac{\left| f(b) \right|}{\left| f(a) \right|}}\exp \left( \frac{i}{n}\Delta_{\gamma_{ab}}\argt f(z) \right)
    \end{align*}
    Это ни что иное, как \eqref{(16.14)}.
\end{enumerate}\theorem
Пусть $f$ удовлетворяет условиям предположения $1$ в $G$. Тогда существуют
регуляртые ветви $\left\{ \sqrt[n]{f(z)} \right\}$тогда и только тогда, когда
\begin{equation}\label{(16.15)}
    \forall \os{\circ}{\gamma} \subseteq G \ \exists k(\os{\circ}{\gamma}) \in \ZZ: \ \Delta_{\os{\circ}{\gamma}}\argt f(z) = 2\pi n k(\os{\circ}{\gamma}))
\end{equation}