\begin{flushright}
    \textit{Лекция 22 (от 17.11)}
\end{flushright}
\section{$\S 27.$ Аналитическое продолжение.}
\Def
Путь задана $f: D \mapsto \CC$, $D \subseteq G \subseteq \CC$, $G$~--- область,
$g: G \mapsto \CC$ регулярна. Если $\forall z \in D \ f(z) = g(z)$, то $g$
называется \textbf{аналитическим продолжением} $f$ на $G$.
\Example
\begin{align*}
  & e^x \to e^z = e^xe^{iy}, \ D = \RR \subseteq \CC \to G = \CC
\end{align*}
\Example
\begin{align*}
  & \sin x \to \sin z = \frac{e^{iz}-e^{-iz}}{2}, \ D = \RR \subseteq \CC \to \CC
\end{align*}
\Def
Пусть $a \in \CC$, $f: B_r(a) \mapsto \CC$, $r>0$~--- регулярная. Тогда $\left(
    B_r(a), f \right)$ называется \textbf{элементом аналитической функции с
  центром в точке $a$}.
\Def
Два элемента $\left( B_r(a), f \right)$ и $\left( B_\rho(b), g \right)$
называются \textbf{эквивалентными}, если $a = b$, $\forall z \in B_{r_0}(a) \
f(z) = g(z)$, $r_0 = \min \left\{ r, \rho \right\}$.
\Def
Пусть $\left( B_r(a), f \right)$~--- элемент аналитической функции, тогда
говорят, что элемент $\left( B_\rho(b), g \right)$ является его
\textbf{непосредственным аналитическим продолжением (НАП)}, если $B_r(a) \cap
B_\rho(b) \neq \varnothing$, $\forall z \in B_r(a) \cap B_\rho(b) \ f(z) =
g(z)$.
\Note
Если определены $B_r(a), B_\rho(b), f$, то по теореме единственности однозначно
определен и $\left( B_\rho(b), g \right)$.
\Def
Пусть $\left( B_r(a), f \right)$~--- элемент. Говорят, что $\left( B_\rho(b), g
\right)$ есть \textbf{аналитическое продолжение элемента $\left( B_r(a), f
  \right)$ вдоль конечной цепочки элементов (кругов)}, если $\exists \left\{
    \left( B_{r_k}(z_k), f_k \right) \right\}_{k=0}^n$ такое, что $\left(
    B_{r_0}(z_0), f_0 \right) \sim \left( B_r(a), f \right)$, $\left(
    B_{r_n}(z_n), f_n \right) \sim \left( B_\rho(b), g \right) $, $\forall k \in
\left\{ 1,\dots, n\right\}$ $\left( B_{r_k}(z_k), f_k \right)$ является
непосредственным аналитическим продолжением $\left( B_{r_{k-1}}(z_{k-1}),
    f_{k-1} \right)$.
\Example
\begin{align*}
  & f_1(z) = \sum_{n=0}^{\infty}z^n
\end{align*}
сходится на $B_1(0)$ и расходится при $\left| z \right| \geq 1$. Пусть $\left(
    B_1(0), f_1 \right)$~--- элемент; тогда
\begin{align*}
  & f_2 = \frac{1}{1-z}
\end{align*}
регулярна в $\CC \setminus \left\{ 1 \right\}$, и для $a \in \CC \setminus [1;
+\infty)$ элемент $\left( B_{\left| a-1 \right|}(a), f_2 \right)$ будет НАП
$\left( B_1(0), f_1 \right)$; для $a_2 \in (1; +\infty)$ элемент $\left(
    B_{\left| a_2-1 \right|}(a_2), f_2 \right)$ не будет НАП $\left( B_1(0), f_1
\right)$, но будет аналитическим продолжением вдоль конечной цепочи кругов.
\Example
Пусть
\begin{align*}
  & f_s(z) = \sqrt{\left| z \right|}\exp\left( \frac{i}{2}\argt_s(z) \right), \ \argt_s(z) \in \left( s-\frac{\pi}{2}, s+\frac{\pi}{2} \right)
\end{align*}
Для элемента $\left( B_1(1), f_0 \right)$ элемент $\left( B_1(i),
    f_{\frac{\pi}{2}} \right)$ будет НАП, а для него, в свою очередь, элемент
$\left( B_1(-1), f_\pi \right)$ будет НАП.
\\
Для элемента $\left( B_1(1), f_0 \right)$ элемент $\left( B_1(-i),
    f_{-\frac{\pi}{2}} \right)$ будет НАП, а для него, в свою очередь, элемент
$\left( B_1(-1), f_{-\pi} \right)$ будет НАП.
\begin{align*}
  & \forall z \in B_1(-1) \ f_\pi(z) = f_{-\pi}(z)
\end{align*}
\Def
Пусть $\left( B_r(a), f \right)$~--- элемент. Говорят, что $\left( B_\rho(b),g
\right)$ является \textbf{аналитическим продолжением вдоль кривой
  $\gamma_{ab}$}, если $\exists r > 0$, $\varphi:[0;l] \mapsto \CC$, $l$~---
длина кривой, $\gamma_{ab}: z = z(s)$, $s \in [0;l]$, а также существует
семейство элементов $\left\{ \left( B_r(z(s)), f(s) \right) \right\}_{s\in
  [0;l]}$, такое, что
\begin{itemize}
\item $\forall s_0 \in [0;l], \ \forall s \in [0;l]\cap [s_0-r; s_0 + r] \
\varphi(z) = f_{s_0}(z)$
\item $\left( B_r(a), f \right)\sim \left( B_r(z(0)), f_0 \right)$, $\left(
    B_\rho(b), g \right) \sim \left( B_r(z(l)), f_l \right)$
\end{itemize}
\Example
Можем рассмотреть те же элементы, что и в примере $4$, $s \in [0;\pi]$, и по
первой кривой $\gamma_1 = e^{is}$ придем к элементу $\left( B_1(e^{i\pi}),
    f_{\pi} \right)$, а по второй кривой $\gamma_2 = e^{-is}$ придем к элементу
$\left( B_1(e^{-i\pi}), f_{-\pi} \right)$.
\theorem
Аналитическое продолжение вдоль конечной цепочки и вдоль кривой эквивалентно.
\pr
По конечной цепочке построим кривую.
\\
По конечной цепочке элементов $\left\{ B_{r_k}(z_k), f_k \right\}$ построим
круги: для каждой точки $z \in [r_k; r_k+1]$ зададим $B_r(z)$, причем $0 <r \leq
\min \left\{ r_k \right\}$. Этот круг будет лежать в объединении исходных
кругов; тогда $\varphi(z) = f_k(z)$.
\\
По кривой построим конечную цепочку.
\\
Разобьем кривую на конечное число участков таких, что $\left| z(s_k) -
    z(s_{k-1}) \right|\leq \dst\frac{r}{2}$ и рассмотрим семейство $\left(
    B_r(z(s_k)), f_{s_k} \right)$. Тогда $z(s_{k})\in B_r(z(s_{k-1}))$,
пересечение непусто и функции на нем совпадают.
\Def
\textbf{Полной аналитической функцией, порожденной начальным элементом $\left(
      B_r(a), f \right)$}, называется совокупность всех аналитических
продолжений по всем кривым с началом в точке $a$, по которым оно возможно.
\Def
\textbf{Аналитической функцией, порожденной начальным элементом $\left(
      B_r(a), f \right)$}, называется связное семейство элементов полной
аналитической функции, начинающейся из этого элемента.
\\
\textbf{Связное семейство элементов}~--- такое семейство, где любые два элемента
могут быть получены один из другого, проходя только по этому семейству.
\theorem (о монодромии)
Пусть аналитическая функция $F(z)$ определена на односвязной области $G$ в
$\CC$. Тогда ее значения в любой точке этой области не зависят от кривой, по
которой получено аналитическим продолжением это значение.
\\
Говорят, что аналитическая функция, заданная на односвязной области, однозначна
в каждой точке и регулярна в области.
\section{$\S 28.$ Полные аналитические функции $\Ln z$, $\sqrt[n]{z}$ и римановы
  поверхности.}
$\forall a \neq 0$ рассмотрим $\left( B_{\left| a \right|}(a), h_a(z) \right)$,
$h_a(z)$~--- регулярная ветвь $\Ln z$ и $\left( B_{\left| a \right|}(a), g_a(z)
\right)$, $g_a(z)$~--- регулярная ветвь $\left\{ \sqrt[n]{z} \right\}$~---
\textbf{элементы, порожденные логарифмом и корнем}.