\begin{flushright}
    \textit{Лекция 6 (от 22.09)}
\end{flushright}
\section{$\S 8.$ Интегральная формула Коши}
\theorem
Пусть $G$~--- ограниченная область в $\CC$ с кусочно гладкой положительно
ориентированной границей $\Gamma$. Пусть $f: \overline{G} \mapsto \CC$ регулярна
на $G$ и непрерывна на $\overline{G}$. Тогда
\begin{equation} \label{(8.1)}
    \forall z \in G f(z) = \frac{1}{2\pi i}\int_{\Gamma}\frac{f(\zeta)}{\zeta - z}d\zeta
\end{equation}
\pr
Зафиксируем
\begin{align*}
  & z \in G: \ \zeta \to \frac{f(\zeta)}{\zeta - z}
\end{align*}
\begin{align*}
  & \exists r_0 > 0: \ \forall r \leq r_0 \ \overline{B_r(z)} \subseteq G
\end{align*}
Пусть
\begin{align*}
  & \gamma_r = \{\zeta \mid \left| \zeta - z \right| = r\}, \ r > 0, \ r \leq r_0
\end{align*}
причем $\gamma_r$ положительно ориентированная. Пусть
\begin{align*}
  & G = \tilde{G} \setminus \overline{B_r(z)}
\end{align*}
Тогда
\begin{align*}
  & \frac{f(\zeta)}{\zeta - z}
\end{align*}
регулярна и непрерывна на $\Gamma_r = \Gamma \cup \gamma_r^{-1}$.
По теореме $3$ $\S 7$
\begin{align*}
  & 0 = \int_{\Gamma_r} \frac{f(\zeta)}{\zeta - z} d \zeta =  \int_{\Gamma} \frac{f(\zeta)}{\zeta - z} d \zeta - \int_{\gamma_r} \frac{f(\zeta)}{\zeta - z} d \zeta
\end{align*}
\begin{align*}
  & \forall r \in (0; r_0) \ J(z) = \frac{1}{2 \pi i} \int_{\Gamma} \frac{f(\zeta)}{\zeta - z} d \zeta = \frac{1}{2 \pi i} \int_{\gamma_r} \frac{f(\zeta)}{\zeta - z} d \zeta
\end{align*}
$f$ непрерывна в $z$ $\Rightarrow$
\begin{align*}
  & \forall \varepsilon > 0 \ \exists \delta > 0: \ \forall \zeta: \ \left| \zeta - z \right| < \delta(\varepsilon) \hookrightarrow \left| f(\zeta) - f(z) \right| < \varepsilon; \ 0 < r < \delta(\varepsilon)
\end{align*}
\begin{align*}
  & \frac{1}{2 \pi i} \int_{\gamma_r} \frac{1}{\zeta - z} d \zeta = 1
\end{align*}
\begin{align*}
  & J(z) - f(z) = \frac{1}{2 \pi i} \int_{\gamma_r} \frac{f(\zeta) - f(z)}{\zeta - z} d \zeta
\end{align*}
\begin{align*}
  & \left| J(z) - f(z) \right| \leq \frac{1}{2 \pi i} \int_{\gamma_r} \frac{\left| f(\zeta) - f(z) \right|}{\left| \zeta - z \right|} \cdot \left| d\zeta \right| \leq \frac{\varepsilon}{2 \pi r} \int_{\gamma_r}\left| d\zeta \right| = \varepsilon
\end{align*}
ч.~т.~д.
\Def
Пусть $\gamma$~--- кусочно гладкая в $\CC$; $q: \gamma \mapsto \CC$ непрерывна.
Тогда \textbf{интегралом Коши} называется выражение
\begin{equation} \label{(8.2)}
    I(z) = \frac{1}{2 \pi i} \int_{\gamma} \frac{q(\zeta)}{\zeta - z} d\zeta, \ z \in \CC \setminus \gamma
\end{equation}
\theorem
В условиях определения интеграла Коши он бесконечно дифференцируем на
$\CC\setminus\gamma$, причем
\begin{equation} \label{(8.3)}
    \forall n \in \NN, \  z \in \CC \setminus \gamma \ I^{(n)}(z) = \frac{n!}{2 \pi i} \int_{\gamma} \frac{q(z)}{(\zeta - z)^{n+1}} d\zeta
\end{equation}
\pr
По индукции.
\\
База: при $n=0$, очевидно, верно.
\\
Шаг: пусть верно для $n-1$, тогда для $n$ фиксируем произвольное $z_0 \in \CC
\setminus \gamma$.
\\
Пусть
\begin{align*}
  & d = \inf\{\left| z_0 - \zeta \right| \mid \zeta \in \gamma\}; \ d < +\infty
\end{align*}
Пусть
\begin{align*}
  & \Delta z: \ 0 < \left| \Delta z \right| < \frac{d}{2}
\end{align*}
Тогда
\begin{align*}
  & \zeta \in \gamma: \ \left| \zeta - (z_0+\Delta z)\right| \geq \left| \zeta - z_0 \right| - \left| \Delta z \right| > d - \frac{d}{2} = \frac{d}{2} > 0
\end{align*}
\begin{align*}
  & \frac{I^{(n-1)}(z_0+\Delta z) - I^{(n-1)}(z_0)}{\Delta z} - \frac{n!}{2 \pi i}\int_{\gamma}\frac{q(\zeta)}{(\zeta-z)^{n+1}}d\zeta = \frac{(n-1)!}{2 \pi i}\int_{\gamma}q(\zeta)\left( \left( \frac{1}{(\zeta-z_0-\Delta z)^n} - \right. \right.\\
  & \left. \left. - \frac{1}{(\zeta-z_0)^n} \right)\frac{1}{\Delta z} - \frac{n}{(\zeta-z_0)^{n+1}}\right)d\zeta
\end{align*}
В силу
\begin{align*}
  & (\zeta-z_0)^n - (\zeta - z_0 - \Delta z)^n = n\Delta z(\zeta - z_0)^{n-1} + O(\Delta z^2)
\end{align*}
\begin{align*}
  & \frac{n(\zeta-z_0)^{n-1}+O(\Delta z)}{(\zeta - z_0)^n(\zeta-z_0-\Delta z)^n} - \frac{n}{(\zeta - z_0)^{n+1}} = \frac{n(\zeta-z_0)^n+O(\Delta z) - (\zeta-z_0-\Delta z)^n}{(\zeta - z_0)^{n+1}(\zeta - z_0 - \Delta z)^n} = \\
  & = n \frac{O(\Delta z)}{(\zeta - z_0)^{n+1}(\zeta - z_0)^n}
\end{align*}
\begin{align*}
  & \exists C > 0: \ \left| O(\Delta z) \right| < C \left| \Delta z \right|
\end{align*}
\begin{align*}
  & \left| \zeta - z_0 \right| \geq d; \ \left| \zeta - z_0 - \Delta z \right| > \frac{d}{2}
\end{align*}
\begin{align*}
  & \left| \frac{\Delta I}{\Delta z} - \frac{1}{2 i\pi} \int_{\gamma}\dots d \zeta\right|\leq \frac{n!}{2 \pi}\int_{\gamma}\left| q(\zeta) \right|\frac{c\left| \Delta z \right|}{d^{n+1}\left( \dst \frac{d}{2} \right)^n}\left| d\zeta \right| \leq A \left| \Delta z \right|
\end{align*}
\theorem
Пусть $f: G \mapsto \CC$ регулярна в $G$, тогда $f$ бесконечно дифференцируема в
$G$ и $\forall z \in \overline{B_r(z_0)} \subseteq G$
\begin{equation} \label{(8.4)}
    f^{(n)}(z) = \frac{n!}{2 \pi i}\int_{\gamma_r}\frac{f(\zeta)}{(\zeta - z)^{n+1}}d\zeta
\end{equation}
\pr
Из \eqref{(8.1)}
\begin{align*}
  f(z) = \frac{1}{2 \pi i}\int_{\gamma_r}\frac{f(\zeta)}{(\zeta - z)}d\zeta
\end{align*}
Тогда по \eqref{(8.3)} выполняется \eqref{(8.4)}.
\section{$\S 9.$ Ряд Тейлора. Теорема Вейерштрасса}
\Def
\textbf{Степенным рядом} называется ряд
\begin{equation} \label{(9.1)}
    \sum_{n=0}^{\infty}c_n(z-a)^n, \ a \in \CC, \ c_n \in \CC
\end{equation}
\theorem (Абеля)
Пусть \eqref{(9.1)} сходится в $z_0 \neq a$.
\\
Тогда
\begin{enumerate}
    \item $\forall z \in B_{\left| z_0 - a \right|}(a)$ \eqref{(9.1)}
    сходится абсолютно;
    \item $\forall r \in (0; \left| z_0-a \right|)$ \eqref{(9.1)} сходится
    равномерно на $\overline{B_r(a)}$.
\end{enumerate}
\pr
\begin{enumerate}
    \item Пусть $z \in B_{\left| z_0-a \right|}(a)$.
    \\
    \eqref{(9.1)} сходится в $z_0 \Rightarrow \dst
    \sum_{n=0}^{\infty}c_n(z-a)^n$ сходится.
    \\
    По критерию Коши
    \begin{align*}
      \lim_{n \to \infty}\left| c_n(z_0-a)^n \right| = 0
    \end{align*}
    \begin{align*}
      \exists \alpha > 0: \ \forall n \in \NN \ \left| c_n(z_0-a)^n \right| \leq \alpha
    \end{align*}
    \\
    Пусть
    \begin{align*}
      z \in B_{\left| z_0-a \right|}(a), \ \left| z-a \right|< \left| z_0-a \right|
    \end{align*}
    \begin{align*}
      q_z = \frac{\left| z-a \right|}{\left| z_0-a \right|} < 1
    \end{align*}
    \begin{align*}
      \left| c_n(z-a)^n \right| = \left| c_n(z_0-a)^n \right|\frac{\left| (z-a)^n \right|}{\left| (z_0-a)^n \right|} \leq \alpha q_z^n
    \end{align*}
    В силу того, что $\dst \sum_{n=0}^{\infty}q_{z}^{n} < \infty$, то по
    признаку сравнения сходится абсолютно.
    \item Имеем
    \begin{align*}
      \forall z \in \overline{B_{r}(a)}, \ r < \left| z_0-a \right|
    \end{align*}
    \begin{align*}
      q = \frac{r}{\left| z_0-a \right|} < 1
    \end{align*}
    \begin{align*}
      \forall z \in \overline{B_r(a)} \ \left| c_n(z-a)^n \right| = \leq \alpha q^n
    \end{align*}
    По признаку Вейерштрасса сходится равномерно.
\end{enumerate}
\begin{align*}
  R = \sup\{\left| z-a \right|\mid \sum_{n=0}^{\infty}c_n(z-a)^n \text{ сходится}\}
\end{align*}
Тогда при $\left| z-a \right|<R$ ряд сходится, при $\left| z-a \right|>R$
расходится.
\begin{align*}
  R = \frac{1}{\overline{\dst \lim_{n \to \infty}} \sqrt[n]{\left| c_n \right|}}
\end{align*}
\example
\begin{align*}
  \sum_{n=0}^{+\infty}z^n, \ \left| z \right|< 1
\end{align*}
\begin{align*}
  f(z) = \frac{1}{1-z}
\end{align*}
\begin{align*}
  S_N = \sum_{n=0}^Nz^n
\end{align*}
\begin{align*}
  (1-z)S_N = 1- z^{N+1}
\end{align*}
\begin{equation}\label{(9.2)}
  S_N = \frac{1}{1-z} - \frac{z^{N+1}}{1-z} \underset{N \to \infty}{\to} \frac{1}{1-z}
\end{equation}
\Def
Пусть $f: B_r(a) \mapsto \CC$ регулярна. Пусть
\begin{equation}\label{(9.3)}
    \exists f^{(n)}(a): \ \sum_{n=0}^{\infty}c_n(z-a)^n, \ c_n = \frac{f^{(n)}(a)}{n!}
\end{equation}
Такой ряд называется \textbf{рядом Тейлора для функции $f$}.
\theorem
Пусть $f: B_r(a) \mapsto \CC$ регулярна. Тогда $\forall z \in B_r(a)$
выполняется разложение $f$ в ряд Тейлора \eqref{(9.3)}, т.~е.
\begin{align*}
  f(z)= \sum_{n=0}^{\infty}c_n(z-a)^n, \ z \in B_r(a)
\end{align*}
\pr
Зафиксируем $z_0 \in B_r(a)$.
\\
Тогда $\exists r_1 > 0: \ \left| a-z_0 \right|<r_1<r$; пусть $\gamma_1 = \{\zeta
\mid \left| \zeta - a \right| = r_1\}$~--- положительно ориентированная. Тогда
по интегральной формуле Коши
\begin{equation}\label{(9.4)}
  f(z_0) = \frac{1}{2 \pi i} \int_{\gamma_1}\frac{f(\zeta)}{\zeta - z_0} d \zeta 
\end{equation}
\begin{align*}
  \frac{1}{\zeta - z_0} = \frac{1}{(\zeta - a) - (z_0 - a)} = \frac{1}{(\zeta - a)\left( 1-\frac{z_0-a}{\zeta - a} \right)} = \frac{1}{\zeta - a} \sum_{n=0}^{\infty}\left( \frac{z_0-a}{\zeta - a} \right)^n
\end{align*}
Это верно, т.~к. $\left| \dst \frac{z_0-a}{\zeta-a} \right| = q < 1$ю
\\
При $\zeta \in \gamma_1$ ряд сходится равномерно.
\begin{align*}
  \frac{f(\zeta)}{z - z_0} = \frac{1}{z - a} \sum_{n=0}^{\infty}\frac{f(\zeta)}{(\zeta - a)^{n+1}}\left( z_0-a\right)^n
\end{align*}
сходится равномерно на $\zeta \in \gamma$. Тогда по теореме $3$ $\S 6$
\begin{align*}
  f(z_0) = \sum_{n=0}^{\infty}\left( \frac{1}{2\pi i} \int_{\gamma} \frac{f(\zeta)}{(\zeta - a)^{n+1}} d \zeta \right)(z_0-a)^n
\end{align*}
По теореме $3$ $\S 8$
\begin{align*}
  \frac{1}{2 \pi i}\int_{\gamma_1}\frac{f(\zeta)}{(\zeta - a)^{n+1}}d \zeta = \frac{f^{(n)}(a)}{n!} = c_n
\end{align*}
не зависит от $z_0$.
\\
ч.~т.~д.
\corollary
Пусть $f: G \mapsto \CC$ регулярна. Пусть $a \in G$, $d = \inf \{\left| a-\zeta
\right| \mid \zeta \in \Gamma\}$.
\\
Тогда $f(z)$ представима в виде ряда Тейлора в $B_d(a)$, т.~е. радиус сходимости
ряда Тейлора не менее $d$.
\note
Как известно, на $\RR$ $f(x) = \dst \frac{1}{x^2+1}$ всюду непрерывна, а ее ряд
Тейлора сходится только при $\left| x \right|< 1$. Это легко понять, перейдя в
$\CC$, ведь там $f(x) = \dst \frac{1}{(x+i)(x-i)}$ и, очевидно, регулярна при
$\left| x \right| < 1$.
\example
$w = e^z$: всюду дифференцируема, $w^{(n)}(z) = e^z = w(z)$. Тогда $\forall R$
ряд Тейлора будет сходиться в $B_R(0)$, причем $c_n = \dst \frac{1}{n!}$.
\begin{align*}
  e^z = \sum_{n=0}^{\infty} \frac{z^n}{n!}, \ R = +\infty
\end{align*}
\example
$\sin z = \dst \frac{e^{iz} - e^{-iz}}{2i}$~--- регулярна в $\CC$.
\begin{align*}
  \sin z = \sum_{n=0}^{\infty} \frac{(-1)^nz^{2n+1}}{(2n+1)!}, \ R = +\infty
\end{align*}
\example
$\cos z = \dst \frac{e^{iz} + e^{-iz}}{2}$~--- регулярна в $\CC$.
\begin{align*}
  \cos z = \sum_{n=0}^{\infty} \frac{(-1)^nz^{2n}}{(2n)!}, \ R = +\infty
\end{align*}
\example
$h(z) = \ln \left| z \right| + i \cdot arg_r(z)$~--- регулярна в $\CC \setminus
(-\infty; 0]$.
\\
Если $h(z+1) = \varphi(z)$, то $\varphi(z)$ регулярна в $\CC\setminus (-\infty; -1]$
\\
$\varphi(z)$ регулярна в $B_1(0)$.
\begin{align*}
  \varphi^{(n)}(z) = \frac{(-1)^{n+1}(n-1)!}{(z+1)^n}
\end{align*}
\begin{align*}
  \varphi(z) = \sum_{n=1}^{\infty}\frac{(-1)^{n+1}z^n}{n!}, \ R < 1
\end{align*}
