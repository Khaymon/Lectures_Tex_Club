\begin{flushright}
    \textit{Лекция 8 (от 29.09)}
\end{flushright}
\section{$\S 11.$ Ряд Лорана}
\Def \textbf{Ряд Лорана с центром в точке $a \in \CCC$}~--- выражение вида
\begin{equation}\label{(11.1)}
    \sum_{n=-\infty}^{+\infty}c_n(z-a)^n
\end{equation}
Это сумма
\begin{equation}\label{(11.2)}
    \sum_{n=0}^{+\infty}c_n(z-a)^n
\end{equation}
и
\begin{equation}\label{(11.3)}
    \sum_{n=-\infty}^{-1}c_n(z-a)^n
\end{equation}
Пусть $t = \dst \frac{1}{z-a}$, тогда \eqref{(11.3)} равносильно
\begin{equation}\label{(11.4)}
    \sum_{m=1}^{+\infty}c_{-m}t^m
\end{equation}
\begin{align*}
  & \left| t \right| < \alpha \Rightarrow \left| z - a \right| > \frac{1}{\alpha} = \rho
\end{align*}
Из \eqref{(11.2)} ($R$~--- радиус сходимости) и \eqref{(11.4)} ($\alpha$~---
радиус сходимости)
\begin{align*}
  & \left| z-a \right|<R, \ \rho < R
\end{align*}
\begin{align*}
  & \rho < \left| z-a \right| < R
\end{align*}
\textbf{Кольцом сходимости} называется
\begin{equation}\label{(11.5)}
    K_{\rho, R}(a) = \{z \mid \rho < \left| z-a \right| < R\}
\end{equation}
$\forall r_1, r_2: \ \rho < r_1 < r_2 < R$ \eqref{(11.1)} сходится
равномерно в кольце $\overline{K_{r_1,r_2}(a)}$.
\theorem (Лорана-Вейерштрасса)
Пусть $f: K_{\rho,R}(a) \mapsto \CC$ регулярна. Тогда она предстваима в виде
сходящегося ряда Лорана с центром в точке $a$:
\begin{equation}\label{(11.6)}
    f(z) = \sum_{-\infty}^{+\infty}c_n(z-a)^n, \ z \in K_{\rho,R}(a)
\end{equation}
\begin{equation}\label{(11.7)}
    c_n = \frac{1}{2\pi i}\int_{\gamma_r}\frac{f(\zeta)}{(\zeta - a)^{n+1}}d\zeta, \ r \in (\rho, R), \ \gamma_r = \{\zeta \mid \left| \zeta - a \right| = r\}
\end{equation}
\pr
~
\begin{enumerate}
    \item Коэффициенты $c_n$ в \eqref{(11.7)} не зависят от $r$.
    \begin{align*}
      & r_1,r_2: \ \rho < r_1 < r_2 < R, \ \gamma_k = \{\zeta \mid \left| \zeta - a \right| = r_k\}
    \end{align*}
    Пусть $\Gamma = \Gamma_1 \cup \Gamma_2$~--- положительно ориентированная
    граница $K_{r_1,r_2}(a)$.
    \\
    $\dst \frac{f(\zeta)}{(\zeta - a)^{n+1}}$ регулярна в $K_{r_1,r_2}(a)$.
    Тогда по обобщенной теореме Коши
    \begin{align*}
      & \int_{\Gamma} \frac{f(\zeta)}{(\zeta - a)^{n+1}}d\zeta = 0 = \int_{\gamma_2} \frac{f(\zeta)}{(\zeta - a)^{n+1}}d\zeta - \int_{\gamma_1} \frac{f(\zeta)}{(\zeta - a)^{n+1}}d\zeta
    \end{align*}
    а значит, $c_n$ не зависит от $r$.
    \item Фисируем произвольную $z_0 \in K_{\rho,R}(a)$.
    \begin{align*}
      & \exists r_1, r_2 \in (\rho,R): \ \rho < r_1 < \left| z_0-a \right|<r_2<R 
    \end{align*}
    значит, для $\gamma_k$ по интегральной формуле Коши
    \begin{align*}
      & f(z_0) = \frac{1}{2\pi i}\int_{\gamma_2} \frac{f(\zeta)}{(\zeta - z_0)}d\zeta - \frac{1}{2\pi i}\int_{\gamma_1} \frac{f(\zeta)}{(\zeta - z_0)}d\zeta = I_2 + I_1
    \end{align*}
    Рассмотрим $I_2$.
    \begin{align*}
      & \frac{1}{\zeta - z_0} = \frac{1}{(\zeta - a) - (z_0-a)} = \frac{1}{(\zeta - a)\left( 1 - \frac{z_0-a}{\zeta-a} \right)}, \ \zeta \in \gamma_2
    \end{align*}
    \begin{align*}
      & \left| \frac{z_0-a}{\zeta - a} \right| = \frac{\left| z_0-a \right|}{r_2} = q_2 < 1
    \end{align*}
    \begin{align*}
      & \frac{1}{2 \pi i}\frac{f(\zeta)}{\zeta - z_0} = \sum_{n=0}^{+\infty}\frac{1}{2\pi i}\frac{f(\zeta)}{(\zeta - a)} \cdot \frac{(z_0-a)^n}{(\zeta - a)^n}, \zeta \in \gamma_2
    \end{align*}
    \begin{align*}
      & M = \sup_{z \in \overline{K_{r_1,r_2}(a)}}\left| f(z) \right|< \infty
    \end{align*}
    По теореме $3$ $\S 6$ интегрируем (сходится равномерно):
    \begin{align*}
      & I_2 = \sum_{n=0}^{+\infty}\int_{\gamma_2} \frac{1}{2 \pi i}\frac{f(\zeta)}{(\zeta - a)}\cdot \frac{(z_0-a)^n}{(\zeta - a)^n} d\zeta = \sum_{n=0}^{+\infty} c_n(z_0-a)^n
    \end{align*}
    Рассмотрим $I_1$.
    \begin{align*}
      & \frac{-1}{\zeta - z_0} = \frac{1}{z_0-\zeta} = \frac{1}{(z_0 - a) - (\zeta-a)} = \frac{1}{(z_0 - a)\left( 1 - \frac{\zeta-a}{z_0-a} \right)}, \ \zeta \in \gamma_1
    \end{align*}
    \begin{align*}
      & \left| \frac{\zeta-a}{z_0 - a} \right| = \frac{r_1}{\left| z_0-a \right|} = q_1 < 1
    \end{align*}
    \begin{align*}
      & -\frac{1}{2 \pi i}\frac{f(\zeta)}{\zeta - z_0} = \sum_{n=0}^{+\infty}\frac{1}{2\pi i}\frac{f(\zeta)}{(z_0 - a)} \cdot \frac{(\zeta-a)^n}{(z_0 - a)^n}, \zeta \in \gamma_2
    \end{align*}
    По теореме $2$ $\S 6$ интегрируем (сходится равномерно):
    \begin{align*}
      & I_1 = \sum_{n=0}^{+\infty}\left( \frac{1}{2\pi i}\int_{\gamma_1}f(\zeta)(\zeta - a)^n d\zeta \right) \frac{1}{(z_0-a)^{n+1}} = \sum_{m = -\infty}^{-1}c_m(z_0-a)^m
    \end{align*}
    Итак, $I_1+I_2$ дают ряд Лорана \eqref{(11.6)} с коэффициентами вида \eqref{(11.7)}.
\end{enumerate}
\corollary
Пусть $f: B_R(a) \mapsto \CC$ регулярна; тогда ее ряд Лорана совпадает с ее
рядом Тейлора.
\pr
~
\begin{enumerate}
    \item $f$ регулярна в круге, а значит,
    \begin{align*}
      & c_m = \frac{1}{2 \pi i}\int_{\gamma_r}\frac{f(\zeta)}{(\zeta - a)^{m+1}}d \zeta
    \end{align*}
    При $m < 0$ $f(\zeta)(\zeta - a)^{-m-1}$ регулярна в этом круге, а значит,
    по теореме Коши $\forall m < 0 \ c_m = 0$.
    \item По \eqref{(11.7)} $\forall n \geq  0 \ c_n$ совпадает с
    коэффициентом ряда Тейлора.
\end{enumerate}
ч.~т.~д.
\theorem (единственность ряда Лорана)
Пусть $f:K_{\rho,R}(a) \mapsto \CC, \ 0 \leq \rho < R$ регулярна и представима в
виде
\begin{equation}\label{(11.8)}
  f(z) = \sum_{-\infty}^{+\infty} b_n(z-a)^n
\end{equation}
Тогда коэффициенты $b_n$ есть коэффициенты $c_n$ из \eqref{(11.7)}.
\pr
\begin{align*}
  & r \in (\rho, R); \gamma_r = \{\zeta \mid \left| \zeta - a \right| = r\}
\end{align*}
Фиксируем $k \in \ZZ$.
\\
Из примера $1$ $\S 6$
\begin{align*}
  & \frac{1}{2 \pi i} \int_{\gamma_r}\frac{d\zeta}{(\zeta - a)^{m+1}} = \left\{ \begin{matrix}
          1, \ m = 0 \\
          0, \ m \in \ZZ \setminus\{0\}
      \end{matrix} \right.
\end{align*}
Ряд Лорана \eqref{(11.8)} сходится локально равномерно в $K_{\rho,R}(a)$,
а значит, равномерно на $\gamma_r$.
\\
Домножим \eqref{(11.8)} на выражение (константное в силу фиксированного
$k$):
\begin{align*}
  & \frac{1}{2\pi i}\frac{f(z)}{(z-a)^{k+1}} = \sum_{n = -\infty}^{+\infty} \frac{1}{2 \pi i}\frac{b_n(z-a)^n}{(z-a)^{k+1}}
\end{align*}
По теореме $3$ $\S 6$ интегрируем по $\gamma_r$:
\begin{align*}
  & c_k = \frac{1}{2 \pi i} \int_{\gamma_r}\frac{f(z)}{(z - a)^{k+1}} = \sum_{n = -\infty}^{+\infty} \frac{1}{2\pi i}b_n \int_{\gamma_r}\frac{dz}{(z-a)^{k-n+1}}
\end{align*}
В правой части ненулевое значение только при $n=k$, а значит, $c_k = b_k$
\\
ч.~т.~д.
\corollary
Разложение регулярной функции $f: B_R(a) \mapsto \CC$ в ряд Тейлора
единственное.
\pr
Это очевидно следует из теоремы $2$ и следствия $1$.
\corollary
Пусть $f:K_{\rho,R}(a) \mapsto \CC$ регулярна и разложена в ряд Лорана. Пусть
$\forall r \in (\rho, R) \ \left| f(z) \right|\leq A_r$ при $z \in \gamma_r$.
\\
Тогда $\forall n \ \left| c_n \right| \leq \dst \frac{A_r}{r_n} \ \forall r \in
(\rho, R)$.
\pr
\begin{align*}
  & \left| c_n \right| = \frac{1}{2 \pi} \int_{\gamma_r}\frac{\left| f(\zeta) \right|}{\left| \zeta - a \right|^{n+1}} \leq \frac{A_r}{2 \pi r^{n+1}} \cdot 2 \pi r = \frac{A_r}{r^n}
\end{align*}
ч.~т.~д.
\section{$\S 12.$ Изолированные особые точки}
\Def
Пусть $f$ определена в $\overset{\circ}{B}_\rho(a), \ a \in \CCC$, регулярна в
$\overset{\circ}{B}_\rho(a)$, но не регулярна в $a$. Тогда $a$~---
\textbf{изолированная особая точка однозначного характера функции $f$ (ИОТ(ОХ),
  ИОТ)}.
\Def
Пусть $a$~--- ИОТ $f$. Тогда $a$ называется:
\begin{enumerate}
    \item \textbf{устранимой особой точкой (УОТ)}, если $\exists \dst \lim_{z
      \to a} f(z) \in \CC$.
    \item \textbf{полюсом}, если $\exists \dst \lim_{z \to a}f(z) = \infty$.
    \item \textbf{существенно особой точкой (СОТ)}, если $\not \! \exists \dst
    \lim_{z \to a} f(z) \in \CCC$.
\end{enumerate}
\begin{itemize}
    \item Если $a \in \CC$, то $\overset{\circ}{B}_\rho(a) = \{z \mid 0 < \left|
        z-a \right| < \rho\}$
    \item Если $a = \infty \in \CCC$, то $\overset{\circ}{B}_\rho(a) = \{z \mid
    \left| z \right| > \rho\}$
\end{itemize}
\begin{enumerate}
    \item Если $a \in \CC$, то существует ряд Лорана с центром в точке $a$:
    \begin{align*}
      & f(z) = \sum_{n=0}^{+\infty}c_n(z-a)^n + \sum_{n=-\infty}^{-1} c_n(z-a)^n
    \end{align*}
    Его первое слагаемое называется \textbf{правильной частью}, второе~---
    \textbf{главной частью}.
    \item Если $a  = \infty \in \CCC$, то существует ряд Лорана с центром в
    точке $a$:
    \begin{align*}
      & f(z) = \sum_{n=-\infty}^{0}c_nz^n + \sum_{n=1}^{+\infty} c_nz^n
    \end{align*}
    Его первое слагаемое называется \textbf{правильной частью}, второе~---
    \textbf{главной частью}.
\end{enumerate}
\theorem
Пусть $a \in \CCC$~--- ИОТ $f$. Тогда для рада Лорана $f$ выполняется:
\begin{enumerate}
    \item $a$~--- УОТ $\Leftrightarrow$ главная часть есть тождественный ноль
    \item $a$~--- полюс $\Leftrightarrow$ главная часть содержит конечное число
    ненулевых слагаемых
    \item $a$~--- СОТ $\Leftrightarrow$ главная часть содержит бесконечно много
    ненулевых слагаемых.
\end{enumerate}
\pr
~
\begin{itemize}
    \item[I] $a \in \CC$
    \begin{enumerate}
        \item $a$~--- УОТ.
        \begin{itemize}
            \item Необходимость.
            \\
            $a$~--- УОТ $\Rightarrow$
            \begin{align*}
              & \exists \lim_{z \to a} \in \CC
            \end{align*}
            \begin{align*}
              & \exists \rho > 0, \ \exists A > 0: \ \left| f(z) \right| < A \ \forall z \in \overset{\circ}{B}_{\rho}(a)
            \end{align*}
            По следствию $11.2.2$
            \begin{align*}
              & \forall r \in (0;\rho) \ \left| c_n \right| \leq \frac{A}{r^n}
            \end{align*}
            а значит, $\forall n < 0$
            \begin{align*}
              & \left| c_n \right| \leq Ar^{\left| n \right|} \underset{r\to 0}{\longrightarrow} 0
            \end{align*}
            А значит, главная часть ряда Лорана тождественно равна нулю.
            \item Достаточность.
            \begin{align*}
              & f(z) = \sum_{n=0}^{+\infty}c_n(z-a)^n = S(z)
            \end{align*}
            По теореме Вейерштрасса ряд сходится локально равномерно, $S(z)$~---
            регулярная при $z \neq a$. Но тогда
            \begin{align*}
              & \lim_{z \to a}f(z) = \lim_{z \to a}S_n(z) = S(a) = c_0 \in \CC
            \end{align*}
        \end{itemize}
        \item $a$~--- полюс.
        \begin{itemize}
            \item Необходимость.
            \\
            $a$~--- полюс $\Rightarrow$
            \begin{align*}
              & \lim_{z \to a}f(z) = \infty
            \end{align*}
            \begin{align*}
              & \exists \rho > 0, \ \exists \delta > 0: \ \left| f(z) \right| > 1 \ \forall z \in \overset{\circ}{B}_{\rho}(a)
            \end{align*}
            Положим $g(z) = \dst \frac{1}{f(z)}$. Эта функция регулярна в
            $\overset{\circ}{B}_{\rho}(a)$, а значит,
            \begin{align*}
              & \lim_{z \to a}g(z) = \lim_{z \to a}\frac{1}{f(z)} = 0
            \end{align*}
            $a$~--- УОТ $g$, а значит, по пункту $1$
            \begin{align*}
              & \exists m \geq 1: \ g(z) = b_m(z-a)^m + b_{m+1}(z-a)^{m+1} + \dots
            \end{align*}
            Тогда $\exists b_m \neq 0$. Тогда
            \begin{align*}
              & g(z) = (z-a)^m \sum_{n=0}^{+\infty}b_{n+m}(z-a)^n = (z-a)^mh(z)
            \end{align*}
            $h(z)$ регулярна в $B_{\rho}(a)$, $h(a) = b_m \neq 0$.
            Тогда
            \begin{align*}
              & \exists \delta_1 \in (0;\delta]: \ h(z) \neq 0 \ \forall z \in B_{\delta_1}(a) 
            \end{align*}
            Тогда $p(z) = \dst \frac{1}{h(z)}$ регулярна в $B_{\delta_1}(a)$,
            $p(a) = \dst \frac{1}{b_m} \neq 0$.
            \begin{equation}\label{(12.1)}
                f(z) = \frac{1}{g(z)} = \frac{p(z)}{(z-a)^m}
            \end{equation}
            \begin{equation}\label{(12.2)}
                \begin{split}
                    & f(z) = \frac{1}{(z-a)^m} \sum_{n=0}^{+\infty}d_n(z-a)^n = \sum_{n=0}^{\infty}d_n(z-a)^{n-m} = \sum_{n=0}^{m-1} d_n(z- \\
                    & - a)^{n-m} + \sum_{n=m}^{+\infty}d_n(z-a)^{n-m}
                \end{split}
            \end{equation}
            \item Достаточность.
            Пусть главна часть содержит конечное число слагаемых (по
            \eqref{(12.2)}), $d_0\neq 0$
            \\
            Тогда из \eqref{(12.2)} и \eqref{(12.1)}:
            \begin{align*}
              & f(z) = \frac{p(z)}{(z-a)^n}, \ p(a) \neq 0
            \end{align*}
            регулярная в $\os{\circ}{B}_{\delta_1}(a)$;
            \begin{align*}
              & \lim_{z \to a} \frac{p(z)}{(z-a)^n} = \infty
            \end{align*}         
        \end{itemize}
        \item $a$~--- СОТ.
        \\
        Существование предела в $\CCC$ дает конечное число слагаемых, и это
        критерий. Значит, отсутствие предела в $\CCC$ равносильно бесконечному
        числу слагаемых в главной части ряда Лорана.
        \end{enumerate}
    \item[II] $a = \infty$: положим $\zeta = \dst \frac{1}{z}$, тогда $\infty$
    перейдет в $0$. Рассмотрим $\tilde{f}(\zeta) = f\left( \dst \frac{1}{\zeta}
    \right)$. Тогда $\zeta = 0$~--- особая точка $\tilde{f}(\zeta)$. Главная
    часть при таком преобразовании переходит в главную часть, и мы можем
    повторить доказательство для точки из $\CC$.
\end{itemize} 