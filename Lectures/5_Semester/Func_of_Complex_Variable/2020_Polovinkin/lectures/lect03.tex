\begin{flushright}
    \textit{Лекция 3 (от 14.09)}
\end{flushright}
\section{$\S 5.$ Теорема об обратной функции}
\theorem
Пусть $f: G \mapsto H \subseteq \CC$, $g: H \mapsto \CC$ регулярны.
Тогда $\zeta(z) = g(f(z)): G \mapsto \CC$ также регулярна, причем
\begin{equation}\label{(5.1)}
    \zeta'(z) = g'(f(z))f'(z) \forall z \in G
\end{equation}
\pr
\begin{align*}
  & z_0 \in G, \ w_0 = f(z_0) \in H
\end{align*}
Из дифференцируемости
\begin{align*}
  & \Delta f = f(z_0)\Delta z + o(\Delta z), \ \Delta g = g'(w_0) \Delta w + 0(\Delta w)
\end{align*}
Пусть $\Delta w = \Delta f$, тогда
\begin{align*}
  & \frac{\Delta \zeta}{\Delta z} = g'(w_0) \frac{\Delta f}{\Delta z} + \frac{o(\Delta f)}{\Delta f} \cdot \frac{\Delta f}{\Delta z} \us{\Delta z \to 0}{\to} g'(w_0)f'(z_0) + 0
\end{align*}
\theorem (об обратной функции)
Пусть $f: G \mapsto \CC$ регулярна и непрерывно дифференцируема на $G$. Пусть
$z_0 \in G$, $w_0 = f(z_0)$, $f'(z_0) \neq 0$. Тогда $\exists B_\delta(z_0),
B_\varepsilon(w_0)$, такие, что:
\begin{enumerate}
    \item $\forall z \in B_\delta(z_0) \ f'(z) \neq 0$;
    \item $\forall \hat{w} \in B_\varepsilon(w_0)$ уравнение $\hat{w} = f(z)$
    имеет в $B_\delta(z_0)$ единственное решение $\hat{z}$, т.~е. на
    $B_\varepsilon(w_0)$ определена обратная функция $g: B_\varepsilon(w_0)
    \mapsto B_\delta(z_0)$, т.~е. $\forall w \in B_\delta(w_0) \hookrightarrow
    f(g(w)) = w$;
    \item $g$ регулярна на $B_\varepsilon(w_0)$, причем
    \begin{align*}
      & \forall w \in B_\varepsilon(w_0) \hookrightarrow g'(w) = \frac{1}{f'(g(w))}
    \end{align*}
\end{enumerate}
\pr
Пусть $f(z) = u(x,y) +iv(x,y)$. Имеем отображение $\RR^2 \mapsto \RR^2$. В силу
непрерывной дифференцируемости этих двух функций запишем якобиан и преобразуем
согласно УКР:
\begin{align*}
  & J(x,y) = \left| \begin{matrix}
          u_x & u_y \\
          v_x & v_y
      \end{matrix} \right| = \left| \begin{matrix}
          u_x & -v_x \\
          v_x & u_x
      \end{matrix} \right| = (u_x)^2+(v_x)^2 = \left| f'(z) \right|^2
\end{align*}
\begin{align*}
  & J(x_0,y_0) \neq 0
\end{align*}
\theorem (Бесов, $\S 12.1$, $1$)
Для такого отображения, как в теореме $5.2$, $\exists B_\delta(z_0): \ \forall z
\in B_\delta(z_0) \ f'(z) \neq 0 $, а также $\exists B_\varepsilon(z_0)$, где
существует обратное отображение
\begin{align*}
  & \left\{ \begin{matrix}
          x = x(u,v) \\
          y = y(u,v)
      \end{matrix} \right.: B_\varepsilon(w_0) \mapsto B_\delta(z_0)
\end{align*}
причем непрерывно дифференцируемое.
\pr
\begin{align*}
  & g(w) = z = x(u,v) + iy(u,v): B_\varepsilon(w_0) \mapsto B _\delta(z_0)
\end{align*}
Докажем, что
\begin{align*}
  &\forall w \in B_\varepsilon(w_0) \ \exists g'(w)
\end{align*}
Действительно,
\begin{align*}
  & \forall \hat{w} \in B_\varepsilon(w_0) \ \exists \hat{z} \in B_\delta(z_0): \ f(\hat{z}) = \hat{w} \Leftrightarrow g(\hat{w}) = \hat{z}
\end{align*}
\begin{align*}
  & \forall \left\{ w_k \right\}_{k=1}^\infty \subseteq B_\varepsilon(w_0): \ \forall k \ w_k \neq \hat{w}, \ w_k \us{k \to \infty}{\to} \hat{w} \ \exists z_k = g(w_k): \ \forall k \ f(z_k) = w_k; \ z_k \us{k \to \infty}{\to} z
\end{align*}
\begin{align*}
  & \frac{g(w_k) - g(\hat{w})}{w_k - \hat{w}} = \frac{z_k - \hat{z}}{w_k - \hat{w}} = \frac{1}{\dst \frac{f(z_k) - f(\hat{z})}{z_k - \hat{z}}} \us{k \to \infty}{\to} \frac{1}{f'(\hat{z})}
\end{align*}
\corollary
Пусть $f: G \mapsto \CC$ \textbf{однолистна} (взаимно однозначна) на $G$,
регулярна на этой области и имеет непрерывную производную. Пусть $\forall z \in
G \ f'(z) \neq 0$. Тогда обратное отображение $g: G^* \mapsto \CC$, причем
$g(G^*) = G$, регулярно на $G^*$.
\pr
Докажем, что $G^*$~--- область. Видим, что $\forall w_0 \in G^* \ \exists z_0
\in G: \ f(z_0) = w_0$. По условию, $f'(z) \neq 0$; тогда по теореме $2$ $\S 5$
$\exists \varepsilon > 0: \ B_\varepsilon(w_0) \subseteq G^*$.
\\
Докажем регулярность.
\begin{align*}
  & w_1, w_2 \in G \Rightarrow \exists z_1, z_2 \in G: \ f(z_k) = w_k
\end{align*}
Заметим, что $\exists \gamma_{z_1z_2} \subseteq G$; тогда $\gamma_{w_1w_2}^* =
\gamma^* = f(\gamma_{z_1z_2}) \subseteq G^*$.
В силу дифференцируемости $g$ в $B_\varepsilon(w_0)$ $g$ будет регулярна на
$G^*$.
\Example
$w = z^n, \ n \geq 2, \ n \in \NN$.
\\
Уравнение $z^n = w_0$ имеет решения:
\begin{align*}
  & z \in \left\{ \sqrt[n]{w_0} \right\} = \left\{ \sqrt[n]{\left| w_0 \right|} \exp \left( \dst \frac{i}{n}\left( \argt w_0 + 2 \pi k \right) \right) \mid k \in \left\{0, \dots, n-1 \right\} \right\}
\end{align*}
\Def
Пусть $\forall z \in G$ поставлено в соответствие некоторое множество $F(z)
\subseteq \CC$. Тогда говорят, что $F$~--- \textbf{многозначная функция}.
\\
Для нашего примера $F(w) = \left\{ \sqrt[n]{w} \right\}, \ w \in \CC$.
\Def
Пусть $F: G \mapsto 2^{\CC}$, $f: G \mapsto \CC$, $\forall z \in G \ f(z)\in
F(z)$. Тогда $f$~--- \textbf{ветвь многозначной функции} $F$.
\Example
\begin{align*}
  & z_1^n = z_2^n \Leftrightarrow \left| z_1 \right|^n \exp \left( in\varphi_1 \right) = \left| z_2 \right|^n \exp \left( in\varphi_2 \right) \Leftrightarrow \left| z_1 \right| = \left|z_2\right|, \ n\left( \varphi_1 - \varphi_2 \right) = 2 \pi k, \ k \in \ZZ
\end{align*}
Рассмотрим область
\begin{align*}
  & G_{\alpha, \beta} = \left\{ z \mid z = re^{i \varphi}, \ r>0, \ \varphi \in \left( \alpha, \beta \right) \right\}
\end{align*}
Если $0 < n(\beta - \alpha) \leq 2 \pi$, то $w = z^n$ будет однолистной. По
следствию $1$ существует обратная $g(w)$, причем регулярная. Эта функция
строится для случая $\alpha = -\frac{\pi}{2}$, $\beta = \frac{\pi}{2}$ так:
\begin{align*}
  & g: G_{-\frac{\pi}{2}, \frac{\pi}{2}} \mapsto G_{-\frac{n\pi}{2}, \frac{n\pi}{2}} \subseteq \CC \setminus (-\infty; 0]; \ l_{\varphi_0} = \left\{ z \mid z = re^{\i \varphi_0}, \ r > 0 \right\}, \ l_{\varphi_0} \mapsto l_{n \varphi_0}
\end{align*}
В явном виде функция будет иметь вид
\begin{align*}
  & g(w) = \sqrt[n]{\left| w \right|}\exp\left( \frac{i}{n} \argm w \right)
\end{align*}
Это \textbf{главная ветвь многозначной функции $\sets{\sqrt[n]{z}}$}.
\Example
$w = e^z$, $w'(z) \neq 0$.
\\
Отыщем область однолистности. Если $z_1 \neq z_2$, а $e^{z_1} = e^{z_2}$, то
$x_1 = x_2$, $y_1 - y_2 = 2 \pi k, \ k \in \ZZ$. Пусть $G_0 = \left\{ z \mid
    \Img z \in (-i\pi, i \pi) \right\}$. Тогда по следствию $5.3.1$ существует
обратная $g: \CC \setminus (-\infty; 0] \mapsto G_0$, причем регулярная. Эта
функция строится так:
\begin{align*}
  & e^z = w_0; \ e^x = \left| w_0 \right|, \ y = \alpha + 2 \pi k
\end{align*}
Имеем логарифм:
\begin{align*}
  & \Ln z = \ln \left| w \right| + i \Arg w
\end{align*}
Функция в явном виде имеет вид:
\begin{align*}
  & g = \ln \left| w \right| + i \argm w
\end{align*}
Это \textbf{главная ветвь многозначной функции $\Ln z$}.
\Example
Производная главной ветви логарифма
\begin{align*}
  & g'(w) = \frac{1}{w}
\end{align*}
а корня
\begin{align*}
  & g'(w) = \frac{1}{n(g(w))^{n-1}}
\end{align*}
\section{$\S 6.$ Интегрирование}
Задание $z(t) = x(t) + iy(t)$, $t \in [t_0;t_1]$, $x, y \in C[t_0, t_1]$
называется \textbf{параметрическим заданием кривой}.
\\
Если $t = \psi(\tau)$, $\tau \in [\tau_0; \tau_1]$ и функция непрерывно моотонно
возрастает, то это называется \textbf{эквивалентными параметризациями}.
\\
Множество эквивалентных параметризаций называется \textbf{кривой}.
\\
Кривая называется \textbf{жордановой (простой)}, если $\tau_1 \neq \tau_2
\Rightarrow z(\tau_1) \neq z(\tau_2)$, \textbf{гладкой}, если она непрерывно
дифференцируема, и \textbf{замкнутой}, если $z(\tau_0) = z(\tau_1)$
(\textbf{начальная} и \textbf{конечная} точки совпадают).
\\
Наконец, кривая называется \textbf{замкнутой жордановой}, если совпадение
разрешено только на концах.
\theorem (Жордана)
Всякая замкнутая ломаная жорданова кривая разбивает комплексную плоскость на
ограниченную (не содержащую $\infty$) и неограниченную, причем в ограниченной
облсти существует триангуляция с ориентацией: можно построить непересекающиеся
отрезки, разбивающие плоскость на треугольники, а на треугольниках задано
направление обхода, что совпадает на границе области с направлением обхода
кривой.
\pr (схема)
\\
Доказывать можем по индукции числа вершин ломаной.
\\
База: $n = 3$, очевидно, есть и область, и триангуляция.
\\
Шаг: допустим, есть вершина $A$ с максимальной действительной частью. Рассмотрим
соседние вершины: предшествующую $B$ и последующую $C$. Если ни в треугольнике,
ни на его границах нет вершин ломаной, то добавляем отрезок $BC$ и ориентацию
ему в обе стороны. Иначе, если внутри треугольника или на стороне $BC$ есть еще
одна вершина (пусть $D$, ближайшая к точке $A$), то включим в триангуляцию
отрезок $AD$.