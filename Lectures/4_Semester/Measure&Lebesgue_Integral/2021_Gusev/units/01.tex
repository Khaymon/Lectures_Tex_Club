\newpage
\lecture{1}{Различные системы множеств.}

В этом курсе имеется дело с функциями, аргументами которых являются множества.
\begin{definition}
	\mdef{Мерой} на множестве $X$ называется функция
	$\mu:\: \F \rightarrow [0,\, \infty]$, где $\F$~--- семейство подмножеств $X$.
\end{definition}

На $\F$ нужно наложить некоторые ограничения, потому как если, к примеру, определена мера для двух множеств,
то логично было бы, чтобы была определена мера и на их сумме, пересечении, объединении. Отсюда вытекают такие понятия как:

\begin{definition}
	Семейство $\F$ подмножеств множества $X$ (далее используется обозначение $\F \subset \mathcal{P} (X)\equiv 2^{X}$, где
	$\mathcal{P} (X)$~--- множество всех подмножеств множества $X$) называется \mdef{$\sigma$"=алгеброй}, если
	\begin{enumerate}[label=\arabic*\degree.]
		\item $\varnothing\in \F$.
		\item $\forall A,\, B\in\F:\: A\cap B\in\F,\,A\cup B\in\F,\, A\setminus B\equiv A\cap B^C\in\F$, где
		      $B^C=X\setminus B$.
		\item $X\in\F$.
		\item $\forall \{A_n\}_{n\in\N}\subset\F$ выполнено, что
		      $\bigcup\limits_{n=1}^{\infty}A_n\in\F,\,\bigcap\limits_{n=1}^{\infty}A_n\in\F$.
	\end{enumerate}
\end{definition}

\begin{definition}
	$\F$~--- \mdef{кольцо}, если выполняются условия $1\degree$ и $2\degree$.
\end{definition}
\begin{definition}
	$\F$~--- \mdef{алгебра}, если выполняются условия $1\degree$, $2\degree$ и $3\degree$.
\end{definition}

\begin{remark}
	Пусть $\F$~--- $\sigma$"=алгебра, тогда $\forall A\in\F:\: A^C=X\setminus A\in\F$. Тогда
	\[
		\bigcup\limits_{n=1}^{\infty}A_n=\left(\bigcup\limits_{n=1}^{\infty}A_n\right)^{CC}=
		\left(\bigcap\limits_{n=1}^{\infty}A_n^C\right)^C.
	\]
	Поэтому можно сказать, что вторая часть в условии $4\degree$ избыточна. Абсолютно аналогично и в обратную сторону,
	то есть эти два требования равносильны.
	\label{lect1remark1}
\end{remark}

\begin{remark}
	Пусть $\F$~--- кольцо,
	тогда $A\cap B=A\cap\left(B^{CC}\right)=A\setminus B^C=A\setminus\left(X\setminus B\right)=
		A\setminus\left(A\setminus B\right)$. То есть требование замкнутости по пересечению в свойстве $2\degree$ избыточно.
\end{remark}

\begin{exercise}
	Пусть $\F$~--- семейство всех ограниченных подмножеств множества $\R$. Тогда $\F$~--- кольцо, но не алгебра.
\end{exercise}

\begin{definition}
	Кольцо $\F$ называется
	\begin{enumerate}[label=\alph*)]
		\item \mdef{$\sigma$"=кольцом}, если $\forall \{A_n\}_{n=1}^{\infty}\subset\F$ выполнено, что
		      $\bigcup\limits_{n=1}^{\infty}A_n\in\F$.
		\item \mdef{$\delta$"=кольцом}, если $\forall \{A_n\}_{n=1}^{\infty}\subset\F$ выполнено, что
		      $\bigcap\limits_{n=1}^{\infty}A_n\in\F$.
	\end{enumerate}
\end{definition}

\begin{remark}
	Любое $\sigma$"=кольцо является $\delta$"=кольцом, но обратное неверно.
\end{remark}

\begin{definition}
	Множество $I\subset \R$ называется \mdef{промежутком}, если $\forall a,\, b\in I:\: [a,\, b]\subset I$.
	Промежуток $I$ называется \mdef{конечным}, если он ограничен.

	Например, $[a,\, b],\, (a,\, b),\, [a,\, b),\, (a,\, b]$~--- промежутки.
\end{definition}

Пусть $K_1$~--- семейство всех конечных промежутков на прямой. Легко заметить, что это не кольцо (объединение промежутков
--- не всегда промежуток). Отсюда вытекает новая структура:

\begin{definition}
	Семейство $\F\subset \mathcal{P} (X)$ называется \mdef{полукольцом}, если
	\begin{enumerate}[label=\arabic*\degree.]
		\item $\varnothing\in \F$.
		\item $\forall A,\, B\in\F:\: A\cap B\in\F$, а $A\setminus B$ представимо в виде
		      конечного дизъюнктного объединения элементов $\F$, то есть
		      \[\exists n\in\N: \: \exists A_1,A_2,\ldots,A_n\in\F: \: A\setminus B=\bigcup_{i=1}^{n}A_i
			      ,\, A_i\cap A_j=\varnothing,\, \forall i,j: i\neq j.\]
		      Если множества попарно не пересекаются, то вводиться обозначение:
		      \[A_1\cup A_2\cup\ldots\cup A_n=A_1\sqcup A_2\sqcup\ldots\sqcup A_n.\]
	\end{enumerate}
\end{definition}

\begin{remark}
	Для любого семейства множеств $\F$ под $\FDU(\F)$ будем понимать семейство всех конечных дизъюнктных
	объединений элементов $\F$ (FDU~--- finite disjoint union).
\end{remark}

\begin{claim}
	$K_1$~--- полукольцо.
\end{claim}

\begin{definition}
	$K_d=\{I_1\times I_2\times \ldots\times I_d,\, \text{где } I_l\in K_1\quad\forall l = \overline{1..d}\}$, $d\in\N$~--- \mdef{семейство клеток
	в $\R^d$}.
\end{definition}

\begin{claim}
	$K_d$~--- полукольцо.
\end{claim}

\begin{definition}
	Пусть $\F$~--- семейство подмножеств множества $X$.
	Введем обозначение: \[\mathcal{R}(\F):=\bigcap\{\mathcal{G}:\:\mathcal{G}\text{~--- кольцо, } \F\in\mathcal{G} \}:=\bigcap M.\]
	$\mathcal{R}(\F)$ называется \mdef{кольцом порожденным $\F$}.
\end{definition}

\begin{theorem}
	\label{lect1thdef}
	$\mathcal{R}(\F)$ является кольцом и $\F\subset \mathcal{R}(\F)$. При этом $\mathcal{R}(\F)$~--- наименьшее по вложению кольцо $\mathcal{S} $,
	такое что $\F\subset \mathcal{S}$.

	\begin{proof}
		\textbf{Шаг 1.} $\mathcal{R} $~--- кольцо, так как
		\begin{enumerate}
			\item $\varnothing\in \mathcal{R}:\: \forall \mathcal{G}$ выполнено $\varnothing\in\mathcal{G}$.
			\item $\forall A,\, B$ верно, что $A,\, B\in \mathcal{G} $, но $\mathcal{G} $~--- кольцо, поэтому
			      $A\cup B\in\mathcal{G} $ и $A\setminus B\in\mathcal{G}$. Таким образом и пересечению данные множества принадлежат.
		\end{enumerate}

		\textbf{Шаг 2.} Если $\mathcal{S} $~--- кольцо и $\F\subset \mathcal{S} $, то $\mathcal{R} \subset \mathcal{S}$, так как
		$\mathcal{S} \in M\Rightarrow \bigcap M\subset \mathcal{S} .$

	\end{proof}
\end{theorem}

\begin{remark}
	Доказательство основано на том факте, что если $\F$ и $\mathcal{G}$~--- кольца, то $\F\cap \mathcal{G} $~--- тоже кольцо.
\end{remark}

Опишем структуру кольца, порожденного полукольцом.

\begin{theorem}
	Пусть $\mathcal{S} $~--- полукольцо. Тогда
	\[
		\mathcal{R} (\mathcal{S} )=\left\{\bigsqcup_{l=1}^{n}A_l:\: n\in\N,\, A_1,\ldots, A_n\in\mathcal{S},\, A_i\cap A_j=\varnothing:
		\: i\neq j\right\} = \FDU(\mathcal{S} ).
	\]
	\label{lect1th1}
\end{theorem}

\begin{lemma} Ослабим условие теоремы выше. Тогда все равно:
	\[
		\mathcal{R} (\mathcal{S} )=\left\{\bigcup_{l=1}^{n}A_l:\: n\in\N,\, A_i\in\mathcal{S}: \: \forall i=\overline{1..n}\right\}
		\cup\{\varnothing\}.
	\]

	\begin{proof}
		Пусть $R=\left\{\bigcup_{l=1}^{n}A_l:\: n\in\N,\, A_i\in\mathcal{S}: \: \forall i=\overline{1..n}\right\}$. Ясно, что
		$R\subset \mathcal{R} (\mathcal{S} )$. Докажем в обратную сторону. Для этого достаточно доказать, что $R$~--- кольцо, тогда
		сразу выполнится $\mathcal{R} (\mathcal{S} )\subset R$.

		Пустое множество очевидно лежит в $R\cup\{\varnothing\}$.

		Пусть $P=\bigcup\limits_{k=1}^{n}A_k,\, A_k\in\mathcal{S} $ и $Q=\bigcup\limits_{l=1}^{m}B_l,\, B_l\in\mathcal{S} $. Тогда, во-первых,
		\[P\cup Q=A_1\cup A_2\cup\ldots\cup A_n\cup B_1\cup B_2\cup\ldots\cup B_m\in R.\]
		Во-вторых,
		\begin{align*}
			P\setminus Q & =\bigcup_{k=1}^{n}\left(A_k\setminus \bigcup_{l=1}^m B_{l}\right)=
			\bigcup_{k=1}^{n}\left(A_k\bigcap \left(\bigcup_{l=1}^m B_{l}\right)^C\right)=                 \\
			             & =\bigcup_{k=1}^{n}\left(A_k\bigcap \left(\bigcap_{l=1}^m B_{l}^C\right)\right)=
			\bigcup_{k=1}^{n}\bigcap_{l=1}^{m}A_k\cap B_l^C=
			\bigcup_{k=1}^{n}\bigcap_{l=1}^{m}A_k\setminus B_l.
		\end{align*}

		Далее имеем $A_k,\, B_l\in \mathcal{S} \Rightarrow A_k\setminus B_l\in \FDU(\mathcal{S})\Rightarrow A_k\setminus
			B_l=\bigsqcup\limits_{i=1}^{N_{k,l}}S_i\Rightarrow \bigcap\limits_{l=1}^{m}A_k\setminus B_l\in\FDU(\mathcal{S} ).$


	\end{proof}
	\label{lect1lemma1}
\end{lemma}