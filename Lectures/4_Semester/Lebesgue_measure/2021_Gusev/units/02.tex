\newpage
\lecture{2}{Сигма-алгебры.}

\subsection{Разбор задач с лекции 1.}

\textbf{1.} Показать, что любое $\sigma$"=кольцо является $\delta$"=кольцом, а также, что обратное неверно.

\begin{proof}

    $\Rightarrow$ Выразим счетное объеденение через счетное пересечение:
    \[
        \bigcap_{i=1}^{\infty} A_i = A_1 \setminus \bigcup_{i=1}^{\infty}\left(A_1\setminus A_{i}\right)
    \]
    $\Leftarrow$\mdef{Контрпример}: Множество всех ограниченных подмножеств прямой или более формально:
    $\F = \{A\subset \R:\: A\text{~--- ограничено}\}$.

\end{proof}

\begin{remark}
    Можно заметить, что $\delta$"=алгебра является $\sigma$"=алгеброй. Доказательство по существу такое же,
    как выше, но вместо $A_1$ фигурирует единица кольца.
\end{remark}
\textbf{2.} Пусть $\F \subset \mathcal{P} (X),\, \forall A,\, B\in\F:\: A\cap B\in\F\text{ и }A\cup B\in \F$.
Верно ли что $\F$~--- полукольцо?

\begin{proof}
    \textbf{Неверно.}
    \mdef{Контрпример}: $\{\varnothing, \, \{a\}, \, \{a,\, b\}\}$

\end{proof}

\textbf{3.} Пусть $\F \subset \mathcal{P} (X)$~--- кольцо, $E\subset X$, $E\notin \F$. Найти (описать)
кольцо, порожденное семейством $\F\cup E$.

\begin{proof}
    \begin{equation}
        \mathcal{R} (\F\cup E) = \left\{A\setminus E,\, E\setminus A,\, A\cap E,\, A\cup E \ |\  A \in \F\cup E\right\} \cup \F.
        \label{lect1task3}
    \end{equation}
    
    Обозначим множество, простроенное в формуле \eqref{lect1task3} $\mathcal{G}$.
    $\mathcal{G} $ является кольцом, так как:
    \begin{enumerate}
        \item $\varnothing\in\F\Rightarrow\varnothing\in \mathcal{G} $;
        \item Замкнутость для любых $A,\, B\in \F$~--- очевидна. Замкнутость для любого $A\in\F$ и $E$ выполняется благодаря
        левой части.
    \end{enumerate}

    Минимальность \mdef{очевидна}. @О.В. Бесов.

\end{proof}

\subsection{Теорема о структуре кольца, порожденного полукольцом.}

\begin{remark}
    Если $\F\subset \mathcal{G} $, то $\mathcal{R} (\F)\subset \mathcal{R} (\mathcal{G} )$.

    Данное замечание можно использовать для упрощения доказательства леммы (1.15).%\ref{lect1lemma1}.
\end{remark}

\begin{lemma}
    Пусть $\CS\subset \CP(X)$~--- полукольцо и $A,\, A_1,\, \ldots,\, A_n\in \CS$. 
    Тогда имеем $A\setminus \bigcup\limits_{i=1}^n A_i\in\FDU(\CS)$. Другими словами 
    $\exists B_1,\,\ldots, \, B_m\in\CS:\: B_i\cap B_j=\varnothing \text{ при } i\neq j$ и 
    $A\setminus \bigcup\limits_{i=1}^n A_i=\bigsqcup\limits_{j=1}^m B_j$. 

    \begin{proof}
        Докажем по индукции по $n$.
        
        При $n=1$ утверждение следует из определения полукольца.

        Допустим утверждение выполнение для $n$, а именно, что 
        $\exists B_1,\,\ldots, \, B_k\in\CS:\: B_i\cap B_j=\varnothing \text{ при } i\neq j$
        и $A\setminus \bigcup\limits_{i=1}^{n}A_i=\bigsqcup\limits_{j=1}^k B_j$. Докажем для $n+1$.
        \[
            A\setminus \bigcup_{i=1}^{n+1}A_i=\left(A\setminus \bigcup_{i=1}^{n}A_i\right)\setminus A_{n+1}=
            \bigsqcup_{i=1}^k \left(B_i\setminus A_{n+1}\right).
        \]
        Далее по определению полукольца имеем: $B_i\setminus A_{n+1}=\bigcup\limits_{l=1}^{N_i}C_{il}$, 
        где $C_{il}\in\CS$ и $B_i\cap B_j=\varnothing,\, i\neq j\Rightarrow C_{il}\cap C_{js}=\varnothing$ если 
        $(i,\, l)\neq (j,\, s)$. Значит можно продолжить 
        \[ 
            \bigsqcup_{i=1}^k \left(B_i\setminus A_{n+1}\right)=\bigsqcup_{i=1}^k \bigsqcup_{l=1}^{N_{i}} C_{il}. 
        \]

        Проще говоря, множество $B_i\setminus A_{n+1}$ можно дизъюнктно разбить на некоторые множества $C_{il}$. И 
        такие можества не будут пересекаться, а значит их дизъюнктное объеденение и даст требуемое.

    \end{proof}
    \label{lect2lemma1}
\end{lemma}

\begin{lemma}
    Пусть $\CS\subset \CP(X)$~--- полукольцо. 
    Если $A_1,\, \ldots,\, A_n\in\CS$, то $\bigcup\limits_{i=1}^nA_i\in\FDU(\CS)$.

    \begin{proof}
        Введем обозначения:
        \begin{align*}
            &B_1 = A_1\in\CS\\
            &B_2 = A_2\setminus A_1\in\FDU(\CS)\text{ (по определению полукольца)}\\
            &\ldots\\
            &B_{k+1}=A_{k+1}\setminus\bigcup_{i=1}^k A_i\in\FDU(\CS)\text{ (по лемме \ref{lect2lemma1})}
        \end{align*}

        По построению $B_i\cap B_j=\varnothing,\, i\neq j$, поэтому 
        $\bigsqcup\limits_{i=1}^n B_i\in\FDU(\CS)$.

    \end{proof}
    \label{lect2lemma2}
\end{lemma}

\begin{remark}
    Если $\F\subset\CP(X)$, то $\forall B_1,\,\ldots,\, B_n\in\FDU(\F)$ таких, что 
    $B_i\cap B_j=\varnothing,\, i\neq j$ выполнено $\bigsqcup\limits_{i=1}^n B_i\in\FDU(\F)$.
\end{remark}

\begin{next0}
    Теорема 1.14.%\ref{lect1th1}.

    \begin{proof}
        По лемме 1.15
        %\ref{lect1lemma1} 
        $A\in\CR(\CS)\Leftrightarrow \exists A_1,\,\ldots,\,A_n\in\CS:\:
        A=\bigcup\limits_{i=1}^n A_i$. И по лемме \ref{lect2lemma2} $\exists$ попарно не пересекающиеся 
        $B_1,\,\ldots,\, B_m\in\CS:\: \bigcup\limits_{i=1}^n A_i=\bigsqcup\limits_{j=1}^m B_j$.

    \end{proof}
\end{next0}

\subsection{Сигма-алгебра.}

Напомним основное понятие:
\begin{definition}
    $\sigma$"=кольцо $\F\subset\CP(X)$ называется $\sigma$"=алгеброй, если $X\in\F$.
\end{definition}

\begin{claim}[Критерий $\sigma$"=алгебры]
    Семейство $\F\subset \CP(X)$ является $\sigma$"=алгеброй тогда и только тогда, когда выполнены следующие свойства: 
    \begin{enumerate}[label=\arabic*\degree.]
        \item $\varnothing\in\F$;
        \item $\forall A\in\F \quad A^C=X\setminus A\in\F$;
        \item $\forall A,\, B\in\F\quad A\cap B\in\F$;
        \item $\forall$ дизъюнктного $\{A_k\}_{k=1}^{\infty}\subset \F$ (то есть $A_k\in\F, \, \forall k$ и
        $A_i\cap A_j=\varnothing,\, \forall i,j,\, i\neq j$) выполнено $\bigsqcup\limits_{k=1}^{\infty}A_k\in\F$.
    \end{enumerate}

    \begin{proof}
        \circled{$\Rightarrow$} Очевидно из определения $\sigma$"=алгебры.

        \circled{$\Leftarrow$} Докажем, что из пунктов $1\degree, \, 2\degree,\, 3\degree$ следует, что $\F$~--- алгебра.
        \begin{itemize}
            \item $\varnothing\in\F$ и $X = \varnothing^C\Rightarrow X\in\F$;
            \item $A\setminus B = A\cap B^C\in\F$ (пункты $1\degree$ и $2\degree$).
        \end{itemize}
        \begin{remark}
            Доказывать замкнутость по объединению не нужно в силу замечания об эквивалентности условий из лекции 1.%(\ref{lect1remark1}).
        \end{remark}

        Осталось доказать, что при условии $4\degree$, то $\F$~--- $\sigma$"=алгебра. Пусть 
        $\{A_n\}_{n\in\N}\subset \F$. Докажем, что $\bigcup\limits_{n=1}^{\infty}\in\F$.
        Рассмотрим:
        \begin{align*}
            &B_1 = A_1\in\CS\\
            &B_2 = A_2\setminus A_1\\
            &\ldots\\
            &B_{k+1}=A_{k+1}\setminus\bigcup_{i=1}^k A_i
        \end{align*}
        Так как было уже доказано, что $\F$~--- алгебра, то все $B_i\in\F$. По построению $B_i\cap B_j=\varnothing$ при
        $i\neq j\Rightarrow$ по пункту $4\degree$ имеем $\bigsqcup\limits_{n=1}^{\infty}B_n\in\F$, с другой стороны 
        $\bigsqcup\limits_{n=1}^{\infty}B_n = \bigcup\limits_{n=1}^{\infty}A_n$.

    \end{proof}
\end{claim}

\begin{definition}
    Пусть $\F\subset \CP(X)$. Сигма-алгеброй, порожденной семейством $\F$, называется семейство 
    \[
        \sigma(\F)=\bigcap\left\{\CG\ |\ \CG\subset\CP(X) \text{~--- } \sigma\text{-алгебра такая, что } \F\subset
        \CG\right\}    
    \]
\end{definition}

\begin{remark}
    Доказательство того, что построенное множество является $\sigma$"=алгеброй по существу аналогично доказательству 
    теоремы 1.13.
    %\ref{lect1thdef}
    Причем $\sigma(\F)$~--- минимальная по вложению $\sigma$"=алгебра $\CG$ такая, что 
    $\F\subset\CG$.
\end{remark}

\begin{definition}
    Пусть $\tau=\{U\subset \R^d:\: U\text{~--- открыто}\}$ (то есть $\tau$~--- евклидова топология).
    
    Множество называется открытым если вместе с любой своей точкой содержит некоторый шарик радиуса $r$, где под
    расстоянием понимается евклидова метрика: $\rho_2(x,\, y) = \sqrt{\sum\limits_{i=1}^d |x_i-y_i|^2}$.
\end{definition}

\begin{definition}
    $\sigma(\tau)$ называется борелевской $\sigma$"=алгеброй. Обозначение $\mathfrak{B} (\R^d)$.
\end{definition}

\begin{remark}
    Аналогичное определение можно ввести для любого метрического пространства.
\end{remark}

\begin{claim}
    Если $K_d$~--- семейство всех клеток в $\R^d$, то $\mathfrak{B} (\R^d)=\sigma(K_d)$.
\end{claim}