\setcounter{section}{11}
\section{Глава 12. Ряды Фурье.}
\subsection{Коэффициенты Фурье.}
Рассматриваем функции, суммируемые с квадратом на промежутке $I\subset \R$ --- это измеримые на $I$ функции, такие что $f^2\in L(I)$, где $L(I)$ --- множество суммируемых на $I$ функций.

\begin{prop}
	Множество суммируемых с квадратом на $I$ функций образуют линейное пространство, причем произведение любых двух таких функций суммируемо на $L(I)$.
\end{prop}

\begin{proof}
	Пусть $f_1, f_2$ суммируемы с квадратом, тогда $|f_1(x)\cdot f_2(x)|\leqslant\dfrac{1}{2} \left({f_1}^2(x)+{f_2}^2(x)\right)$. Произведение измеримых функций является измеримой функцией, это произведение по модулю не превосходит суммируемой функции, тогда по признаку суммируемости, это произведение является измеримой функцией. Для доказательства линейности пространства, докажем, что $f_1(x)+f_2(x)$ тоже суммируема с квадратом. Сумма измеримых функций --- измерима. $(f_1(x)+f_2(x))^2={f_1}^2(x)+2\cdot f_1(x)\cdot f_2(x) + {f_2}^2(x)$ --- каждое слагаемое суммируемое, значит и все выражение суммируемо.
\end{proof}

Попытаемся ввести скалярное произведение двух функций, как $$<f_1, f_2>=\int\limits_{I}f_1(x)\cdot f_2(x)d\mu(x).$$ При проверке свойств скалярного произведения, получим, что свойство $<f,f>=0\Leftrightarrow f=0$ не выполняется, так как $<f,f>=0\Leftrightarrow \int\limits_{I}f^2(x)\cdot d\mu(x)\Leftrightarrow f=0$ почти всюду. Поэтому необходимо ввести отношение эквивалентности и отождествить функции которые совпадают почти всюду.

\begin{Def}
	$L_2(I)$ --- множество суммируемых с квадратом функций, с отождествлением эквивалентных (то есть совпадающих почти всюду на $I$) функций. Причем, оно является евклидовым пространством.
\end{Def}

Теперь, к примеру, функции Дирихле или Римана не отличимы от тождественного нуля.
\begin{prop}[Неравенство Коши-Буняковского-Шварца]\ \\
	$|<f_1,f_2>|^2\leqslant <f_1,f_1>\cdot<f_2,f_2>$
	
	$\left(\int\limits_{I}f_1(x)\cdot f_2(x)d\mu(x)\right)^2\leqslant\int\limits_{I}{f_1}^2(x)d\mu(x)\cdot \int\limits_{I}{f_2}^2(x)d\mu(x)$.
\end{prop}

\begin{Def}
	Ортогональная система функций из $L_2(I)$ --- это последовательность $\{e_n(x)\}_{n=1}^\infty$ ненулевых элементов $L_2(I)$, такая что $<e_n,e_m>=0, m\ne m$. Если при этом $<e_n,e_n>=1, n=1,\ldots$, то $\{e_n(x)\}_{n=1}^\infty$ называется ортонормированной системой.
\end{Def}

\begin{example}[ортогональных систем]\ 
	\begin{enumerate}
		\item $I=[-1,1]$ Многочлены Лежандра $P_n(x)=\dfrac{1}{n!\cdot2^n}\dfrac{d^n}{dx^n}\left((x^2-1)^n\right), P_0(x)=1$.
		\begin{proof}(Первый вариант доказательства)
			Необходимо проверить, что при $n>k\geqslant 0 \Rightarrow <P_n, P_k>=0$. 
			\begin{multline*}
				\int\limits_{-1}^1\dfrac{d^n}{dx^n}\left((x^2-1)^n\right)\dfrac{d^k}{dx^k}\left((x^2-1)^k\right)dx=[\text{интегрируем по частям}]=\\=\underbrace{\dfrac{d^{n-1}}{dx^{n-1}}((x^2-1)^n)\dfrac{d^k}{dx^k}((x^2-1)^k)|_{-1}^1}_{=0, \text{см. дальше}}-\int\limits_{-1}^1\dfrac{d^{n-1}}{dx^{n-1}}((x^2-1)^n)\dfrac{d^{k+1}}{dx^{k+1}}((x^2-1)^k)dx.
			\end{multline*}
			Заметим, что $(x^2-1)^n$ --- представляет собой многочлен степени $2n$. Раскладывая многочлен по формуле Тейлора с центром в единице, получим конструкцию $(x-1)^n\cdot(\ldots)$. А значит все производные от $(x^2-1)^n$ в единичке, вплоть до $n-1$ порядка, равны нуля. Аналогично в $-1$. Продолжая интегрирование по частям получим: $$(-1)^n\int\limits_{-1}^1(x^2-1)^n\dfrac{d^{n+k}}{dx^{n+k}}((x^2-1)^k)dx.$$
			Так как $n>k$, то $n+k>2k$ и производная $n+k$ степени от $(x^2-1)^k$ равна тождественному нулю.
		\end{proof}
		\begin{proof}(Второй вариант доказательства)
			Необходимо проверить, что при $n>k\geqslant 0 \Rightarrow <P_n, P_k>=0 \\
			\mathcal{J}=\int\limits_{-1}^1\dfrac{d^n}{dx^n}\left((x^2-1)^n\right)\dfrac{d^k}{dx^k}\left((x^2-1)^k\right)dx=0$.
			
			Введем вспомогательное утверждение: $\dfrac{d^j}{dx^j}((x^2-1)^n)=(x-1)^{n-j}g_j(x)$, где $g_j(x)$ --- многочлены степени $n$, для $j=0,1,\ldots, n$. Доказательство по индукции. 
			
			При $j=0$ никаких производных не берем, $g_0(x)=(x+1)^n$. 
			
			$\dfrac{d^{j+1}}{dx^{j+1}}((x^2-1)^n)=\left((x-1)^{n-j}g_j(x)\right)'=(n-j)(x-1)^{n-j-1}g_j(x)+(x-1)^{n-j}g_j'(x)=(x-1)^{n-(j+1)}((n-j)g_j(x)+(x-1)g_j'(x))$. В качестве $g_{j+1}(x)=(n-j)g_j(x)+(x-1)g_j'(x)$.
			
			Следующее утверждение будет также справедливо: $\dfrac{d^j}{dx^j}((x^2-1)^n)=(x+1)^{n-j}g_j(x)$, где $g_j(x)$ --- многочлены степени $n$, для $j=0,1,\ldots, n$.
			
			Возвращаясь к доказательству получаем, что
			
			$\dfrac{d^{j}}{dx^j}((x^2-1)^n)|_{x=\pm1}=0, j=0,\ldots,n-1$. Интегрируя по частям $\mathcal{J}$, загоняя первый сомножитель под дифференциал, получаем: 
			\begin{multline*}
				\mathcal{J}=-\int\limits_{-1}^1\dfrac{d^{n-1}}{dx^{n-1}}((x^2-1)^n)\dfrac{d^{k+1}}{dx^{k+1}}((x^2-1)^k)dx=\overset{n \text{ раз интегрируя по частям}}{\ldots}=\\=(-1)^n\int\limits_{-1}^1(x^2-1)^n\dfrac{d^{k+1}}{dx^{k+n}}((x^2-1)^k)dx=0,
			\end{multline*}
			так как берем производную порядка больше чем степень многочлена:$k+n>2\cdot  k$.
		\end{proof}
		\item Стандартная тригонометрическая система. $I=[a,a+2\pi]$. Система функций: $\dfrac{1}{2}, cos(x),sin(x),cos(2x),sin(2x),\ldots$.
		\begin{proof}
			Необходимо проверить пять типов равенств:
			
			$\int\limits_{I}\cos(nx)\cos(mx)dx=0, m\ne n.$
			
			$\int\limits_{I}\sin(nx)\sin(mx)dx=0, m\ne n.$
			
			$\int\limits_{I}\cos(nx)\sin(mx)dx=0.$
			
			$\int\limits_{I}\frac{1}{2}\cos(nx)dx=0.$
			
			$\int\limits_{I}\frac{1}{2}\sin(nx)dx=0.$
			
			Проверим
			\begin{multline*}
				\int\limits_{-\pi}^{\pi}\cos(nx)\cos(mx)dx=\frac{1}{2}\int\limits_{-\pi}^{\pi}(\cos((m+n)x)+\cos((m-n)x))dx=\\= \frac{1}{2(m+n)}\sin((m+n)x)|_{-\pi}^{\pi}+\frac{1}{2(m-n)}\sin((m-n)x)|_{-\pi}^{\pi}=0, m\ne n
			\end{multline*} 
		\end{proof}
	\end{enumerate}
\end{example}

\begin{Def}
	Если $\{e_n(x)\}_{n=1}^\infty$ --- ортогональная система в $L_2(I)$, то $\forall f\in L_2(I)$ коэффициентами Фурье называются числа $\alpha_n=\dfrac{<f,e_n>}{<e_n,e_n>}$. Функциональный ряд $\sum\limits_{n=1}^\infty\alpha_n e_n(x)$ называется рядом Фурье функции $f$ по системе $\{e_n(x)\}_{n=1}^\infty$.
\end{Def}

Рассмотрим ряды Фурье в стандартной тригонометрической системе: 
$$\begin{aligned}
	a_n&=\dfrac{\int\limits_I f(x)\cos nx d\mu(x)}{\int\limits_I \cos^2nxdx}=\dfrac{1}{\pi}\int\limits_{[-\pi,\pi]}f(x)\cos nx d\mu(x), n=1,2,\ldots\\
	a_0&=\dfrac{\int\limits_I \dfrac{1}{2}f(x) d\mu(x)}{\int\limits_I {\left(\dfrac{1}{2}\right)}^2dx}=\dfrac{1}{\pi}\int\limits_{[-\pi,\pi]}f(x) d\mu(x)\\
	b_n&=\dfrac{\int\limits_{I}f(x)\sin nx d\mu(x)}{\int\limits_{I}\sin^2nxdx}=\dfrac{1}{\pi}\int\limits_{[-\pi,\pi]}f(x)\sin nxd\mu(x), n=1,2,\ldots
\end{aligned}
\eqno(1)
$$

Тригонометрический ряд Фурье:
$$f(x)\sim \dfrac{a_0}{2}+\sum\limits_{n=1}^\infty(a_n\cos nx+b_n\sin nx)
\eqno(2)
$$

Предположение о том что $f\in L_2$ в тригонометрической системы излишнее, достаточно предполагать, что
$f\in L_1[-\pi,\pi]$, где $L_1[-\pi,\pi]$ --- линейное нормированное пространство, состоящее из классов эквивалентностых суммируемых на $[-\pi,\pi]$ функций, с нормой 
$\parallel f \parallel_{L_1[-\pi,\pi]}=\int\limits_{[-\pi,\pi]}|f(x)|d\mu(x),$ так как если функция $f(x)$ --- суммируемая, то по признаку суммируемости $|f(x)\cos nx|\leqslant |f(x)|\Rightarrow f(x)\cos nx$ --- тоже суммируемая функция.

В случае не $2\pi$-периодичности, а $2l$-периодичности --- производим замену: растяжения или сжатия.
$$
\begin{aligned}
	a_n&=\dfrac{1}{l}\int\limits_{[-l,l]}f(x)\cos \dfrac{n\pi x}{l}d\mu(x), n=0,1,\ldots\\
	b_n&=\dfrac{1}{l}\int\limits_{[-l,l]}f(x)\sin \dfrac{n\pi x}{l}d\mu(x), n=1,2,\ldots\\
	f(x)&\sim \dfrac{a_0}{2}+\sum\limits_{n=1}^\infty\left(a_n\cos \dfrac{n\pi x}{l}+b_n\sin \dfrac{n\pi x}{l}\right).
\end{aligned}$$

\begin{Def}
	Носителем функции $f(x)$ называется замыкание множества тех $x$, для которых $f(x)\ne 0$ и обозначается $\textrm{supp} f$. Функции с ограниченным носителем называются финитными.
\end{Def}

\begin{lemma}
	Множество\label{lemma_1} непрерывных финитных функций всюду плотно в $L_1(\R)$ $($то есть его замыкание совпадает с $L_1(\R))$.
\end{lemma}
\begin{proof}
	Заменим $\R$ на отрезок $[a,b]$. То есть докажем, что множество непрерывных на $[a,b]$ функций всюду плотно в $L_1[a,b]$. Имеем суммируемую функцию $f(x)\in L_1[a,b]$. Всюду плотность означает, что ${(\forall\varepsilon >0)} {(\exists g\in C[a,b])} \parallel f-g\parallel_{L_1[a,b]}<\varepsilon$.
	
  Любая суммируемая функция представима в виде разности двух неотрицательных суммируемых функций. Значит, задача сводится к доказательству утверждения для неотрицательных суммируемых функций.
 
 Напомню, что срезка $f_{[N]} (x) := \begin{cases}
 	f(x),\ f(x)\leqslant N\\
 	N,\ f(x) > N
 \end{cases}$ --- ограниченная функция и для неотрицательной суммируемой функции верно, что $\int\limits_{[a,b]}f(x)d\mu(x)=\lim\limits_{N\to\infty}\int\limits_{[a,b]}f_{[N]}d\mu(x)$.  Отсюда следует, что интеграл от функции очень мало отличается от интеграла от срезки, при достаточно большом $N$. Значит задача свелась к доказательству утверждения для ограниченной неотрицательной суммируемой функции.
 
 По теореме о представлении неотрицательной измеримой функции пределом последовательности ступенчатых функций, $\exists \{h_n\}\  h_n \uparrow f$ ($h_n$ неубывая стремятся к $f$). В этой теореме не предполагается, что $f(x)$ ограниченная, а в нашем случае $f(x)$ ограниченная и следовательно, $h_n(x)$ принимают конечное число значений. По теореме Леви: 
		$	\int\limits_{[a,b]} h_n(x)d\mu(x)\rightarrow \int\limits_{[a,b]}f(x)d\mu(x)\Rightarrow
			\int\limits_{[a,b]}|f(x)-h_n(x)|d\mu(x)=\int\limits_{[a,b]}(f(x)-h_n(x))d\mu(x)\underset{n\to\infty}{\to}0.$
Значит наша задача свелась к доказательству утверждения для ступенчатых функций с конечным множеством значений.

Любая ступенчатая $h_n(x)=\sum\limits_{k=1}^N C_k\chi_{E_k}(x)$, где $C_k$ --- некоторый коэффициент, $\chi_{E_k}$ --- характеристическая функция измеримого множества $E_k$. Значит нам достаточно приблизить любую характеристическую функцию измеримого множества непрерывными.
		
По критерию измеримости по Лебегу $(\forall\varepsilon>0)\ (\exists M_\varepsilon\subset[a,b])\  \mu(E\Delta M_\varepsilon)<\varepsilon$. Тогда ${\parallel \chi_E - \chi_{M_\varepsilon}\parallel_{L_1([a,b])}=\int\limits_{[a,b]}|\chi_{E}(x)-\chi_{M_\varepsilon}(x)|d\mu(x)=\int\limits_{[a,b]}\chi_{E\Delta M_\varepsilon}(x)d\mu(x)=\mu(E\Delta M_\varepsilon)<\varepsilon}$. Задача свелась к случаю характеристической функции элементарного множества.
		
Характеристическая функция элементарного множества --- это сумма характеристических функций одномерных брусьев, то есть промежутков. Поскольку включение концов не имеет значения, можно считать $M_\varepsilon$ интервалом, то есть мы должны научиться приближать характеристическую функцию интервала, а это достаточно легко сделать: 
\begin{figure}[h]
	\begin{center}
		\begin{tikzpicture}[
			scale=4,
					axis/.style={very thick, ->, >=stealth'},
					important line/.style={thick},
					funk/.style={color=red, very thick},
					dashed line/.style={dashed, thin},
					pile/.style={thick, ->, >=stealth', shorten <=2pt, shorten
						>=2pt},
					every node/.style={color=black}
					]
					
					\draw[axis] (-0.1,0)  -- (1.1,0) node(xline)[right] {x};
					\draw[axis] (0,-0.1) -- (0,0.9) node(yline)[above] {y};
					% Lines
					\draw[funk] (.15,.0) coordinate (A) -- (.30,.55)
					coordinate (B) node[right, text width=5em] {};
					\draw[funk] (.70,.55) coordinate (C) -- (.85,.0)
					coordinate (D) node[right, text width=5em] {};
					\draw[important line] (.15,.55) coordinate (C) -- (.85,.55)
					coordinate (D) node[right, text width=5em] {};
					\draw[important line] (-.03,.55) coordinate (C) -- (.03,.55)
					coordinate (D) node[right, text width=5em] {};
					\draw[funk] (-0.1,.0) coordinate (C) -- (.15,.0)
					coordinate (D) node[right, text width=5em] {};
					\draw[funk] (.30,.55) coordinate (C) -- (.70,.55)
					coordinate (D) node[right, text width=5em] {};
					\draw[funk] (.85,.0) coordinate (C) -- (1.0,.0)
					coordinate (D) node[right, text width=5em] {};
					\draw	(.15,-0.05) node[anchor=north] {c};
					\draw	(-0.1,.58) node[anchor=north] {1};
					\draw	(.30,0) node[anchor=north] {c+$\frac{1}{m}$};
					\draw	(.70,0) node[anchor=north] {d-$\frac{1}{m}$};
					\draw	(.85,-0.05) node[anchor=north] {d};
					\draw	(.85,0.07) node[anchor=north] {)};
					\draw	(.15,0.07) node[anchor=north] {(};
					\draw[dotted] (.30,.0) -- (.30,.55);
					\draw[dotted] (.70,.0) -- (.70,.55);
					\node[draw, scale=0.7] at (2.0,0.60) {
						$\begin{aligned}
							\varphi (x) &:= \begin{cases}
								1,\ x\in [c+\frac{1}{m},d-\frac{1}{m}]\\
								0,\ x \not\in [c+\frac{1}{m},d-\frac{1}{m}]\\
								\text{линейна на }[c,c+\frac{1}{m}]\text{ и }[d-\frac{1}{m},d]\\
							\end{cases}\\
							\chi_{(c,d)} (x) &:= \begin{cases}
								1,\ x\in (c,d)\\
								0,\ x \not\in (c,d)
							\end{cases}
						\end{aligned}$
					
				};
				\end{tikzpicture}
			\end{center}
		\end{figure}
	рассмотрим промежуток $(c,d)\subset[a,b],\ \varphi(x)$ --- непрерывная функция.\\
	Тогда $\int\limits_{[a,b]} |\chi_{(c,d)}(x)-\varphi(x)|d\mu(x)=\dfrac{2}{m}<\varepsilon$ --- при достаточно большом $m$, этот инеграл может быть сколь угодно маленьким. Возвращаясь по цепочке, получаем, что с помощью непрерывных функций в метрике $L_1$, мы можем приблизить любую финитную функцию.
	
	Теперь рассмотрим общий случай: $f(x)\in L_1(\R)$. По определению интеграла Лебега по множеству бесконечной меры
	$\int\limits_{\R}|f(x)|d\mu(x)=\lim\limits_{N\to\infty}\int\limits_{[-N,N]}|f(x)|d\mu(x)$. Выберем такое $N$, что $\int\limits_{\R\backslash[-N,N]}|f(x)|d\mu(x)<\dfrac{\varepsilon}{3}$.
	
	 На отрезке $[-N,N]$ по доказанному $(\exists g(x)\in C[-N,N]) \parallel f-g\parallel_{L_1[-N,N]}<\dfrac{\varepsilon}{3}$. Доопределим $g(x)$ как показано на рисунке.
	 
	 \begin{figure}[h]
	 	\begin{center}
	 		\begin{tikzpicture}[
	 			scale=4,
	 			axis/.style={very thick, ->, >=stealth'},
	 			important line/.style={thick},
	 			funk/.style={color=red, very thick},
	 			dashed line/.style={dashed, thin},
	 			pile/.style={thick, ->, >=stealth', shorten <=2pt, shorten
	 				>=2pt},
	 			every node/.style={color=black}
	 			]
	 			
	 			\draw[axis] (-1.2,0)  -- (1.1,0) node(xline)[right] {$x$};
	 			\draw[axis] (0,-0.3) -- (0,0.9) node(yline)[above] {$y$};
	 			% Lines
	 			\draw[thick] plot [smooth,tension=1] coordinates{(-.7,-.4) (-.6,-.3) (-.4, -.3) (-.2,.4) (.1,.5) (.3, .4) (.6,.35)};
	 			\draw	(-.7,-0.05) node[anchor=north] {$-N$};
	 			\draw	(.6,0.17) node[anchor=north] {$N$};
	 			\draw	(-1,0.17) node[anchor=north] {$(-N-\gamma)$};
	 			\draw	(.8,-0.05) node[anchor=north] {($N+\gamma)$};
	 			\draw[thick] plot [smooth,tension=1] coordinates{(-.7, -0.03) (-.7, 0.03)};
	 			\draw[thick] plot [smooth,tension=1] coordinates{(.6, -0.03) (.6, 0.03)};
	 			\draw[thick] plot [smooth,tension=1] coordinates{(-1.0, -0.03) (-1.0, 0.03)};
	 			\draw[thick] plot [smooth,tension=1] coordinates{(.8, -0.03) (.8, 0.03)};
	 			\draw	(.2,0.7) node[anchor=north] {$g(x)$};
	 			\draw[thick] plot [smooth,tension=1] coordinates{(-.7,-.4) (-1., 0)};
	 			\draw[thick] plot [smooth,tension=1] coordinates{(.6,.35) (.8, 0)};
	 			\node[draw, scale=0.9] at (2.0,0.60) {
	 				$\begin{aligned}
	 					g (x) &:= \begin{cases}
	 						g(x),\ x\in [-N,N]\\
	 						0,\ x\not\in [-N,N]\\
	 						\text{линейна на }[-N-\gamma,-N]\text{ и }[N,N+\gamma]\\
	 					\end{cases}
	 				\end{aligned}$
	 			};
	 		\end{tikzpicture}
	 	\end{center}
	 \end{figure}
	 За счет выбора $\gamma$, получим $\int\limits_{\R\backslash[-N,N]}|g(x)|d\mu(x) =\text{площади двух треугольников} <\dfrac{\varepsilon}{3}$. Тогда
	 $\int\limits_{\R}|f(x)-g(x)|d\mu(x)\leqslant \int\limits_{[-N,N]}|f(x)-g(x)|d\mu(x)+\int\limits_{\R\backslash[-N,N]}(|f(x)|+|g(x)|)d\mu(x)<\varepsilon$.
\end{proof}

\begin{lemma}
	Каждая\label{lemma_2} суммируемая на $\R$ функция $f(x)$ непрерывна в среднем относительно сдвига, то есть $\lim\limits_{t\to 0}\int\limits_{\R}|f(x+t)-f(x)|d\mu(x)=0$.
\end{lemma}
\begin{proof}
	Докажем, что для $f\in L_1[a,b]$: $\lim\limits_{\delta\to 0}\sup\limits_{0\leqslant h\leqslant \delta}\int\limits_{a}^{b-h}|f(x+h)-f(x)|d\mu(x)=0$.\\
	Из \hyperref[lemma_1]{Леммы 1} следует $(\forall\varepsilon >0)(\exists g\in C[a,b])\parallel f-g\parallel_{L_1[a,b]}=\int\limits_{a}^b|f(x)-g(x)|d\mu(x)<\dfrac{\varepsilon}{3}$.
	
	Т. к. $g\in C[a,b]\overset{\text{т. Кантора}}{\Rightarrow} (\forall\varepsilon>0)(\exists\delta>0)(\forall x_1,x_2\in[a,b],|x_1-x_2|\leqslant\delta)\ |g(x_1)-g(x_2)|<\frac{\varepsilon}{3\cdot (b-a)}$.
	
	Тогда $\forall h, 0\leqslant h \leqslant \delta$
	\begin{multline*}
		\int\limits_{a}^{b-h}|f(x+h)-f(x)|d\mu(x)\leqslant \underbrace{\int\limits_{a}^{b-h}|f(x+h)-g(x+h)|d\mu(x)}_{<\dfrac{\varepsilon}{3}\text{, см. пояснение}}+\\+
		\int\limits_{a}^{b-h}\underbrace{|g(x+h)-g(x)|}_{<\dfrac{\varepsilon}{3\cdot (b-a)}}d\mu(x)+
		\underbrace{\int\limits_{a}^{b-h}|g(x)-f(x)|d\mu(x)}_{<\dfrac{\varepsilon}{3}\text{, см. пояснение}}<\varepsilon. 
	\end{multline*}
Пояснение: оба эти интеграла меньше нормы $\parallel f-g\parallel_{L_1[a,b]}$, так как в ней интегрирование происходит по всему отрезку $[a,b]$, а в интегралах по части отрезка $[a,b]$.

Пусть $f\in L_1(\R)$. Из определения интеграла Лебега по множеству бесконечной меры $\Rightarrow\exists N \int\limits_{|x|\geqslant N}|f(x)|d\mu(x)<\varepsilon$. Рассмотрим сужение функции $f: g(x)=f(x)\cdot \chi_{[-N-1,N+1]}(x)$. Положив $a=-N-1, b=N+1$, по доказанному, получаем, что $$\forall\varepsilon>0\exists\delta\in(0,1)\sup\limits_{0\leqslant h\leqslant \delta}\int\limits_{[-N-1,N+1-h]}|g(x+h)-g(x)|d\mu(x)<\varepsilon.$$ При $|t|<\delta$ 
\begin{multline*}
	\int\limits_{\R}|f(x+t)-f(x)|d\mu(x)=\underbrace{\int\limits_{\{x,x+t\}\subset[-N-1,N+1]}|g(x+t)-g(x)|d\mu(x)}_{<\varepsilon}+\\+
	\underbrace{\int\limits_{\{x,x+t\}\not\subset[-N-1,N+1]}|f(x+t)-f(x)|d\mu(x)}_{<2\varepsilon, \text{ см. пояснение}}<3\varepsilon.
\end{multline*}
Пояснение: какая-то точка из множества $\{x,x+t\}$ не попала в отрезок $[-N-1,N+1]$, другая точка отстоит от нее на $t$, а $|t|<\delta, \delta\in(0,1)$. Значит обе точки не попали в отрезок $[-N,N]$. Следовательно можно воспользоваться оценкой $\int\limits_{|x|\geqslant N}|f(x)|d\mu(x)<\varepsilon$ и тем, что интеграл от модуля разности не превосходит суммы интегралов от модуля каждого.
\end{proof}

\begin{theorem}[Римана об осцилляции]
	Если $f\in L_1(I)$ ($I$ --- конечный или бесконечный промежуток), то $\lim\limits_{\lambda\to\infty}\int\limits_{I}f(x)\cos\lambda xd\mu(x)=\lim\limits_{\lambda\to\infty}\int\limits_{I}f(x)\sin\lambda xd\mu(x)=0$.
\end{theorem}

\begin{proof}
	Пусть $f$ определена на $I$. Доопределим $f(x)=0, x\in(\R\backslash I)$. Пусть $\mathcal{J}=\int\limits_{\R}f(x)\cos\lambda xd\mu(x)$. Сделаем замену $x=t+\frac{\pi}{\lambda}$. 
	Тогда $\mathcal{J}=\int\limits_{\R}f(t+\frac{\pi}{\lambda})\cos(\lambda t+\pi)d\mu(t)=\\=-\int\limits_{\R}f(x+\frac{\pi}{\lambda})\cos(\lambda x)d\mu(x)=-\frac{1}{2}\int\limits_{\R}(f(x+\frac{\pi}{\lambda})-f(x))\cos\lambda xd\mu(x)$. Следовательно \\$|\mathcal{J}|\leqslant \int\limits_{\R}|f(x+\frac{\pi}{\lambda})-f(x)|d\mu(x)\to 0$, при $\lambda\to\infty$ по \hyperref[lemma_2]{Лемме 2}.
\end{proof}

\begin{corollary}
	Если $f\in L_1[-\pi,\pi], 2\pi$-периодическая, то ее коэффициенты Фурье образуют бесконечно малые последовательности, то есть $\lim\limits_{n\to\infty}a_n=\lim\limits_{n\to\infty}b_n=0$.
\end{corollary}

Если $f$ --- четная, $f\in L_1[-\pi,\pi]\Rightarrow$
$$\begin{aligned}
 	a_n&=\dfrac{2}{\pi}\int\limits_{0}^{\pi}f(x)\cos nxd\mu(x), n=0,1,2,\ldots\\
 	b_n&=0, n=1,2,\ldots
 \end{aligned}
 \eqno(3)
 $$
Если $f$ --- нечетная, $f\in L_1[-\pi,\pi]\Rightarrow$
$$\begin{aligned}
	b_n&=\dfrac{2}{\pi}\int\limits_{0}^{\pi}f(x)\sin nxd\mu(x), n=1,2,\ldots\\
	a_n&=0, n=0,1,\ldots
\end{aligned}
\eqno(4)
$$

\subsubsection{Классические ортогональные многочлены и связанные с ними ортогональные системы.}

\begin{enumerate}
	\item На отрезке $[-1,1]$ ортогональна система: $(1-x)^{\frac{\alpha}{2}}(1+x)^{\frac{\beta}{2}}P_n^{(\alpha,\beta)}(x)$,\\где $P_n^{(\alpha,\beta)}(x)=c_n(1-x)^{-\alpha}(1+x)^{-\beta} \frac{d^n}{dx^n}\left((1-x)^{\alpha+n}(1+x)^{\beta+n}\right)$ --- многочлены Якоби, $\alpha>-1, \beta>-1, n=0,1,\ldots$.
	
	При 
	\begin{enumerate}
		\item $\alpha=\beta, P_n$ --- многочлены Гегенбауэра (ультрасферические).
		\item $\alpha=\beta=0, P_n$ --- многочлены Лежандра.
		\item $\alpha=\beta=-\frac{1}{2}, P_n$ --- многочлены Чебышева 1-го рода $T_n(x)$.
		\item $\alpha=\beta=-\frac{1}{2}, P_n$ --- многочлены Чебышева 2-го рода $U_n(x)$
	\end{enumerate}
	\item На отрезке $[0,\infty]$ ортогональна система: $x^{\frac{\alpha}{2}}e^{-\frac{x}{2}}L_n^{(\alpha)}(x)$, где $L_n^{(\alpha)}(x)=x^{-\alpha}e^x\frac{d^n}{dx^n}\left(x^{\alpha+n}e^{-x}\right)$ --- многочлены Лагерра, $\alpha>-1$.
	\item На $(-\infty,\infty)$ ортогональна система: $e^{-\frac{x^2}{2}}H_n(x)$, где $H_n(x)=e^{x^2}\frac{d^n}{dx^n}(x^ne^{-x^2})$ --- многочлены Эрмита.
\end{enumerate}

\subsection{Сходимость тригонометрических рядов Фурье.}

Будем писать $f\in L_{2\pi}\Leftrightarrow f\in L_1[-\pi,\pi]$ и $2\pi$-периодическая. 
\begin{lemma}
	Если $f\in L_{2\pi}$, то $n$-я частичная сумма тригонометрического ряда Фурье $S_n(f,x)=\frac{a_0}{2}+\sum\limits_{k=1}^n(a_k\cos kx+b_k\sin kx)$ может быть представлена следующим образом: $S_n(f,x)\overset{(a)}{=}\frac{1}{\pi}\int\limits_{-\pi}^\pi f(t)D_n(x-t)d\mu(t)\overset{(b)}{=}\frac{1}{\pi}\int\limits_{-\pi}^\pi f(x+u)D_n(u)d\mu(u)$, где $D_n(u)$ --- ядро Дирихле, $D_n(u)=\frac{1}{2}+\sum\limits_{k=1}^n \cos ku\overset{(c)}{=}\dfrac{\sin(n+\frac{1}{2})u}{2\sin\frac{u}{2}}$. 
\end{lemma}

\begin{proof}
	Из формул $(1)$ и $(2)$:
	$$S_n(f,x)=\frac{1}{\pi}\int\limits_{-\pi}^\pi f(t)\left(\frac{1}{2}+\sum\limits_{k=1}^n(\underbrace{\cos kx \cos kt+\sin kx\sin kt}_{\cos k(x-t)})\right)d\mu(t)=\frac{1}{\pi}\int\limits_{-\pi}^\pi f(t)D_n(x-t)d\mu(t), $$ где $D_n(u)=\dfrac{1}{2}+\sum\limits_{k=1}^n\cos ku$. Тем самым равенство $(a)$ проверено, равенство $(b)$ получается из простой заменой переменной $t=x+u$ и из того, что $D_n$ --- четная функция, причем сдвиг пределов интегрирования не происходит, так как интеграл от любой $2\pi$ периодической функции по любому отрезку длины $2\pi$ одинаковый.\\
	Проверим равенство $(c)$:
	\begin{multline*}
		D_n(u)=\dfrac{1}{2}+\sum\limits_{k=1}^n\cos ku=\dfrac{1}{2}+\dfrac{1}{2}\sum\limits_{k=1}^n(e^{iku}+e^{-iku})=\dfrac{1}{2}\sum\limits_{k=-n}^ne^{iku}=\text{[сумма геом. прогрессии]}=\\=\dfrac{1}{2}e^{-inu}\dfrac{e^{i(2n+1)u}-1}{e^{iu}-1}=\frac{1}{2}e^{-inu}\dfrac{e^{i(2n+1)\frac{u}{2}}-e^{-i(2n+1)\frac{u}{2}}}{e^{i\frac{u}{2}}-e^{-i\frac{u}{2}}}\cdot\dfrac{e^{i(2n+1)\frac{u}{2}}}{e^{i\frac{u}{2}}}=\dfrac{\sin(n+\frac{1}{2})u}{2\sin\frac{u}{2}}.
	\end{multline*}
\end{proof}

\begin{theorem}[Признак Дини.]
	Если $f\in L_{2\pi}$ и $\varphi_{x_0}\in L_1(0,\delta),\delta>0$, где\\ $\varphi_{x_0}(t)=\dfrac{f(x_0+t)+f(x_0-t)-2S(x_0)}{den}$, то тригонометрический ряд Фурье функции $f(x)$ сходится к $S(x_0)$.
\end{theorem}

\begin{proof}
	Рассмотрим $S_n(f,x_0)-S(x_0)=\frac{1}{\pi}$
\end{proof}



























