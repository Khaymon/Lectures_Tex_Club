\subsection{Приложения теории Галуа}

В данном разделе с помощью теории Галуа мы докажем алгебраическую замкнутость поля $\Cm$, завершим изучение построений циркулем и линейкой и исследуем разрешимость уравнений в радикалах.

\begin{proposition}
	Пусть $G$ "--- конечная группа порядка $2^n$, $n \in \N$. Тогда в $G$ есть подгруппа $H$ такая, что $[G : H] = 2$.
\end{proposition}

\begin{proof}
	Проведем индукцию по $n$. База, $n = 1$, тривиальна, докажем переход.
	Группа $G$ является $p$-группой, поэтому ее центр $Z(G)$ нетривиален. Поскольку $Z(G) \normal G$, можно рассмотреть группу $G / Z(G)$. Это $2$-группа меньшего порядка, и по предположению индукции в $G / Z(G)$ есть подгруппа $H$ такая, что $[G / Z(G) : H] = 2$. По теореме о соответствии, $H = K / Z(G)$ для некоторой группы $K$ такой, что $Z(G) \le K \le G$. Значит, $[G : K] = 2$, и переход доказан.
\end{proof}

\begin{corollary}
	Пусть $G$ "--- группа, $|G| = 2^n$. Тогда в $G$ существует такая цепочка подгрупп $G = G_n \ge G_{n-1} \ge \dotsb \ge G_0 = \{e\}$, что $\forall i \in \{0, \dotsc, n-1\}: [G_{i + 1} : G_i] = 2$.
\end{corollary}

\begin{theorem}[основная теорема алгебры]
	Поле $\Cm$ алгебраически замкнуто.
\end{theorem}

\begin{proof}
	Заметим, что верны два следующих утверждения:
	\begin{enumerate}
		\item Каждый многочлен $f \in \R[x]$ нечетной степени имеет корень.
		\item Каждый многочлен $g \in \Cm[x]$ степени $2$ имеет корень.
	\end{enumerate}

	Из этого следует, что у $\R$ нет нетривиальных конечных расширений нечетной степени, а у $\Cm$ нет конечных расширений степени $2$. Действительно, по теореме о примитивном элементе, такое расширение порождалось бы одним элементом, и его минимальный многочлен был бы неприводим и при этом имел бы корень.
	
	Пусть $L \supset \Cm$ "--- конечное расширение. Тогда расширение $L \supset \R$ тоже конечно, поэтому $L = \R(\gamma)$ для некоторого $\gamma \in L$. Рассмотрим $K \supset L$ "--- поле разложения минимального многочлена $m_\gamma \in \R[x]$, тогда $K$ нормально над $\R$ и над $\Cm$. Выберем силовскую $2$-подгруппу $H$ в $\Aut_{\R}K$ и соответствующее ей поле $M$ такое, что $\R \subset M \subset K$. Тогда $[M : \R] \hm= [\Aut_{\R}K : H] \equiv_2 1$, поэтому $M = \R$. Значит, $[K : \R] = [K : M] = |H| = 2^n$, $n \in \N$. Тогда $[K : \Cm] = 2^{n - 1}$. Если $n \ge 2$, то, аналогично, выберем подгруппу индекса $2$ в $\Aut_{\Cm}K$ и получим расширение поля $\Cm$ порядка $2$, что невозможно. Значит, $n = 1$ и $K = L = \Cm$.
	
	Таким образом, у поля $\Cm$ нет нетривиальных конечных расширений, поэтому оно алгебраически замкнуто.
\end{proof}

\begin{theorem}
	Точку $z \in \Cm$ можно построить циркулем и линейкой $\lra$ поле разложения ее минимального многочлена $m_{z} \in \Q[x]$ имеет степень $2^n$, $n \in \N$.
\end{theorem}

\begin{proof}
	Обозначим через $L$ поле разложения многочлена $m_{z}$.
	\begin{itemize}
		\item[$\la$] Пусть $[L : \Q] = 2^n$, $n \in \N$. Тогда в группе $G := \Aut_\Q{L}$ порядка $2^n$ существует такая цепочка подгрупп $G = G_n \ge \dotsb \ge G_0 = \{e\}$, что для всех $i \in \{0, \dotsc, n - 1\}$ выполнено $[G_{i+1} : G_{i}] = 2$. Выберем соответствующую цепочке подгрупп башню расширений $\Q = L_n \subset \dotsb \subset L_0 = L$, тогда $\forall i \in \{0, \dotsc, n-1\}: [L_i : L_{i+1}] = 2$. Но $z \in L$, поэтому $z$ можно построить.
		
		\item[$\ra$] Пусть $z$ можно построить, тогда существует башня расширений $L_0 \supset \dotsb \supset L_n = \Q$ такая, что $z \in L_n$ и $\forall i \in \{0, \dotsc, n-1\}: [L_{i} : L_{i + 1}] = 2$. Оставим без доказательства тот факт, что каждое из расширений $L_i$ можно дополнить до расширения ${L}'_i$ такого, что $L'_0 \supset \dotsb \supset L'_n = \Q$, $z \in L'_0$ и $\forall i \in \{0, \dotsc, n - 1\}: [L'_i : L'_{i + 1}] = 2^{k_i}$, $k_i \in \N$, причем $L'_i$ нормально над $\Q$. Тогда $z$ лежит в нормальном над $\Q$ расширении $L'_0$ степени $2^{k_0\dotsm k_{n-1}}$, поэтому поле разложения многочлена $m_z$ имеет степень $2^n$, $n \in \N$.\qedhere
	\end{itemize}
\end{proof}

\begin{corollary}
	Пусть $p$ "--- простое число. Тогда правильный $p$-угольник можно построить циркулем и линейкой $\lra$ $p = 2^k + 1$, $k \in \N$.
\end{corollary}

\begin{proof}
	Число $\xi_p$ является корнем многочлена $f_p := x^{p - 1} + \dotsb + x + 1 \in \Q[x]$. Этот многочлен неприводим, поскольку после замены $x = t + 1$ к нему применим признак Эйзенштейна по модулю $p$. Значит, $f_p$ является минимальным многочленом для $\xi_p$, при этом $\Q(\xi_p)$ содержит все его корни, поскольку они имеют вид $\xi_p, \xi_p^2, \dotsc, \xi_p^{p - 1}$. Следовательно, $\Q(\xi_p)$ "--- это поле разложения многочлена $f_p$, и $[\Q(\xi_p) : \Q] = p - 1$. Остается воспользоваться теоремой выше.
\end{proof}

Займемся теперь разрешимостью уравнений в радикалах. \textbf{До конца раздела} зафиксируем поле $F$, алгебраическое замыкание которого сепарабельно.

\begin{definition}
	Пусть $f \in F[x]$ "--- многочлен, $L$ "--- его поле разложения. Уравнение $f(x) = 0$ называется \textit{разрешимым в радикалах}, если выполнено одно из следующих условий:
	\begin{enumerate}
		\item Существует башня расширений $L = L_0 \supset \dotsb \supset L_n = F$ такая, что для каждого $i \in \{0, \dotsc, n - 1\}$ выполнено $L_{i} = L_{i + 1}(\sqrt[\leftroot{2}\uproot{4}m_{i + 1}]{a_{i + 1}})$, $m_{i + 1} \in \N, a_{i + 1} \in L_{i + 1}$
		\item Существует башня расширений $L_0 \supset \dotsb \supset L_n = F$ такая, что $L_0 \supset L$ и для каждого $i \in \{0, \dotsc, n - 1\}$ поле $L_{i}$ "--- это поле разложения многочлена $x^{m_{i + 1}} - a_{i + 1} \in L_{i + 1}[x]$
	\end{enumerate}
\end{definition}

\begin{proposition}
	Условия в определении разрешимости в радикалах эквивалентны.
\end{proposition}

\begin{proof}~
	\begin{itemize}
		\item\imp{1}{2}Положим $N := [m_1, \dotsc, m_n]$ и рассмотрим $L_n' \supset F$ "--- поле разложения многочлена $x^N - 1$. Тогда $L_n' = F(\xi_N)$, где $\xi_N$ "--- корень $N$-ной степени из $1$ над $F$. Построим башню расширений $L_0' = L_0(\xi_N) \supset \dotsb \supset L_n' = F(\xi_N)$, тогда для каждого $i \in \{0, \dotsc, n - 1\}$ поле $L_{i}'$ является полем разложения многочлена $x^{m_{i+1}} - a_{i+1} \in L_{i+1}'[x]$.
		\item\imp{2}{1}Заметим, что для всех $i \in \{0, \dotsc, n - 1\}$ выполнено $L_{i} = L_{i+1}(\sqrt[\leftroot{2}\uproot{4}m_{i+1}]{a_{i+1}}, \xi_{m_{i + 1}})$, $m_{i+1} \in \N, a_{i+1} \in L_{i+1}$. Значит, в башню достаточно добавить промежуточные расширения вида $L_{i} \supset L_{i + 1}(\xi_{m_{i + 1}}) \supset L_{i + 1}$.\qedhere
	\end{itemize}
\end{proof}

\begin{note}
	Нетрудно проверить, что во втором условии требование $L_0 \supset L$ можно заменить на требование $L_0 = L$, а также добиться того, чтобы для каждого $i \in \{0, \dotsc, n - 1\}$ расширение $L_i \supset F$ было нормальным. Именно таким вариантом определения мы будем пользоваться далее.
\end{note}

\begin{definition}
	\textit{Группой Галуа} многочлена $f \in F[x]$ называется группа автоморфизмов, сохраняющих $F$, его поля разложения. Обозначение "--- $\Gal{f}$.
\end{definition}

\begin{definition}
	Группа $G$ называется \textit{разрешимой}, если для некоторого $n \in \N$ ее $n$-ный коммутант $G^{(n)}$ тривиален.
\end{definition}

\begin{note}
	Мы будем пользоваться без доказательства утверждением из теории групп о том, что разрешимость эквивалентна следующему условию: в $G$ существует цепочка нормальных подгрупп $G = G_n\varnormal\dotsb\varnormal G_0 = \{e\}$ такая, что для каждого $i \in \{0, \dotsc, n - 1\}$ факторгруппа $G_{i+1}/G_i$ "--- абелева.
\end{note}

\begin{theorem}
	Пусть $f \in F[x]$. Тогда уравнение $f(x) = 0$ разрешимо в радикалах $\lra$ группа $\Gal{f}$ разрешима.
\end{theorem}

\begin{proof}~
	\begin{itemize}
		\item[$\ra$] Пусть $L = L_0 \supset \dotsb \supset L_n = F$ "--- башня расширений из определения разрешимости в радикалах. Проверим, что $\Gal{f} = \Aut_{L_n}{L} \ge \dotsb \ge \Aut_{L_0}L = \{\id\}$ при добавлении нескольких промежуточных подгрупп образует цепочку нормальных подгрупп с абелевыми факторами.
		
		Для произвольного $i \in \{0, \dotsc, n - 1\}$ рассмотрим многочлен $x^{m_{i+1}} - a_{i+1} \in L_{i + 1}[x]$, полем разложения которого является $L_i$, а также $K_i$ "--- поле разложения многочлена $x^N - 1 \in L_{i + 1}[x]$. Тогда $L_i = L_{i + 1}(\sqrt[\leftroot{2}\uproot{4}m_{i+1}]{a_{i+1}}, \xi_{m_{i+1}}) \supset K_i = L_{i + 1}(\xi_{m_{i+1}}) \supset L_{i + 1}$. Заметим, что $\Aut_{L_{i+1}}K_i \subset \Z_{m_{i+1}}^*$, поскольку каждый автоморфизм $\phi \in \Aut_{L_{i+1}}K_i$ однозначно задается образом элемента $\xi_{m_{i+1}}$, а $\xi_{m_{i+1}}$ может переходить только в $\xi_{m_{i+1}}^m$ для некоторого $m \in \N$ такого, что $(m, {m_{i+1}}) = 1$. Значит, группа $\Aut_{L_{i+1}}K_i$ "--- абелева.
		
		Аналогично, $\Aut_{K_i}L_i \subset \Z_n$, поскольку под действием автоморфизма $\phi \in \Aut_{K_i}L_i$ элемент $\sqrt[\leftroot{2}\uproot{4}m_{i+1}]{a_{i+1}}$ может переходить только в один из элементов вида $\sqrt[\leftroot{2}\uproot{4}m_{i+1}]{a_{i+1}}\xi_{m_{i+1}}^m$, $m \in \{0, \dotsc, m_{i+1} - 1\}$. Значит, группа $\Aut_{K_i}L_i$ "--- тоже абелева.
		
		Заметим теперь, что $\Aut_{K_i}L \normal \Aut_FL$, $\Aut_{L_{i + 1}}L \normal \Aut_FL$, поскольку расширения $K_i$ и $L_{i + 1}$ нормальны над $F$. Оставим без доказательства тот факт, что $\Aut_{L_{i+1}}K_i \hm\cong \Aut_{L_{i + 1}}L / \Aut_{K_i}L$ и $\Aut_{K_i}L_i \cong \Aut_{K_i}L / \Aut_{L_i}L$. Пользуясь им, получим, что $\Gal{f} = \Aut_{L_n}{L} \varnormal \Aut_{K_{n-1}}L \varnormal \dotsb \varnormal \Aut_{K_0}L \varnormal \Aut_{L_0}L = \{\id\}$ "--- это цепочка нормальных подгрупп с абелевыми факторами.
		
		\item[$\la$] Оставим этот факт без доказательства.\qedhere
	\end{itemize}
\end{proof}

\begin{corollary}
	Пусть $f \in \Q[x]$ "--- неприводимый многочлен степени $5$, имеющий над $\R$ ровно $3$ корня. Тогда уравнение $f(x) = 0$ не разрешимо в радикалах.
\end{corollary}

\begin{proof}
	Можно считать, что каждый автоморфизм из $\Gal{f}$ осуществляет перестановку корней многочлена $f$ над $\overline{\Q}$, поэтому $\Gal{f} \subset S_5$. Пусть корни $\alpha_1, \alpha_2 \in \overline{\Q}$ многочлена $f$ "--- комплексные, а остальные "--- вещественные. Тогда комплексное сопряжение, то есть $(12) \in S_5$, лежит в $\Gal{f}$. В силу утверждения о продолжении гомоморфизма, для любых $i, j \in \{1, \dotsc, 5\}$ существует автоморфизм из $\Gal{f}$ такой, что $\alpha_i \mapsto \alpha_j$. Докажем теперь, что из этого следует равенство $\Gal{f} = S_5$.
	
	Рассмотрим граф транспозиций в $\Gal{f}$, то есть граф с множеством вершин $\{1, \dotsc, 5\}$ и множеством ребер $\{\{i, j\} : (ij) \in \Gal{f}\}$. Поскольку $\Gal{f}$ "--- группа, то компоненты связности в этом графе являются кликами. Заметим теперь, что при сопряжении перестановкой $\sigma \in \Gal{f}$ такой, что $\sigma(i) = j$, компонента, содержащая $i$, переходит в компоненту, содержащую $j$. Значит, все компоненты имеют одинаковый размер. Но число $5$ "--- простое, поэтому в графе есть лишь одна компонента, и, следовательно, $\Gal{f} = S_5$. Эта группа неразрешима, поэтому и уравнение $f(x) = 0$ не разрешимо в радикалах.
\end{proof}

\begin{example}
	Многочлен $f := x^5 - 10x + 5 \in \Q[x]$ не разрешим в радикалах. Действительно, исследуя $f$ на экстремумы с помощью производной, легко проверить, что он имеет ровно три корня над $\R$. Неприводимость многочлена $f$ следует из признака Эйзенштейна по модулю $5$. Значит, применимо утверждение выше.
\end{example}

Наконец, завершим исследование построимости правильных $n$-угольников.

\begin{proposition}
	Пусть $f \in \Z_p[x]$. Тогда $\forall x \in \Z_p: f(x^p) = (f(x))^p$.
\end{proposition}

\begin{proof}
	Пусть $f = a_nx^n + \dotsc + a_1x + a_0$, где $a_0, \dotsc, a_n \in \Z_p$. Тогда:
	\[(f(x))^p = a_n^px^{np} + \dotsb + a_1^px^p + a_0^p = a_nx^{np} + \dotsb + a_1x^p + a_0 = f(x^p)\]
	
	Первое равенство выше выполено в силу того, что $\forall a, b \in \Z_p: (a + b)^p = a^p + b^p$, второе равенство --- по малой теореме Ферма.
\end{proof}

\begin{theorem}
	Степень минимального многочлена числа $\xi_n \in \Cm$ над $\Q$ равна $\phi(n)$.
\end{theorem}

\begin{proof}
	Для произвольного $n \in \N$ положим $\psi_n(x) := \prod_{1 \le d \le n, (d, n) = 1}(x - \xi_n^d)$. Докажем, что это минимальный многочлен для $\xi_n$ над $\Q$. Проверим сначала, что $\psi_n \in \Z[x]$. Действительно, поскольку выполнено равенство $\prod_{k \mid n}\psi_k(x) = {x^n - 1}$, то целочисленность всех коэффициентов многочлена $\psi_n$ легко проверяется по индукции.
	\pagebreak
	
	Пусть теперь $m \in \Z[x]$ "--- минимальный многочлен для $\xi_n$ (возможно, отличный от $\psi_n$), $\xi$ "--- его произвольный корень. Покажем, что $m$ должен также иметь корни $\xi^p$ для всех простых чисел $p \nmid n$. В силу минимальности, $m \mid x^n - 1$, то есть $x^n - 1 = m(x)g(x)$ для некоторого $g \in \Q[x]$, причем легко проверить, что коэффициенты в $g$ целочисленны. Если $m(\xi^p) \ne 0$, то $g(\xi^p) = 0$. Тогда $m(\xi) \equiv_p 0$ и $g(\xi^p) = g(\xi)^p \equiv_p 0 \ra g(\xi) \equiv_p 0$. Значит, $\xi$ является кратным корнем многочлена $x^n - 1$ над $\Z_p$, но формальная производная этого многочлена равна $nx^{n - 1}$ и не имеет над $\Z_p$ корней, отличных от нуля, поскольку $p \nmid n$. Следовательно, $m(\xi^p) = 0$. Но тогда для всех $d$ таких, что $1 \le d \le n, (d, n) = 1$ число $\xi_n^d$ является корнем многочлена $m$. Таким образом, $m = \psi_n$.
\end{proof}

\begin{corollary}
	Пусть $n \in \N$. Тогда правильный $n$-угольник можно построить циркулем и линейкой $\lra$ $\phi(n) = 2^k$, $k \in \N$.
\end{corollary}