\subsection{Основные понятия теории делимости}

В данном разделе зафиксируем $K$ "--- область целостности.

\begin{definition}
	Пусть $a, b \in K$. Говорят, что \textit{$a$ делит $b$}, или \textit{$b$ делится на $a$}, если $\exists c \in K: b = ac$. Обозначение "--- $a \mid b$.
\end{definition}

\begin{note}
	Отношение делимости является отношением предпорядка, но не является отношением порядка, поскольку для него не выполнена антисимметричность: если $a \mid b$ и $b \mid a$, то $b = ac$ и $a = bd$, $c, d \in K$, причем $cd = 1$, то есть $c, d \in K^*$.
\end{note}

\begin{definition}
	Группа $K^*$ действует на $K$ умножениями. Орбиты этого действия называются \textit{классами ассоциированности}. Элементы $a, b \in K$ называются \textit{ассоциированными}, если они лежат в одном классе ассоциированности, то есть $\exists c \in K^*: a = bc$. Обозначение "--- $a \sim b$.
\end{definition}

\begin{proposition}
	Отношение ассоциированности является отношением эквивалентности на $K$.
\end{proposition}

\begin{proof}
	Отношение ассоциированности совпадает с отношением эквивалентности относительно действия $K^*$ на $K$ умножениями.
\end{proof}

\begin{proposition}
	Пусть $a, b \in K$. Тогда эквивалентны следующие утверждения:
	\begin{enumerate}
		\item $a \sim b$
		\item $\{c \in K: a \mid c\} = \{c \in K: b \mid c\}$
		\item $\{c \in K: c \mid a\} = \{c \in K: c \mid b\}$
	\end{enumerate}
\end{proposition}

\begin{proof} Докажем, что $1 \lra 2$, поскольку эквивалентность $1 \lra 3$ доказывается аналогично:
	\begin{itemize}
		\item[$\ra$] Если $a = bd$, $d \in K^*$, то, в силу обратимости $d$, $a \mid c \lra \exists \alpha \in K: c = a\alpha \lra \exists \alpha \in K: c = bd\alpha \lra \exists \beta \in K: c = b\beta \lra b \mid c$
		\item[$\la$] Из условия следует, что $a \mid b$ и $b \mid a$, поэтому $a \sim b$
	\end{itemize}
\end{proof}

\begin{definition}
	\textit{Расширение кольца $\Z$ числом $\alpha \in \Cm$} "--- это наименьшее по включению подкольцо в $\Cm$, содержащее $\Z$ и $\alpha$. Обозначение "--- $\Z[\alpha]$.
\end{definition}

\begin{proposition}
	$\forall \alpha \in \Cm: \Z[\alpha] = \{p(\alpha): p \in \Z[x]\}$.
\end{proposition}

\begin{proof}
	Обозначим множество справа через $S$. С одной стороны, очевидно, что $S \subset \Z[\alpha]$. С другой стороны, $S$ "--- кольцо, поскольку оно замкнуто относительно сложения и умножения, поэтому $\Z[\alpha] \subset S$.
\end{proof}

\begin{example}
	Рассмотрим $\Z[i] = \{a + bi: a, b \in \Z\}$ "--- \textit{гауссовы числа}. Покажем, что $\Z[i]^* \hm= \{\pm1, \pm i\}$. Для этого определим на $\Z[i]$ \textit{норму} следующим образом: $\forall a + bi \in \Z[i]: N(a + bi) := a^2 + b^2$. Заметим, что $\forall z, w \in \Z[i]: N(zw) = N(z)N(w)$. Тогда, поскольку нормы всех чисел в $\Z[i]$ "--- целые числа, $z \in \Z[i]^* \lra N(z) = 1 \lra z \in \{\pm1, \pm i\}$.
	
	\pagebreak
	На комплексной плоскости гауссовы числа образуют решетку следующего вида:
	\begin{center}
		\scalebox{0.7}{
		\begin{tikzpicture}
			\clip (-3.8, -3.8) rectangle (3.8, 3.8);
			\begin{scope}[thick,font=\small]
				\draw [->] (-3.8, 0) -- (3.8, 0);
				\draw [->] (0, -3.8) -- (0, 3.8);
				
				\draw (1,3pt) -- (1,-3pt) node [below] {$1$};
				\draw (3pt,1) -- (-3pt,1) node [left] {$i$};
			\end{scope}
			
			\foreach \x in {-5, -4, ..., 5}{
				\foreach \y in {-5, -4, ..., 5}{
					\node[draw,circle,inner sep=1pt,fill] at (\x, \y) {};
				}
			}
		
			\node[draw,circle,inner sep=1.5pt,fill,blue] at (0, 1) {};
			\node[draw,circle,inner sep=1.5pt,fill,blue] at (1, 0) {};
			\node[draw,circle,inner sep=1.5pt,fill,blue] at (0, -1) {};
			\node[draw,circle,inner sep=1.5pt,fill,blue] at (-1, 0) {};
		\end{tikzpicture}
		}
	\end{center}
\end{example}

\begin{example}
	Рассмотрим $\Z[\omega] = \{a + b\omega: a, b \in \Z\}$, где $\omega$ "--- нетривиальный корень уравнения $z^3 = 1$ ($\omega = -\frac12 + \frac{\sqrt{3}}2i$), "--- \textit{числа Эйзенштейна}. Покажем, что $\Z[\omega]^* \hm= \{\pm1, \pm \omega, \pm(1 + \omega)\}$.
	
	
	На комплексной плоскости числа Эйзенштейна образуют решетку следующего вида:
	\begin{center}
		\scalebox{0.7}{
			\begin{tikzpicture}
				\clip (-3.8, -3.8) rectangle (3.8, 3.8);
				\begin{scope}[thick,font=\small]
					\draw [->] (-3.8, 0) -- (3.8, 0);
					\draw [->] (0, -3.8) -- (0, 3.8);
					
					\draw (1,3pt) -- (1,-3pt) node [below] {$1$};
					\draw (3pt,1) -- (-3pt,1) node [left] {$i$};
				\end{scope}
				
				\def\aa{-1/2}
				\def\bb{1.732/2}
				
				\foreach \x in {-5, -4, ..., 5}{
					\foreach \y in {-5, -4, ..., 5}{
						\node[draw,circle,inner sep=1pt,fill] at (\x + \y* \aa, \y * \bb) {};
					}
				}
				
				\node[draw,circle,inner sep=1.5pt,fill,blue] at (-1/2, 1.732/2) {} node [left] at (-1/2, 1.732/2) {$\omega$};
				\node[draw,circle,inner sep=1.5pt,fill,blue] at (1/2, 1.732/2) {};
				\node[draw,circle,inner sep=1.5pt,fill,blue] at (1, 0) {};
				\node[draw,circle,inner sep=1.5pt,fill,blue] at (-1, 0) {};
				\node[draw,circle,inner sep=1.5pt,fill,blue] at (1/2, -1.732/2) {};
				\node[draw,circle,inner sep=1.5pt,fill,blue] at (-1/2, -1.732/2) {};
			\end{tikzpicture}
		}
	\end{center}
\end{example}