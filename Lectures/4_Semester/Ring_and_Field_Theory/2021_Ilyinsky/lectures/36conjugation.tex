\subsection{Сопряженные элементы}

\begin{definition}
	Пусть $F$ "--- поле. Элементы $\alpha, \beta \in \overline{F}$ называются \textit{сопряженными}, если их минимальные многочлены $m_\alpha, m_\beta$ над $F$ совпадают.
\end{definition}

\begin{note}
	Легко проверить по определению, что минимальные многочлены $m_\alpha, m_\beta$ совпадают $\lra m_\alpha(\beta) = 0 \lra m_\beta(\alpha) = 0$.
\end{note}

В данном разделе мы докажем, что каждое конечное расширение поля $F$ порождается одним элементом при некоторых ограничениях на поле $F$. Чтобы ослабить эти ограничения, сначала введем определение.

\begin{definition}
	Пусть $F$ "--- поле, $L \supset F$ "--- его алгебраическое расширение. Элемент $\alpha \in L$ называется \textit{сепарабельным}, если его минимальный многочлен $m_\alpha$ не имеет кратных корней над $\overline{F}$. Расширение $L$ называется \textit{сепарабельным}, если все его элементы сепарабельны.
\end{definition}

\begin{proposition}
	Пусть $F$ "--- поле нулевой характеристики, $f \in F[x]$ "--- неприводимый многочлен. Тогда $f$ не имеет кратных корней над $\overline{F}$.
\end{proposition}

\begin{proof}
	Заметим, что $f$ не имеет кратных корней над $\overline{F}$ $\lra$ $(f, f') \sim 1$ над $\overline{F}[x]$, где $f'$ "--- формальная производная многочлена $f$. В силу алгоритма Евклида для многочленов, $f$ и $f'$ взаимно просты над $F[x]$ и $\overline{F}[x]$ одновременно, поэтому $(f, f') \sim 1$ над $\overline{F}[x]$ $\lra$ $(f, f') \sim 1$ над $F[x]$. Но последнее условие выполнено всегда, поскольку $f$ неприводим над $F[x]$ и $0 < \deg{f'} < \deg{f}$.
\end{proof}

\begin{corollary}
	 Если $F$ "--- поле нулевой характеристики, то любое его алгебраическое расширение сепарабельно.
\end{corollary}

\begin{note}
	Аналогичное рассуждение не работает для полей ненулевой характеристики. Если $\cha{F} = p$, то $(x - 1)^p = x^p - 1$ и $(x^p - 1)' = 0$ --- многочлен и его формальная производная уже необязательно одновременно взаимно просты над $F[x]$ и $\overline{F}[x]$.
\end{note}

\begin{theorem}[о примитивном элементе]
	Пусть $F$ "--- бесконечное поле, $K \supset F$ "--- сепарабельное конечное расширение поля $F$. Тогда существует $\gamma \in K$ такой, что $K = F(\gamma)$.
\end{theorem}

\begin{proof}
	Пусть $(\alpha_1, \dotsc, \alpha_n)$ "--- базис в пространстве $K$ над $F$. Достаточно показать, что $\forall \alpha, \beta \in K: \exists \gamma \in K: F(\alpha, \beta) = F(\gamma)$, чтобы затем провести индукцию. Будем искать элемент $\gamma$ в виде $\gamma = \alpha +c\beta$, $c \in F$. Если мы покажем, что для некоторого $c \in F$ выполнено $\beta \in F(\gamma)$, то тогда мы докажем, что $\alpha = \gamma - c\beta \in F(\gamma)$, и, следовательно, $F(\alpha, \beta) = F(\gamma)$.
	
	Пусть $\alpha = \alpha_1, \dotsc, \alpha_n \in \overline{F}$, $\beta = \beta_1, \dotsc, \beta_m \in \overline{F}$ "--- классы сопряженности элементов $\alpha, \beta$, $m_\alpha, m_\beta \in F[x]$ "--- их минимальные многочлены. Выберем элемент $c \in F$, отличный от элементов $-(\alpha - \alpha_i)(\beta - \beta_j)^{-1}$ для всех $i \in \{2, \dotsc, n\}$ и $j \in \{2, \dotsc, m\}$. Элемент $\beta$ является корнем многочленов $m_\beta$ и $f = m_\alpha(\gamma - cx) \in F(\gamma)[x]$. Проверим, что это единственный их общий корень. Если $\delta \in \overline{F}$ "--- общий корень $m_\beta$ и $f$, отличный от $\beta$, то $\delta$ сопряжен с $\beta$, а $\gamma - c\delta$ сопряжен с $\alpha$. Значит, $\gamma = \alpha_i + c\beta_j$ для некоторых $i \in \{2, \dotsc, n\}$ и $j \in \{2, \dotsc, m\}$, что невозможно по выбору элемента $c$. Таким образом, $(m_\beta, f) = x - \beta$ над $\overline{F}[x]$ $\ra$ $(m_\beta, f) \hm= x - \beta$ над $F(\gamma)[x] \ra x - \beta \in F(\gamma)[x] \ra \beta \in F(\gamma)$, что и требовалось.
\end{proof}

\begin{note}
	Из доказательства теоремы также следует алгоритм поиска примитивного элемента: если $c \in F$ отлично от всех элементов вида $-(\alpha - \overline{\alpha})(\beta - \overline{\beta})^{-1}$, где $\overline{\alpha}, \overline{\beta}$ "--- сопряженные к $\alpha, \beta$, то $F(\alpha + c\beta) = F(\alpha, \beta)$.
\end{note}