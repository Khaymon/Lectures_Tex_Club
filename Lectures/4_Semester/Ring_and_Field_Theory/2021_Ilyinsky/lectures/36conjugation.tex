\subsection{Теорема о примитивном элементе}

\begin{definition}
	Пусть $F$ "--- поле. Элементы $\alpha, \beta \in \overline{F}$ называются \textit{сопряженными} над $F$, если их минимальные многочлены $m_\alpha, m_\beta \in F[x]$ совпадают.
\end{definition}

\begin{note}
	Легко проверить по определению, что минимальные многочлены $m_\alpha, m_\beta$ совпадают $\lra m_\alpha(\beta) = 0 \lra m_\beta(\alpha) = 0$.
\end{note}

В данном разделе мы докажем, что каждое конечное расширение поля $F$ порождается одним элементом при некоторых ограничениях на поле $F$. Чтобы ослабить эти ограничения, введем следующее определение.

\begin{definition}
	Пусть $F$ "--- поле, $L \supset F$ "--- его алгебраическое расширение. Элемент $\alpha \in L$ называется \textit{сепарабельным}, если его минимальный многочлен $m_\alpha$ не имеет кратных корней над $\overline{F}$. Расширение $L$ называется \textit{сепарабельным}, если все его элементы сепарабельны.
\end{definition}

\begin{proposition}
	Пусть $F$ "--- поле, $f, g \in F[x]$. Тогда $(f, g)$ над $F[x]$ совпадает с $(f, g)$ над $\overline{F}[x]$.
\end{proposition}

\begin{proof}
	Пусть $(f, g)$ над $F[x]$ равен $h \in F[x]$. С одной стороны, очевидно, что $h \mid (f, g)$ над $\overline{F}[x]$. С другой стороны, наибольший общий делитель представим в виде линейной комбинации $f$ и $g$ в силу алгоритма Евклида: $\exists \widetilde{f}, \widetilde{g} \in F[x]: f\widetilde{f} + g\widetilde{g} = h$. Значит, любой делитель $f, g$ над $\overline{F}[x]$ также делит $h$.
\end{proof}

\begin{proposition}
	Пусть $F$ "--- поле нулевой характеристики, $f \in F[x]$ "--- неприводимый многочлен. Тогда $f$ не имеет кратных корней над $\overline{F}$.
\end{proposition}

\begin{proof}
	Заметим, что $f$ не имеет кратных корней над $\overline{F}$ $\lra$ $(f, f') \sim 1$ над $\overline{F}[x]$, где $f'$ "--- формальная производная многочлена $f$. Как уже было доказано, $(f, f') \sim 1$ над $\overline{F}[x]$ $\lra$ $(f, f') \sim 1$ над $F[x]$. Но последнее условие выполнено всегда, поскольку $f$ неприводим над $F[x]$ и $\deg{f'} = \deg{f} - 1$.
\end{proof}

\begin{corollary}
	 Если $F$ "--- поле нулевой характеристики, то любое его алгебраическое расширение сепарабельно.
\end{corollary}

\begin{note}
	Аналогичное рассуждение не работает для полей ненулевой характеристики. Если $\cha{F} = p$, то $(x - 1)^p = x^p - 1$ и $(x^p - 1)' = 0$ --- многочлен и его формальная производная уже необязательно одновременно взаимно просты.
\end{note}

\begin{theorem}[о примитивном элементе]
	Пусть $F$ "--- бесконечное поле, $K \supset F$ "--- сепарабельное конечное расширение поля $F$. Тогда существует $\gamma \in K$ такой, что $K = F(\gamma)$.
\end{theorem}

\begin{proof}
	Пусть $(\alpha_1, \dotsc, \alpha_n)$ "--- базис в пространстве $K$ над $F$. Достаточно показать, что $\forall \alpha, \beta \in K: \exists \gamma \in K: F(\alpha, \beta) = F(\gamma)$, чтобы затем провести индукцию. Будем искать элемент $\gamma$ в виде $\gamma = \alpha +c\beta$, $c \in F$. Если мы покажем, что для некоторого $c \in F$ выполнено $\beta \in F(\gamma)$, то из этого будет следовать, что при данном $c \in F$ выполнено $\alpha = \gamma - c\beta \in F(\gamma)$ и $F(\alpha, \beta) = F(\gamma)$.
	
	Пусть $\alpha = \alpha_1, \dotsc, \alpha_n \in \overline{F}$, $\beta = \beta_1, \dotsc, \beta_m \in \overline{F}$ "--- классы сопряженности элементов $\alpha, \beta$, $m_\alpha, m_\beta \in F[x]$ "--- их минимальные многочлены. Выберем элемент $c \in F$, отличный от элементов $-(\alpha - \alpha_i)(\beta - \beta_j)^{-1}$ для всех $i \in \{2, \dotsc, n\}$ и $j \in \{2, \dotsc, m\}$. Элемент $\beta$ является корнем многочленов $m_\beta$ и $f := m_\alpha(\gamma - cx) \in F(\gamma)[x]$. Проверим, что это единственный их общий корень. Пусть $\delta \in \overline{F}$ "--- общий корень $m_\beta$ и $f$, отличный от $\beta$, тогда $\delta$ сопряжен с $\beta$, а $\gamma - c\delta$ сопряжен с $\alpha$. Значит, $\gamma = \alpha_i + c\beta_j$ для некоторых $i \in \{2, \dotsc, n\}$ и $j \in \{2, \dotsc, m\}$, что невозможно по выбору элемента $c$. Таким образом, $(m_\beta, f) = x - \beta$ над $\overline{F}[x]$ $\ra$ $(m_\beta, f) \hm= x - \beta$ над $F(\gamma)[x] \ra x - \beta \in F(\gamma)[x] \ra \beta \in F(\gamma)$, что и требовалось.
\end{proof}

\begin{note}
	Из доказательства теоремы также следует алгоритм поиска примитивного элемента: если $c \in F$ отличен от всех элементов вида $-(\alpha - \widetilde{\alpha})(\beta - \widetilde{\beta})^{-1}$, где $\widetilde{\alpha}, \widetilde{\beta}$ "--- сопряженные к $\alpha, \beta$, то $F(\alpha + c\beta) = F(\alpha, \beta)$.
\end{note}

\begin{note}
	Теорема о примитивном элементе верна и для конечных полей, причем для всех, поскольку любое алгебраическое расширение конечного поля сепарабельно. Это будет доказано позднее.
\end{note}