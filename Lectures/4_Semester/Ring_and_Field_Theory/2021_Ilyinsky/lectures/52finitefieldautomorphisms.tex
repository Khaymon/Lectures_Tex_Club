\subsection{Автоморфизмы конечных полей}

\begin{proposition}
	Пусть $F$ "--- конечное поле, $f \in F[x]$ "--- неприводимый многочлен. Тогда $f$ не имеет кратных корней над $\overline{F}$.
\end{proposition}

\begin{proof}
	Пусть это не так, тогда $(f, f') \not\sim 1$. Тогда, в силу неприводимости многочлена $f$, $f' = 0$. Значит, $f$ имеет вид $f = a_mx^{mp} + \dotsc + a_1x^p + a_0$ для некоторых $a_0, \dotsc, a_m \in F$. Но тогда $f = (a_mx^{m} + \dotsc + a_1x + a_0)^p$, что противоречит его неприводимости.
\end{proof}

В силу утверждения выше и того факта, что любое конечное поле $\F_{p^n}$ является полем разложения многочлена из $\Z_p[x]$, все результаты теории Галуа справедливы и для конечных полей. \textbf{До конца раздела} зафиксируем конечное поле $\F_{p^n}$ и заметим, что любой автоморфизм $\phi \in \Aut\F_{p^n}$ сохраняет $\Z_p \subset \F_{p^n}$.

\begin{theorem}
	$\Aut\F_{p^n} \cong \Z_n$.
\end{theorem}

\begin{proof}
	Обозначим через $\phi \in \Aut\F_{p^n}$ \textit{автоморфизм Фробениуса}, имеющий вид $\phi(x) := x^p$ для всех $x \in \F_{p^n}$. Очевидно, $\phi$ сохраняет операции, инъективность же выполнена потому, что $\ke\phi = \{0\}$, и ее достаточно для биективности, поскольку $\F_{p^n}$ конечно. Заметим также, что $\F_{p^n}^\phi = \{a \in \F_{p^n} : a^p = a\} = \Z_p$.
	
	С одной стороны, $\phi^n = \id$. С другой стороны, $\forall k \in \{1, 
	\dotsc, n - 1\}:$ равенство $\phi^k(x) = x$ выполнено только для таких элементов поля $\F_{p^n}$, которые являются корнями многочлена $x^{p^k} - x \in \Z_p[x]$.  Но $|\Aut\F_{p^n}| = |\Aut_{\Z_p}\F_{p^n}| = [\F_{p^n} : \Z_p] = n$ в силу нормальности расширения $\F_{p^n} \supset \Z_p$, поэтому $\Aut\F_{p^n} = \gl\phi\gr = \{\id, \phi, \dotsc, \phi^{n-1}\} \cong \Z_n$.
\end{proof}

\begin{corollary}
	Подполя в $\F_{p^n}$ имеют вид $\F_{p^k}$, где $k \mid n$, причем для каждого $k \mid n$ подполе порядка $p^k$ существует и единственно.
\end{corollary}

\begin{proof}
	Подполя в $\F_{p^n}$ биективно соответствуют подгруппам в $\Aut\F_{p^n} \cong \Z_n$. Но все подгруппы в $\Z_n$ имеют вид $k\Z_n$ для $k \mid n$, причем подгруппа каждого порядка единственна. Легко проверить, что тогда подгруппа в $\Aut\F_{p^n}$ индекса $k$ соответствует в точности подполю $\F_{p^k} \subset \F_{p^n}$ при каждом $k \mid n$, причем единственному, и других подполей в $\F_{p^n}$ нет.
\end{proof}

\begin{theorem}
	Обозначим через $d_n$ число многочленов степени $n$, неприводимых над $\Z_p[x]$. Тогда для любого $n \in \N$ выполено следующее равенство:
	\[d_n = \frac1n\sum_{k \mid n}p^k\mu\left(\frac nk\right)\]
\end{theorem}

\begin{proof}
	Поле $\F_{p^n}$ является полем разложения многочлена $x^{p^n} - x \in \Z_p[x]$. Рассмотрим разложение $x^{p^n} - x = f_1\dotsb f_s$ на неприводимые сомножители и докажем, что $f_1, \dotsc, f_s \in \Z_p[x]$ "--- это в точности все различные неприводимые над $\Z_p[x]$ многочлены степеней $k$ таких, что $k \mid n$.
	
	Многочлены $f_1, \dotsc, f_s$ различны потому, что у $x^{p^n} - x$ нет кратных корней. Рассмотрим теперь многочлен $f_i$ степени $k_i$ для произвольного $i \in \{1, \dotsc, s\}$ и его произвольный корень $\alpha \in \F_{p^n}$. Тогда расширение $\Z_p(\alpha)$ имеет степень $k_i$. Но $\Z_p(\alpha) \subset \F_{p^n}$, поэтому $k_i \mid n$. С другой стороны, если $f \in \Z_p[x]$ "--- неприводимый многочлен степени $k \mid n$ и $\alpha \in \overline{\Z_p}$ "--- его корень, то $|\Z_p(\alpha)| = p^k$, поэтому $\alpha \in \F_{p^k} \subset \F_{p^n}$. Значит, $f = f_i$ для некоторого $i \in \{1, \dotsc, s\}$.
	
	Подсчитывая число корней в левой и правой части равенства $x^{p^n} - x = f_1\dotsb f_s$, получаем, что $p^n = \sum_{k \mid n}kd_k$. Тогда, по формуле обращения Мебиуса, $nd_n = \sum_{k \mid n}p^k\mu(\frac nk)$, что и требовалось.
\end{proof}

\begin{corollary}
	Для любого $n \in \N$ сущестувет многочлен степени $n$, неприводимый над $\Z_p[x]$.
\end{corollary}

\begin{note}
	Утверждение выше позволяет строить конечные поля как факторкольца кольца $\Z_p[x]$ по идеалам, порожденным подходящими неприводимыми многочленами.
\end{note}