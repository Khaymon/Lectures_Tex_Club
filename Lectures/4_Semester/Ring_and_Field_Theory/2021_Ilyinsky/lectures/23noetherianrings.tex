\subsection{Нетеровы кольца}

\begin{definition}
	Область целостности $K$ называется \textit{нетеровым кольцом}, если выполнено одно из следующих утверждений:
	\begin{enumerate}
		\item Если $I_0 \subset I_1 \subset I_2 \subset \dots$ "--- последовательность идеалов в $K$, то $\exists m \in N: \forall n \ge m: I_n = I_m$.
		\item Любой идеал в $K$ является конечнопорожденным.
		\item Если $(a_0) \subset (a_0, a_1) \subset (a_0, a_1, a_2) \dots$ "--- последовательность идеалов в $K$, то $\exists m \in N: \forall n \ge m: a_n \in (a_1, \dotsc, a_m)$.
	\end{enumerate}
\end{definition}

\begin{proposition}
	Утверждения в определении нетерова кольца эквивалентны.
\end{proposition}

\begin{proof}~
	\begin{itemize}
		\item\imp{1}{3}Тривиально, это частный случай более общего утверждения.
		\item\imp{3}{2}Пусть $I$ "--- ненулевой идеал в $K$. Выберем $a_0 \in I$. Если $I = (a_0)$, то получено требуемое. Иначе --- выберем $a_1 \in I \backslash (a_0)$. Если $I = (a_0, a_1)$, то получено требуемое. Иначе --- выберем $a_2 \in I \backslash (a_0, a_1)$ и продолжим процесс. Предположим, что этот процесс не завершается, тогда в $I$ есть нестабилизирующаяся последовательность идеалов $(a_0) \subsetneq (a_0, a_1) \subsetneq (a_0, a_1, a_2) \subsetneq \dotsc$, что невозможно.
		\item\imp{2}{1}Пусть $I_0 \subset I_1 \subset I_2 \subset \dots$ "--- последовательность идеалов в $K$. Как уже было доказано, $I := \bigcup_{j = 0}^\infty I_j$ "--- тоже идеал, тогда $I$ конечнопорожден, то есть $I = (a_1, \dotsc, a_n)$ для некоторых $a_1, \dotsc, a_n \in K$. Но тогда $\exists m := \min\{j \in \N \cup \{0\} : a_1, \dotsc, a_n \in I_j\}: \forall n \ge m: I_n = I_m$.\qedhere
	\end{itemize}
\end{proof}

\begin{corollary}
	Если $K$ "--- нетерово кольцо, то у каждого элемента $x \in K \backslash \{0\}$ существует разложение.
\end{corollary}

\begin{proof}
	Аналогично случаю кольца главных идеалов.
\end{proof}

\begin{proposition}
	Пусть $\phi: K \to L$ "--- гомоморфизм коммутативных колец, причем $K$ "--- нетерово. Тогда $\phi(K)$ "--- тоже нетерово.
\end{proposition}

\begin{proof}
	Гомоморфизм $\phi$ осуществляет биекцию между идеалами в $K$, содержащими $\ke\phi$, и идеалами в $\phi(K)$. В частности, если идеал $I$ в $K$ конечнопорожден, то есть $I = (a_1, \dotsc, a_n)$, $a_1, \dotsc, a_n \in K$, то тогда $\phi(I) = (\phi(a_1), \dotsc, \phi(a_n))$.
\end{proof}

\begin{corollary}
	Если $I$ "--- идеал в нетеровом кольце $K$, то кольцо $K / I$ "--- тоже нетерово.
\end{corollary}

\begin{theorem}[Гильберта, о базисе]
	Если $K$ "--- нетерово кольцо, то $K[x]$ "--- тоже нетерово кольцо.
\end{theorem}

\begin{proof}
	Предположим, что в $K[x]$ есть идеал $I$, не являющийся конечнопорожденным. Выберем в $I$ многочлены следующего вида:
	\begin{align*}
		&f_0 \in I,~d_0 := \deg{f_0} = \min\{\deg{f} : f \in I\}\\
		&f_1 \in I\backslash(f_0),~d_1 := \deg{f_1} = \min\{\deg{f} : f \in I\backslash(f_0)\} \ge d_1\\
		&f_2 \in I\backslash(f_0, f_1),~d_2 := \deg{f_2} = \min\{\deg{f} : f \in I\backslash(f_0, f_1)\} \ge d_2\\
		&\dotsc
	\end{align*}
	
	Обозначим через $a_i$ старший коэффициент многочлена $f_i$ для всех $i \in \N \cup \{0\}$. Поскольку $K$ "--- нетерово кольцо, то $\exists m \in \N: \forall n \ge m: a_n \in (a_0, \dotsc, a_m)$. По предположению, $\exists n \ge m: f_n \not\in (f_0, \dotsc, f_m)$. Но $a_n \in (a_0, \dotsc, a_m)$, поэтому $\exists b_0, \dotsc, b_m \in K: a_n = b_0a_0 + \dotsb + b_ma_m$. Рассмотрим многочлен $g := f_n - (b_0f_0x^{d_n - d_0} + \dotsb + b_mf_mx^{d_n - d_m})$. По предположению, $g \in I\backslash(f_0, \dotsc, f_m)$, но коэффициент этого многочлена при $x^{d_n}$ равен $a_n - (b_0a_0 + \dotsb + b_ma_m) = 0$, поэтому $\deg{g} < \deg{f_n}$ --- противоречие.
\end{proof}