\subsection{Конечные расширения}

\begin{proposition}
	Пусть $L \supset F$ "--- конечное расширение. Тогда каждый элемент $\alpha \in L$ является алгебраическим над $F$.
\end{proposition}

\begin{proof}
	Расширение $L$ конечно, причем $L \supset F(\alpha) \supset F$, поэтому расширение $F(\alpha)$ также конечно.
\end{proof}

\begin{definition}
	Расширение $L \supset F$ называется \textit{алгебраическим}, если все его элементы являются алгебраическими над $F$.
\end{definition}

\begin{proposition}
	Пусть $L \supset F$ "--- конечное расширение. Тогда если $(\alpha_1, \dotsc, \alpha_n)$ "--- базис в $L$, то $L = F(\alpha_1, \dotsc, \alpha_n)$.
\end{proposition}

\begin{proof}
	С одной стороны, $F(\alpha_1, \dotsc, \alpha_n) \supset L$ по определению расширения поля $F$ элементами $\alpha_1, \dotsc, \alpha_n$. С другой стороны, очевидно, $F(\alpha_1, \dotsc, \alpha_n) \subset L$.
\end{proof}

\begin{proposition}
	Пусть $F \subset L \subset K$ "--- башня конечных расширений. Тогда выполнено равенство $[K : F] = [K : L][L : F]$.
\end{proposition}

\begin{proof}
	Пусть $(x_1, \dotsc, x_n)$ "--- базис в $L$ над $F$, $(y_1, \dotsc, y_m)$ "--- базис в $K$ над $L$. Докажем, что попарные произведения элементов этих базисов образуют базис в $K$ над $F$. Очевидно, что полученная система выражает все элементы из $K$ линейными комбинациями с коэффициентами из $F$.
	
	Проверим, что система линейно независима. Пусть $\sum_{i = 1}^n\sum_{j = 1}^m \lambda_{ij}x_iy_j = 0$ для некоторых $\lambda_{ij} \in F$. Тогда, в силу линейной независимости $(y_1, \dotsc, y_m)$ над $L$, для всех $j \in \{1, \dotsc, m\}$ выполнено $\sum_{i = 1}^n\lambda_{ij}x_i = 0$. Значит, в силу линейной независимости $(x_1, \dotsc, x_n)$ над $F$, для всех $i \in \{1, \dotsc, n\}$ и $j \in \{1, \dotsc, m\}$ выполнено $\lambda_{ij} = 0$.
\end{proof}

\begin{proposition}
	Пусть $F$ "--- поле. Для любого $f \in F[x]$ у $F$ есть конечное расширение, в котором $f$ раскладывается на линейные сомножители.
\end{proposition}

\begin{proof}
	Пусть $f = g_1\dotsm g_s$ "--- разложение $f$ на неприводимые над $F[x]$ сомножители. Как уже было доказано, в поле $L_1 := F[x] / (g_1) \supset F$ у многочлена $g_1$ есть корень. Над $L_1[x]$ часть многочленов в разложении многочлена $f$ распадется на неприводимые многочлены меньшей степени. Продолжим процесс и за конечное число шагов мы получим требуемое расширение, которое будет конечным в силу мультипликативности степени расширения.
\end{proof}

\begin{definition}
	\textit{Полем разложения} многочлена $f \in F[x]$ называется минимальное по включению расширение $K \supset F$ такое, что над $K[x]$ многочлен $f$ раскладывается на линейные сомножители.
\end{definition}

\begin{note}
	В силу последнего утверждения, поле разложения $K$ любого многочлена $f \in F[x]$ существует, причем неравенство $[K : F] \le \deg{f}!$ выполнено по построению.
\end{note}

\begin{example}
	Рассмотрим многочлен $x^3 - 2 \in \Q[x]$, корнями которого являются $\sqrt[3]{2}$, $\sqrt[3]{2}\omega$ и $\sqrt[3]{2}\omega^2$. Тогда при построении его поля разложения будет получена следующая последовательность расширений:
	\[\Q \subset \Q(\sqrt[3]{2}) \subset \Q(\sqrt[3]{2}, \sqrt[3]{2}\omega) \subset \Q(\sqrt[3]{2}, \sqrt[3]{2}\omega, \sqrt[3]{2}\omega^2)\]
	
	Заметим, что выполнены равенства $[\Q : \Q(\sqrt[3]{2})] = 3$, $[\Q(\sqrt[3]{2}) : \Q(\sqrt[3]{2}, \sqrt[3]{2}\omega)] = 2$ и $[\Q(\sqrt[3]{2}, \sqrt[3]{2}\omega) : \Q(\sqrt[3]{2}, \sqrt[3]{2}\omega, \sqrt[3]{2}\omega^2)] = 1$. Значит, поле разложения имеет степень $6$.
\end{example}