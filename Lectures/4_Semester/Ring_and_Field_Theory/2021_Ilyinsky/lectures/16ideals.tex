\subsection{Кольца главных идеалов}

\begin{definition}
	Пусть $K$ "--- коммутативное кольцо. Множество $I \subset K$ называется \textit{идеалом} кольца $K$, если:
	\begin{enumerate}
		\item $(I, +)$ "--- подгруппа в $(K, +)$
		\item $\forall a \in K: aI \subset I$
	\end{enumerate}
\end{definition}

\begin{example}
	Рассмотрим несколько примеров идеалов в коммутативном кольце $K$:
	\begin{itemize}
		\item $\{0\}, K$ "--- \textit{тривиальные} идеалы
		\item $(a) := \{ax : x \in K\}$, где $a \in K$, "--- \textit{главные} идеалы
		\item $(a_1, \dotsc, a_n) := \{a_1x_1 + \dotsb + a_nx_n : x_1, \dotsc, x_n \in K\}$, где $a_1, \dotsc, a_n \in K$, "--- \textit{конечнопорожденные} идеалы
	\end{itemize}
\end{example}

\begin{note}
	Заметим, что $(a)$ "--- это множество элементов, делящихся на $a$. Тогда:
	\begin{itemize}
		\item $\forall a, b \in K \backslash \{0\}: (a) \subset (b) \lra b \mid a$
		\item $\forall a, b \in K \backslash \{0\}: (a) = (b) \lra a \sim b$
		\item Если $K$ "--- евклидово, то $\forall a, b \in K \backslash \{0\}: (a, b) = ((a, b))$
	\end{itemize}
\end{note}

\begin{definition}
	Область целостности $K$ называется \textit{кольцом главных идеалов}, если любой идеал в $K$ является главным.
\end{definition}

\begin{theorem}
	Если $K$ "--- евклидово кольцо, то $K$ "--- кольцо главных идеалов.
\end{theorem}

\begin{proof}
	Пусть $I$ "--- идеал в $K$. Если $I = \{0\}$, то $I = (0)$. Если это не так, то в $I$ можно выбрать элемент $d \in I$ с минимальной нормой. Проверим, что тогда $I = (d)$. Рассмотрим произвольный элемент $x \in I \backslash \{0\}$ и разделим его на $d$ с остатком: $x = qd + r$. Случай $N(r) < N(d)$ невозможен, поскольку $r = x - qd \in I$, поэтому $r = 0$ и $d \mid x$.
\end{proof}

Теперь до конца раздела зафиксируем $K$ "--- кольцо главных идеалов.

\begin{proposition}
	Любая возрастающая последовательность идеалов в $K$ стабилизируется.
\end{proposition}

\begin{proof}
	Пусть $(a_0) \subset (a_1) \subset (a_2) \subset \dotsc$ "--- возрастающая последовательность идеалов в $K$. Проверим, что $I := \bigcup_{i = 0}^\infty (a_i)$ "--- тоже идеал. Действительно, если $b, c \in I$, то $b, c \in (a_k)$ для некоторого $k \in \N$, поэтому $b + c \in (a_k) \subset I$ и $\forall x \in K: bx, cx \in (a_k) \subset I$.
	
	Поскольку $K$ "--- кольцо главных идеалов, то $I = (y)$ для некоторого $y \in K$. Но тогда $y \in (a_m)$ для некоторого $m \in \N$, поэтому последовательность стабилизируется, начиная с номера $m$, то есть $\forall n \ge m: I_n = I_m$.
\end{proof}

\begin{proposition}
	У каждого элемента $x \in K \backslash \{0\}$ существует разложение.
\end{proposition}

\begin{proof}
	Предположим, что это неверно для элемента $a \in K \backslash \{0\}$. Тогда $a$ можно представить в виде $a = a_1b_1$, где $a, b_1 \not\in K^* \cup \{0\}$, и у $a_1$ тоже нет разложения. Повторяя этот процесс для $a_1, a_2, \dotsc$, получим последовательность идеалов $(a) \subset (a_1) \subset (a_2) \subset \dots$, причем все вложения в ней строгие к силу попарной неассоциированности порождающих элементов. Это противоречит тому, что последовательность должна стабилизироваться. Значит, разложение есть у каждого ненулевого элемента.
\end{proof}

\begin{proposition}
	Каждый неразложимый элемент $p$ в $K$ прост.
\end{proposition}

\begin{proof}
	Пусть $p \mid ab$ для некоторых $a, b \in K$. Положим $I := \{x \in K: p \mid xb\}$. Легко видеть, что $I$ "--- идеал, и $a, p \in I$. Поскольку любой идеал в $K$ "--- главный, то $I = (d)$ для некоторого $d \in K \backslash \{0\}$, причем $d \mid a, p$. Поскольку $p$ неразложим, то либо $d \sim 1$, тогда $I = K$ и $p \mid b$, либо $d \sim p$, тогда $p \mid a$.
\end{proof}

\begin{theorem}
	Кольцо $K$ факториально.
\end{theorem}

\begin{proof}
	В $K$ существует разложение для каждого ненулевого элемента, и каждый неразложимый элемент в $K$ прост, поэтому $K$ факториально.
\end{proof}

\begin{note}
	Доказать, что заданное кольцо не является евклидовым "--- это нетривиальная задача. Однако часто бывает легко проверить, что кольцо не является кольцом главных идеалов, рассмотрев конкретный идеал. Например, скоро мы покажем, что кольцо $\Q[x, y]$ факториально, но при этом очевидно, что идеал $(x, y)$ "--- не главный в $\Q[x, y]$.
\end{note}