\subsection{Неразложимые элементы}

В данном разделе зафиксируем кольцо $\Z[u]$, где $u \in \{i, \omega, \sqrt{2}i\}$. Норма в $\Z[u]$ мультипликативна, то есть $\forall a, b \in \Z[u]: N(ab) = N(a)N(b)$.

\begin{proposition}
	Элемент $x \in \Z[u]$ обратим $\lra$ $N(x) = 1$.
\end{proposition}

\begin{proof}~
	\begin{itemize}
		\item[$\ra$] Если $x$ обратим, то $N(x)N(x^{-1}) = N(xx^{-1}) = N(1) = 1$, откуда $N(x) = 1$ в силу целочисленности $N$
		\item[$\la$] Если $N(x) = 1$, то $\forall b \in \Z[u]: N(ab) = N(b)$ и, как уже было доказано, $x$ обратим
	\end{itemize}
\end{proof}

\begin{proposition}
	Если для $x \in \Z[u]$ выполнено $N(x) = p$, где $p$ "--- простое число, то $x$ неразложим в $\Z[u]$.
\end{proposition}

\begin{proof}
	Пусть $x = ab$, где $a, b \in \Z[u]$. Тогда $N(a)N(b) = N(ab) = N(x) = p$, откуда либо $N(a) = 1$, либо $N(b) = 1$. Значит, один из элементов $a, b$ "--- обратимый, поэтому $x$ неразложим.
\end{proof}

\begin{proposition}
	Если для $x \in \Z[u]$ выполнено $N(x) = p^2$, где $p$ "--- простое число, и в $\Z[u]$ нет элементов с нормой $p$, то $x$ неразложим в $\Z[u]$.
\end{proposition}

\begin{proof}
	Пусть $x = ab$, где $a, b \in \Z[u]$. Тогда $N(a)N(b) = N(ab) = N(x) = p^2$, и, поскольку в $\Z[u]$ нет элементов с нормой $p$, либо $N(a) = 1$, либо $N(b) = 1$. Следовательно, $x$ неразложим.
\end{proof}

\begin{proposition}
	Элемент $x$ неразложим в $\Z[u]$ $\lra$ $N(x)$ "--- простое число или $x \sim p$, где $p$ "--- простое число, неразложимое в $\Z[u]$.
\end{proposition}