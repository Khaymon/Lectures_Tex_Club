\subsection{Неразложимые элементы}

В данном разделе зафиксируем кольцо $\Z[u]$, где $u \in \{i, \omega, \sqrt{2}i\}$. Норма в $\Z[u]$ мультипликативна, то есть $\forall a, b \in \Z[u]: N(ab) = N(a)N(b)$.

\begin{proposition}
	Элемент $x \in \Z[u]$ обратим $\lra$ $N(x) = 1$.
\end{proposition}

\begin{proof}~
	\begin{itemize}
		\item[$\ra$] Если $x$ обратим, то $N(x)N(x^{-1}) = N(xx^{-1}) = N(1) = 1$, откуда $N(x) = 1$ в силу целочисленности и неотрицательности нормы $N$.
		\item[$\la$] Если $N(x) = 1$, то $\forall b \in \Z[u]: N(ab) = N(b)$. Тогда аналогично уже доказанному легко убедиться, что $x$ обратим.\qedhere
	\end{itemize}
\end{proof}

\begin{proposition}
	Если для $x \in \Z[u]$ выполнено $N(x) = p$, где $p$ "--- простое число, то $x$ неразложим в $\Z[u]$.
\end{proposition}

\begin{proof}
	Пусть $x = ab$, где $a, b \in \Z[u]$. Тогда $N(a)N(b) = N(ab) = N(x) = p$, откуда либо $N(a) = 1$, либо $N(b) = 1$. Значит, один из элементов $a, b$ "--- обратимый, поэтому $x$ неразложим.
\end{proof}

\begin{proposition}
	Если для $x \in \Z[u]$ выполнено $N(x) = p^2$, где $p$ "--- простое число, и в $\Z[u]$ нет элементов с нормой $p$, то $x$ неразложим в $\Z[u]$.
\end{proposition}

\begin{proof}
	Пусть $x = ab$, где $a, b \in \Z[u]$. Тогда $N(a)N(b) = N(ab) = N(x) = p^2$, и, поскольку в $\Z[u]$ нет элементов с нормой $p$, либо $N(a) = 1$, либо $N(b) = 1$. Следовательно, $x$ неразложим.
\end{proof}

\begin{theorem}
	Элемент $x$ неразложим в $\Z[u]$ $\lra$ либо $N(x)$ "--- простое число, либо $x \sim p$, где $p$ "--- простое число, неразложимое в $\Z[u]$.
\end{theorem}

\begin{proof}~
	\begin{itemize}
		\item[$\la$] Если $N(x)$ "--- простое число, то $x$ неразложим, как уже было доказано. Если же $x \sim p$, где $p$ "--- простое число, неразложимое в $\Z[u]$, то $N(x) = p^2$, причем в $\Z[u]$ нет элементов с нормой $p$. Значит, $x$ снова неразложим по уже доказанному.
		\item[$\ra$] Пусть $x$ неразложим. Тогда $x$ прост, и, поскольку $N(x) = x\overline{x}$, $x \mid p$, где $p$ "--- некоторый простой делитель $N(x)$, то есть $p = xw$ для некоторого $w \in \Z[u]$, и $N(x)N(w) \hm= N(p) = p^2$. Если $N(x) = 1$, то $x$ обратим, что невозможно. Если $N(x) = p$, то получено требуемое. Если же $N(x) = p^2$, то $N(w) = 1$, то есть $w$ обратим, поэтому $x \sim p$ и $p$ неразложим.\qedhere
	\end{itemize}
\end{proof}

\begin{note}
	Пусть $u$ "--- иррациональный корень многочлена $p \in \Z[x]$ второй степени со старшим коэффициентом, равным 1, $\overline{u}$ "--- другой корень этого же многочлена. Тогда на $\Z[u]$ можно определить функцию $N: \Z[u] \to \N \cup \{0\}$, $\forall z = a + bu \in \Z[u]: N(z) := |(a + bu)(a + b\overline{u})|$. Если эта функция окажется нормой на $\Z[u]$, то в этом кольце будут справедливы все утверждения, доказанные выше.
\end{note}

\begin{example}
	Легко заметить, что число 3 разложимо в $\Z[\omega]$, поскольку $3 = N(1 - \omega)$. Найдем, какие простые числа являются неразложимыми в $\Z[\omega]$:
	\begin{enumerate}
		\item Пусть $p$ "--- простое число вида $3k + 2$. Покажем, что $p$ неразложимо в $\Z[\omega]$. Пусть это не так, тогда $\exists a, b \in \Z: N(a + b\omega) = p$. Но $N(a + b\omega) = a^2 - ab + b^2 \equiv_3 (a + b)^2$, и, следовательно, $(a + b)^2 \equiv_3 p \equiv_3 2$, что невозможно.
		\item Пусть $p$ "--- простое число вида $3k + 1$. Покажем, что $p$ разложимо в $\Z[\omega]$. Для этого заметим следующее:
		\[\left(\frac{-3}p\right) = \left(\frac{-1}p\right)\left(\frac{3}p\right) = \left((-1)^{\frac{p-1}2}\right)^2\left(\frac p3\right) = \left(\frac13\right) = 1\]
		
		Значит, $\exists x \in \Z: x^2 \equiv_p -3$. Предположим, что элемент $p$ неразложим в $\Z[\omega]$, тогда он прост в $\Z[\omega]$, поэтому $p \mid x^2 + 3 = x^2 - (2\omega+1)^2 \ra p \mid (x + 1) + 2\omega$ или $p \mid (x - 1) - 2\omega$, но $p \nmid 2$ "--- противоречие.
	\end{enumerate}
	
	Таким образом, неразложимые простые числа в $\Z[\omega]$ "--- это в точности простые числа вида $3k + 2$.
\end{example}