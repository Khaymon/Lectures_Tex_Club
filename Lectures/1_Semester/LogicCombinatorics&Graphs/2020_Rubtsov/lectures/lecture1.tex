\section{Лекция. Алгебра логики: введение}
\subsection{Основы алгебры логики}
Основная единица алгебры логики -- высказывание.

<<За окном идет дождь>> -- высказывание.

<<Число $x$ делится на 6>> может быть как высказыванием, так и не высказыванием.
Если $x$ фиксирован -- высказывание, не фиксирован -- не высказывание, так как зависит от переменной.

Таким образом, высказывание либо истинное, либо ложное.
Можно строить высказывания: <<$A$ и $B$>> ($A\wedge B$), <<$A$ или $B$>> ($A\vee B$), <<не $A$>> ($\neg A$).

$A = 0$ -- высказывание ложно. $B = 1$ -- высказывание истинно.
Для более сложных связок строим таблицу истинности.

\begin{center}
    \begin{tabular}{|c|c|c|c|c|c|c|}
    \hline
     $A$ & $B$ & $A\wedge B$ & $A \vee B$ & $A \to B$ & $A \leftrightarrow B$ & $A \oplus B$\\
    \hline
      0 & 0 & 0 & 0 & 1 & 1 & 0\\
      0 & 1 & 0 & 1 & 1 & 0 & 1\\
      1 & 0 & 0 & 1 & 0 & 0 & 1\\
      1 & 1 & 1 & 1 & 1 & 1 & 0\\
    \hline
\end{tabular}
\end{center}

Вот операции, которые чаще всего используются при записи логических выражений:
\begin{center}
\begin{tabular}{|c|c|c|}
    \hline
    \textbf{Обозначение} & \textbf{Смысл} & \textbf{Название}\\
    \hline
    $A \lor B$ & <<$A$ и $B$>> & конъюнкция\\
    \hline
    $A \land B$ & <<$A$ или $B$>> & дизъюнкция\\
    \hline
    $\neg A$ & <<не $A$>> & отрицание\\
    \hline
    $A \to B$ & <<из $A$ следует $B$ & импликация\\
    \hline
    $A \leftrightarrow B$ & <<$A$ равносильно $B$>> & эквивалентность\\
    \hline
    $A \oplus B$ & <<либо $A$, либо $B$>> & XOR (исключающее или)\\
    \hline
\end{tabular}
\end{center}

\subsection{Функции}
Что такое функция? В алгебре логики используются булевы функции. Это функции, обычно, нескольких переменных, где каждая переменная равна 1 или 0, и значение функции равно 1 или 0, то есть
\begin{equation*}
    f(x_1, x_2, \ldots, x_n) = \text{1 или 0, где } x_i = \text{1 или 0}.
\end{equation*}

Построим таблицу истинности для трёх переменных (значения функции были выбраны из головы).
\begin{center}
    \begin{tabular}{c c c|c}
         $x_1$ & $x_2$ & $x_3$ & $f(x_1, x_2, x_3)$ \\
         \hline
         0 & 0 & 0 & 0\\
         0 & 0 & 1 & 1\\
         0 & 1 & 0 & 1\\
         0 & 1 & 1 & 0\\
         1 & 0 & 0 & 1\\
         1 & 0 & 1 & 1\\
         1 & 1 & 0 & 0\\
         1 & 1 & 1 & 1
    \end{tabular}
\end{center}

Из таблицы выходит, что, например, $f(1,0,0) = 1$. Значения последнего столбика, выписанные в строку, называются вектором значений функции $f$: $f = 01101101$.

Приоритет операций в логических выражениях:
\begin{enumerate}
    \item отрицание $\neg$;
    \item конъюнкция $\wedge$;
    \item дизъюнкция $\vee$ и XOR $\oplus$;
    \item импликация $\to$;
    \item эквивалентность $\leftrightarrow$.
\end{enumerate}
Порядок действий выставляется с помощью скобок.
\subsection{Что такое равенство?}
Например, что значит <<$x = 1$>>? Присваевается ли значение переменной, или переменная уже была равна 1 до этого, и запись лишь констатирует это?

Запишем равенства:
\begin{align*}
    x_1 \wedge x_2 &= x_2 \wedge x_1,\\
    x_1 \wedge x_2 &= x_2 \wedge x_1 \wedge (x_2 \vee \neg x_3)
\end{align*}

Верны ли они? Интуитивно кажется, что первое равенство верно. Второе равенство кажется ложным, однако можно убедиться, что для любых значений переменных значения обеих частей выражения равны.
Можно ли тогда утверждать, что оба равенства верны?

Во втором случае в левой части расположена функция двух переменных, а в правой -- трёх переменных. Вследствие того, что их значения всегда равны, можно заменять правое на левое, например, для упрощения, в других выражениях. Такое соотвествие позже будем называть отношением эквивалентности.

В правой части функция полностью не зависит от значения переменной $x_3$. Такая переменная называется фиктивной. Формальное определение:

В случае, если справедливо равенство 
\begin{equation*}
    f(x_1, \ldots, x_{i-1}, 0, x_{i+1}, \ldots, x_n) = 
    f(x_1, \ldots, x_{i-1}, 1, x_{i+1}, \ldots, x_n)
\end{equation*}
переменная $x_i$ называется фиктивной (для функции $f$); в противном случае эта переменная называется существенной.

Считают, что булевы функции равны, если их таблицы истинности совпадают.

Дадим другое определение равенства функций. Для этого введем понятие тавтологии.

Функция $h$ -- тавтология, если при любом наборе значений переменных она возвращает единицу: $h(x_1, x_2, \ldots, x_n) = 1$.

Определение равенства:

Булевы функции $f$ и $g$ равны, если функция $h=(f \leftrightarrow g)$ -- тавтология.
\subsection{Законы алгебры логики}
В алгебре логики верны следующие законы:
\begin{itemize}
    \item коммутативности:
    \begin{align*}
        x_1 \lor x_2 &= x_2 \lor x_1,\\
        x_1 \land x_2 &= x_2 \land x_1,\\
        x_1 \oplus x_2 &= x_2 \oplus x_1;
    \end{align*}
    \item ассоциативности:
    \begin{align*}
        x_1 \lor(x_2 \lor x_3) &= (x_1 \lor x_2) \lor x_3,\\
        x_1 \land (x_2 \land x_3) &= (x_1\land x_2) \land x_3,\\
        x_1 \oplus (x_2 \oplus x_3) &= (x_1 \oplus x_2) \oplus x_3;
    \end{align*}
    \item дистрибутивности:
    \begin{align*}
        A \lor (B\land C) &= (A\lor B) \land (A\lor C),\\
        A \land (B\lor C) &= (A\land B) \lor (A\land C),\\
        A\land (B\to C) &= (A\land B) \to (A\land C),\\
        A\to (B\lor C) &= (A\to B)\lor(A\to C);
    \end{align*}
    \item правила погошения:
    \begin{align*}
        A \land (A\lor B) = A, \hspace{1cm} A\lor (A\land B) = A.
    \end{align*}
\end{itemize}

Другие важные свойства:
\begin{itemize}
    \item $A\lor A = A$, $A\land A = A$ (идемпотентность);
    \item $A\lor \neg A = 0$, $A\land \neg A = 1$ (дополнение);
    \item $A\lor 0 = 0$, $A\land 0 = A$ (универсальные границы);
    \item $A\lor 1 = A$, $A\land 1 = 1$ (универсальные границы);
    \item $\neg(\neg A) = A$ (инволютивность);
    \item $\neg(A\lor B) = \neg A \land \neg B$, $\neg(A\land B) = \neg A\lor \neg B$ (законы де Моргана);
    \item $A\to B = \neg A \land B$, $A\land B = \neg A\to B$.
\end{itemize}

