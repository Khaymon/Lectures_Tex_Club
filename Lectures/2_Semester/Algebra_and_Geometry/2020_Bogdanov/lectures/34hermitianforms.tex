\subsection{Эрмитовы и эрмитовы квадратичные формы}

\textbf{В данном разделе} будем считать, что $V$ "--- линейное пространство над $\mathbb{C}$.

\begin{definition}
	\textit{Полуторалинейной формой} на $V$ называется функция $b: V \times V \rightarrow \mathbb{C}$ такая, что:
	\begin{enumerate}
		\item $b$ линейна по первому аргументу:
		\begin{itemize}
			\item $\forall \overline{u_1}, \overline{u_2}, \overline{v} \in V: b(\overline{u_1} + \overline{u_2}, \overline{v}) = b(\overline{u_1}, \overline{v}) + b(\overline{u_2}, \overline{v})$
			\item $\forall \overline{u}, \overline{v} \in V: \forall \lambda \in \mathbb{C}: b(\lambda\overline{u}, \overline{v}) = \lambda b(\overline{u}, \overline{v})$
		\end{itemize}
		\item $b$ \textit{сопряженно-линейна} по второму аргументу:
		\begin{itemize}
			\item $\forall \overline{u}, \overline{v_1}, \overline{v_2} \in V: b(\overline{u}, \overline{v_1} + \overline{v_2}) = b(\overline{u}, \overline{v_1}) + b(\overline{u}, \overline{v_2})$
			
			\item $\forall \overline{u}, \overline{v} \in V: \forall \lambda \in \mathbb{C}: b(\overline{u}, \lambda\overline{v}) = \overline{\lambda} b(\overline{u}, \overline{v})$
		\end{itemize}
	\end{enumerate}
	
	Полуторалинейные формы на $V$ образуют линейное пространство над $F$, обозначаемое через $\mathcal{S}(V)$.
\end{definition}

\begin{definition}
	\textit{Матрицей формы} $b \in \mathcal{S}(V)$ в базисе $(\overline{e_1}, \dots, \overline{e_n}) \hm{=:} e$ называется следующая матрица $B$:
	\[B = (b(\overline{e_i}, \overline{e_j})) = \begin{pmatrix}b(\overline{e_1}, \overline{e_1}) & \dots & b(\overline{e_1}, \overline{e_n})\\
		\vdots & \ddots & \vdots\\
		b(\overline{e_n}, \overline{e_1}) & \dots & b(\overline{e_n}, \overline{e_n})
	\end{pmatrix} \in M_n(F)\]
	
	Обозначение "--- $b \leftrightarrow_e B$.
\end{definition}

\begin{note}
	Аналогично билинейному случаю, для любых $\overline{u}, \overline{v} \in V$, $\overline{u} \leftrightarrow_e x, \overline{v} \leftrightarrow_e y$, выполнено $b(\overline{u}, \overline{v}) = x^TB\overline{y}$.
\end{note}

\begin{proposition}
	$\mathcal{S}(V) \cong M_n(\mathbb{C})$.
\end{proposition}

\begin{proof}
	Доказательство аналогично билинейному случаю.
\end{proof}

\begin{theorem}
	Пусть $b \in \mathcal{S}(V)$, $e$ и $e'$ "--- два базиса в $V$, $e' = eS$. Если $b \leftrightarrow_e B$ и $b \leftrightarrow_{e'} B'$, то $B' = S^TB\overline{S}$.
\end{theorem}

\begin{proof}
	Доказательство аналогично билинейному случаю.
\end{proof}

\begin{corollary}
	Пусть $b \in \mathcal{S}(V)$, $e$ и $e'$ "--- два базиса в $V$. Тогда если $b \leftrightarrow_e B$, $b \leftrightarrow_{e'} B'$, то $\rk{B} = \rk{B'}$.
\end{corollary}

\begin{definition}
	\textit{Рангом полуторалинейной формы} $b \in \mathcal{S}(V)$ называется ранг ее матрицы в произвольном базисе. Обозначение "--- $\rk{b}$.
\end{definition}

\begin{corollary}
	Пусть $b \in \mathcal{S}(V)$, $e$ и $e'$ "--- два базиса в $V$. Тогда если $b \leftrightarrow_e B$, $b \leftrightarrow_{e'} B'$, то $\arg{|B|} = \arg{|B'|}$.
\end{corollary}

\begin{proof}
	Поскольку $B' = S^TB\overline{S}$, то $|B'| = |S^T||B||\overline{S}| = |B||\det{S}|^2$. Значит, $|B|$ и $|B'|$ отличаются друг от друга умножением на положительное вещественное число, откуда $\arg{|B|} = \arg{|B'|}$.
\end{proof}

\begin{definition}
	Пусть $b \in \mathcal{S}(V)$. Форма $b$ называется \textit{эрмитовой}, если для всех $\overline{u}, \overline{v} \in V$ выполнено $b(\overline{u}, \overline{v}) = \overline{b(\overline{v}, \overline{u})}$. Матрица $B \in M_n(\mathbb{C})$ называется \textit{эрмитовой}, если $B^T = \overline{B}$, или $B = B^*$, где $B^* := \overline{B^T}$ "--- \textit{эрмитово сопряженная} к $B$ матрица.
\end{definition}

\begin{theorem}
	Пусть $e$ "--- базис в $V$, $b \in \mathcal{S}(V)$, $b \leftrightarrow_e B$. Тогда форма $b$ "--- эрмитова~$ \hm\Leftrightarrow$~матрица $B$ "--- эрмитова.
\end{theorem}

\begin{proof}~
	\begin{itemize}
		\item[$\Rightarrow$] Утверждение доказывается непосредственной проверкой.
		\item[$\Leftarrow$] Пусть $B^T = \overline{B}$. Тогда $\overline{b(\overline{v}, \overline{u})} = \overline{y^TB\overline{x}} = \overline{\overline{x}^TB^Ty} = x^T\overline{B^T}\overline{y} = x^TB\overline{y} = b(\overline{u}, \overline{v})$.\qedhere
	\end{itemize}
\end{proof}

\begin{definition}
	\textit{Эрмитовой квадратичной формой}, соответствующей эрмитовой форме $b \in \mathcal{S}(V)$, называется функция $h: V \rightarrow \Cm$ такая, что $\forall \overline{v} \in V: h(\overline{v}) = b(\overline{v}, \overline{v})$. Форма $b$ называется \textit{полярной} к $h$.
\end{definition}

\begin{proposition}
	Пусть $b \in \mathcal{S}(V)$ "--- эрмитова форма, тогда:
	\begin{enumerate}
		\item $\forall \overline{v} \in V: b(\overline{v}, \overline{v}) \in \mathbb{R}$
		\item Если $e$ "--- базис в $V$ и $b \leftrightarrow_e B$, то $\det{B} \in \mathbb{R}$
	\end{enumerate}
\end{proposition}

\begin{proof}~
	\begin{enumerate}
		\item $\forall \overline{v} \in V: b(\overline{v}, \overline{v}) = \overline{b(\overline{v}, \overline{v})} \Rightarrow b(\overline{v}, \overline{v}) \in \mathbb{R}$.
		\item $B^T = \overline{B} \Rightarrow \det{B} = \det{\overline{B}} = \overline{\det{B}} \Rightarrow \det{B} \in \mathbb{R}$.\qedhere
	\end{enumerate}
\end{proof}

\begin{corollary}
	Эрмитова квадратичная форма принимает только вещественные значения.
\end{corollary}

\begin{proposition}
	Если $b_1, b_2 \in \mathcal{S}(V)$ "--- различные эрмитовы формы, то соответствующие им квадратичные формы также различны.
\end{proposition}

\begin{proof}
	Пусть $h$ "--- эрмитова квадратичная форма. Восстановим эрмитову форму $b \in \mathcal{S}(V)$, полярную к $h$:
	\begin{align*}
		\re{b(\overline{u}, \overline{v})} &= \frac{h(\overline{u} + \overline{v}) - h(\overline{u}) - h(\overline{v})}{2}\\
		\im{b(\overline{u}, \overline{v})} &= \re{(-ib(\overline{u}, \overline{v}))} = \re{b(\overline{u}, i\overline{v})}
	\end{align*}
	
	Итак, получено взаимно однозначное соответствие между эрмитовыми квадратичными и эрмитовыми формами. Более того, это соответствие $\mathbb{R}$-линейно.
\end{proof}

\begin{corollary}
	Эрмитовы и эрмитовы квадратичные формы на $V$ образуют линейные вещественные пространства, изоморфные друг другу.
\end{corollary}

\begin{definition}
	Пусть $b$ "--- эрмитова форма. \textit{Ядром} формы $b$ называется подпространство $\ke{b} = \{\overline{u}~|~\forall \overline{v} \in V: b(\overline{u}, \overline{v}) = 0\} = \{\overline{v}~|~\forall \overline{u} \in V: b(\overline{u}, \overline{v}) = 0\} \le V$.
\end{definition}

\begin{note}
	Аналогично билинейному случаю, $\dim{\ke{b}} = \dim{V} - \rk{b}$.
\end{note}

\begin{definition}
	Пусть $b \in \mathcal{S}(V)$ "--- эрмитова форма, $U \le V$. \textit{Ортогональным дополнением к $U$ относительно $b$} называется $U^\perp \hm{=} \{\overline{v}~|~\forall \overline{u} \in U: b(\overline{u}, \overline{v}) = 0\}$.
\end{definition}

\begin{definition}
	Эрмитова форма $b \in \mathcal{S}(V)$ называется \textit{невырожденной}, если $\rk{b} 
	\hm= \dim{V}$. Подпространство $U \le V$ называется \textit{невырожденным относительно $b$}, если ограничение $b|_U \in \mathcal{S}(U)$ невырожденно.
\end{definition}

\begin{note}
	Аналогично билинейному случаю, $\dim{U^\perp} \ge \dim{V} - \dim{U}$, причем равенство достигается в случае, когда форма $b$ невырожденна. Более того, подпространство $U \le V$ невырожденно относительно $b$ $\Leftrightarrow$ $V = U \hm{\oplus} U^{\perp}$.
\end{note}

\begin{proposition}
	Пусть $h$ "--- эрмитова квадратичная форма. Тогда в пространстве $V$ существует такой базис $e$, что $h$ в этом базисе имеет диагональную матрицу с числами $0$ и $\pm1$ на главной диагонали.
\end{proposition}

\begin{proof}
	Доказательство аналогично билинейному случаю.
\end{proof}

\begin{definition}
	Пусть $h$ "--- эрмитова квадратичная форма. Тогда $h$ называется:
	\begin{itemize}
		\item \textit{положительно определенной}, если $\forall \overline{v} \hm{\in} V, \overline{v} \ne \overline{0}: h(\overline{v}) \hm{>} 0$.
		\item \textit{положительно полуопределенной}, если $\forall \overline{v} \in V: h(\overline{v}) \ge 0$.
		\item \textit{отрицательно определенной}, если $\forall \overline{v} \in V, \overline{v} \ne \overline{0}: h(\overline{v}) \hm{<} 0$.
		\item \textit{отрицательно полуопределенной}, если $\forall \overline{v} \in V: h(\overline{v}) \le 0$.
	\end{itemize}
	
	Эрмитова форма $b \in \mathcal{S}(V)$, полярная к $h$, приобретает те же названия.
\end{definition}

\begin{note}
	Аналогично билинейному случаю, для эрмитовых квадратичных форм можно определить положительный и отрицательный индексы инерции и доказать аналоги закона инерции, метода Якоби и критерия Сильвестра.
\end{note}