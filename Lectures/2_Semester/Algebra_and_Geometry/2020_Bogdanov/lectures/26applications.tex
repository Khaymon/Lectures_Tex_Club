\subsection{Некоторые приложения жордановой нормальной формы}

\begin{note}
	Доказанные теоремы дают возможность для любой пары матриц $A$ и $B$ с характеристическими многочленами вида $\eqref*{charpol}$ определить, подобны они или нет: если их жордановы формы $A'$ и $B'$ совпадают с точностью до перестановки клеток, они подобны, иначе "--- нет (поскольку жорданова нормальная форма оператора единственна).
\end{note}

\begin{proposition}
	Пусть $\phi \in \mathcal{L}(V)$, $\chi_\phi$ имеет вид $\eqref*{charpol}$. Тогда его минимальный многочлен имеет вид $\mu_\phi = \prod_{i = 1}^{k}(\lambda_i - \lambda)^{\beta_i}$,
	где $\beta_i$ "--- наибольший размер клетки c собственным значением $\lambda_i$ в жордановой нормальной форме оператора $\phi$.
\end{proposition}

\begin{proof}
	Уже было доказано, что $\mu_\phi$ имеет такой вид, как в утверждении, с показателями степеней $\beta_i \le \alpha_i$ Для завершения доказательства остается лишь определить, чему равны $\beta_i$. 	Заметим, что наблюдение о независимом возведении жордановых клеток в степень можно обобщить на случай произвольного многочлена $p \in F[x]$:
	\[p\left(\begin{array}{@{}cccc@{}}
		\cline{1-1}
		\multicolumn{1}{|c|}{J_{k_1}} & 0 & \dots & 0\\
		\cline{1-2}
		0 & \multicolumn{1}{|c|}{J_{k_2}} & \dots & 0\\
		\cline{2-2}
		\vdots & \vdots & \ddots & \vdots\\
		\cline{4-4}
		0 & 0 & \dots & \multicolumn{1}{|c|}{J_{k_s}}\\
		\cline{4-4}
	\end{array}\right) = \left(\begin{array}{@{}cccc@{}}
		\cline{1-1}
		\multicolumn{1}{|c|}{p(J_{k_1})} & 0 & \dots & 0\\
		\cline{1-2}
		0 & \multicolumn{1}{|c|}{p(J_{k_2})} & \dots & 0\\
		\cline{2-2}
		\vdots & \vdots & \ddots & \vdots\\
		\cline{4-4}
		0 & 0 & \dots & \multicolumn{1}{|c|}{p(J_{k_s})}\\
		\cline{4-4}
	\end{array}\right)\]
	
	Значит, $p$ "--- аннулирующий $\Leftrightarrow$ $p$ обнуляет каждую клетку жордановой нормальной формы. Пусть $s_1$ "--- наибольший размер клетки, соответствующий $\lambda_1$ в жордановой нормальной форме $\phi$, тогда:
	\[0 = \mu_\phi(J_{s_1}(\lambda_1)) = \prod_{i = 1}^k(\lambda_iE - J_{s_1}(\lambda_1))^{\beta_i} = (-J_{s_1})^{\beta_1}\prod_{i = 2}^k(\lambda_iE - J_{s_1}(\lambda_1))^{\beta_i}\]
	
	Поскольку все матрицы в произведении кроме первой невырожденные, их произведение тоже невырожденное, поэтому необходимо, чтобы $(J_{s_1})^{\beta_1} = 0$, значит, $\beta_1 \ge s_1$. С другой стороны, $\beta_1 = s_1$ достаточно, чтобы обнулить все клетки, соответствующие $\lambda_1$. Аналогичные рассуждения справедливы для остальных $\beta_i$.
\end{proof}

\begin{proposition}
	Если минимальный многочлен $\mu_\phi$ оператора $\phi \in \mathcal{L}(V)$ раскладывается на линейные сомножители, то и $\chi_\phi$ раскладывается на линейные сомножители, то есть имеет вид $\eqref*{charpol}$.
\end{proposition}

\begin{proof}
	Минимальный многочлен, конечно, является аннулирующим, причем $\mu_\phi = \prod_{i = 1}^k(\lambda_i - \lambda)^{\gamma_i}$, поэтому, как уже было доказано, $V = V_1 \oplus \dots \oplus V_k$, где $V_i = \ke{(\phi_{\lambda_i})^{\gamma_i}}$, причем все $V_i$ "--- инвариантные. Тогда справедливы дальнейшие рассуждения доказательства существования жордановой нормальной формы у оператора $\phi$, и, поскольку жорданова нормальная форма "--- это верхнетреугольная матрица, то $\chi_\phi$ раскладывается на линейные сомножители.
\end{proof}

\begin{note}
	Жорданова нормальная форма позволяет быстро находить $A^k$, если $\chi_A$ имеет вид $\eqref*{charpol}$. Действительно, если у $B$ "--- жорданова нормальная форма $A$, то для некоторой $S \in \GL_n(F)$ выполнено $A \hm{=} S^{-1}BS \hm{\Rightarrow} A^n = S^{-1}B^nS$. Поскольку жордановы клетки возводятся в степень независимо, достаточно уметь находить степень каждой из них. Для этого заметим следующее:
	\[(J_k(\lambda_0))^n = (\lambda_0E + J_k)^n = \sum_{i = 0}^nC_n^i\lambda_0^{n - i}J_k^i = \sum_{i = 0}^{k - 1}C_n^i\lambda_0^{n - i}J_k^i\]
	
	Бином Ньютона для данных двух матриц справедлив, поскольку они коммутируют.
\end{note}

\begin{note}
	Можно непосредственно убедиться, что справедливо более сильное утверждение: если $\cha{F} = 0$, то $\forall p \in F[x]: p(A) \hm{=} S^{-1}p(B)S$, причем для каждой жордановой клетки $J_k(\lambda)$ в $B$ справедлива следующая формула:
	\[p(J_k(\lambda)) = p(\lambda)E + p'(\lambda)J_k + \frac{p''(\lambda)}{2}J_k^2 + \dots + \frac{p^{(k-1)}(\lambda)}{(k-1)!}J_k^{k-1}\]
\end{note}